%% 
%% This is file, `file.tex',
%% generated with the extract package.
%% 
%% Generated on :  2015/07/31,19:14
%% From source  :  mpFormulaC.tex
%% Using options:  active,generate=file,extract-cmd={chapter,section},extract-env={mpFunctionsExtract}
%% 
\documentclass[12pt,a4paper,openany]{book}

\begin{document}

\chapter{Preface}

\chapter{Introduction}

\section{Overview: Features and Setup}

\section{License}

\section{No Warranty}

\section{Related Software}

\chapter{Tutorials}

\section{Why multi-precision arithmetic?}

\section{Graphics using Latex}

\section{Graphics using .NET Framework}

\chapter{Python: Built-in numerical types}

\section{Truth Value Testing}

\section{Boolean Operations: and, or, not}

\section{Comparisons}

\section{Numeric Types - int, float, complex}

\section{Long integers}

\section{Fractions}

\chapter{Basic Usage}

\section{Number types}

\section{Precision and representation issues}

\section{Conversion and printing}

\section{Rounding}

\begin{mpFunctionsExtract}
\mpFunctionOne
{ceil? mpNum?  a number down to the nearest integer.}
{x? mpNum? A real number.}
\end{mpFunctionsExtract}

\begin{mpFunctionsExtract}
\mpFunctionOne
{floor? mpNum?  a number down to the nearest integer.}
{x? mpNum? A real number.}
\end{mpFunctionsExtract}

\begin{mpFunctionsExtract}
\mpFunctionOne
{nint? mpNum?  a number down to the nearest integer.}
{x? mpNum? A real number.}
\end{mpFunctionsExtract}

\begin{mpFunctionsExtract}
\mpFunctionOne
{frac? mpNum? the fractional part of $x$.}
{x? mpNum? A colpmex or real number.}
\end{mpFunctionsExtract}

\section{Components of Real and Complex Numbers}

\begin{mpFunctionsExtract}
\mpFunctionTwo
{ldexp? mpNum? $x \cdot 2^{y}$}
{x? mpNum? A real number.}
{y? mpNum? A real number.}
\end{mpFunctionsExtract}

\begin{mpFunctionsExtract}
\mpFunctionOne
{frexp? mpNumList? returns simultaneously significand and exponent of $x$}
{x? mpNum? A real number.}
\end{mpFunctionsExtract}

\begin{mpFunctionsExtract}
\mpFunctionTwo
{mpc? mpNum? a complex number $z$ build from the real components $x$ and $y$ as $z=x+iy$.}
{x? mpNum? A real number.}
{y? mpNum? A real number.}
\end{mpFunctionsExtract}

\begin{mpFunctionsExtract}
\mpFunctionOne
{polar? mpNum? Returns the polar representation of the complex number $z$.}
{z? mpNum? A complex or real number.}
\end{mpFunctionsExtract}

\begin{mpFunctionsExtract}
\mpFunctionTwo
{rect? mpNum? the complex number represented by polar coordinates $(r,\phi)$.}
{x? mpNum? A real number.}
{y? mpNum? A real number.}
\end{mpFunctionsExtract}

\begin{mpFunctionsExtract}
\mpFunctionOne
{re? mpNum? the real part of $x$, $\Re(x)$.}
{z? mpNum? A complex number.}
\end{mpFunctionsExtract}

\begin{mpFunctionsExtract}
\mpFunctionOne
{im? mpNum? the imaginary part of $x$, $\Im(x)$.}
{z? mpNum? A complex number.}
\end{mpFunctionsExtract}

\begin{mpFunctionsExtract}
\mpFunctionOne
{abs? mpNum? the absolute value of $z=x+iy$}
{z? mpNum? A real or complex number.}
\end{mpFunctionsExtract}

\begin{mpFunctionsExtract}
\mpFunctionOne
{fabs? mpNum? the absolute value of $z=x+iy$}
{z? mpNum? A real or complex number.}
\end{mpFunctionsExtract}

\begin{mpFunctionsExtract}
\mpFunctionOne
{arg? mpNum? the argument of $z=x+iy$}
{z? mpNum? A complex number.}
\end{mpFunctionsExtract}

\begin{mpFunctionsExtract}
\mpFunctionOne
{phase? mpNum? the argument of $z=x+iy$}
{z? mpNum? A complex number.}
\end{mpFunctionsExtract}

\begin{mpFunctionsExtract}
\mpFunctionOne
{sign? mpNum? the value of the sign of $x, \text{sign}(x)$.}
{x? mpNum? A real or complex number.}
\end{mpFunctionsExtract}

\begin{mpFunctionsExtract}
\mpFunctionOne
{conj? mpNum? the complex conjugate of $z$, $\overline{z}$}
{z? mpNum? A complex number.}
\end{mpFunctionsExtract}

\section{Arithmetic operations}

\begin{mpFunctionsExtract}
\mpFunctionThree
{fadd? mpNum? the sum of the numbers x and y, giving a floating-point result, optionally using a custom precision and rounding mode..}
{x? mpNum? A complex number.}
{y? mpNum? A complex number.}
{Keywords? String? prec, dps, exact, rounding.}
\end{mpFunctionsExtract}

\begin{mpFunctionsExtract}
\mpFunctionTwo
{fsum? mpNum? the sum of the numbers x and y, giving a floating-point result, optionally using a custom precision and rounding mode..}
{terms? mpNum? a finite number of terms.}
{Keywords? String? absolute=False, squared=False.}
\end{mpFunctionsExtract}

\begin{mpFunctionsExtract}
\mpFunctionThree
{fsub? mpNum? the sum of the numbers x and y, giving a floating-point result, optionally using a custom precision and rounding mode..}
{x? mpNum? A complex number.}
{y? mpNum? A complex number.}
{Keywords? String? prec, dps, exact, rounding.}
\end{mpFunctionsExtract}

\begin{mpFunctionsExtract}
\mpFunctionTwo
{fneg? mpNum? the sum of the numbers x and y, giving a floating-point result, optionally using a custom precision and rounding mode..}
{x? mpNum? A complex number.}
{Keywords? String? prec, dps, exact, rounding.}
\end{mpFunctionsExtract}

\begin{mpFunctionsExtract}
\mpFunctionThree
{fmul? mpNum? the sum of the numbers x and y, giving a floating-point result, optionally using a custom precision and rounding mode..}
{x? mpNum? A complex number.}
{y? mpNum? A complex number.}
{Keywords? String? prec, dps, exact, rounding.}
\end{mpFunctionsExtract}

\begin{mpFunctionsExtract}
\mpFunctionTwo
{fprod? mpNum? a product containing a finite number of factors}
{factors? mpNum? a finite number of factors}
{Keywords? String? prec, dps, exact, rounding.}
\end{mpFunctionsExtract}

\begin{mpFunctionsExtract}
\mpFunctionTwo
{fdot? mpNum? a product containing a finite number of factors}
{factors? mpNum? a finite number of factors}
{Keywords? String? prec, dps, exact, rounding.}
\end{mpFunctionsExtract}

\begin{mpFunctionsExtract}
\mpFunctionThree
{fdiv? mpNum? the sum of the numbers x and y, giving a floating-point result, optionally using a custom precision and rounding mode..}
{x? mpNum? A complex number.}
{y? mpNum? A complex number.}
{Keywords? String? prec, dps, exact, rounding.}
\end{mpFunctionsExtract}

\begin{mpFunctionsExtract}
\mpFunctionTwo
{fmod? mpReal? the remainder of $x/y$}
{x? mpReal? A real number.}
{y? mpReal? A real number.}
\end{mpFunctionsExtract}

\section{Logical Operators }

\section{Comparison Operators and Sorting}

\begin{mpFunctionsExtract}
\mpFunctionTwo
{chop? mpNum? Chops off small real or imaginary parts, or converts numbers close to zero to exact zeros}
{x? mpNum? A real or complex number.}
{Keywords? String? tol=None}
\end{mpFunctionsExtract}

\begin{mpFunctionsExtract}
\mpFunctionThree
{almosteq? mpNum? Determine whether the difference between $s$ and $t$ is smaller than a given epsilon, either relatively or absolutely.}
{s? mpNum? A real or complex number.}
{t? mpNum? A real or complex number.}
{Keywords? String? rel\_eps=None, abs\_eps=None}
\end{mpFunctionsExtract}

\section{Properties of numbers}

\begin{mpFunctionsExtract}
\mpFunctionOne
{isnormal? Boolean?  Determine whether x is 'normal' in the sense of floating-point representation; that is, return False if x is zero, an infinity or NaN; otherwise return True. By extension, a complex number x is considered 'normal' if its magnitude is normal}
{Number1? mpNum? A real or complex number.}
\end{mpFunctionsExtract}

\begin{mpFunctionsExtract}
\mpFunctionOne
{isfinite? Boolean?  Return True if x is a finite number, i.e. neither an infinity or a NaN}
{Number1? mpNum? A real or complex number.}
\end{mpFunctionsExtract}

\begin{mpFunctionsExtract}
\mpFunctionOne
{isinf? Boolean?  True if the absolute value of x is infinite; otherwise return False}
{Number1? mpNum? A real or complex number.}
\end{mpFunctionsExtract}

\begin{mpFunctionsExtract}
\mpFunctionOne
{isnan? Boolean?  Return True if x is a NaN (not-a-number), or for a complex number, whether either the real or complex part is NaN; otherwise return False}
{Number1? mpNum? A real or complex number.}
\end{mpFunctionsExtract}

\begin{mpFunctionsExtract}
\mpFunctionTwo
{isint? Boolean? Return True if x is integer-valued; otherwise return False.}
{x? mpNum? A real number.}
{Kexwords? String? gaussian=False.}
\end{mpFunctionsExtract}

\begin{mpFunctionsExtract}
\mpFunctionOne
{mag? mpNum? Quick logarithmic magnitude estimate of a number.}
{x? mpNum? A real number.}
\end{mpFunctionsExtract}

\begin{mpFunctionsExtract}
\mpFunctionOne
{.nint\_distance? mpNum? Return $(n,d)$ where $n$ is the nearest integer to $x$ and $d$ is an estimate of $\log_2(|x-n|)$.}
{x? mpNum? A real number.}
\end{mpFunctionsExtract}

\section{Number generation}

\begin{mpFunctionsExtract}
\mpFunctionZero
{rand? mpNum? Returns an mpf with value chosen randomly from $[0,1)$. The number of randomly generated bits in the mantissa is equal to the working precision.}
\end{mpFunctionsExtract}

\begin{mpFunctionsExtract}
\mpFunctionTwo
{fraction? mpNum?  Given Python integers $(p,q)$, returns a lazy mpf representing the fraction $p/q$. The value is updated with the precision.}
{p? mpNum? an integer.}
{q? mpNum? an integer.}
\end{mpFunctionsExtract}

\begin{mpFunctionsExtract}
\mpFunctionThree
{arange? mpNum?  This is a generalized version of Python's range() function that accepts fractional endpoints and step sizes and returns a list of mpf instance.}
{a? mpNum? a real number.}
{b? mpNum? a real number.}
{h? mpNum? a real number.}
\end{mpFunctionsExtract}

\begin{mpFunctionsExtract}
\mpFunctionFour
{linspace? mpNum?  This is a generalized version of Python's range() function that accepts fractional endpoints and step sizes and returns a list of mpf instance.}
{a? mpNum? a real number.}
{b? mpNum? a real number.}
{h? mpNum? a real number.}
{Keywords? String? endpoint=True.}
\end{mpFunctionsExtract}

\section{Matrices}

\begin{mpFunctionsExtract}
\mpFunctionTwo
{matrix? mpNum?  This is a generalized version of Python's range() function that accepts fractional endpoints and step sizes and returns a list of mpf instance.}
{data? Object? an object specifying the matrix.}
{Keywords? String? random, random-symmetric, random-complex, random-hermitian, zeros, ones, eye, row-vector, col-vector, diagonal.}
\end{mpFunctionsExtract}

\begin{mpFunctionsExtract}
\mpFunctionThree
{MatrixAdd? mpNum? the sum of the numbers x and y, giving a floating-point result, optionally using a custom precision and rounding mode..}
{x? mpNum? A complex number.}
{y? mpNum? A complex number.}
{Keywords? String? prec, dps, exact, rounding.}
\end{mpFunctionsExtract}

\chapter{Elementary Functions}

\section{Constants}

\begin{mpFunctionsExtract}
\mpFunctionZero
{pi? mpNum?  pi: 3.14159...}
\end{mpFunctionsExtract}

\begin{mpFunctionsExtract}
\mpFunctionZero
{degree? mpNum?  degree = 1 deg = pi / 180: 0.0174533...}
\end{mpFunctionsExtract}

\begin{mpFunctionsExtract}
\mpFunctionZero
{e? mpNum?  the base of the natural logarithm, e = exp(1): 2.71828...}
\end{mpFunctionsExtract}

\begin{mpFunctionsExtract}
\mpFunctionZero
{phi? mpNum?  Golden ratio phi: 1.61803...}
\end{mpFunctionsExtract}

\begin{mpFunctionsExtract}
\mpFunctionZero
{euler? mpNum?  Euler's constant: 0.577216...}
\end{mpFunctionsExtract}

\begin{mpFunctionsExtract}
\mpFunctionZero
{catalan? mpNum?  Catalan's constant: 0.915966...}
\end{mpFunctionsExtract}

\begin{mpFunctionsExtract}
\mpFunctionZero
{apery? mpNum?  Apery's constant: 1.20206...}
\end{mpFunctionsExtract}

\begin{mpFunctionsExtract}
\mpFunctionZero
{khinchin? mpNum?  Khinchin's constant: 2.68545...}
\end{mpFunctionsExtract}

\begin{mpFunctionsExtract}
\mpFunctionZero
{glaisher? mpNum?  Glaisher's constant: 1.28243...}
\end{mpFunctionsExtract}

\begin{mpFunctionsExtract}
\mpFunctionZero
{mertens? mpNum?  Mertens' constant: 0.261497...}
\end{mpFunctionsExtract}

\begin{mpFunctionsExtract}
\mpFunctionZero
{twinprime? mpNum?  Twin prime constant: 0.660162...}
\end{mpFunctionsExtract}

\begin{mpFunctionsExtract}
\mpFunctionZero
{inf? mpNum? the value of the representation of  $+\infty$ in the current precision.}
\end{mpFunctionsExtract}

\begin{mpFunctionsExtract}
\mpFunctionZero
{nan? mpNum? the value of the representation of Not a Number (NaN) in the current precision.}
\end{mpFunctionsExtract}

\section{Exponential and Logarithmic Functions}

\begin{mpFunctionsExtract}
\mpFunctionOne
{exp? mpNum? the complex exponential of $z$}
{z? mpNum? A complex number.}
\end{mpFunctionsExtract}

\begin{mpFunctionsExtract}
\mpFunctionOne
{expj? mpNum?  $10^z$}
{z? mpNum? A complex number.}
\end{mpFunctionsExtract}

\begin{mpFunctionsExtract}
\mpFunctionOne
{expjpi? mpNum?  $10^z$}
{z? mpNum? A complex number.}
\end{mpFunctionsExtract}

\begin{mpFunctionsExtract}
\mpFunctionOne
{expm1? mpNum?  $10^z$}
{z? mpNum? A complex number.}
\end{mpFunctionsExtract}

\begin{mpFunctionsExtract}
\mpFunctionOne
{exp10? mpNum?  $10^z$}
{z? mpNum? A complex number.}
\end{mpFunctionsExtract}

\begin{mpFunctionsExtract}
\mpFunctionOne
{exp2? mpNum?  $2^z$}
{z? mpNum? A complex number.}
\end{mpFunctionsExtract}

\begin{mpFunctionsExtract}
\mpFunctionTwo
{Logb? mpNum? the value of the logarithm  to base $b$: $\text{logb}(x) = \log_{b}(x)$.}
{x? mpNum? A real number.}
{b? mpNum? A real number.}
\end{mpFunctionsExtract}

\begin{mpFunctionsExtract}
\mpFunctionTwo
{log? mpNum? the complex natural logarithm of $z$}
{z? mpNum? A complex number.}
{base? mpNum? the base of the logarithm. A real number.}
\end{mpFunctionsExtract}

\begin{mpFunctionsExtract}
\mpFunctionOne
{ln? mpNum? the complex natural logarithm of $z$}
{z? mpNum? A complex number.}
\end{mpFunctionsExtract}

\begin{mpFunctionsExtract}
\mpFunctionOne
{log10? mpNum? $\log_{10}(z)$}
{z? mpNum? A complex number.}
\end{mpFunctionsExtract}

\begin{mpFunctionsExtract}
\mpFunctionOne
{log2? mpNum? $\log_{2}(z)$}
{z? mpNum? A complex number.}
\end{mpFunctionsExtract}

\begin{mpFunctionsExtract}
\mpFunctionOne
{lnp1? mpNum? the value of the function $\ln(1+x)$.}
{x? mpNum? A real number.}
\end{mpFunctionsExtract}

\section{Roots and Power Functions}

\begin{mpFunctionsExtract}
\mpFunctionOne
{square? mpNum? the square of $z$.}
{z? mpNum? A complex number.}
\end{mpFunctionsExtract}

\begin{mpFunctionsExtract}
\mpFunctionTwo
{power? mpNum? an complex power of $z$}
{z1? mpNum? A complex number.}
{z2? mpNum? A complex number.}
\end{mpFunctionsExtract}

\begin{mpFunctionsExtract}
\mpFunctionTwo
{powm1? mpNum? an integer power of $z$}
{z? mpNum? A complex number.}
{k? mpNum?  A complex number.}
\end{mpFunctionsExtract}

\begin{mpFunctionsExtract}
\mpFunctionOne
{sqrt? mpNum? the square root of $z$}
{z? mpNum? A complex number.}
\end{mpFunctionsExtract}

\begin{mpFunctionsExtract}
\mpFunctionTwo
{hypot? mpNum? the value of $\sqrt{x^2+y^2}$.}
{x? mpNum? A real number.}
{y? mpNum? A real number.}
\end{mpFunctionsExtract}

\begin{mpFunctionsExtract}
\mpFunctionOne
{cbrt? mpNum? the square root of $z$}
{z? mpNum? A complex number.}
\end{mpFunctionsExtract}

\begin{mpFunctionsExtract}
\mpFunctionTwo
{root? mpNum? the value of the $n^{th}$ root of $x$, $\sqrt[n]{x}, n=2,3,...$.}
{z? mpNum? A complex number.}
{n? mpNum? An integer.}
\end{mpFunctionsExtract}

\begin{mpFunctionsExtract}
\mpFunctionTwo
{nthroot? mpNum? the value of the $n^{th}$ root of $x$, $\sqrt[n]{x}, n=2,3,...$.}
{n? mpNum? An integer.}
{y? mpNum? A real number.}
\end{mpFunctionsExtract}

\section{Trigonometric Functions}

\begin{mpFunctionsExtract}
\mpFunctionOne
{degrees? mpNum? the value of $x$ converted to degrees, with the input $x$ in radians.}
{x? mpNum? A real number.}
\end{mpFunctionsExtract}

\begin{mpFunctionsExtract}
\mpFunctionOne
{radians? mpNum? the value of $x$ converted to radians, with the input $x$ in degrees.}
{x? mpNum? A real number.}
\end{mpFunctionsExtract}

\begin{mpFunctionsExtract}
\mpFunctionOne
{sin? mpNum? complex sine of $z$}
{z? mpNum? A complex number.}
\end{mpFunctionsExtract}

\begin{mpFunctionsExtract}
\mpFunctionOne
{cos? mpNum? complex cosine of $z$}
{z? mpNum? A complex number.}
\end{mpFunctionsExtract}

\begin{mpFunctionsExtract}
\mpFunctionOne
{tan? mpNum? complex tangent of $z$}
{z? mpNum? A complex number.}
\end{mpFunctionsExtract}

\begin{mpFunctionsExtract}
\mpFunctionOne
{sec? mpNum? the complex secant of $z$}
{z? mpNum? A complex number.}
\end{mpFunctionsExtract}

\begin{mpFunctionsExtract}
\mpFunctionOne
{csc? mpNum? the complex cosecant of $z$}
{z? mpNum? A complex number.}
\end{mpFunctionsExtract}

\begin{mpFunctionsExtract}
\mpFunctionOne
{cot? mpNum? the complex cotangent of $z$}
{z? mpNum? A complex number.}
\end{mpFunctionsExtract}

\section{Hyperbolic Functions}

\begin{mpFunctionsExtract}
\mpFunctionOne
{sinh? mpNum? the complex hyperbolic sine of $z$}
{z? mpNum? A complex number.}
\end{mpFunctionsExtract}

\begin{mpFunctionsExtract}
\mpFunctionOne
{cosh? mpNum? the complex hyperbolic cosine of $z$}
{z? mpNum? A complex number.}
\end{mpFunctionsExtract}

\begin{mpFunctionsExtract}
\mpFunctionOne
{tanh? mpNum? the complex hyperbolic tangent of $z$}
{z? mpNum? A complex number.}
\end{mpFunctionsExtract}

\begin{mpFunctionsExtract}
\mpFunctionOne
{sech? mpNum? the complex hyperbolic secant of $z$}
{z? mpNum? A complex number.}
\end{mpFunctionsExtract}

\begin{mpFunctionsExtract}
\mpFunctionOne
{csch? mpNum? the complex hyperbolic cosecant of $z$}
{z? mpNum? A complex number.}
\end{mpFunctionsExtract}

\begin{mpFunctionsExtract}
\mpFunctionOne
{coth? mpNum? the complex hyperbolic cotangent of $z$}
{z? mpNum? A complex number.}
\end{mpFunctionsExtract}

\section{Inverse Trigonometric Functions}

\begin{mpFunctionsExtract}
\mpFunctionOne
{asin? mpNum? the inverse complex sine of $z$}
{z? mpNum? A complex number.}
\end{mpFunctionsExtract}

\begin{mpFunctionsExtract}
\mpFunctionOne
{acos? mpNum? the inverse complex cosine of $z$}
{z? mpNum? A complex number.}
\end{mpFunctionsExtract}

\begin{mpFunctionsExtract}
\mpFunctionOne
{atan? mpNum? the inverse complex tangent of $z$}
{z? mpNum? A complex number.}
\end{mpFunctionsExtract}

\begin{mpFunctionsExtract}
\mpFunctionTwo
{Atan2? mpNum? the value of the arc-tangent of $x$ in radians.}
{x? mpNum? A real number.}
{y? mpNum? A real number.}
\end{mpFunctionsExtract}

\begin{mpFunctionsExtract}
\mpFunctionOne
{acot? mpNum? the inverse complex cotangent of $z$}
{z? mpNum? A complex number.}
\end{mpFunctionsExtract}

\section{Inverse Hyperbolic Functions}

\begin{mpFunctionsExtract}
\mpFunctionOne
{asinh? mpNum? the inverse complex hyperbolic sine of $z$}
{z? mpNum? A complex number.}
\end{mpFunctionsExtract}

\begin{mpFunctionsExtract}
\mpFunctionOne
{acosh? mpNum? the inverse complex hyperbolic cosine of $z$}
{z? mpNum? A complex number.}
\end{mpFunctionsExtract}

\begin{mpFunctionsExtract}
\mpFunctionOne
{atanh? mpNum? the inverse complex hyperbolic tangent of $z$}
{z? mpNum? A complex number.}
\end{mpFunctionsExtract}

\begin{mpFunctionsExtract}
\mpFunctionOne
{acoth? mpNum? the inverse complex hyperbolic cotangent of $z$}
{z? mpNum? A complex number.}
\end{mpFunctionsExtract}

\chapter{Linear Algebra}

\section{Norms}

\begin{mpFunctionsExtract}
\mpFunctionTwo
{norm? mpNumList? the entrywise $p$-norm of an iterable x, i.e. the vector norm.}
{Y? mpNum[]? An array of real numbers.}
{Keywords? String?  p=2.}
\end{mpFunctionsExtract}

\begin{mpFunctionsExtract}
\mpFunctionTwo
{mnorm? mpNumList? the matrix (operator) $p$-norm of A. Currently p=1 and p=inf are supported.}
{A? mpNum[]? An array of real numbers.}
{Keywords? String?  p=1.}
\end{mpFunctionsExtract}

\section{Decompositions}

\begin{mpFunctionsExtract}
\mpFunctionTwo
{cholesky? mpNum? the Cholesky decomposition of a symmetric positive-definite matrix $A$.}
{A? mpNum[]? A symmetric matrix.}
{Keywords? String?  tol=None.}
\end{mpFunctionsExtract}

\section{Linear Equations}

\begin{mpFunctionsExtract}
\mpFunctionThree
{lu\_solve? mpNum? solves a linear equation system using a LU decomposition.}
{A? mpNum[]? A symmetric matrix.}
{b? mpNum[]? A symmetric matrix.}
{Keywords? String?  tol=None.}
\end{mpFunctionsExtract}

\begin{mpFunctionsExtract}
\mpFunctionFour
{residual? mpNum? the residual  $||Ax-b||$.}
{A? mpNum[]? A square matrix.}
{b? mpNum[]? A vector.}
{x? mpNum[]? A vector.}
{Keywords? String?  tol=None.}
\end{mpFunctionsExtract}

\section{Matrix Factorization}

\begin{mpFunctionsExtract}
\mpFunctionTwo
{lu? mpNum? an explicit LU factorization of a matrix, returning P, L, U}
{A? mpNum[]? A square matrix.}
{Keywords? String?  tol=None.}
\end{mpFunctionsExtract}

\begin{mpFunctionsExtract}
\mpFunctionTwo
{qr? mpNum? an explicit QR factorization of a matrix, returning Q, R}
{A? mpNum[]? A square matrix.}
{Keywords? String?  tol=None.}
\end{mpFunctionsExtract}

\chapter{Factorials and gamma functions}

\section{Factorials}

\begin{mpFunctionsExtract}
\mpFunctionOne
{Factorial? mpNum? the factorial, $x!$.}
{z? mpNum? A real or complex number.}
\end{mpFunctionsExtract}

\begin{mpFunctionsExtract}
\mpFunctionOne
{fac? mpNum? the factorial, $x!$.}
{z? mpNum? A real or complex number.}
\end{mpFunctionsExtract}

\begin{mpFunctionsExtract}
\mpFunctionOne
{fac2? mpNum? the double factorial $x!!$.}
{z? mpNum? A real or complex number.}
\end{mpFunctionsExtract}

\section{Binomial coefficient}

\begin{mpFunctionsExtract}
\mpFunctionTwo
{binomial? mpNum? the binomial coefficient.}
{n? mpNum? A real or complex number.}
{k? mpNum? A real or complex number.}
\end{mpFunctionsExtract}

\section{Pochhammer symbol, Rising and falling factorials}

\begin{mpFunctionsExtract}
\mpFunctionTwoNotImplemented
{RelativePochhammerMpMath? mpNum? the relative Pochhammer symbol.}
{a? mpNum? An integer.}
{x? mpNum? An integer.}
\end{mpFunctionsExtract}

\begin{mpFunctionsExtract}
\mpFunctionTwo
{rf? mpNum? the rising factorial.}
{x? mpNum? A real or complex number.}
{n? mpNum? A real or complex number.}
\end{mpFunctionsExtract}

\begin{mpFunctionsExtract}
\mpFunctionTwo
{ff? mpNum? the falling factorial.}
{x? mpNum? A real or complex number.}
{n? mpNum? A real or complex number.}
\end{mpFunctionsExtract}

\section{Super- and hyperfactorials}

\begin{mpFunctionsExtract}
\mpFunctionOne
{superfac? mpNum? the superfactorial.}
{z? mpNum? A real or complex number.}
\end{mpFunctionsExtract}

\begin{mpFunctionsExtract}
\mpFunctionOne
{hyperfac? mpNum? the hyperfactorial.}
{z? mpNum? A real or complex number.}
\end{mpFunctionsExtract}

\begin{mpFunctionsExtract}
\mpFunctionOne
{barnesg? mpNum? the Barnes G-function.}
{z? mpNum? A real or complex number.}
\end{mpFunctionsExtract}

\section{Gamma functions}

\begin{mpFunctionsExtract}
\mpFunctionOne
{gamma? mpNum? the gamma function, $\Gamma(x)$.}
{z? mpNum? A real or complex number.}
\end{mpFunctionsExtract}

\begin{mpFunctionsExtract}
\mpFunctionOne
{rgamma? mpNum? the reciprocal of the gamma function, $1/\Gamma(z)$.}
{z? mpNum? A real or complex number.}
\end{mpFunctionsExtract}

\begin{mpFunctionsExtract}
\mpFunctionTwo
{gammaprod? mpNum?  the  product / quotient of gamma functions.}
{a? mpNum? A real or complex iterables.}
{b? mpNum? A real or complex iterables.}
\end{mpFunctionsExtract}

\begin{mpFunctionsExtract}
\mpFunctionOne
{loggamma? mpNum? the principal branch of the log-gamma function, $\ln\Gamma(z)$.}
{z? mpNum? A real or complex number.}
\end{mpFunctionsExtract}

\begin{mpFunctionsExtract}
\mpFunctionFour
{gammainc? mpNum? the incomplete gamma function with integration limits $[a, b]$.}
{z? mpNum? A real or complex number.}
{a? mpNum? A real or complex number (default = 0).}
{b? mpNum? A real or complex number (default = inf).}
{Keywords? String?  regularized=False.}
\end{mpFunctionsExtract}

\begin{mpFunctionsExtract}
\mpFunctionTwoNotImplemented
{GammaPDerivativeMpMath? mpNum? the partial derivative with respect to $x$ of the incomplete gamma function $P(a,x)$.}
{a? mpNum? A real number.}
{x? mpNum? A real number.}
\end{mpFunctionsExtract}

\begin{mpFunctionsExtract}
\mpFunctionTwoNotImplemented
{GammaPMpMath? mpNum? the normalised incomplete gamma function $P(a,x)$.}
{a? mpNum? A real number.}
{x? mpNum? A real number.}
\end{mpFunctionsExtract}

\begin{mpFunctionsExtract}
\mpFunctionTwoNotImplemented
{GammaQMpMath? mpNum? the normalised incomplete gamma function $Q(a,x)$.}
{a? mpNum? A real number.}
{x? mpNum? A real number.}
\end{mpFunctionsExtract}

\begin{mpFunctionsExtract}
\mpFunctionTwoNotImplemented
{NonNormalisedGammaPMpMath? mpNum? the non-normalised incomplete gamma function $\Gamma(a,x)$.}
{a? mpNum? A real number.}
{x? mpNum? A real number.}
\end{mpFunctionsExtract}

\begin{mpFunctionsExtract}
\mpFunctionTwoNotImplemented
{NonNormalisedGammaQMpMath? mpNum? the non-normalised incomplete gamma function $\gamma(a,x)$.}
{a? mpNum? A real number.}
{x? mpNum? A real number.}
\end{mpFunctionsExtract}

\begin{mpFunctionsExtract}
\mpFunctionTwoNotImplemented
{TricomiGammaMpMath? mpNum? Tricomi's entire incomplete gamma function $\gamma^*(a,x)$.}
{a? mpNum? A real number.}
{x? mpNum? A real number.}
\end{mpFunctionsExtract}

\begin{mpFunctionsExtract}
\mpFunctionTwoNotImplemented
{GammaPinvMpMath? mpNum? the inverse of the normalised incomplete gamma function $P(a,x)$.}
{a? mpNum? A real number.}
{p? mpNum? A real number.}
\end{mpFunctionsExtract}

\begin{mpFunctionsExtract}
\mpFunctionTwoNotImplemented
{GammaQinvMpMath? mpNum? the inverse of the normalised incomplete gamma function $Q(a,x)$.}
{a? mpNum? A real number.}
{q? mpNum? A real number.}
\end{mpFunctionsExtract}

\section{Polygamma functions and harmonic numbers}

\begin{mpFunctionsExtract}
\mpFunctionTwo
{polygamma? mpNum? the polygamma function of order $m$ of $z$, $\psi^{(m)}(z)$.}
{m? mpNum? A real or complex number.}
{z? mpNum? A real or complex number.}
\end{mpFunctionsExtract}

\begin{mpFunctionsExtract}
\mpFunctionTwo
{psi? mpNum? the polygamma function of order $m$ of $z$, $\psi^{(m)}(z)$.}
{m? mpNum? A real or complex number.}
{z? mpNum? A real or complex number.}
\end{mpFunctionsExtract}

\begin{mpFunctionsExtract}
\mpFunctionOne
{digamma? mpNum? the digamma function.}
{z? mpNum? A real or complex number.}
\end{mpFunctionsExtract}

\begin{mpFunctionsExtract}
\mpFunctionOne
{harmonic? mpNum? a floating-point approximation of the $n$-th harmonic number $H(n)$.}
{n? mpNum? An  real or complex number.}
\end{mpFunctionsExtract}

\section{Beta Functions}

\begin{mpFunctionsExtract}
\mpFunctionTwo
{beta? mpNum? the beta function, $B(x,y)=\Gamma(x) \Gamma(y)/\Gamma(x+y)$.}
{x? mpNum? A real or complex number.}
{y? mpNum? A real or complex number.}
\end{mpFunctionsExtract}

\begin{mpFunctionsExtract}
\mpFunctionTwoNotImplemented
{LnBetaMpMath? mpNum? the logarithm of the beta function $\ln B(a,b)|$ with $a,b \neq 0,-1,-2,\ldots$.}
{a? mpNum? A real number.}
{b? mpNum? A real number.}
\end{mpFunctionsExtract}

\begin{mpFunctionsExtract}
\mpFunctionFive
{betainc? mpNum? the generalized incomplete beta function.}
{a? mpNum? A real or complex number.}
{b? mpNum? A real or complex number.}
{x1? mpNum? A real or complex number (default = 0).}
{x2? mpNum? A real or complex number (default = 1).}
{Keywords? String?  regularized=False.}
\end{mpFunctionsExtract}

\begin{mpFunctionsExtract}
\mpFunctionThreeNotImplemented
{IBetaNonNormalizedMpMath? mpNum? the non-normalised incomplete beta function.}
{a? mpNum? A real number.}
{b? mpNum? A real number.}
{x? mpNum? A real number.}
\end{mpFunctionsExtract}

\begin{mpFunctionsExtract}
\mpFunctionThreeNotImplemented
{IBetaMpMath? mpNum? the normalised incomplete beta function.}
{a? mpNum? A real number.}
{b? mpNum? A real number.}
{x? mpNum? A real number.}
\end{mpFunctionsExtract}

\chapter{Exponential integrals and error functions}

\section{Exponential integrals}

\begin{mpFunctionsExtract}
\mpFunctionOne
{ei? mpNum? the exponential integral.}
{z? mpNum? A real or complex number.}
\end{mpFunctionsExtract}

\begin{mpFunctionsExtract}
\mpFunctionOne
{e1? mpNum? the exponential integral $\text{E}_1(x)$.}
{z? mpNum? A real or complex number.}
\end{mpFunctionsExtract}

\begin{mpFunctionsExtract}
\mpFunctionTwo
{expint? mpNum? the generalized exponential integral or En-function.}
{n? mpNum? A real or complex number.}
{z? mpNum? A real or complex number.}
\end{mpFunctionsExtract}

\begin{mpFunctionsExtract}
\mpFunctionTwoNotImplemented
{GeneralizedExponentialIntegralEpMpMath? mpNum? the generalized exponential integrals $\text{E}_n(x)$ of real order $p$.}
{x? mpNum? A real number.}
{p? mpNum? A real number.}
\end{mpFunctionsExtract}

\section{Logarithmic integral}

\begin{mpFunctionsExtract}
\mpFunctionOne
{li? mpNum? the logarithmic integral.}
{z? mpNum? A real or complex number.}
\end{mpFunctionsExtract}

\section{Trigonometric integrals}

\begin{mpFunctionsExtract}
\mpFunctionOne
{ci? mpNum? the cosine integral.}
{z? mpNum? A real or complex number.}
\end{mpFunctionsExtract}

\begin{mpFunctionsExtract}
\mpFunctionOne
{si? mpNum? the sine integral.}
{z? mpNum? A real or complex number.}
\end{mpFunctionsExtract}

\section{Hyperbolic integrals}

\begin{mpFunctionsExtract}
\mpFunctionOne
{chi? mpNum? the hyperbolic cosine integral.}
{z? mpNum? A real or complex number.}
\end{mpFunctionsExtract}

\begin{mpFunctionsExtract}
\mpFunctionOne
{shi? mpNum? the hyperbolic sine integral.}
{z? mpNum? A real or complex number.}
\end{mpFunctionsExtract}

\section{Error functions}

\begin{mpFunctionsExtract}
\mpFunctionOne
{erf? mpNum? the error function, $\text{erf}(x)$.}
{z? mpNum? A real or complex number.}
\end{mpFunctionsExtract}

\begin{mpFunctionsExtract}
\mpFunctionOne
{erfc? mpNum? the complementary error function, $\text{erfc}(x)=1-\text{erf}(x)$.}
{z? mpNum? A real or complex number.}
\end{mpFunctionsExtract}

\begin{mpFunctionsExtract}
\mpFunctionOne
{erfi? mpNum? the imaginary error function, $\text{erfi}(x)$.}
{z? mpNum? A real or complex number.}
\end{mpFunctionsExtract}

\begin{mpFunctionsExtract}
\mpFunctionOne
{erfinv? mpNum? the inverse error function, $\text{erfinv}(x)$.}
{x? mpNum? A real number.}
\end{mpFunctionsExtract}

\section{The normal distribution}

\begin{mpFunctionsExtract}
\mpFunctionThree
{npdf? mpNum? the normal probability density function.}
{x? mpNum? A real number.}
{mu? mpNum? A real number.}
{sigma? mpNum? A real number.}
\end{mpFunctionsExtract}

\begin{mpFunctionsExtract}
\mpFunctionThree
{ncdf? mpNum? the normal cumulative distribution function.}
{x? mpNum? A real number.}
{mu? mpNum? A real number.}
{sigma? mpNum? A real number.}
\end{mpFunctionsExtract}

\section{Fresnel integrals}

\begin{mpFunctionsExtract}
\mpFunctionOne
{fresnels? mpNum? the Fresnel sine integral.}
{z? mpNum? A real or complex number.}
\end{mpFunctionsExtract}

\begin{mpFunctionsExtract}
\mpFunctionOne
{fresnelc? mpNum? the Fresnel cosine integral.}
{z? mpNum? A real or complex number.}
\end{mpFunctionsExtract}

\section{Other Special Functions}

\begin{mpFunctionsExtract}
\mpFunctionTwo
{lambertw? mpNum? the Lambert W function.}
{z? mpNum? A real or complex number.}
{Keywords? String? k=0.}
\end{mpFunctionsExtract}

\begin{mpFunctionsExtract}
\mpFunctionTwo
{agm? mpNum? the arithmetic-geometric mean of $a$ and $b$.}
{a? mpNum? A real or complex number.}
{b? mpNum? A real or complex number.}
\end{mpFunctionsExtract}

\chapter{Bessel functions and related functions}

\section{Bessel functions}

\begin{mpFunctionsExtract}
\mpFunctionTwoNotImplemented
{BesselIeMpMath? mpNum? $I_{\nu, e}(x) = I_{\nu}(x) \exp(-|x|)$, the exponentially scaled modified Bessel function $I_{\nu}(z)$ of the first kind of order $\nu$, $x \geq 0$ if $\nu$ is not an integer.}
{x? mpNum? A real number.}
{$\nu$? mpNum? A real number.}
\end{mpFunctionsExtract}

\begin{mpFunctionsExtract}
\mpFunctionTwoNotImplemented
{BesselKeMpMath? mpNum? $K_{\nu, e}(x) = K_{\nu}(x) \exp(x)$, the exponentially scaled modified Bessel function $K_{\nu}(z)$ of the first kind of order $\nu$, $x > 0$.}
{x? mpNum? A real number.}
{$\nu$? mpNum? A real number.}
\end{mpFunctionsExtract}

\begin{mpFunctionsExtract}
\mpFunctionThree
{besselj? mpNum? the Bessel function of the first kind $J_n(x)$.}
{n? mpNum? A real or complex number.}
{x? mpNum? A real or complex number.}
{Keywords? String? derivative=0.}
\end{mpFunctionsExtract}

\begin{mpFunctionsExtract}
\mpFunctionOne
{j0? mpNum? the Bessel function $J_0(x)$.}
{x? mpNum? A real or complex number.}
\end{mpFunctionsExtract}

\begin{mpFunctionsExtract}
\mpFunctionOne
{j1? mpNum? the Bessel function $J_1(x)$.}
{x? mpNum? A real or complex number.}
\end{mpFunctionsExtract}

\begin{mpFunctionsExtract}
\mpFunctionThree
{bessely? mpNum? the Bessel function of the second kind $Y_n(x)$.}
{n? mpNum? A real or complex number.}
{x? mpNum? A real or complex number.}
{Keywords? String? derivative=0.}
\end{mpFunctionsExtract}

\begin{mpFunctionsExtract}
\mpFunctionThree
{besseli? mpNum? the modified Bessel function of the first kind $J_n(x)$.}
{n? mpNum? A real or complex number.}
{x? mpNum? A real or complex number.}
{Keywords? String? derivative=0.}
\end{mpFunctionsExtract}

\begin{mpFunctionsExtract}
\mpFunctionTwo
{besselk? mpNum? the modified Bessel function of the second kind $K_n(x)$.}
{n? mpNum? A real or complex number.}
{x? mpNum? A real or complex number.}
\end{mpFunctionsExtract}

\section{Bessel function zeros}

\begin{mpFunctionsExtract}
\mpFunctionThree
{besseljzero? mpNum? the m-th positive zero of the Bessel function of the first kind}
{v? mpNum? A real or complex number.}
{m? mpNum? A real or complex number.}
{Keywords? String? derivative=0.}
\end{mpFunctionsExtract}

\begin{mpFunctionsExtract}
\mpFunctionThree
{besselyzero? mpNum? the m-th positive zero of the Bessel function of the second kind}
{v? mpNum? A real or complex number.}
{m? mpNum? A real or complex number.}
{Keywords? String? derivative=0.}
\end{mpFunctionsExtract}

\section{Hankel functions}

\begin{mpFunctionsExtract}
\mpFunctionTwo
{hankel1? mpNum? the Hankel function of the first kind}
{n? mpNum? A real or complex number.}
{x? mpNum? A real or complex number.}
\end{mpFunctionsExtract}

\begin{mpFunctionsExtract}
\mpFunctionTwo
{hankel2? mpNum? the Hankel function of the second kind}
{n? mpNum? A real or complex number.}
{x? mpNum? A real or complex number.}
\end{mpFunctionsExtract}

\section{Kelvin functions}

\begin{mpFunctionsExtract}
\mpFunctionTwo
{ber? mpNum? the Kelvin function ber}
{n? mpNum? A real or complex number.}
{z? mpNum? A real or complex number.}
\end{mpFunctionsExtract}

\begin{mpFunctionsExtract}
\mpFunctionTwo
{bei? mpNum? the Kelvin function bei}
{n? mpNum? A real or complex number.}
{z? mpNum? A real or complex number.}
\end{mpFunctionsExtract}

\begin{mpFunctionsExtract}
\mpFunctionTwo
{ker? mpNum? the Kelvin function ker}
{n? mpNum? A real or complex number.}
{z? mpNum? A real or complex number.}
\end{mpFunctionsExtract}

\begin{mpFunctionsExtract}
\mpFunctionTwo
{kei? mpNum? the Kelvin function kei}
{n? mpNum? A real or complex number.}
{z? mpNum? A real or complex number.}
\end{mpFunctionsExtract}

\section{Struve Functions}

\begin{mpFunctionsExtract}
\mpFunctionTwo
{struveh? mpNum? the Struve function H}
{n? mpNum? A real or complex number.}
{z? mpNum? A real or complex number.}
\end{mpFunctionsExtract}

\begin{mpFunctionsExtract}
\mpFunctionTwo
{struvel? mpNum? the modified Struve function L}
{n? mpNum? A real or complex number.}
{z? mpNum? A real or complex number.}
\end{mpFunctionsExtract}

\section{Anger-Weber functions}

\begin{mpFunctionsExtract}
\mpFunctionTwo
{angerj? mpNum? the Anger function J}
{v? mpNum? A real or complex number.}
{z? mpNum? A real or complex number.}
\end{mpFunctionsExtract}

\begin{mpFunctionsExtract}
\mpFunctionTwo
{webere? mpNum? the Weber function E}
{v? mpNum? A real or complex number.}
{z? mpNum? A real or complex number.}
\end{mpFunctionsExtract}

\section{Lommel functions}

\begin{mpFunctionsExtract}
\mpFunctionThree
{lommels1? mpNum? the First Lommel functions s}
{u? mpNum? A real or complex number.}
{v? mpNum? A real or complex number.}
{z? mpNum? A real or complex number.}
\end{mpFunctionsExtract}

\begin{mpFunctionsExtract}
\mpFunctionThree
{lommels2? mpNum? the Second Lommel functions S}
{u? mpNum? A real or complex number.}
{v? mpNum? A real or complex number.}
{z? mpNum? A real or complex number.}
\end{mpFunctionsExtract}

\section{Airy and Scorer functions}

\begin{mpFunctionsExtract}
\mpFunctionTwo
{airyai? mpNum? the Airy function Ai}
{z? mpNum? A real or complex number.}
{Keywords? String? derivative=0.}
\end{mpFunctionsExtract}

\begin{mpFunctionsExtract}
\mpFunctionTwo
{airybi? mpNum? the Airy function Bi}
{z? mpNum? A real or complex number.}
{Keywords? String? derivative=0.}
\end{mpFunctionsExtract}

\begin{mpFunctionsExtract}
\mpFunctionTwo
{airyaizero? mpNum? the $k$-th zero of the Airy Ai-function}
{k? mpNum? An integer.}
{Keywords? String? derivative=0.}
\end{mpFunctionsExtract}

\begin{mpFunctionsExtract}
\mpFunctionTwo
{airybizero? mpNum? the $k$-th zero of the Airy Bi-function}
{k? mpNum? An integer.}
{Keywords? String? derivative=0, complex=0.}
\end{mpFunctionsExtract}

\begin{mpFunctionsExtract}
\mpFunctionOne
{scorergi? mpNum? the Scorer function Gi}
{z? mpNum? A real or complex number.}
\end{mpFunctionsExtract}

\begin{mpFunctionsExtract}
\mpFunctionOne
{scorerhi? mpNum? the Scorer function Gi}
{z? mpNum? A real or complex number.}
\end{mpFunctionsExtract}

\section{Coulomb wave functions}

\begin{mpFunctionsExtract}
\mpFunctionThree
{coulombf? mpNum? the regular Coulomb wave function}
{l? mpNum? A real or complex number.}
{eta? mpNum? A real or complex number.}
{z? mpNum? A real or complex number.}
\end{mpFunctionsExtract}

\begin{mpFunctionsExtract}
\mpFunctionThree
{coulombg? mpNum? the irregular Coulomb wave function}
{l? mpNum? A real or complex number.}
{eta? mpNum? A real or complex number.}
{z? mpNum? A real or complex number.}
\end{mpFunctionsExtract}

\begin{mpFunctionsExtract}
\mpFunctionTwo
{coulombc? mpNum? the normalizing Gamow constant for Coulomb wave functions}
{l? mpNum? A real or complex number.}
{eta? mpNum? A real or complex number.}
\end{mpFunctionsExtract}

\section{Parabolic cylinder functions}

\begin{mpFunctionsExtract}
\mpFunctionTwo
{pcfd? mpNum? the parabolic cylinder function D}
{n? mpNum? A real or complex number.}
{z? mpNum? A real or complex number.}
\end{mpFunctionsExtract}

\begin{mpFunctionsExtract}
\mpFunctionTwo
{pcfu? mpNum? the parabolic cylinder function U}
{a? mpNum? A real or complex number.}
{z? mpNum? A real or complex number.}
\end{mpFunctionsExtract}

\begin{mpFunctionsExtract}
\mpFunctionTwo
{pcfv? mpNum? the parabolic cylinder function V}
{a? mpNum? A real or complex number.}
{z? mpNum? A real or complex number.}
\end{mpFunctionsExtract}

\begin{mpFunctionsExtract}
\mpFunctionTwo
{pcfw? mpNum? Computes the parabolic cylinder function W}
{a? mpNum? A real or complex number.}
{z? mpNum? A real or complex number.}
\end{mpFunctionsExtract}

\chapter{Orthogonal polynomials}

\section{Legendre functions}

\begin{mpFunctionsExtract}
\mpFunctionTwo
{legendre? mpNum? the Legendre polynomial $P_n(x)$}
{n? mpNum? A real or complex number.}
{x? mpNum? A real or complex number.}
\end{mpFunctionsExtract}

\begin{mpFunctionsExtract}
\mpFunctionFour
{legenp? mpNum? the (associated) Legendre function of the first kind of degree $n$ and order $m$, $P_n^m(z)$.}
{n? mpNum? A real or complex number.}
{m? mpNum? A real or complex number.}
{z? mpNum? A real or complex number.}
{Keywords? String? type=2.}
\end{mpFunctionsExtract}

\begin{mpFunctionsExtract}
\mpFunctionFour
{legenq? mpNum? the (associated) Legendre function of the second kind of degree $n$ and order $m$, $Q_n^m(z)$.}
{n? mpNum? A real or complex number.}
{m? mpNum? A real or complex number.}
{z? mpNum? A real or complex number.}
{Keywords? String? type=2.}
\end{mpFunctionsExtract}

\begin{mpFunctionsExtract}
\mpFunctionFour
{spherharm? mpNum? the spherical harmonic $Y_l^m(\theta,\phi)$}
{l? mpNum? A real or complex number.}
{m? mpNum? A real or complex number.}
{theta? mpNum? A real or complex number.}
{phi? mpNum? A real or complex number.}
\end{mpFunctionsExtract}

\section{Chebyshev polynomials}

\begin{mpFunctionsExtract}
\mpFunctionTwo
{chebyt? mpNum? the Chebyshev polynomial of the first kind $T_n(x)$}
{n? mpNum? A real or complex number.}
{x? mpNum? A real or complex number.}
\end{mpFunctionsExtract}

\begin{mpFunctionsExtract}
\mpFunctionTwo
{chebyu? mpNum? the Chebyshev polynomial of the second kind $U_n(x)$}
{n? mpNum? A real or complex number.}
{x? mpNum? A real or complex number.}
\end{mpFunctionsExtract}

\section{Jacobi and Gegenbauer polynomials}

\begin{mpFunctionsExtract}
\mpFunctionFour
{jacobi? mpNum? the Jacobi polynomial $P_n^{(a,b)}$}
{n? mpNum? A real or complex number.}
{a? mpNum? A real or complex number.}
{b? mpNum? A real or complex number.}
{z? mpNum? A real or complex number.}
\end{mpFunctionsExtract}

\begin{mpFunctionsExtract}
\mpFunctionThreeNotImplemented
{ZernikeRadialMpMath? mpNum? the Zernike radial polynomials $R^m_n(r)$.}
{n? mpNum? An Integer.}
{m? mpNum? An Integer.}
{x? mpNum? A real number.}
\end{mpFunctionsExtract}

\begin{mpFunctionsExtract}
\mpFunctionThree
{gegenbauer? mpNum? the Gegenbauer polynomial $C_n^{(a)}(z)$}
{n? mpNum? A real or complex number.}
{a? mpNum? A real or complex number.}
{z? mpNum? A real or complex number.}
\end{mpFunctionsExtract}

\section{Hermite and Laguerre polynomials}

\begin{mpFunctionsExtract}
\mpFunctionTwo
{hermite? mpNum? the Hermite polynomial $H_n(z)$}
{n? mpNum? A real or complex number.}
{z? mpNum? A real or complex number.}
\end{mpFunctionsExtract}

\begin{mpFunctionsExtract}
\mpFunctionThree
{laguerre? mpNum? the generalized Laguerre polynomial $L_n^{\alpha}(z)$}
{n? mpNum? A real or complex number.}
{a? mpNum? A real or complex number.}
{z? mpNum? A real or complex number.}
\end{mpFunctionsExtract}

\begin{mpFunctionsExtract}
\mpFunctionTwoNotImplemented
{LaguerreLMpMath? mpNum? $L_n (x)$, the Laguerre polynomials of degree $n \geq 0$.}
{n? mpNum? An Integer.}
{x? mpNum? A real number.}
\end{mpFunctionsExtract}

\begin{mpFunctionsExtract}
\mpFunctionThreeNotImplemented
{AssociatedLaguerreMpMath? mpNum? $L^m_n (x)$, the associated Laguerre polynomials of degree $n \geq 0$ and order $m \geq 0$.}
{n? mpNum? An Integer.}
{m? mpNum? An Integer.}
{x? mpNum? A real number.}
\end{mpFunctionsExtract}

\chapter{Hypergeometric functions}

\section{Confluent Hypergeometric Limit Function}

\begin{mpFunctionsExtract}
\mpFunctionTwo
{hyp0f1? mpNum? the hypergeometric function ${}_0F_1$}
{a? mpNum? A real or complex number.}
{z? mpNum? A real or complex number.}
\end{mpFunctionsExtract}

\begin{mpFunctionsExtract}
\mpFunctionTwoNotImplemented
{Hypergeometric0F1RegularizedMpMath? mpNum? the regularized confluent hypergeometric limit function ${}_0F_1(b; x)$.}
{b? mpNum? A real number.}
{x? mpNum? A real number.}
\end{mpFunctionsExtract}

\section{Kummer's Confluent Hypergeometric Functions and related functions}

\begin{mpFunctionsExtract}
\mpFunctionThree
{hyp1f1? mpNum? the confluent hypergeometric function of the first kind ${}_1F_1(a,b;z)$}
{a? mpNum? A real or complex number.}
{b? mpNum? A real or complex number.}
{z? mpNum? A real or complex number.}
\end{mpFunctionsExtract}

\begin{mpFunctionsExtract}
\mpFunctionThreeNotImplemented
{Hypergeometric1F1RegularizedMpMath? mpNum? Kummer's regularized confluent hypergeometric function ${}_1F_1(a, b; x)$.}
{a? mpNum? A real number.}
{b? mpNum? A real number.}
{z? mpNum? A real number.}
\end{mpFunctionsExtract}

\begin{mpFunctionsExtract}
\mpFunctionThree
{hyperu? mpNum? the Tricomi confluent hypergeometric function $U$}
{a? mpNum? A real or complex number.}
{b? mpNum? A real or complex number.}
{z? mpNum? A real or complex number.}
\end{mpFunctionsExtract}

\begin{mpFunctionsExtract}
\mpFunctionThree
{hyp2f0? mpNum? the hypergeometric function ${}_2F_0$}
{a? mpNum? A real or complex number.}
{b? mpNum? A real or complex number.}
{z? mpNum? A real or complex number.}
\end{mpFunctionsExtract}

\section{Whittaker functions M and W}

\begin{mpFunctionsExtract}
\mpFunctionThree
{whitm? mpNum? the Whittaker function $M$}
{k? mpNum? A real or complex number.}
{m? mpNum? A real or complex number.}
{z? mpNum? A real or complex number.}
\end{mpFunctionsExtract}

\begin{mpFunctionsExtract}
\mpFunctionThree
{whitw? mpNum? the Whittaker function $W$}
{k? mpNum? A real or complex number.}
{m? mpNum? A real or complex number.}
{z? mpNum? A real or complex number.}
\end{mpFunctionsExtract}

\section{Gauss Hypergeometric Function}

\begin{mpFunctionsExtract}
\mpFunctionFour
{hyp2f1? mpNum? the square of $z$.}
{a? mpNum? A real or complex number.}
{b? mpNum? A real or complex number.}
{c? mpNum? A real or complex number.}
{z? mpNum? A real or complex number.}
\end{mpFunctionsExtract}

\begin{mpFunctionsExtract}
\mpFunctionFourNotImplemented
{Hypergeometric2F1RegularizedMpMath? mpNum? the regularized Gauss hypergeometric function ${}_2F_1(a, b; c; x)$.}
{a? mpNum? A real number.}
{b? mpNum? A real number.}
{c? mpNum? A real number.}
{z? mpNum? A real number.}
\end{mpFunctionsExtract}

\section{Additional Hypergeometric Functions}

\begin{mpFunctionsExtract}
\mpFunctionFour
{hyp1f2? mpNum? the the hypergeometric function ${}_1F_2(a_1; b_1, b_2; z)$}
{a1? mpNum? A real or complex number.}
{b1? mpNum? A real or complex number.}
{b2? mpNum? A real or complex number.}
{z? mpNum? A real or complex number.}
\end{mpFunctionsExtract}

\begin{mpFunctionsExtract}
\mpFunctionFive
{hyp2f2? mpNum? the hypergeometric function  ${}_2F_2(a_1; b_1, b_2; z)$.}
{a1? mpNum? A real or complex number.}
{a2? mpNum? A real or complex number.}
{b1? mpNum? A real or complex number.}
{b2? mpNum? A real or complex number.}
{z? mpNum? A real or complex number.}
\end{mpFunctionsExtract}

\begin{mpFunctionsExtract}
\mpFunctionSix
{hyp2f3? mpNum? the hypergeometric function ${}_2F_3(a_1,a2;b_1,b_2,b_3;z)$.}
{a1? mpNum? A real or complex number.}
{a2? mpNum? A real or complex number.}
{b1? mpNum? A real or complex number.}
{b2? mpNum? A real or complex number.}
{b3? mpNum? A real or complex number.}
{z? mpNum? A real or complex number.}
\end{mpFunctionsExtract}

\begin{mpFunctionsExtract}
\mpFunctionSix
{hyp3f2? mpNum? hypergeometric function ${}_3F_2$.}
{a1? mpNum? A real or complex number.}
{a2? mpNum? A real or complex number.}
{a3? mpNum? A real or complex number.}
{b1? mpNum? A real or complex number.}
{b2? mpNum? A real or complex number.}
{z? mpNum? A real or complex number.}
\end{mpFunctionsExtract}

\section{Generalized hypergeometric functions}

\begin{mpFunctionsExtract}
\mpFunctionThree
{hyper? mpNum? the generalized hypergeometric function${}_pF_q$}
{as? mpNum? list of real or complex numbers.}
{bs? mpNum? list of real or complex numbers.}
{z? mpNum? A real or complex number.}
\end{mpFunctionsExtract}

\begin{mpFunctionsExtract}
\mpFunctionFour
{hypercomb? mpNum? a weighted combination of hypergeometric functions}
{f? mpFunction? a real or function.}
{params? mpNum? list of real or complex numbers.}
{z? mpNum? A real or complex number.}
{Keywords? String ? discardknownzeros=True.}
\end{mpFunctionsExtract}

\section{Meijer G-function}

\begin{mpFunctionsExtract}
\mpFunctionFour
{meijerg? mpNum? the  Meijer G-function}
{as? mpNum? list of real or complex numbers.}
{bs? mpNum? list of real or complex numbers.}
{z? mpNum? A real or complex number.}
{Keywords? String?  r=1, series=1.}
\end{mpFunctionsExtract}

\section{Bilateral hypergeometric series}

\begin{mpFunctionsExtract}
\mpFunctionFour
{bihyper? mpNum? the  bilateral hypergeometric series}
{as? mpNum? list of real or complex numbers.}
{bs? mpNum? list of real or complex numbers.}
{z? mpNum? A real or complex number.}
{Keywords? String?  r=1, series=1.}
\end{mpFunctionsExtract}

\section{Hypergeometric functions of two variables}

\begin{mpFunctionsExtract}
\mpFunctionFour
{hyper2d? mpNum? the sum the generalized 2D hypergeometric series}
{a? mpNum? A real or complex number.}
{b? mpNum? A real or complex number.}
{x? mpNum? A real or complex number.}
{y? mpNum? A real or complex number.}
\end{mpFunctionsExtract}

\begin{mpFunctionsExtract}
\mpFunctionSix
{appellf1? mpNum? the Appell F1 hypergeometric function of two variables.}
{a? mpNum? A real or complex number.}
{b1? mpNum? A real or complex number.}
{b2? mpNum? A real or complex number.}
{c? mpNum? A real or complex number.}
{x? mpNum? A real or complex number.}
{y? mpNum? A real or complex number.}
\end{mpFunctionsExtract}

\begin{mpFunctionsExtract}
\mpFunctionSeven
{appellf2? mpNum? the Appell F2 hypergeometric function of two variables.}
{a? mpNum? A real or complex number.}
{b1? mpNum? A real or complex number.}
{b2? mpNum? A real or complex number.}
{c1? mpNum? A real or complex number.}
{c2? mpNum? A real or complex number.}
{x? mpNum? A real or complex number.}
{y? mpNum? A real or complex number.}
\end{mpFunctionsExtract}

\begin{mpFunctionsExtract}
\mpFunctionSeven
{appellf3? mpNum? the Appell F3 hypergeometric function of two variables.}
{a1? mpNum? A real or complex number.}
{a2? mpNum? A real or complex number.}
{b1? mpNum? A real or complex number.}
{b2? mpNum? A real or complex number.}
{c? mpNum? A real or complex number.}
{x? mpNum? A real or complex number.}
{y? mpNum? A real or complex number.}
\end{mpFunctionsExtract}

\begin{mpFunctionsExtract}
\mpFunctionSix
{appellf4? mpNum? the Appell F4 hypergeometric function of two variables.}
{a? mpNum? A real or complex number.}
{b? mpNum? A real or complex number.}
{c1? mpNum? A real or complex number.}
{c2? mpNum? A real or complex number.}
{x? mpNum? A real or complex number.}
{y? mpNum? A real or complex number.}
\end{mpFunctionsExtract}

\chapter{Elliptic functions}

\section{Elliptic arguments}

\begin{mpFunctionsExtract}
\mpFunctionOne
{qfrom? mpNum? the elliptic nome $q$.}
{Keywords? String? m=x; k=x; tau=x; qbar=x.}
\end{mpFunctionsExtract}

\begin{mpFunctionsExtract}
\mpFunctionOne
{qbarfrom? mpNum? the number-theoretic nome $\overline{q}$.}
{Keywords? String? m=x; k=x; tau=x; q=x.}
\end{mpFunctionsExtract}

\begin{mpFunctionsExtract}
\mpFunctionOne
{mfrom? mpNum? the elliptic parameter $m$.}
{Keywords? String? k=x; tau=x; q=x; qbar=x.}
\end{mpFunctionsExtract}

\begin{mpFunctionsExtract}
\mpFunctionOne
{kfrom? mpNum? the elliptic modulus $k$.}
{Keywords? String? m=x; tau=x; q=x; qbar=x.}
\end{mpFunctionsExtract}

\begin{mpFunctionsExtract}
\mpFunctionOne
{taufrom? mpNum? the elliptic half-period ratio $\tau$.}
{Keywords? String? m=x; k=x; q=x; qbar=x.}
\end{mpFunctionsExtract}

\section{Legendre elliptic integrals}

\begin{mpFunctionsExtract}
\mpFunctionOne
{ellipk? mpNum? the complete elliptic integral of the first kind, $K(m)$.}
{m? mpNum? A real or complex number.}
\end{mpFunctionsExtract}

\begin{mpFunctionsExtract}
\mpFunctionTwo
{ellipf? mpNum? the Legendre incomplete elliptic integral of the first kind $F(\phi,m)$.}
{phi? mpNum? A real or complex number.}
{m? mpNum? A real or complex number.}
\end{mpFunctionsExtract}

\begin{mpFunctionsExtract}
\mpFunctionOne
{ellipe? mpNum? the Legendre complete elliptic integral of the second kind $E(m)$.}
{m? mpNum? A real or complex number.}
\end{mpFunctionsExtract}

\begin{mpFunctionsExtract}
\mpFunctionTwo
{ellipef? mpNum? the Legendre incomplete elliptic integral of the second kind $E(\phi,m)$.}
{phi? mpNum? A real or complex number.}
{m? mpNum? A real or complex number.}
\end{mpFunctionsExtract}

\begin{mpFunctionsExtract}
\mpFunctionTwo
{ellippi? mpNum? the complete elliptic integral of the third kind $\Pi(n,m$.}
{n? mpNum? A real or complex number.}
{m? mpNum? A real or complex number.}
\end{mpFunctionsExtract}

\begin{mpFunctionsExtract}
\mpFunctionThree
{ellippif? mpNum? the Legendre incomplete elliptic integral of the third kind $\Pi(n;\phi,m)$.}
{n? mpNum? A real or complex number.}
{phi? mpNum? A real or complex number.}
{m? mpNum? A real or complex number.}
\end{mpFunctionsExtract}

\section{Carlson symmetric elliptic integrals}

\begin{mpFunctionsExtract}
\mpFunctionThree
{elliprf? mpNum? the Carlson symmetric elliptic integral of the first kind.}
{x? mpNum? A real or complex number.}
{y? mpNum? A real or complex number.}
{z? mpNum? A real or complex number.}
\end{mpFunctionsExtract}

\begin{mpFunctionsExtract}
\mpFunctionThree
{elliprc? mpNum? the degenerate Carlson symmetric elliptic integral of the first kind.}
{x? mpNum? A real or complex number.}
{y? mpNum? A real or complex number.}
{Keywords? String? pv=True.}
\end{mpFunctionsExtract}

\begin{mpFunctionsExtract}
\mpFunctionFour
{elliprj? mpNum? the Carlson symmetric elliptic integral of the third kind.}
{x? mpNum? A real or complex number.}
{y? mpNum? A real or complex number.}
{z? mpNum? A real or complex number.}
{p? mpNum? A real or complex number.}
\end{mpFunctionsExtract}

\begin{mpFunctionsExtract}
\mpFunctionThree
{elliprd? mpNum? the Carlson symmetric elliptic integral of the second kind.}
{x? mpNum? A real or complex number.}
{y? mpNum? A real or complex number.}
{z? mpNum? A real or complex number.}
\end{mpFunctionsExtract}

\begin{mpFunctionsExtract}
\mpFunctionThree
{elliprg? mpNum? the Carlson completely symmetric elliptic integral of the second kind.}
{x? mpNum? A real or complex number.}
{y? mpNum? A real or complex number.}
{z? mpNum? A real or complex number.}
\end{mpFunctionsExtract}

\section{Jacobi theta functions}

\begin{mpFunctionsExtract}
\mpFunctionFour
{jtheta? mpNum? the Jacobi theta function $\vartheta_n(z,q)$.}
{n? mpNum? An integer, where $n=1,2,3,4$.}
{z? mpNum? A real or complex number.}
{q? mpNum? A real or complex number.}
{Keywords? String? derivative=0.}
\end{mpFunctionsExtract}

\section{Jacobi elliptic functions}

\begin{mpFunctionsExtract}
\mpFunctionFour
{ellipfun? mpNum? any of the Jacobi elliptic functions.}
{kind? String? A function identifier.}
{u? mpNum? A real or complex number.}
{m? mpNum? A real or complex number.}
{Keywords? String?  q=None, k=None, tau=None.}
\end{mpFunctionsExtract}

\section{Klein j-invariant}

\begin{mpFunctionsExtract}
\mpFunctionOne
{kleinj? mpNum? the Klein j-invariant.}
{tau? mpNum? A real or complex number.}
\end{mpFunctionsExtract}

\chapter{Zeta functions, L-series and polylogarithms}

\section{Riemann and Hurwitz zeta functions}

\begin{mpFunctionsExtract}
\mpFunctionTwo
{zeta? mpNum? the Riemann zeta function}
{s? mpNum? A real or complex number.}
{Keywords? String? derivative=0.}
\end{mpFunctionsExtract}

\begin{mpFunctionsExtract}
\mpFunctionThree
{hurwitz? mpNum? the  Hurwitz  zeta function}
{s? mpNum? A real or complex number.}
{a? mpNum? A real or complex number.}
{Keywords? String? derivative=0.}
\end{mpFunctionsExtract}

\section{Dirichlet L-series}

\begin{mpFunctionsExtract}
\mpFunctionOne
{altzeta? mpNum? the Dirichlet eta function, $\eta(s)$}
{s? mpNum? A real or complex number.}
\end{mpFunctionsExtract}

\begin{mpFunctionsExtract}
\mpFunctionOneNotImplemented
{DirichletEtam1MpMath? mpNum? the Dirichlet function $\eta(s) - 1$.}
{x? mpNum? A real number.}
\end{mpFunctionsExtract}

\begin{mpFunctionsExtract}
\mpFunctionOneNotImplemented
{DirichletBetaMpMath? mpNum? the Dirichlet function $\beta(s)$.}
{s? mpNum? A real number.}
\end{mpFunctionsExtract}

\begin{mpFunctionsExtract}
\mpFunctionOneNotImplemented
{DirichletLambdaMpMath? mpNum? the Dirichlet function $\beta(s)$.}
{s? mpNum? A real number.}
\end{mpFunctionsExtract}

\begin{mpFunctionsExtract}
\mpFunctionThree
{dirichlet? mpNum? the Dirichlet L-function}
{s? mpNum? A real or complex number.}
{chi? mpNum? A periodic sequence.}
{Keywords? mpNum? derivative=0.}
\end{mpFunctionsExtract}

\section{Stieltjes constants}

\begin{mpFunctionsExtract}
\mpFunctionTwo
{stieltjes? mpNum? the $n$-th Stieltjes constant}
{n? mpNum? A real or complex number.}
{a? mpNum? A real or complex number.}
\end{mpFunctionsExtract}

\section{Zeta function zeros}

\begin{mpFunctionsExtract}
\mpFunctionTwo
{zetazero? mpNum? the $n$-th nontrivial zero of $\zeta(s)$ on the critical line}
{n? mpNum? An integer.}
{Keywords? String? verbose=False.}
\end{mpFunctionsExtract}

\begin{mpFunctionsExtract}
\mpFunctionOne
{nzeros? mpNum? the number of zeros of the Riemann zeta function in $(0,1) \times (0,t)$, usually denoted by $N(t)$.}
{t? mpNum? An integer.}
\end{mpFunctionsExtract}

\section{Riemann-Siegel Z function and related functions}

\begin{mpFunctionsExtract}
\mpFunctionOne
{siegelz? mpNum? the Riemann-Siegel Z function}
{t? mpNum? A real or complex number.}
\end{mpFunctionsExtract}

\begin{mpFunctionsExtract}
\mpFunctionOne
{siegeltheta? mpNum? the Riemann-Siegel theta function}
{t? mpNum? A real or complex number.}
\end{mpFunctionsExtract}

\begin{mpFunctionsExtract}
\mpFunctionOne
{grampoint? mpNum? the $n$-th Gram point $g_n$, defined as the solution to the equation $\theta(g_n)=\pi n$ where $\theta(t)$ is the Riemann-Siegel theta function}
{n? mpNum? A real or complex number.}
\end{mpFunctionsExtract}

\begin{mpFunctionsExtract}
\mpFunctionOne
{backlunds? mpNum? the function $S(t) = \text{arg}\zeta\left(\tfrac{1}{2}+it\right)/\pi$.}
{t? mpNum? A real or complex number.}
\end{mpFunctionsExtract}

\section{Lerch transcendent and related functions}

\begin{mpFunctionsExtract}
\mpFunctionThree
{lerchphi? mpNum? the Lerch transcendent}
{z? mpNum? A real or complex number.}
{s? mpNum? A real or complex number.}
{a? mpNum? A real or complex number.}
\end{mpFunctionsExtract}

\begin{mpFunctionsExtract}
\mpFunctionTwoNotImplemented
{FermiDiracIntMpMath? mpNum? the complete Fermi-Dirac integrals $F_n(x)$ of integer order.}
{x? mpNum? A real number.}
{n? mpNum? An Integer.}
\end{mpFunctionsExtract}

\begin{mpFunctionsExtract}
\mpFunctionOneNotImplemented
{FermiDiracPHalfMpMath? mpNum? the complete Fermi-Dirac integral $F_{-1/2}(x)$.}
{s? mpNum? A real number.}
\end{mpFunctionsExtract}

\begin{mpFunctionsExtract}
\mpFunctionOneNotImplemented
{FermiDiracHalfMpMath? mpNum? the complete Fermi-Dirac integral $F_{1/2}(x)$.}
{s? mpNum? A real number.}
\end{mpFunctionsExtract}

\begin{mpFunctionsExtract}
\mpFunctionOneNotImplemented
{FermiDirac3HalfMpMath? mpNum? the complete Fermi-Dirac integral $F_{3/2}(x)$.}
{s? mpNum? A real number.}
\end{mpFunctionsExtract}

\begin{mpFunctionsExtract}
\mpFunctionTwoNotImplemented
{LegendreChiMpMath? mpNum? the Legendre Chi-Function function $\chi_s(x)$.}
{s? mpNum? A real number.}
{x? mpNum? A real number.}
\end{mpFunctionsExtract}

\begin{mpFunctionsExtract}
\mpFunctionOneNotImplemented
{InverseTangentMpMath? mpNum? the inverse-tangent integral.}
{x? mpNum? A real number.}
\end{mpFunctionsExtract}

\section{Polylogarithms and Clausen functions}

\begin{mpFunctionsExtract}
\mpFunctionTwo
{polylog? mpNum? the polylogarithm}
{s? mpNum? A real or complex number.}
{z? mpNum? A real or complex number.}
\end{mpFunctionsExtract}

\begin{mpFunctionsExtract}
\mpFunctionOne
{dilog? mpNum? the dilogarithm function $\text{Li}_2(x)$.}
{x? mpNum? A real number.}
\end{mpFunctionsExtract}

\begin{mpFunctionsExtract}
\mpFunctionTwoNotImplemented
{DebyeMpMath? mpNum? the Debye function of order $n$.}
{n? mpNum? An Integer.}
{x? mpNum? A real number.}
\end{mpFunctionsExtract}

\begin{mpFunctionsExtract}
\mpFunctionTwo
{clsinlog? mpNum? the Clausen sine function}
{s? mpNum? A real or complex number.}
{z? mpNum? A real or complex number.}
\end{mpFunctionsExtract}

\begin{mpFunctionsExtract}
\mpFunctionTwo
{clcos? mpNum? the Clausen cosine function}
{s? mpNum? A real or complex number.}
{z? mpNum? A real or complex number.}
\end{mpFunctionsExtract}

\begin{mpFunctionsExtract}
\mpFunctionTwo
{polyexp? mpNum? the polyexponential function}
{s? mpNum? A real or complex number.}
{z? mpNum? A real or complex number.}
\end{mpFunctionsExtract}

\section{Zeta function variants}

\begin{mpFunctionsExtract}
\mpFunctionOne
{primezeta? mpNum? the prime zeta function.}
{s? mpNum? A real or complex number.}
\end{mpFunctionsExtract}

\begin{mpFunctionsExtract}
\mpFunctionTwo
{secondzeta? mpNum? the secondary zeta function}
{s? mpNum? A real or complex number.}
{Keywords? String? a=0.015, error=False.}
\end{mpFunctionsExtract}

\chapter{Number-theoretical, combinatorial and integer functions}

\section{Fibonacci numbers}

\begin{mpFunctionsExtract}
\mpFunctionTwo
{fibonacci? mpNum? the $n$-th Fibonacci number, $F(n)$}
{n? mpNum? A real or complex number.}
{Keywords? String? derivative=0.}
\end{mpFunctionsExtract}

\begin{mpFunctionsExtract}
\mpFunctionTwo
{fib? mpNum? the $n$-th Fibonacci number, $F(n)$}
{n? mpNum? A real or complex number.}
{Keywords? String? derivative=0.}
\end{mpFunctionsExtract}

\section{Bernoulli numbers and polynomials}

\begin{mpFunctionsExtract}
\mpFunctionOne
{bernoulli? mpNum? the $n$th Bernoulli number, $B_n$, for any integer $n>0$}
{n? mpNum? An integer}
\end{mpFunctionsExtract}

\begin{mpFunctionsExtract}
\mpFunctionOne
{bernfrac? mpNum? a tuple of integers $(p,q)$ such that $p/q=B_n$ exactly, where $B_n$denotes the $n$-th Bernoulli number.}
{n? mpNum? An integer}
\end{mpFunctionsExtract}

\begin{mpFunctionsExtract}
\mpFunctionTwo
{bernpoly? mpNum? the Bernoulli polynomial $B_n(z)$}
{n? mpNum? A real or complex number.}
{z? mpNum? A real or complex number.}
\end{mpFunctionsExtract}

\section{Euler numbers and polynomials}

\begin{mpFunctionsExtract}
\mpFunctionOne
{eulernum? mpNum? the $n$-th Euler number}
{n? mpNum? An integer}
\end{mpFunctionsExtract}

\begin{mpFunctionsExtract}
\mpFunctionTwo
{eulerpoly? mpNum? the Euler polynomial $E_n(z)$}
{n? mpNum? A real or complex number.}
{z? mpNum? A real or complex number.}
\end{mpFunctionsExtract}

\section{Bell numbers and polynomials}

\begin{mpFunctionsExtract}
\mpFunctionTwo
{bell? mpNum? the Bell polynomial $B_n(x)$}
{n? mpNum? A non-negative integer.}
{x? mpNum? A real or complex number.}
\end{mpFunctionsExtract}

\section{Stirling numbers}

\begin{mpFunctionsExtract}
\mpFunctionThree
{stirling1? mpNum? the Stirling number of the first kind $s(n,k)$}
{n? mpNum? A real or complex number.}
{k? mpNum? A real or complex number.}
{Keywords? String? exact=False.}
\end{mpFunctionsExtract}

\begin{mpFunctionsExtract}
\mpFunctionThree
{stirling2? mpNum? the Stirling number of the second kind $s(n,k)$}
{n? mpNum? A real or complex number.}
{k? mpNum? A real or complex number.}
{Keywords? String? exact=False.}
\end{mpFunctionsExtract}

\section{Prime counting functions}

\begin{mpFunctionsExtract}
\mpFunctionOne
{primepi? mpNum? the prime counting function}
{x? mpNum? A real number}
\end{mpFunctionsExtract}

\begin{mpFunctionsExtract}
\mpFunctionOne
{primepi2? mpNum? an interval (as an mpi instance) providing bounds for the value of the prime counting function $\pi(x)$}
{x? mpNum? A real number}
\end{mpFunctionsExtract}

\begin{mpFunctionsExtract}
\mpFunctionOne
{riemannr? mpNum? the Riemann R function, a smooth approximation of the prime counting function $\pi(x)$}
{x? mpNum? A real number}
\end{mpFunctionsExtract}

\section{Miscellaneous functions}

\begin{mpFunctionsExtract}
\mpFunctionTwo
{cyclotomic? mpNum? the cyclotomic polynomial $\Phi_n(x)$}
{n? mpNum? A real or complex number.}
{x? mpNum? A real or complex number.}
\end{mpFunctionsExtract}

\begin{mpFunctionsExtract}
\mpFunctionOne
{mangoldt? mpNum? the von Mangoldt function}
{n? mpNum? An integer}
\end{mpFunctionsExtract}

\chapter{q-functions}

\section{q-Pochhammer symbol}

\begin{mpFunctionsExtract}
\mpFunctionThree
{qp? mpNum? the q-Pochhammer symbol (or q-rising factorial)}
{a? mpNum? A real or complex number.}
{q? mpNum? A real or complex number.}
{n? mpNum? An integer.}
\end{mpFunctionsExtract}

\section{q-gamma and factorial}

\begin{mpFunctionsExtract}
\mpFunctionTwo
{qgamma? mpNum? the q-gamma function}
{z? mpNum? A real or complex number.}
{q? mpNum? A real or complex number.}
\end{mpFunctionsExtract}

\begin{mpFunctionsExtract}
\mpFunctionTwo
{qfac? mpNum? the  q-factorial}
{z? mpNum? A real or complex number.}
{q? mpNum? A real or complex number.}
\end{mpFunctionsExtract}

\section{Hypergeometric q-series}

\begin{mpFunctionsExtract}
\mpFunctionFour
{qhyper? mpNum? the hypergeometric q-series}
{as? mpNum? A real or complex number.}
{bs? mpNum? A real or complex number.}
{q? mpNum? A real or complex number.}
{z? mpNum? A real or complex number.}
\end{mpFunctionsExtract}

\chapter{Matrix functions}

\section{Matrix exponential}

\begin{mpFunctionsExtract}
\mpFunctionTwo
{expm? mpNum? the matrix exponential of a square matrix $A$}
{A? mpNum? A real or complex matrix.}
{Keywords? String? method='taylor'.}
\end{mpFunctionsExtract}

\section{Matrix cosine}

\begin{mpFunctionsExtract}
\mpFunctionOne
{cosm? mpNum? the matrix cosine of a square matrix $A$}
{A? mpNum? A real or complex matrix.}
\end{mpFunctionsExtract}

\section{Matrix sine}

\begin{mpFunctionsExtract}
\mpFunctionOne
{sinm? mpNum? the matrix sine of a square matrix $A$}
{A? mpNum? A real or complex matrix.}
\end{mpFunctionsExtract}

\section{Matrix square root}

\begin{mpFunctionsExtract}
\mpFunctionTwo
{sqrtm? mpNum? a square root of a square matrix $A$}
{A? mpNum? A real or complex matrix.}
{Keywords? String?  mayrotate=2.}
\end{mpFunctionsExtract}

\section{Matrix logarithm}

\begin{mpFunctionsExtract}
\mpFunctionOne
{logm? mpNum? the matrix logarithm of a square matrix $A$}
{A? mpNum? A real or complex matrix.}
\end{mpFunctionsExtract}

\section{Matrix power}

\begin{mpFunctionsExtract}
\mpFunctionTwo
{powm? mpNum? $A^r=\exp(A \log r)$ for a matrix $A$ and complex number $r$}
{A? mpNum? A real or complex matrix.}
{r? mpNum? A real or complex number.}
\end{mpFunctionsExtract}

\chapter{Eigensystems and related Decompositions}

\section{Singular value decomposition}

\begin{mpFunctionsExtract}
\mpFunctionTwo
{svd? mpNum? the singular value decomposition of matrix A}
{A? mpNum? A real or complex number.}
{Keywords? String? compute\_uv = True.}
\end{mpFunctionsExtract}

\section{The Schur decomposition}

\begin{mpFunctionsExtract}
\mpFunctionOne
{schur? mpNum? the Schur decomposition of a square matrix $A$}
{A? mpNum? A real or complex matrix.}
\end{mpFunctionsExtract}

\section{The eigenvalue problem}

\begin{mpFunctionsExtract}
\mpFunctionTwo
{eig? mpNum? the solution of the (ordinary) eigenvalue problem for a real or complex square matrix $A$}
{A? mpNum? A real or complex number.}
{Keywords? String? left = False, right = False.}
\end{mpFunctionsExtract}

\section{The symmetric eigenvalue problem}

\begin{mpFunctionsExtract}
\mpFunctionTwo
{eigh? mpNum? the solution of the (ordinary) eigenvalue problem for a real symmetric or complex hermitian square matrix $A$}
{A? mpNum? A real or complex number.}
{Keywords? String? eigvals\_only = False.}
\end{mpFunctionsExtract}

\chapter{GMP, MPD, and related libraries: an overview}

\section{Integer Types and Fractions}

\section{FloatingPoint Types}

\section{Arithmetic Operators}

\section{Comparison Operators and Sorting}

\section{Vectors, Matrices and Tables}

\chapter{FMPZ}

\section{Multiprecision Rational Numbers (GMP: MPQ)}

\section{Arithmetic Operators }

\begin{mpFunctionsExtract}
\mpFunctionOne
{intNeg? mpNum?  $-n$}
{n? mpNum? An Integer.}
\end{mpFunctionsExtract}

\begin{mpFunctionsExtract}
\mpFunctionTwo
{intAdd? mpNum?  $n_1 + n_2.$.}
{n1? mpNum? An Integer.}
{n2? mpNum? An Integer.}
\end{mpFunctionsExtract}

\begin{mpFunctionsExtract}
\mpFunctionTwo
{intSub? mpNum?  $n_1 - n_2.$.}
{n1? mpNum? An Integer.}
{n2? mpNum? An Integer.}
\end{mpFunctionsExtract}

\begin{mpFunctionsExtract}
\mpFunctionTwo
{intMul? mpNum?  $n_1  \times n_2$.}
{n1? mpNum? An Integer.}
{n2? mpNum? An Integer.}
\end{mpFunctionsExtract}

\begin{mpFunctionsExtract}
\mpFunctionThree
{intFma? mpNum? $(n_1 \times n_2) + n_3$.}
{n1? mpNum? An Integer.}
{n2? mpNum? An Integer.}
{n3? mpNum? An Integer.}
\end{mpFunctionsExtract}

\begin{mpFunctionsExtract}
\mpFunctionThree
{intFms? mpNum? $(n_1 \times n_2) - n_3$.}
{n1? mpNum? An Integer.}
{n2? mpNum? An Integer.}
{n3? mpNum? An Integer.}
\end{mpFunctionsExtract}

\begin{mpFunctionsExtract}
\mpFunctionTwo
{intLSH? mpNum? the product of $n$ and $2^k$}
{n? mpNum? An Integer.}
{k? mpNum? An Integer.}
\end{mpFunctionsExtract}

\begin{mpFunctionsExtract}
\mpFunctionTwo
{intRSH? mpNum? the quotient of $n$ and $2^k$}
{n? mpNum? An Integer.}
{k? mpNum? An Integer.}
\end{mpFunctionsExtract}

\begin{mpFunctionsExtract}
\mpFunctionTwo
{intDivExact? mpNum? $n/d$}
{n? mpNum? An Integer.}
{d? mpNum? An Integer.}
\end{mpFunctionsExtract}

\begin{mpFunctionsExtract}
\mpFunctionTwo
{intMod? mpNum? $n$ mod $d$.}
{n? mpNum? An Integer.}
{d? mpNum? An Integer.}
\end{mpFunctionsExtract}

\section{Divisions, forming quotients and/or remainder}

\begin{mpFunctionsExtract}
\mpFunctionTwo
{intCDivQ? mpNum? the quotient of $n$ and $d$, rounded up towards $+\infty$.}
{n? mpNum? An Integer.}
{d? mpNum? An Integer.}
\end{mpFunctionsExtract}

\begin{mpFunctionsExtract}
\mpFunctionTwo
{intCDivQ2exp? mpNum? the quotient of $n$ and $2^b$, rounded up towards $+\infty$.}
{n? mpNum? An Integer.}
{b? mpNum? An Integer.}
\end{mpFunctionsExtract}

\begin{mpFunctionsExtract}
\mpFunctionTwo
{intCDivR? mpNum? the remainder, once the quotient of $n$ and $d$, rounded up towards $+\infty$, has been obtained.}
{n? mpNum? An Integer.}
{d? mpNum? An Integer.}
\end{mpFunctionsExtract}

\begin{mpFunctionsExtract}
\mpFunctionTwo
{intCDivR2exp? mpNum? the remainder, once the quotient of $n$ and $2^b$, rounded up towards $+\infty$, has been obtained.}
{n? mpNum? An Integer.}
{b? mpNum? An Integer.}
\end{mpFunctionsExtract}

\begin{mpFunctionsExtract}
\mpFunctionTwo
{intCDivQR? mpNumList[2]? the quotient of $n$ and $d$, rounded up towards $+\infty$, and the remainder.}
{n? mpNum? An Integer.}
{d? mpNum? An Integer.}
\end{mpFunctionsExtract}

\begin{mpFunctionsExtract}
\mpFunctionTwo
{intFDivQ? mpNum? the quotient of $n$ and $d$, rounded down towards $-\infty$.}
{n? mpNum? An Integer.}
{d? mpNum? An Integer.}
\end{mpFunctionsExtract}

\begin{mpFunctionsExtract}
\mpFunctionTwo
{intFDivQ2exp? mpNum? the quotient of $n$ and $2^b$, rounded down towards $-\infty$.}
{n? mpNum? An Integer.}
{b? mpNum? An Integer.}
\end{mpFunctionsExtract}

\begin{mpFunctionsExtract}
\mpFunctionTwo
{intFDivR? mpNum? the remainder, once the quotient of $n$ and $d$, rounded down towards $-\infty$, has been obtained.}
{n? mpNum? An Integer.}
{d? mpNum? An Integer.}
\end{mpFunctionsExtract}

\begin{mpFunctionsExtract}
\mpFunctionTwo
{intFDivR2exp? mpNum? the remainder, once the quotient of $n$ and $2^b$, rounded down towards $-\infty$, has been obtained.}
{n? mpNum? An Integer.}
{b? mpNum? An Integer.}
\end{mpFunctionsExtract}

\begin{mpFunctionsExtract}
\mpFunctionTwo
{intFDivQR? mpNumList[2]? the quotient of $n$ and $d$, rounded down towards $-\infty$, and the remainder.}
{n? mpNum? An Integer.}
{d? mpNum? An Integer.}
\end{mpFunctionsExtract}

\begin{mpFunctionsExtract}
\mpFunctionTwo
{intTDivQ? mpNum? the quotient of $n$ and $d$, rounded towards zero.}
{n? mpNum? An Integer.}
{d? mpNum? An Integer.}
\end{mpFunctionsExtract}

\begin{mpFunctionsExtract}
\mpFunctionTwo
{intTDivQ2exp? mpNum? the quotient of $n$ and $2^b$, rounded towards zero.}
{n? mpNum? An Integer.}
{b? mpNum? An Integer.}
\end{mpFunctionsExtract}

\begin{mpFunctionsExtract}
\mpFunctionTwo
{intTDivR? mpNum? the remainder, once the quotient of $n$ and $d$,rounded towards zero, has been obtained.}
{n? mpNum? An Integer.}
{d? mpNum? An Integer.}
\end{mpFunctionsExtract}

\begin{mpFunctionsExtract}
\mpFunctionTwo
{intTDivR2exp? mpNum? the remainder, once the quotient of $n$ and $2^b$, rounded towards zero, has been obtained.}
{n? mpNum? An Integer.}
{b? mpNum? An Integer.}
\end{mpFunctionsExtract}

\begin{mpFunctionsExtract}
\mpFunctionTwo
{intTDivQr? mpNumList[2]? the quotient of $n$ and $d$, rounded towards zero, and the remainder.}
{n? mpNum? An Integer.}
{d? mpNum? An Integer.}
\end{mpFunctionsExtract}

\section{Logical Operators }

\begin{mpFunctionsExtract}
\mpFunctionTwo
{intAND? mpNum? $n_1$ bitwise-and $n_2$.}
{n1? mpNum? An Integer.}
{n2? mpNum? An Integer.}
\end{mpFunctionsExtract}

\begin{mpFunctionsExtract}
\mpFunctionTwo
{intIOR? mpNum? $n_1$ bitwise-inclusive-or $n_2$.}
{n1? mpNum? An Integer.}
{n2? mpNum? An Integer.}
\end{mpFunctionsExtract}

\begin{mpFunctionsExtract}
\mpFunctionTwo
{intXOR? mpNum? $n_1$ bitwise-exclusive-or $n_2$.}
{n1? mpNum? An Integer.}
{n2? mpNum? An Integer.}
\end{mpFunctionsExtract}

\section{Bit-Oriented Functions}

\begin{mpFunctionsExtract}
\mpFunctionOne
{intComplement? mpNum? the one's complement of $n$.}
{n? mpNum? An Integer.}
\end{mpFunctionsExtract}

\begin{mpFunctionsExtract}
\mpFunctionTwo
{intHamDist? mpNum? the hamming distance between the two operands}
{n1? mpNum? An Integer.}
{n2? mpNum? An Integer.}
\end{mpFunctionsExtract}

\begin{mpFunctionsExtract}
\mpFunctionTwo
{intTestBit? mpNum? 1 or 0 according to whether bit $k$ in $n$ is set or not.}
{n? mpNum? An Integer.}
{k? mpNum? An Integer.}
\end{mpFunctionsExtract}

\begin{mpFunctionsExtract}
\mpFunctionTwo
{intComBit? mpNum? n with the complement bit $k$ set in $n$.}
{n? mpNum? An Integer.}
{k? mpNum? An Integer.}
\end{mpFunctionsExtract}

\begin{mpFunctionsExtract}
\mpFunctionTwo
{intClearBit? mpNum? $n$ with the bit $k$ cleared in $n$.}
{n? mpNum? An Integer.}
{k? mpNum? An Integer.}
\end{mpFunctionsExtract}

\begin{mpFunctionsExtract}
\mpFunctionTwo
{intSetBit? mpNum? $n$ with the bit $k$ set in $n$.}
{n? mpNum? An Integer.}
{k? mpNum? An Integer.}
\end{mpFunctionsExtract}

\begin{mpFunctionsExtract}
\mpFunctionTwo
{intScan0? mpNum? the index of the found bit 0, starting from bit $k$.}
{n? mpNum? An Integer.}
{k? mpNum? An Integer.}
\end{mpFunctionsExtract}

\begin{mpFunctionsExtract}
\mpFunctionTwo
{intScan1? mpNum? the index of the found bit 1, starting from bit $k$.}
{n? mpNum? An Integer.}
{k? mpNum? An Integer.}
\end{mpFunctionsExtract}

\begin{mpFunctionsExtract}
\mpFunctionOne
{intPopCount? mpNum? the population count of $n$.}
{n? mpNum? An Integer.}
\end{mpFunctionsExtract}

\section{Sign, Powers and Roots}

\begin{mpFunctionsExtract}
\mpFunctionOne
{intSgn? mpNum? the sign of $n$.}
{n? mpNum? An Integer.}
\end{mpFunctionsExtract}

\begin{mpFunctionsExtract}
\mpFunctionOne
{intAbs? mpNum? the absolute value of $n$.}
{n? mpNum? An Integer.}
\end{mpFunctionsExtract}

\begin{mpFunctionsExtract}
\mpFunctionTwo
{intPow? mpNum? the value of $n^k$. The case $0^0$ yields 1.}
{n? mpNum? An Integer.}
{k? mpNum? An Integer.}
\end{mpFunctionsExtract}

\begin{mpFunctionsExtract}
\mpFunctionThree
{intPowMod? mpNum? the value of $n^k \text{ mod } m$.}
{n? mpNum? An Integer.}
{k? mpNum? An Integer.}
{m? mpNum? An Integer.}
\end{mpFunctionsExtract}

\begin{mpFunctionsExtract}
\mpFunctionOne
{intSqrt? mpNum? the truncated integer part of the square root of $n$.}
{n? mpNum? An Integer.}
\end{mpFunctionsExtract}

\begin{mpFunctionsExtract}
\mpFunctionOne
{intSqrtRem? mpNumList[2]? the truncated integer part of the square root of $n$, and the remainder.}
{n? mpNum? An Integer.}
\end{mpFunctionsExtract}

\begin{mpFunctionsExtract}
\mpFunctionTwo
{intRoot? mpNum? the truncated integer part of the $n^{th}$ root of $m$}
{n? mpNum? An Integer.}
{m? mpNum? An Integer.}
\end{mpFunctionsExtract}

\begin{mpFunctionsExtract}
\mpFunctionTwo
{intRootRem? mpNumList[2]? the truncated integer part of the $n^{th}$ root of $m$, with remainder}
{n? mpNum? An Integer.}
{m? mpNum? An Integer.}
\end{mpFunctionsExtract}

\section{Numbertheoretic Functions}

\begin{mpFunctionsExtract}
\mpFunctionOne
{intFactorial? mpNum?  $n!$, the factorial of $n$}
{n? mpNum? An Integer.}
\end{mpFunctionsExtract}

\begin{mpFunctionsExtract}
\mpFunctionTwo
{intBinCoeff? mpNum? the binomial coefficient}
{n? mpNum? An Integer.}
{k? mpNum? An Integer.}
\end{mpFunctionsExtract}

\begin{mpFunctionsExtract}
\mpFunctionOne
{intNextprime? Integer?  the next prime greater than $n$.}
{n? Integer? An Integer.}
\end{mpFunctionsExtract}

\begin{mpFunctionsExtract}
\mpFunctionTwo
{intGcd? mpNum? the greatest common divisor of $n_1$ and $n_2$}
{n1? mpNum? An Integer.}
{n2? mpNum? An Integer.}
\end{mpFunctionsExtract}

\begin{mpFunctionsExtract}
\mpFunctionTwo
{intGcdExt? mpNumList[3]? the extended greatest common divisor of $n_1$ and $n_2$}
{n1? mpNum? An Integer.}
{n2? mpNum? An Integer.}
\end{mpFunctionsExtract}

\begin{mpFunctionsExtract}
\mpFunctionTwo
{intLcm? mpNum? the least common multiple of $n_1$ and $n_2$.}
{n1? mpNum? An Integer.}
{n2? mpNum? An Integer.}
\end{mpFunctionsExtract}

\begin{mpFunctionsExtract}
\mpFunctionTwo
{intInvertMod? mpNum? the inverse of $n_1$ modulo $n_2$}
{n1? mpNum? An Integer.}
{n2? mpNum? An Integer.}
\end{mpFunctionsExtract}

\begin{mpFunctionsExtract}
\mpFunctionTwo
{intRemoveFactor? mpNum? $n$ with all occurrences of the factor $f$ removed from $n$.}
{n? mpNum? An Integer.}
{f? mpNum? An Integer.}
\end{mpFunctionsExtract}

\begin{mpFunctionsExtract}
\mpFunctionTwo
{intLegendreSymbol? mpNum? the Legendre symbol $\left(\frac{a}{p}\right)$.}
{a? mpNum? An Integer.}
{p? mpNum? An Integer.}
\end{mpFunctionsExtract}

\begin{mpFunctionsExtract}
\mpFunctionTwo
{intJacobiSymbol? mpNum? the Jacobi symbol $\left(\frac{a}{b}\right)$}
{a? mpNum? An Integer.}
{b? mpNum? An Integer.}
\end{mpFunctionsExtract}

\begin{mpFunctionsExtract}
\mpFunctionTwo
{intKroneckerSymbol? mpNum? the Kronecker symbol $\left(\frac{a}{b}\right)$}
{a? mpNum? An Integer.}
{b? mpNum? An Integer.}
\end{mpFunctionsExtract}

\begin{mpFunctionsExtract}
\mpFunctionOne
{intFibonacci? mpNum? the $n^{th}$ Fibonacci number.}
{n? mpNum? An Integer.}
\end{mpFunctionsExtract}

\begin{mpFunctionsExtract}
\mpFunctionOne
{intLucas? mpNum? the $n^{th}$ Lucas number.}
{n? mpNum? An Integer.}
\end{mpFunctionsExtract}

\section{Additional Numbertheoretic Functions}

\begin{mpFunctionsExtract}
\mpFunctionOne
{intIsBpswPrp? mpNum? True if n is a Baillie-Pomerance-Selfridge-Wagstaff probable prime.}
{n? mpNum? An Integer.}
\end{mpFunctionsExtract}

\begin{mpFunctionsExtract}
\mpFunctionTwo
{intIsEulerPrp? mpNum? True if n is an Euler (also known as Solovay-Strassen) probable}
{n? mpNum? An Integer.}
{a? mpNum? An Integer.}
\end{mpFunctionsExtract}

\begin{mpFunctionsExtract}
\mpFunctionTwo
{intIsExtraStrongLucasPrp? mpNum? True if n is an extra strong Lucas probable prime}
{n? mpNum? An Integer.}
{p? mpNum? An Integer.}
\end{mpFunctionsExtract}

\begin{mpFunctionsExtract}
\mpFunctionTwo
{intIsFermatPrp? mpNum? True if n is a Fermat probable prime to the base a}
{n? mpNum? An Integer.}
{a? mpNum? An Integer.}
\end{mpFunctionsExtract}

\begin{mpFunctionsExtract}
\mpFunctionThree
{intIsFibonacciPrp? mpNum? True if n is an Fibonacci probable prime with parameters (p,q).}
{n? mpNum? An Integer.}
{p? mpNum? An Integer.}
{q? mpNum? An Integer.}
\end{mpFunctionsExtract}

\begin{mpFunctionsExtract}
\mpFunctionThree
{intIsLucasPrp? mpNum? True if n is a Lucas probable prime with parameters (p,q).}
{n? mpNum? An Integer.}
{p? mpNum? An Integer.}
{q? mpNum? An Integer.}
\end{mpFunctionsExtract}

\begin{mpFunctionsExtract}
\mpFunctionOne
{intIsSelfridgePrp? mpNum? True if n is a Lucas probable prime with Selfidge parameters (p,q).}
{a? mpNum? An Integer.}
\end{mpFunctionsExtract}

\begin{mpFunctionsExtract}
\mpFunctionOne
{intIsStrongBpswPrp? mpNum? True if n is a strong Baillie-Pomerance-Selfridge-Wagstaff probable prime}
{a? mpNum? An Integer.}
\end{mpFunctionsExtract}

\begin{mpFunctionsExtract}
\mpFunctionThree
{intIsStrongLucasPrp? mpNum? True if n is a strong Lucas probable prime with parameters (p,q).}
{n? mpNum? An Integer.}
{p? mpNum? An Integer.}
{q? mpNum? An Integer.}
\end{mpFunctionsExtract}

\begin{mpFunctionsExtract}
\mpFunctionTwo
{intIsStrongPrp? mpNum? True if n is an strong (also known as Miller-Rabin) probable prime}
{n? mpNum? An Integer.}
{a? mpNum? An Integer.}
\end{mpFunctionsExtract}

\begin{mpFunctionsExtract}
\mpFunctionOne
{intIsStrongSelfridgePrp? mpNum? True if n is a strong Lucas probable prime with Selfidge parameters}
{a? mpNum? An Integer.}
\end{mpFunctionsExtract}

\begin{mpFunctionsExtract}
\mpFunctionThree
{intLucasU? mpNum? the k-th element of the Lucas U sequence defined by p,q}
{p? mpNum? An Integer.}
{q? mpNum? An Integer.}
{k? mpNum? An Integer.}
\end{mpFunctionsExtract}

\begin{mpFunctionsExtract}
\mpFunctionFour
{intLucasModU? mpNum? the k-th element of the Lucas U sequence defined by p,q (mod n)}
{p? mpNum? An Integer.}
{q? mpNum? An Integer.}
{k? mpNum? An Integer.}
{n? mpNum? An Integer.}
\end{mpFunctionsExtract}

\begin{mpFunctionsExtract}
\mpFunctionThree
{intLucasV? mpNum? the k-th element of the Lucas V sequence defined by p,q}
{p? mpNum? An Integer.}
{q? mpNum? An Integer.}
{k? mpNum? An Integer.}
\end{mpFunctionsExtract}

\begin{mpFunctionsExtract}
\mpFunctionFour
{intLucasModV? mpNum? the k-th element of the Lucas V sequence defined by p,q (mod n)}
{p? mpNum? An Integer.}
{q? mpNum? An Integer.}
{k? mpNum? An Integer.}
{n? mpNum? An Integer.}
\end{mpFunctionsExtract}

\section{Random Numbers}

\begin{mpFunctionsExtract}
\mpFunctionOne
{intUrandomb? mpNum? a uniformly distributed random integer in the range 0 to $2^n - 1$, inclusive.}
{n? mpNum? An Integer.}
\end{mpFunctionsExtract}

\begin{mpFunctionsExtract}
\mpFunctionOne
{intUrandomm? mpNum? a uniformly distributed random integer in the range 0 to $n - 1$, inclusive.}
{n? mpNum? An Integer.}
\end{mpFunctionsExtract}

\begin{mpFunctionsExtract}
\mpFunctionOne
{intRrandomb? mpNum? a random integer with long strings of zeros and ones in the binary representation.}
{n? mpNum? An Integer.}
\end{mpFunctionsExtract}

\section{Information Functions for Integers}

\begin{mpFunctionsExtract}
\mpFunctionThree
{IsCongruent? mpNum? TRUE if $n$ is congruent to $c$ modulo $d$, and FALSE otherwise.}
{n? mpNum? An Integer.}
{d? mpNum? An Integer.}
{m? mpNum? An Integer.}
\end{mpFunctionsExtract}

\begin{mpFunctionsExtract}
\mpFunctionThree
{IsCongruent2exp? mpNum? TRUE if $n$ is congruent to $c$ modulo $d$, and FALSE otherwise.}
{n? mpNum? An Integer.}
{c? mpNum? An Integer.}
{b? mpNum? An Integer.}
\end{mpFunctionsExtract}

\begin{mpFunctionsExtract}
\mpFunctionTwo
{IsProbablyPrime? mpNum? 2 if $n$ is definitely prime, returns 1 if $n$ is probably prime (without being certain), and returns 0 if n is definitely composite.}
{n? mpNum? An Integer.}
{reps? mpNum? An Integer.}
\end{mpFunctionsExtract}

\begin{mpFunctionsExtract}
\mpFunctionTwo
{IsDivisible? mpNum? TRUE if $n$ is exactly divisible by $d$.}
{n? mpNum? An Integer.}
{d? mpNum? An Integer.}
\end{mpFunctionsExtract}

\begin{mpFunctionsExtract}
\mpFunctionTwo
{IsDivisible2exp? mpNum? TRUE if $n$ is exactly divisible by $2^b$.}
{n? mpNum? An Integer.}
{b? mpNum? An Integer.}
\end{mpFunctionsExtract}

\begin{mpFunctionsExtract}
\mpFunctionOne
{IsPerfectPower? mpNum? TRUE if $n$ is a perfect power.}
{n? mpNum? An Integer.}
\end{mpFunctionsExtract}

\begin{mpFunctionsExtract}
\mpFunctionOne
{IsPerfectSquare? mpNum? non-zero if $n$ is a perfect square.}
{n? mpNum? An Integer.}
\end{mpFunctionsExtract}

\chapter{FMPQ}

\chapter{ARB}

\section{Multiprecision Ball Arithmetic (ARB)}

\section{Information Functions for Intervals}

\begin{mpFunctionsExtract}
\mpFunctionOne
{IsEmpty? mpNum? TRUE  if $x$ is empty (its endpoints are in reverse order), and FALSE otherwise.}
{x? mpNum? A real number.}
\end{mpFunctionsExtract}

\begin{mpFunctionsExtract}
\mpFunctionTwo
{IsInside? mpNum? TRUE  if $x$ is contained in $y$, and FALSE otherwise.}
{x? mpNum? A real number.}
{y? mpNum? A real number.}
\end{mpFunctionsExtract}

\begin{mpFunctionsExtract}
\mpFunctionTwo
{IsStrictlyInside? mpNum? TRUE  if the second interval $y$ is contained in the interior of $x$, and FALSE otherwise.}
{x? mpNum? A real number.}
{y? mpNum? A real number.}
\end{mpFunctionsExtract}

\begin{mpFunctionsExtract}
\mpFunctionOne
{IsStrictlyNeg? mpNum? TRUE if $x$ contains only negative numbers, and FALSE otherwise.}
{x? mpNum? A real number.}
\end{mpFunctionsExtract}

\begin{mpFunctionsExtract}
\mpFunctionOne
{IsStrictlyPos? mpNum? TRUE if $x$ contains only positive numbers, and FALSE otherwise.}
{x? mpNum? A real number.}
\end{mpFunctionsExtract}

\chapter{ACB}

\chapter{BLAS Support (based on Eigen)}

\section{BLAS Level 1 Support and related Functions}

\begin{mpFunctionsExtract}
\mpFunctionTwo
{RDot? mpNum? the real scalar product $\boldsymbol{x}^T \boldsymbol{y}$ for the real vectors $\boldsymbol{x}$ and $\boldsymbol{y}$.}
{x? mpNum[]? A vector of real numbers.}
{y? mpNum[]? A vector of real numbers.}
\end{mpFunctionsExtract}

\begin{mpFunctionsExtract}
\mpFunctionTwo
{cplxDotu? mpNum? the complex  scalar product $\boldsymbol{x}^T \boldsymbol{y}$ for the complex  vectors $\boldsymbol{x}$ and $\boldsymbol{y}$.}
{x? mpNum[]? A vector of complex numbers.}
{y? mpNum[]? A vector of complex numbers.}
\end{mpFunctionsExtract}

\begin{mpFunctionsExtract}
\mpFunctionTwo
{cplxDotc? mpNum? the complex conjugate scalar product $\boldsymbol{x}^H \boldsymbol{y}$ for the complex  vectors $\boldsymbol{x}$ and $\boldsymbol{y}$.}
{x? mpNum[]? A vector of complex numbers.}
{y? mpNum[]? A vector of complex numbers.}
\end{mpFunctionsExtract}

\begin{mpFunctionsExtract}
\mpFunctionTwo
{RNrm2? mpNum? the Euclidean norm $||\boldsymbol{x}||_2$ of the real vector $\boldsymbol{x}$.}
{x? mpNum[]? A vector of real numbers.}
{y? mpNum[]? A vector of real numbers.}
\end{mpFunctionsExtract}

\begin{mpFunctionsExtract}
\mpFunctionTwo
{cplxNrm2? mpNum? the Euclidean norm $||\boldsymbol{x}||_2$ of the complex vector $\boldsymbol{x}$.}
{x? mpNum[]? A vector of complex numbers.}
{y? mpNum[]? A vector of complex numbers.}
\end{mpFunctionsExtract}

\begin{mpFunctionsExtract}
\mpFunctionTwo
{RAsum? mpNum? the the absolute sum of the elements of the real vector $\boldsymbol{x}$.}
{x? mpNum[]? A vector of real numbers.}
{y? mpNum[]? A vector of real numbers.}
\end{mpFunctionsExtract}

\begin{mpFunctionsExtract}
\mpFunctionTwo
{cplxAsum? mpNum? the  sum of the magnitudes of the real and imaginary parts of the complex vector $\boldsymbol{x}$.}
{x? mpNum[]? A vector of complex numbers.}
{y? mpNum[]? A vector of complex numbers.}
\end{mpFunctionsExtract}

\begin{mpFunctionsExtract}
\mpFunctionThree
{RAxpy? mpNum?  the sum $\alpha \boldsymbol{x} + \boldsymbol{y}$ for the real scalar $\alpha$ and the real vectors $\boldsymbol{x}$ and $\boldsymbol{y}$.}
{$\alpha$? mpNum? A real scalar.}
{x? mpNum[]? A vector of real numbers.}
{y? mpNum[]? A vector of real numbers.}
\end{mpFunctionsExtract}

\begin{mpFunctionsExtract}
\mpFunctionThree
{cplxAxpy? mpNum? the sum $\alpha \boldsymbol{x} + \boldsymbol{y}$ for the complex scalar $\alpha$ and the complex vectors $\boldsymbol{x}$ and $\boldsymbol{y}$.}
{$\alpha$? mpNum? A complex scalar.}
{x? mpNum[]? A vector of complex numbers.}
{y? mpNum[]? A vector of complex numbers.}
\end{mpFunctionsExtract}

\section{BLAS Level 2 Support}

\begin{mpFunctionsExtract}
\mpFunctionSix
{RGemv? mpNum? the matrix-vector product and sum for a general matrix.}
{TransA? Integer? An indicator specifying the multiplication.}
{$\alpha$? mpNum? A real scalar.}
{A? mpNum[,]? A matrix of real numbers.}
{x? mpNum[]? A vector of real numbers.}
{$\beta$? mpNum? A real scalar.}
{y? mpNum[]? A vector of real numbers.}
\end{mpFunctionsExtract}

\begin{mpFunctionsExtract}
\mpFunctionSix
{cplxGemv? mpNum? the matrix-vector product and sum for a general matrix.}
{TransA? Integer? An indicator specifying the multiplication.}
{$\alpha$? mpNum? A complex scalar.}
{A? mpNum[,]? A matrix of complex numbers.}
{x? mpNum[]? A vector of complex numbers.}
{$\beta$? mpNum? A complex scalar.}
{y? mpNum[]? A vector of complex numbers.}
\end{mpFunctionsExtract}

\begin{mpFunctionsExtract}
\mpFunctionFive
{RTrmv? mpNum?  the matrix-vector product for a triangular matrix.}
{Uplo? Integer? An indicator specifying whether the upper or lower triangle will be used.}
{TransA? Integer? An indicator specifying the multiplication.}
{Diag? Integer? An indicator specifying the use of the diagonal.}
{A? mpNum[,]? A matrix of real numbers.}
{x? mpNum[]? A vector of real numbers.}
\end{mpFunctionsExtract}

\begin{mpFunctionsExtract}
\mpFunctionFive
{cplxTrmv? mpNum?  the matrix-vector product for a triangular matrix.}
{Uplo? Integer? An indicator specifying whether the upper or lower triangle will be used.}
{TransA? Integer? An indicator specifying the multiplication.}
{Diag? Integer? An indicator specifying the use of the diagonal.}
{A? mpNum[,]? A matrix of complex numbers.}
{x? mpNum[]? A vector of complex numbers.}
\end{mpFunctionsExtract}

\begin{mpFunctionsExtract}
\mpFunctionFive
{RTrsv? mpNum?  the inverse matrix-vector product for a triangular matrix.}
{Uplo? Integer? An indicator specifying whether the upper or lower triangle will be used.}
{TransA? Integer? An indicator specifying the multiplication.}
{Diag? Integer? An indicator specifying the use of the diagonal.}
{A? mpNum[,]? A matrix of real numbers.}
{x? mpNum[]? A vector of real numbers.}
\end{mpFunctionsExtract}

\begin{mpFunctionsExtract}
\mpFunctionFive
{cplxTrsv? mpNum?  the inverse matrix-vector product for a triangular matrix.}
{Uplo? Integer? An indicator specifying whether the upper or lower triangle will be used.}
{TransA? Integer? An indicator specifying the multiplication.}
{Diag? Integer? An indicator specifying the use of the diagonal.}
{A? mpNum[,]? A matrix of complex numbers.}
{x? mpNum[]? A vector of complex numbers.}
\end{mpFunctionsExtract}

\begin{mpFunctionsExtract}
\mpFunctionSix
{RSymv? mpNum? the matrix-vector product and sum for a symmetric matrix.}
{Uplo? Integer? An indicator specifying whether the upper or lower triangle will be used.}
{$\alpha$? mpNum? A real scalar.}
{A? mpNum[,]? A matrix of real numbers.}
{x? mpNum[]? A vector of real numbers.}
{$\beta$? mpNum? A real scalar.}
{y? mpNum[]? A vector of real numbers.}
\end{mpFunctionsExtract}

\begin{mpFunctionsExtract}
\mpFunctionSix
{cplxHemv? mpNum? the matrix-vector product and sum for a hermitian matrix.}
{Uplo? Integer? An indicator specifying whether the upper or lower triangle will be used.}
{$\alpha$? mpNum? A complex scalar.}
{A? mpNum[,]? A matrix of complex numbers.}
{x? mpNum[]? A vector of complex numbers.}
{$\beta$? mpNum? A complex scalar.}
{y? mpNum[]? A vector of complex numbers.}
\end{mpFunctionsExtract}

\begin{mpFunctionsExtract}
\mpFunctionFour
{RGer? mpNum? the rank-1 update for a general matrix}
{$\alpha$? mpNum? A real scalar.}
{x? mpNum[]? A vector of real numbers.}
{y? mpNum[]? A vector of real numbers.}
{A? mpNum[,]? A matrix of real numbers.}
\end{mpFunctionsExtract}

\begin{mpFunctionsExtract}
\mpFunctionFour
{cplxGeru? mpNum? the rank-1 update for a general matrix}
{$\alpha$? mpNum? A complex scalar.}
{x? mpNum[]? A vector of complex numbers.}
{y? mpNum[]? A vector of complex numbers.}
{A? mpNum[,]? A matrix of complex numbers.}
\end{mpFunctionsExtract}

\begin{mpFunctionsExtract}
\mpFunctionFour
{cplxGerc? mpNum? the rank-1 update for a general matrix}
{$\alpha$? mpNum? A complex scalar.}
{x? mpNum[]? A vector of complex numbers.}
{y? mpNum[]? A vector of complex numbers.}
{A? mpNum[,]? A matrix of complex numbers.}
\end{mpFunctionsExtract}

\begin{mpFunctionsExtract}
\mpFunctionFour
{RSyr? mpNum? the Rank-1 update for a symmetric matrix.}
{Uplo? Integer? An indicator specifying whether the upper or lower triangle will be used.}
{$\alpha$? mpNum? A real scalar.}
{x? mpNum[]? A vector of real numbers.}
{A? mpNum[,]? A matrix of real numbers.}
\end{mpFunctionsExtract}

\begin{mpFunctionsExtract}
\mpFunctionFour
{cplxHer? mpNum? the Rank-1 update for a hermitian matrix.}
{Uplo? Integer? An indicator specifying whether the upper or lower triangle will be used.}
{$\alpha$? mpNum? A complex scalar.}
{x? mpNum[]? A vector of complex numbers.}
{A? mpNum[,]? A matrix of complex numbers.}
\end{mpFunctionsExtract}

\begin{mpFunctionsExtract}
\mpFunctionFive
{RSyr2? mpNum? the Rank-1 update for a symmetric matrix.}
{Uplo? Integer? An indicator specifying whether the upper or lower triangle will be used.}
{$\alpha$? mpNum? A real scalar.}
{x? mpNum[]? A vector of real numbers.}
{y? mpNum[]? A vector of real numbers.}
{A? mpNum[,]? A matrix of real numbers.}
\end{mpFunctionsExtract}

\begin{mpFunctionsExtract}
\mpFunctionFive
{cplxHer2? mpNum? the Rank-1 update for a hermitian matrix.}
{Uplo? Integer? An indicator specifying whether the upper or lower triangle will be used.}
{$\alpha$? mpNum? A complex scalar.}
{x? mpNum[]? A vector of complex numbers.}
{y? mpNum[]? A vector of complex numbers.}
{A? mpNum[,]? A matrix of complex numbers.}
\end{mpFunctionsExtract}

\section{BLAS Level 3 Support}

\begin{mpFunctionsExtract}
\mpFunctionSeven
{RGemm? mpNum? the matrix-matrix product and sum for a general matrix.}
{TransA? Integer? An indicator specifying the multiplication.}
{TransB? Integer? An indicator specifying the multiplication.}
{$\alpha$? mpNum? A real scalar.}
{A? mpNum[,]? A matrix of real numbers.}
{B? mpNum[,]? A matrix of real numbers.}
{$\beta$? mpNum? A real scalar.}
{C? mpNum[,]? A matrix of real numbers.}
\end{mpFunctionsExtract}

\begin{mpFunctionsExtract}
\mpFunctionSeven
{cplxGemm? mpNum? the matrix-matrix product and sum for a general matrix.}
{TransA? Integer? An indicator specifying the multiplication.}
{TransB? Integer? An indicator specifying the multiplication.}
{$\alpha$? mpNum? A real scalar.}
{A? mpNum[,]? A matrix of real numbers.}
{B? mpNum[,]? A matrix of real numbers.}
{$\beta$? mpNum? A real scalar.}
{C? mpNum[,]? A matrix of real numbers.}
\end{mpFunctionsExtract}

\begin{mpFunctionsExtract}
\mpFunctionSeven
{RSymm? mpNum? the matrix-matrix product and sum for a symmetric matrix.}
{Side? Integer? An indicator specifying the order of the multiplication.}
{Uplo? Integer? An indicator specifying whether the upper or lower triangle will be used.}
{$\alpha$? mpNum? A real scalar.}
{A? mpNum[,]? A matrix of real numbers.}
{B? mpNum[,]? A matrix of real numbers.}
{$\beta$? mpNum? A real scalar.}
{C? mpNum[,]? A matrix of real numbers.}
\end{mpFunctionsExtract}

\begin{mpFunctionsExtract}
\mpFunctionSeven
{cplxSymm? mpNum? the matrix-matrix product and sum for a symmetric matrix.}
{Side? Integer? An indicator specifying the order of the multiplication.}
{Uplo? Integer? An indicator specifying whether the upper or lower triangle will be used.}
{$\alpha$? mpNum? A complex scalar.}
{A? mpNum[,]? A matrix of complex numbers.}
{B? mpNum[,]? A matrix of complex numbers.}
{$\beta$? mpNum? A complex scalar.}
{C? mpNum[,]? A matrix of complex numbers.}
\end{mpFunctionsExtract}

\begin{mpFunctionsExtract}
\mpFunctionSeven
{cplxHemm? mpNum? the matrix-matrix product and sum for a hermitian matrix.}
{Side? Integer? An indicator specifying the order of the multiplication.}
{Uplo? Integer? An indicator specifying whether the upper or lower triangle will be used.}
{$\alpha$? mpNum? A complex scalar.}
{A? mpNum[,]? A matrix of complex numbers.}
{B? mpNum[,]? A matrix of complex numbers.}
{$\beta$? mpNum? A complex scalar.}
{C? mpNum[,]? A matrix of complex numbers.}
\end{mpFunctionsExtract}

\begin{mpFunctionsExtract}
\mpFunctionSix
{RTrmm? mpNum? the matrix-matrix produc for a triangular matrix.}
{Side? Integer? An indicator specifying the order of the multiplication.}
{Uplo? Integer? An indicator specifying whether the upper or lower triangle will be used.}
{TransA? Integer? An indicator specifying the multiplication.}
{$\alpha$? mpNum? A real scalar.}
{A? mpNum[,]? A matrix of real numbers.}
{B? mpNum[,]? A matrix of real numbers.}
\end{mpFunctionsExtract}

\begin{mpFunctionsExtract}
\mpFunctionSix
{cplxTrmm? mpNum? the matrix-matrix product for a triangular matrix.}
{Side? Integer? An indicator specifying the order of the multiplication.}
{Uplo? Integer? An indicator specifying whether the upper or lower triangle will be used.}
{TransA? Integer? An indicator specifying the multiplication.}
{$\alpha$? mpNum? A complex scalar.}
{A? mpNum[,]? A matrix of complex numbers.}
{B? mpNum[,]? A matrix of complex numbers.}
\end{mpFunctionsExtract}

\begin{mpFunctionsExtract}
\mpFunctionSix
{RTrsm? mpNum? the inverse matrix-matrix product for a triangular matrix.}
{Side? Integer? An indicator specifying the order of the multiplication.}
{Uplo? Integer? An indicator specifying whether the upper or lower triangle will be used.}
{TransA? Integer? An indicator specifying the multiplication.}
{$\alpha$? mpNum? A real scalar.}
{A? mpNum[,]? A matrix of real numbers.}
{B? mpNum[,]? A matrix of real numbers.}
\end{mpFunctionsExtract}

\begin{mpFunctionsExtract}
\mpFunctionSix
{cplxTrsm? mpNum? the inverse matrix-matrix product for a triangular matrix.}
{Side? Integer? An indicator specifying the order of the multiplication.}
{Uplo? Integer? An indicator specifying whether the upper or lower triangle will be used.}
{TransA? Integer? An indicator specifying the multiplication.}
{$\alpha$? mpNum? A complex scalar.}
{A? mpNum[,]? A matrix of complex numbers.}
{B? mpNum[,]? A matrix of complex numbers.}
\end{mpFunctionsExtract}

\begin{mpFunctionsExtract}
\mpFunctionSix
{Rsyrk? mpNum? a rank-k update for a symmetric matrix.}
{Uplo? Integer? An indicator specifying whether the upper or lower triangle will be used.}
{Trans? Integer? An indicator specifying the multiplication.}
{$\alpha$? mpNum? A real scalar.}
{A? mpNum[,]? A matrix of real numbers.}
{$\beta$? mpNum? A real scalar.}
{C? mpNum[,]? A matrix of real numbers.}
\end{mpFunctionsExtract}

\begin{mpFunctionsExtract}
\mpFunctionSix
{cplxSyrk? mpNum? a rank-k update for a symmetric matrix.}
{Uplo? Integer? An indicator specifying whether the upper or lower triangle will be used.}
{Trans? Integer? An indicator specifying the multiplication.}
{$\alpha$? mpNum? A complex scalar.}
{A? mpNum[,]? A matrix of complex numbers.}
{$\beta$? mpNum? A complex scalar.}
{C? mpNum[,]? A matrix of complex numbers.}
\end{mpFunctionsExtract}

\begin{mpFunctionsExtract}
\mpFunctionSix
{cplxHerk? mpNum? a rank-k update for a hermitian matrix.}
{Uplo? Integer? An indicator specifying whether the upper or lower triangle will be used.}
{Trans? Integer? An indicator specifying the multiplication.}
{$\alpha$? mpNum? A complex scalar.}
{A? mpNum[,]? A matrix of complex numbers.}
{$\beta$? mpNum? A complex scalar.}
{C? mpNum[,]? A matrix of complex numbers.}
\end{mpFunctionsExtract}

\begin{mpFunctionsExtract}
\mpFunctionSeven
{Rsyr2k? mpNum? a rank-k update for a symmetric matrix.}
{Uplo? Integer? An indicator specifying whether the upper or lower triangle will be used.}
{Trans? Integer? An indicator specifying the multiplication.}
{$\alpha$? mpNum? A real scalar.}
{A? mpNum[,]? A matrix of real numbers.}
{B? mpNum[,]? A matrix of real numbers.}
{$\beta$? mpNum? A real scalar.}
{C? mpNum[,]? A matrix of real numbers.}
\end{mpFunctionsExtract}

\begin{mpFunctionsExtract}
\mpFunctionSeven
{cplxSyr2k? mpNum? a rank-k update for a symmetric matrix.}
{Uplo? Integer? An indicator specifying whether the upper or lower triangle will be used.}
{Trans? Integer? An indicator specifying the multiplication.}
{$\alpha$? mpNum? A complex scalar.}
{A? mpNum[,]? A matrix of complex numbers.}
{B? mpNum[,]? A matrix of complex numbers.}
{$\beta$? mpNum? A complex scalar.}
{C? mpNum[,]? A matrix of complex numbers.}
\end{mpFunctionsExtract}

\begin{mpFunctionsExtract}
\mpFunctionSeven
{cplxHer2k? mpNum? a rank-k update for a hermitian matrix.}
{Uplo? Integer? An indicator specifying whether the upper or lower triangle will be used.}
{Trans? Integer? An indicator specifying the multiplication.}
{$\alpha$? mpNum? A complex scalar.}
{A? mpNum[,]? A matrix of complex numbers.}
{B? mpNum[,]? A matrix of complex numbers.}
{$\beta$? mpNum? A complex scalar.}
{C? mpNum[,]? A matrix of complex numbers.}
\end{mpFunctionsExtract}

\chapter{Linear Solvers (based on Eigen)}

\section{Cholesky Decomposition without Pivoting}

\begin{mpFunctionsExtract}
\mpFunctionFour
{DecompCholeskyLLT? mpNumList? the Cholesky decomposition $A = LL^* = U^*U$ of a matrix.}
{A? mpNum[,]? the real matrix of which we are computing the $LL^T$ Cholesky decomposition.}
{B? mpNum[,]? A vector or matrix of real numbers.}
{UpLo? Integer? the triangular part that will be used for the decompositon: Lower (default) or Upper. The other triangular part won't be read.}
{Output? String? A string specifying the output options.}
\end{mpFunctionsExtract}

\begin{mpFunctionsExtract}
\mpFunctionFour
{cplxDecompCholeskyLLT? mpNumList? the Cholesky decomposition $A = LL^* = U^*U$ of a matrix.}
{A? mpNum[,]? the complex matrix of which we are computing the $LL^T$ Cholesky decomposition.}
{B? mpNum[,]? A vector or complex of real numbers.}
{UpLo? Integer? the triangular part that will be used for the decompositon: Lower (default) or Upper. The other triangular part won't be read.}
{Output? String? A string specifying the output options.}
\end{mpFunctionsExtract}

\begin{mpFunctionsExtract}
\mpFunctionThree
{SolveCholeskyLLT? mpNum[]? the solution $x$ of $A x = b$ , based on a Cholesky decomposition.}
{A? mpNum[,]? A symmetric positive definite real matrix.}
{B? mpNum[,]? A real vector or matrix.}
{UpLo? Integer? the triangular part that will be used for the decompositon: Lower (default) or Upper. The other triangular part won't be read.}
\end{mpFunctionsExtract}

\begin{mpFunctionsExtract}
\mpFunctionThree
{cplxSolveCholeskyLLT? mpNum[]? the solution $x$ of $A x = b$ , based on a Cholesky decomposition.}
{A? mpNum[,]? A symmetric positive definite complex matrix.}
{B? mpNum[,]? A complex vector or matrix.}
{UpLo? Integer? the triangular part that will be used for the decompositon: Lower (default) or Upper. The other triangular part won't be read.}
\end{mpFunctionsExtract}

\begin{mpFunctionsExtract}
\mpFunctionTwo
{InvertCholeskyLLT? mpNum[]? $A^{-1}$, the inverse of $A$, based on a Cholesky decomposition.}
{A? mpNum[,]? A symmetric positive definite real matrix.}
{UpLo? Integer? the triangular part that will be used for the decompositon: Lower (default) or Upper. The other triangular part won't be read.}
\end{mpFunctionsExtract}

\begin{mpFunctionsExtract}
\mpFunctionTwo
{cplxInvertCholeskyLLT? mpNum[]? $A^{-1}$, the inverse of $A$, based on a Cholesky decomposition.}
{A? mpNum[,]? A symmetric positive definite complex matrix.}
{UpLo? Integer? the triangular part that will be used for the decompositon: Lower (default) or Upper. The other triangular part won't be read.}
\end{mpFunctionsExtract}

\begin{mpFunctionsExtract}
\mpFunctionTwo
{DetCholeskyLLT? mpNum? $|A|$, the determinant of $A$, based on a Cholesky decomposition.}
{A? mpNum[,]? A symmetric positive definite real matrix.}
{UpLo? Integer? the triangular part that will be used for the decompositon: Lower (default) or Upper. The other triangular part won't be read.}
\end{mpFunctionsExtract}

\begin{mpFunctionsExtract}
\mpFunctionTwo
{cplxDetCholeskyLLT? mpNum? $|A|$, the determinant of $A$, based on a Cholesky decomposition.}
{A? mpNum[,]? A symmetric positive definite complex matrix.}
{UpLo? Integer? the triangular part that will be used for the decompositon: Lower (default) or Upper. The other triangular part won't be read.}
\end{mpFunctionsExtract}

\section{Cholesky Decomposition with Pivoting}

\begin{mpFunctionsExtract}
\mpFunctionFour
{DecompCholeskyLDLT? mpNumList? the Cholesky decomposition with pivoting of $A = LL^* = U^*U$.}
{A? mpNum[,]? the real matrix of which we are computing the $LL^T$ Cholesky decomposition.}
{B? mpNum[,]? A vector or matrix of real numbers.}
{UpLo? Integer? the triangular part that will be used for the decompositon: Lower (default) or Upper. The other triangular part won't be read.}
{Output? String? A string specifying the output options.}
\end{mpFunctionsExtract}

\begin{mpFunctionsExtract}
\mpFunctionFour
{cplxDecompCholeskyLDLT? mpNumList? the Cholesky decomposition with pivoting of $A = LL^* = U^*U$.}
{A? mpNum[,]? the complex matrix of which we are computing the $LL^T$ Cholesky decomposition.}
{B? mpNum[,]? A vector or complex of real numbers.}
{UpLo? Integer? the triangular part that will be used for the decompositon: Lower (default) or Upper. The other triangular part won't be read.}
{Output? String? A string specifying the output options.}
\end{mpFunctionsExtract}

\begin{mpFunctionsExtract}
\mpFunctionThree
{SolveCholeskyLDLT? mpNum[]? the solution $x$ of $A x = b$ , based on a Cholesky decomposition with pivoting.}
{A? mpNum[,]? A symmetric positive definite real matrix.}
{B? mpNum[,]? A real vector or matrix.}
{UpLo? Integer? the triangular part that will be used for the decompositon: Lower (default) or Upper. The other triangular part won't be read.}
\end{mpFunctionsExtract}

\begin{mpFunctionsExtract}
\mpFunctionThree
{cplxSolveCholeskyLDLT? mpNum[]? the solution $x$ of $A x = b$ , based on a Cholesky decomposition with pivoting.}
{A? mpNum[,]? A symmetric positive definite complex matrix.}
{B? mpNum[,]? A complex vector or matrix.}
{UpLo? Integer? the triangular part that will be used for the decompositon: Lower (default) or Upper. The other triangular part won't be read.}
\end{mpFunctionsExtract}

\begin{mpFunctionsExtract}
\mpFunctionTwo
{InvertCholeskyLDLT? mpNum[]? $A^{-1}$, the inverse of $A$, based on a Cholesky decomposition with pivoting.}
{A? mpNum[,]? A symmetric positive definite real matrix.}
{UpLo? Integer? the triangular part that will be used for the decompositon: Lower (default) or Upper. The other triangular part won't be read.}
\end{mpFunctionsExtract}

\begin{mpFunctionsExtract}
\mpFunctionTwo
{cplxInvertCholeskyLDLT? mpNum[]? $A^{-1}$, the inverse of $A$, based on a Cholesky decomposition with pivoting.}
{A? mpNum[,]? A symmetric positive definite complex matrix.}
{UpLo? Integer? the triangular part that will be used for the decompositon: Lower (default) or Upper. The other triangular part won't be read.}
\end{mpFunctionsExtract}

\begin{mpFunctionsExtract}
\mpFunctionTwo
{DetCholeskyLDLT? mpNum? $|A|$, the determinant of $A$, based on a Cholesky decomposition.}
{A? mpNum[,]? A symmetric positive definite real matrix.}
{UpLo? Integer? the triangular part that will be used for the decompositon: Lower (default) or Upper. The other triangular part won't be read.}
\end{mpFunctionsExtract}

\begin{mpFunctionsExtract}
\mpFunctionTwo
{cplxDetCholeskyLDLT? mpNum? $|A|$, the determinant of $A$, based on a Cholesky decomposition.}
{A? mpNum[,]? A symmetric positive definite complex matrix.}
{UpLo? Integer? the triangular part that will be used for the decompositon: Lower (default) or Upper. The other triangular part won't be read.}
\end{mpFunctionsExtract}

\section{LU Decomposition with partial Pivoting}

\begin{mpFunctionsExtract}
\mpFunctionThree
{DecompPartialPivLU? mpNumList? the LU decomposition with partial pivoting of $A = PLU$.}
{A? mpNum[,]? the square real matrix of which we are computing the $LU$ decomposition.}
{B? mpNum[,]? A vector or matrix of real numbers.}
{Output? String? A string specifying the output options.}
\end{mpFunctionsExtract}

\begin{mpFunctionsExtract}
\mpFunctionThree
{cplxDecompPartialPivLU? mpNumList? the LU decomposition with partial pivoting of $A = PLU$.}
{A? mpNum[,]? the square complex matrix of which we are computing the $LU$ decomposition.}
{B? mpNum[,]? A vector or complex of real numbers.}
{Output? String? A string specifying the output options.}
\end{mpFunctionsExtract}

\begin{mpFunctionsExtract}
\mpFunctionTwo
{SolvePartialPivLU? mpNum[]? the solution $x$ of $A x = b$ , based on a LU decomposition with partial pivoting.}
{A? mpNum[,]? A square real matrix.}
{B? mpNum[,]? A real vector or matrix.}
\end{mpFunctionsExtract}

\begin{mpFunctionsExtract}
\mpFunctionTwo
{cplxSolvePartialPivLU? mpNum[]? the solution $x$ of $A x = b$ , based on a LU decomposition with partial pivoting.}
{A? mpNum[,]? A square complex matrix.}
{B? mpNum[,]? A complex vector or matrix.}
\end{mpFunctionsExtract}

\begin{mpFunctionsExtract}
\mpFunctionOne
{InvertPartialPivLU? mpNum[]? $A^{-1}$, the inverse of $A$, based on a LU decomposition with partial pivoting.}
{A? mpNum[,]? A square real matrix.}
\end{mpFunctionsExtract}

\begin{mpFunctionsExtract}
\mpFunctionOne
{cplxInvertPartialPivLU? mpNum[]? $A^{-1}$, the inverse of $A$, based on a LU decomposition with partial pivoting.}
{A? mpNum[,]? A square complex matrix.}
\end{mpFunctionsExtract}

\begin{mpFunctionsExtract}
\mpFunctionOne
{DetPartialPivLU? mpNum? $|A|$, the determinant of $A$, based on a LU decomposition with partial pivoting.}
{A? mpNum[,]? A square real matrix.}
\end{mpFunctionsExtract}

\begin{mpFunctionsExtract}
\mpFunctionOne
{cplxDetPartialPivLU? mpNum? $|A|$, the determinant of $A$, based on a LU decomposition with partial pivoting.}
{A? mpNum[,]? A square complex matrix.}
\end{mpFunctionsExtract}

\section{LU Decomposition with full Pivoting}

\begin{mpFunctionsExtract}
\mpFunctionThree
{DecompFullPivLU? mpNumList? the LU decomposition with full pivoting of $A = PLUQ$.}
{A? mpNum[,]? the square real matrix of which we are computing the $LU$ decomposition.}
{B? mpNum[,]? A vector or matrix of real numbers.}
{Output? String? A string specifying the output options.}
\end{mpFunctionsExtract}

\begin{mpFunctionsExtract}
\mpFunctionThree
{cplxDecompFullPivLU? mpNumList? the LU decomposition with full pivoting of $A = PLUQ$.}
{A? mpNum[,]? the square complex matrix of which we are computing the $LU$ decomposition.}
{B? mpNum[,]? A vector or complex of real numbers.}
{Output? String? A string specifying the output options.}
\end{mpFunctionsExtract}

\begin{mpFunctionsExtract}
\mpFunctionTwo
{SolveFullPivLU? mpNum[]? the solution $x$ of $A x = b$ , based on a LU decomposition with full pivoting.}
{A? mpNum[,]? A square real matrix.}
{B? mpNum[,]? A real vector or matrix.}
\end{mpFunctionsExtract}

\begin{mpFunctionsExtract}
\mpFunctionTwo
{cplxSolveFullPivLU? mpNum[]? the solution $x$ of $A x = b$ , based on a LU decomposition with full pivoting.}
{A? mpNum[,]? A square complex matrix.}
{B? mpNum[,]? A complex vector or matrix.}
\end{mpFunctionsExtract}

\begin{mpFunctionsExtract}
\mpFunctionOne
{InvertFullPivLU? mpNum[]? $A^{-1}$, the inverse of $A$, based on a LU decomposition with full pivoting.}
{A? mpNum[,]? A square real matrix.}
\end{mpFunctionsExtract}

\begin{mpFunctionsExtract}
\mpFunctionOne
{cplxInvertFullPivLU? mpNum[]? $A^{-1}$, the inverse of $A$, based on a LU decomposition with full pivoting.}
{A? mpNum[,]? A square complex matrix.}
\end{mpFunctionsExtract}

\begin{mpFunctionsExtract}
\mpFunctionOne
{DetFullPivLU? mpNum? $|A|$, the determinant of $A$, based on a LU decomposition with full pivoting.}
{A? mpNum[,]? A square real matrix.}
\end{mpFunctionsExtract}

\begin{mpFunctionsExtract}
\mpFunctionOne
{cplxDetFullPivLU? mpNum? $|A|$, the determinant of $A$, based on a LU decomposition with full pivoting.}
{A? mpNum[,]? A square complex matrix.}
\end{mpFunctionsExtract}

\section{QR Decomposition without Pivoting}

\begin{mpFunctionsExtract}
\mpFunctionThree
{DecompQR? mpNumList? the QR decomposition without pivoting of $A = QR$.}
{A? mpNum[,]? the square real matrix of which we are computing the $LU$ decomposition.}
{B? mpNum[,]? A vector or matrix of real numbers.}
{Output? String? A string specifying the output options.}
\end{mpFunctionsExtract}

\begin{mpFunctionsExtract}
\mpFunctionThree
{cplxDecompQR? mpNumList? the QR decomposition without pivoting of $A = QR$.}
{A? mpNum[,]? the square complex matrix of which we are computing the $LU$ decomposition.}
{B? mpNum[,]? A vector or complex of real numbers.}
{Output? String? A string specifying the output options.}
\end{mpFunctionsExtract}

\begin{mpFunctionsExtract}
\mpFunctionTwo
{SolveQR? mpNum[]? the solution $x$ of $A x = b$ , based on a QR decomposition without pivoting.}
{A? mpNum[,]? A square real matrix.}
{B? mpNum[,]? A real vector or matrix.}
\end{mpFunctionsExtract}

\begin{mpFunctionsExtract}
\mpFunctionTwo
{cplxSolveQR? mpNum[]? the solution $x$ of $A x = b$ , based on a QR decomposition without pivoting.}
{A? mpNum[,]? A square complex matrix.}
{B? mpNum[,]? A complex vector or matrix.}
\end{mpFunctionsExtract}

\begin{mpFunctionsExtract}
\mpFunctionOne
{InvertQR? mpNum[]? $A^{-1}$, the inverse of $A$, based on a QR decomposition without pivoting.}
{A? mpNum[,]? A square real matrix.}
\end{mpFunctionsExtract}

\begin{mpFunctionsExtract}
\mpFunctionOne
{cplxInvertQR? mpNum[]? $A^{-1}$, the inverse of $A$, based on a QR decomposition without pivoting.}
{A? mpNum[,]? A square complex matrix.}
\end{mpFunctionsExtract}

\begin{mpFunctionsExtract}
\mpFunctionOne
{DetQR? mpNum? $|A|$, the determinant of $A$, based on a QR decomposition without pivoting.}
{A? mpNum[,]? A square real matrix.}
\end{mpFunctionsExtract}

\begin{mpFunctionsExtract}
\mpFunctionOne
{cplxDetQR? mpNum? $|A|$, the determinant of $A$, based on a QR decomposition without pivoting.}
{A? mpNum[,]? A square complex matrix.}
\end{mpFunctionsExtract}

\section{QR Decomposition with column Pivoting}

\begin{mpFunctionsExtract}
\mpFunctionThree
{DecompColPivQR? mpNumList? the QR decomposition with column-pivoting of $A = QR$.}
{A? mpNum[,]? the square real matrix of which we are computing the $LU$ decomposition.}
{B? mpNum[,]? A vector or matrix of real numbers.}
{Output? String? A string specifying the output options.}
\end{mpFunctionsExtract}

\begin{mpFunctionsExtract}
\mpFunctionThree
{cplxDecompColPivQR? mpNumList? the QR decomposition with column-pivoting of $A = QR$.}
{A? mpNum[,]? the square complex matrix of which we are computing the $LU$ decomposition.}
{B? mpNum[,]? A vector or complex of real numbers.}
{Output? String? A string specifying the output options.}
\end{mpFunctionsExtract}

\begin{mpFunctionsExtract}
\mpFunctionTwo
{SolveColPivQR? mpNum[]? the solution $x$ of $A x = b$ , based on a QR decomposition with column-pivoting.}
{A? mpNum[,]? A square real matrix.}
{B? mpNum[,]? A real vector or matrix.}
\end{mpFunctionsExtract}

\begin{mpFunctionsExtract}
\mpFunctionTwo
{cplxSolveColPivQR? mpNum[]? the solution $x$ of $A x = b$ , based on a QR decomposition with column-pivoting.}
{A? mpNum[,]? A square complex matrix.}
{B? mpNum[,]? A complex vector or matrix.}
\end{mpFunctionsExtract}

\begin{mpFunctionsExtract}
\mpFunctionOne
{InvertColPivQR? mpNum[]? $A^{-1}$, the inverse of $A$, based on a QR decomposition with column-pivoting.}
{A? mpNum[,]? A square real matrix.}
\end{mpFunctionsExtract}

\begin{mpFunctionsExtract}
\mpFunctionOne
{cplxInvertColPivQR? mpNum[]? $A^{-1}$, the inverse of $A$, based on a QR decomposition with column-pivoting.}
{A? mpNum[,]? A square complex matrix.}
\end{mpFunctionsExtract}

\begin{mpFunctionsExtract}
\mpFunctionOne
{DetColPivQR? mpNum? $|A|$, the determinant of $A$, based on a QR decomposition with column-pivoting.}
{A? mpNum[,]? A square real matrix.}
\end{mpFunctionsExtract}

\begin{mpFunctionsExtract}
\mpFunctionOne
{cplxDetColPivQR? mpNum? $|A|$, the determinant of $A$, based on a QR decomposition with column-pivoting.}
{A? mpNum[,]? A square complex matrix.}
\end{mpFunctionsExtract}

\section{QR Decomposition with full Pivoting}

\begin{mpFunctionsExtract}
\mpFunctionThree
{DecompFullPivQR? mpNumList? the QR decomposition with full pivoting of $AP = QR$.}
{A? mpNum[,]? the square real matrix of which we are computing the $LU$ decomposition.}
{B? mpNum[,]? A vector or matrix of real numbers.}
{Output? String? A string specifying the output options.}
\end{mpFunctionsExtract}

\begin{mpFunctionsExtract}
\mpFunctionThree
{cplxDecompFullPivQR? mpNumList? the QR decomposition with full pivoting of $AP = QR$.}
{A? mpNum[,]? the square complex matrix of which we are computing the $LU$ decomposition.}
{B? mpNum[,]? A vector or complex of real numbers.}
{Output? String? A string specifying the output options.}
\end{mpFunctionsExtract}

\begin{mpFunctionsExtract}
\mpFunctionTwo
{SolveFullPivQR? mpNum[]? the solution $x$ of $A x = b$ , based on a QR decomposition with full pivoting.}
{A? mpNum[,]? A square real matrix.}
{B? mpNum[,]? A real vector or matrix.}
\end{mpFunctionsExtract}

\begin{mpFunctionsExtract}
\mpFunctionTwo
{cplxSolveFullPivQR? mpNum[]? the solution $x$ of $A x = b$ , based on a QR decomposition with full pivoting.}
{A? mpNum[,]? A square complex matrix.}
{B? mpNum[,]? A complex vector or matrix.}
\end{mpFunctionsExtract}

\begin{mpFunctionsExtract}
\mpFunctionOne
{InvertFullPivQR? mpNum[]? $A^{-1}$, the inverse of $A$, based on a QR decomposition with full pivoting.}
{A? mpNum[,]? A square real matrix.}
\end{mpFunctionsExtract}

\begin{mpFunctionsExtract}
\mpFunctionOne
{cplxInvertFullPivQR? mpNum[]? $A^{-1}$, the inverse of $A$, based on a QR decomposition with full pivoting.}
{A? mpNum[,]? A square complex matrix.}
\end{mpFunctionsExtract}

\begin{mpFunctionsExtract}
\mpFunctionOne
{DetFullPivQR? mpNum? $|A|$, the determinant of $A$, based on a QR decomposition with full pivoting.}
{A? mpNum[,]? A square real matrix.}
\end{mpFunctionsExtract}

\begin{mpFunctionsExtract}
\mpFunctionOne
{cplxDetFullPivQR? mpNum? $|A|$, the determinant of $A$, based on a QR decomposition with full pivoting.}
{A? mpNum[,]? A square complex matrix.}
\end{mpFunctionsExtract}

\section{Singular Value Decomposition}

\begin{mpFunctionsExtract}
\mpFunctionFour
{DecompJacobiSVD? mpNumList? the Cholesky decomposition $A = LL^* = U^*U$ of a matrix.}
{A? mpNum[,]? the real matrix of which we are computing the $LL^T$ Cholesky decomposition.}
{B? mpNum[,]? A vector or matrix of real numbers.}
{computationOptions? Integer? An optional parameter allowing to specify if you want full or thin U or V unitaries to be computed.}
{Output? String? A string specifying the output options.}
\end{mpFunctionsExtract}

\begin{mpFunctionsExtract}
\mpFunctionFour
{cplxDecompJacobiSVD? mpNumList? the Cholesky decomposition $A = LL^* = U^*U$ of a matrix.}
{A? mpNum[,]? the complex matrix of which we are computing the $LL^T$ Cholesky decomposition.}
{B? mpNum[,]? A vector or complex of real numbers.}
{computationOptions? Integer? An optional parameter allowing to specify if you want full or thin U or V unitaries to be computed.}
{Output? String? A string specifying the output options.}
\end{mpFunctionsExtract}

\begin{mpFunctionsExtract}
\mpFunctionTwo
{SolveJacobiSVD? mpNum[]? the solution $x$ of $A x = b$ , based on a singular value decomposition.}
{A? mpNum[,]? A symmetric positive definite real matrix.}
{B? mpNum[,]? A real vector or matrix.}
\end{mpFunctionsExtract}

\begin{mpFunctionsExtract}
\mpFunctionTwo
{cplxSolveJacobiSVD? mpNum[]? the solution $x$ of $A x = b$ , based on a singular value decomposition.}
{A? mpNum[,]? A symmetric positive definite complex matrix.}
{B? mpNum[,]? A complex vector or matrix.}
\end{mpFunctionsExtract}

\begin{mpFunctionsExtract}
\mpFunctionOne
{InvertJacobiSVD? mpNum[]? $A^{-1}$, the inverse of $A$, based on a singular value decomposition.}
{A? mpNum[,]? A square real matrix.}
\end{mpFunctionsExtract}

\begin{mpFunctionsExtract}
\mpFunctionOne
{cplxInvertJacobiSVD? mpNum[]? $A^{-1}$, the inverse of $A$, based on a singular value decomposition.}
{A? mpNum[,]? A square complex matrix.}
\end{mpFunctionsExtract}

\begin{mpFunctionsExtract}
\mpFunctionOne
{DetJacobiSVD? mpNum? $|A|$, the determinant of $A$, based on a singular value decomposition.}
{A? mpNum[,]? A square real matrix.}
\end{mpFunctionsExtract}

\begin{mpFunctionsExtract}
\mpFunctionOne
{cplxDetJacobiSVD? mpNum? $|A|$, the determinant of $A$, based on a singular value decomposition.}
{A? mpNum[,]? A square complex matrix.}
\end{mpFunctionsExtract}

\section{Householder Transformations}

\chapter{Eigensystems, (based on Eigen)}

\section{Symmetric/Hermitian Eigensystems}

\begin{mpFunctionsExtract}
\mpFunctionOne
{EigenSymm? mpNum? the eigenvalues of a real symmetric matrix.}
{A? mpNum[,]? the real matrix of which we are computing the eigenvalues.}
\end{mpFunctionsExtract}

\begin{mpFunctionsExtract}
\mpFunctionOne
{EigenSymmv? mpNum? the eigenvalues and eigenvectors of a real symmetric matrix.}
{A? mpNum[,]? the real matrix of which we are computing the eigenvalues.}
\end{mpFunctionsExtract}

\begin{mpFunctionsExtract}
\mpFunctionOne
{MatSymmInverseSqrt? mpNum? the inverse matrix square root of a real symmetric matrix.}
{A? mpNum[,]? the real matrix of which we are computing the eigenvalues.}
\end{mpFunctionsExtract}

\begin{mpFunctionsExtract}
\mpFunctionOne
{MatSymmSqrt? mpNum? the matrix square root of a real symmetric matrix.}
{A? mpNum[,]? the real matrix of which we are computing the eigenvalues.}
\end{mpFunctionsExtract}

\begin{mpFunctionsExtract}
\mpFunctionOne
{cplxEigenHerm? mpNum? the eigenvalues of a complex hermitian matrix.}
{A? mpNum[,]? the complex hermitian  matrix of which we are computing the eigenvalues.}
\end{mpFunctionsExtract}

\begin{mpFunctionsExtract}
\mpFunctionOne
{cplxEigenHermv? mpNum? the eigenvalues and eigenvectors of a complex hermitian matrix.}
{A? mpNum[,]? the complex hermitian  matrix of which we are computing the eigenvalues.}
\end{mpFunctionsExtract}

\section{General (Nonsymmetric) Eigensystems}

\begin{mpFunctionsExtract}
\mpFunctionOne
{EigenNonsymm? mpNum[]? the eigenvalues of a real general (non-symmetric) matrix.}
{A? mpNum[,]? the real general (non-symmetric) matrix of which we are computing the eigenvalues.}
\end{mpFunctionsExtract}

\begin{mpFunctionsExtract}
\mpFunctionOne
{EigenNonsymmv? mpNumList[2]? the eigenvalues and eigenvectors of a real general (non-symmetric) matrix.}
{A? mpNum[,]? the real general (non-symmetric) matrix of which we are computing the eigenvalues.}
\end{mpFunctionsExtract}

\begin{mpFunctionsExtract}
\mpFunctionOne
{PseudoEigenNonsymm? mpNum[]? the pseudoeigenvalues of a real general (non-symmetric) matrix.}
{A? mpNum[,]? the real general (non-symmetric) matrix of which we are computing the pseudoeigenvalues.}
\end{mpFunctionsExtract}

\begin{mpFunctionsExtract}
\mpFunctionOne
{PseudoEigenNonsymmv? mpNumList[2]? the pseudoeigenvalues and pseudoeigenvectors of a real general (non-symmetric) matrix.}
{A? mpNum[,]? the real general (non-symmetric) matrix of which we are computing the pseudoeigenvalues and pseudoeigenvectors.}
\end{mpFunctionsExtract}

\begin{mpFunctionsExtract}
\mpFunctionOne
{cplxEigenNonsymm? mpNum[]? the eigenvalues of a complex general (non-symmetric) matrix.}
{A? mpNum[,]? the complex general (non-symmetric) matrix of which we are computing the eigenvalues.}
\end{mpFunctionsExtract}

\begin{mpFunctionsExtract}
\mpFunctionOne
{cplxEigenNonsymmv? mpNumList[2]? the eigenvalues and eigenvectors of a complex general (non-symmetric) matrix.}
{A? mpNum[,]? the complex general (non-symmetric) matrix of which we are computing the eigenvalues.}
\end{mpFunctionsExtract}

\section{Generalized Eigensystems}

\begin{mpFunctionsExtract}
\mpFunctionTwo
{EigenGensymm? mpNum? the eigenvalues of a real Generalized Symmetric-Definite Eigensystem.}
{A? mpNum[,]? Selfadjoint matrix in matrix pencil. Only the lower triangular part of the matrix is referenced.}
{B? mpNum[,]? Positive-definite matrix in matrix pencil. Only the lower triangular part of the matrix is referenced.}
\end{mpFunctionsExtract}

\begin{mpFunctionsExtract}
\mpFunctionTwo
{EigenGensymmv? mpNum? the eigenvalues and eigenvectors of a real Generalized Symmetric-Definite Eigensystem.}
{A? mpNum[,]? Selfadjoint matrix in matrix pencil. Only the lower triangular part of the matrix is referenced.}
{B? mpNum[,]? Positive-definite matrix in matrix pencil. Only the lower triangular part of the matrix is referenced.}
\end{mpFunctionsExtract}

\begin{mpFunctionsExtract}
\mpFunctionTwo
{cplxEigenGenherm? mpNum? the eigenvalues of a Complex Hermitian Generalized Symmetric-Definite Eigensystem.}
{A? mpNum[,]? Selfadjoint matrix in matrix pencil. Only the lower triangular part of the matrix is referenced.}
{B? mpNum[,]? Positive-definite matrix in matrix pencil. Only the lower triangular part of the matrix is referenced.}
\end{mpFunctionsExtract}

\begin{mpFunctionsExtract}
\mpFunctionTwo
{cplxEigenGenhermv? mpNum? the eigenvalues and eigenvectors of a Complex Hermitian Generalized Symmetric-Definite Eigensystem.}
{A? mpNum[,]? Selfadjoint matrix in matrix pencil. Only the lower triangular part of the matrix is referenced.}
{B? mpNum[,]? Positive-definite matrix in matrix pencil. Only the lower triangular part of the matrix is referenced.}
\end{mpFunctionsExtract}

\begin{mpFunctionsExtract}
\mpFunctionTwo
{EigenGenNonsymm? mpNum? the eigenvalues of a real Generalized Non-Symmetric Eigensystem.}
{A? mpNum[,]? Selfadjoint matrix in matrix pencil. Only the lower triangular part of the matrix is referenced.}
{B? mpNum[,]? Positive-definite matrix in matrix pencil. Only the lower triangular part of the matrix is referenced.}
\end{mpFunctionsExtract}

\begin{mpFunctionsExtract}
\mpFunctionTwo
{EigenGenNonsymmv? mpNum? the eigenvalues and eigenvectors of a real Generalized Non-Symmetric Eigensystem.}
{A? mpNum[,]? Selfadjoint matrix in matrix pencil. Only the lower triangular part of the matrix is referenced.}
{B? mpNum[,]? Positive-definite matrix in matrix pencil. Only the lower triangular part of the matrix is referenced.}
\end{mpFunctionsExtract}

\section{Decompositions}

\section{Matrix Functions}

\begin{mpFunctionsExtract}
\mpFunctionOne
{MatSqrt? mpNum? an expression representing the matrix square root of the real matrix M.}
{M? mpNum[,]? the real matrix of which we are computing the matrix square root.}
\end{mpFunctionsExtract}

\begin{mpFunctionsExtract}
\mpFunctionOne
{cplxMatSqrt? mpNum? an expression representing the matrix square root of the complex matrix M.}
{M? mpNum[,]? the complex matrix of which we are computing the matrix square root.}
\end{mpFunctionsExtract}

\begin{mpFunctionsExtract}
\mpFunctionOne
{MatExp? mpNum? an expression representing the matrix exponential of the real matrix M.}
{M? mpNum[,]? the real matrix of which we are computing the matrix exponential.}
\end{mpFunctionsExtract}

\begin{mpFunctionsExtract}
\mpFunctionOne
{cplxMatExp? mpNum? an expression representing the matrix exponential of the complex matrix M.}
{M? mpNum[,]? the complex matrix of which we are computing the matrix exponential.}
\end{mpFunctionsExtract}

\begin{mpFunctionsExtract}
\mpFunctionOne
{MatLog? mpNum? an expression representing the matrix logarithm of the real matrix M.}
{M? mpNum[,]? the real matrix of which we are computing the matrix logarithm.}
\end{mpFunctionsExtract}

\begin{mpFunctionsExtract}
\mpFunctionOne
{cplxMatLog? mpNum? an expression representing the matrix logarithm of the complex matrix M.}
{M? mpNum[,]? the complex matrix of which we are computing the matrix logarithm.}
\end{mpFunctionsExtract}

\begin{mpFunctionsExtract}
\mpFunctionTwo
{MatPow? mpNum? an expression representing the matrix power of the real matrix M.}
{M? mpNum[,]? M base of the matrix power, should be a square matrix.}
{p? mpNum? exponent of the matrix power, should be real.}
\end{mpFunctionsExtract}

\begin{mpFunctionsExtract}
\mpFunctionTwo
{cplxMatPow? mpNum? an expression representing the matrix power of the complex matrix M.}
{M? mpNum[,]? M base of the matrix power, should be a square matrix.}
{p? mpNum? exponent of the matrix power, should be real.}
\end{mpFunctionsExtract}

\begin{mpFunctionsExtract}
\mpFunctionTwo
{MatGeneralFunction? mpNum? an expression representing f applied to the real matrix M.}
{M? mpNum[,]? argument of matrix function, should be a square matrix.}
{f? mpFunction? f an entire function; f(x,n) should compute the n-th derivative of f at x.}
\end{mpFunctionsExtract}

\begin{mpFunctionsExtract}
\mpFunctionTwo
{cplxMatGeneralFunction? mpNum? an expression representing f applied to the complex matrix M.}
{M? mpNum[,]? argument of matrix function, should be a square matrix.}
{f? mpFunction? f an entire function; f(x,n) should compute the n-th derivative of f at x.}
\end{mpFunctionsExtract}

\begin{mpFunctionsExtract}
\mpFunctionOne
{MatSin? mpNum? an expression representing the matrix sine of the real matrix M.}
{M? mpNum[,]? the real matrix of which we are computing the matrix sine.}
\end{mpFunctionsExtract}

\begin{mpFunctionsExtract}
\mpFunctionOne
{cplxMatSin? mpNum? an expression representing the matrix sine of the complex matrix M.}
{M? mpNum[,]? the complex matrix of which we are computing the matrix sine.}
\end{mpFunctionsExtract}

\begin{mpFunctionsExtract}
\mpFunctionOne
{MatCos? mpNum? an expression representing the matrix cosine of the real matrix M.}
{M? mpNum[,]? the real matrix of which we are computing the matrix cosine.}
\end{mpFunctionsExtract}

\begin{mpFunctionsExtract}
\mpFunctionOne
{cplxMatCos? mpNum? an expression representing the matrix cosine of the complex matrix M.}
{M? mpNum[,]? the complex matrix of which we are computing the matrix cosine.}
\end{mpFunctionsExtract}

\begin{mpFunctionsExtract}
\mpFunctionOne
{MatSinh? mpNum? an expression representing the matrix hyperbolic sine of the real matrix M.}
{M? mpNum[,]? the real matrix of which we are computing the matrix hyperbolic sine.}
\end{mpFunctionsExtract}

\begin{mpFunctionsExtract}
\mpFunctionOne
{cplxMatSinh? mpNum? an expression representing the matrix hyperbolic sine of the complex matrix M.}
{M? mpNum[,]? the complex matrix of which we are computing the matrix hyperbolic sine.}
\end{mpFunctionsExtract}

\begin{mpFunctionsExtract}
\mpFunctionOne
{MatCosh? mpNum? an expression representing the matrix hyperbolic cosine of the real matrix M.}
{M? mpNum[,]? the real matrix of which we are computing the matrix hyperbolic cosine.}
\end{mpFunctionsExtract}

\begin{mpFunctionsExtract}
\mpFunctionOne
{cplxMatCosh? mpNum? an expression representing the matrix hyperbolic cosine of the complex matrix M.}
{M? mpNum[,]? the complex matrix of which we are computing the matrix hyperbolic cosine.}
\end{mpFunctionsExtract}

\chapter{Polynomials (based on Eigen)}

\section{Polynomial Evaluation}

\begin{mpFunctionsExtract}
\mpFunctionTwo
{PolynomialEvaluation? mpNum? the value of a polynomial for the real variable $x$ with real coefficients $c$.}
{x? mpNum? A real number.}
{c? mpNum[]? A vector of real coefficients.}
\end{mpFunctionsExtract}

\begin{mpFunctionsExtract}
\mpFunctionTwo
{cplxPolynomialEvaluation? mpNum? the value of a polynomial for the complex variable $z$ with complex coefficients $c$.}
{z? mpNum? A complex number.}
{c? mpNum[]? A vector of complex coefficients.}
\end{mpFunctionsExtract}

\section{Quadratic Equations}

\begin{mpFunctionsExtract}
\mpFunctionThree
{QuadraticEquation? mpNum[]? a real vector containing the real roots of the quadratic equation.}
{a? mpNum? A real number.}
{b? mpNum? A real number.}
{c? mpNum? A real number.}
\end{mpFunctionsExtract}

\begin{mpFunctionsExtract}
\mpFunctionThree
{cplxQuadraticEquation? mpNum[]? a complex vector containing the complex roots of the quadratic equation.}
{a? mpNum? A real or complex number.}
{b? mpNum? A real or complex number.}
{c? mpNum? A real or complex number.}
\end{mpFunctionsExtract}

\section{Cubic Equations}

\begin{mpFunctionsExtract}
\mpFunctionFour
{CubicEquation? mpNum[]? a real vector containing the real roots of the cubic equation.}
{a? mpNum? A real number.}
{b? mpNum? A real number.}
{c? mpNum? A real number.}
{d? mpNum? A real number.}
\end{mpFunctionsExtract}

\begin{mpFunctionsExtract}
\mpFunctionFour
{cplxCubicEquation? mpNum[]? a complex vector containing the complex roots of the cubic equation.}
{a? mpNum? A real or complex number.}
{b? mpNum? A real or complex number.}
{c? mpNum? A real or complex number.}
{d? mpNum? A real or complex number.}
\end{mpFunctionsExtract}

\section{Quartic Equations}

\begin{mpFunctionsExtract}
\mpFunctionFive
{QuarticEquation? mpNum[]? a real vector containing the real roots of the quartic equation.}
{a? mpNum? A real number.}
{b? mpNum? A real number.}
{c? mpNum? A real number.}
{d? mpNum? A real number.}
{e? mpNum? A real number.}
\end{mpFunctionsExtract}

\begin{mpFunctionsExtract}
\mpFunctionFive
{cplxQuarticEquation? mpNum[]? a complex vector containing the complex roots of the quartic equation.}
{a? mpNum? A real or complex number.}
{b? mpNum? A real or complex number.}
{c? mpNum? A real or complex number.}
{d? mpNum? A real or complex number.}
{e? mpNum? A real or complex number.}
\end{mpFunctionsExtract}

\section{General Polynomial Equations}

\begin{mpFunctionsExtract}
\mpFunctionOne
{GeneralPolynomialEquation? mpNum[]? a real vector containing the real roots of the general real polynomial.}
{a? mpNum[]? The real coefficients of the polynomial.}
\end{mpFunctionsExtract}

\begin{mpFunctionsExtract}
\mpFunctionOne
{cplxGeneralPolynomialEquation? mpNum[]? a complex vector containing the complex roots of the general complex polynomial.}
{c? mpNum[]? The complex coefficients of the polynomial.}
\end{mpFunctionsExtract}

\chapter{Fast Fourier Transform (based on Eigen)}

\section{Discrete Fourier Transforms}

\begin{mpFunctionsExtract}
\mpFunctionOne
{FFTW\_FORWARD? mpNum[]? a complex vector containing the forward complex discrete Fourier transform of $X$.}
{X? mpNum[]? A complex vector.}
\end{mpFunctionsExtract}

\begin{mpFunctionsExtract}
\mpFunctionOne
{FFTW\_BACKWARD? mpNum[]? a complex vector containing the backward complex discrete Fourier transform of $X$.}
{X? mpNum[]? A complex vector.}
\end{mpFunctionsExtract}

\begin{mpFunctionsExtract}
\mpFunctionOne
{FFTW\_R2C? mpNum[]? a complex vector containing the forward complex discrete Fourier transform of $X$.}
{X? mpNum[]? A real vector.}
\end{mpFunctionsExtract}

\begin{mpFunctionsExtract}
\mpFunctionOne
{FFTW\_C2R? mpNum[]? a real vector containing the backward  discrete Fourier transform of $X$.}
{X? mpNum[]? A complex hermitian vector.}
\end{mpFunctionsExtract}

\begin{mpFunctionsExtract}
\mpFunctionOne
{FFTW\_REDFT00? mpNum[]? a real vector containing the REDFT00 transform (type-I DCT) transform of $X$.}
{X? mpNum[]? A real vector.}
\end{mpFunctionsExtract}

\begin{mpFunctionsExtract}
\mpFunctionOne
{FFTW\_REDFT10? mpNum[]? a real vector containing the REDFT10 transform (type-II DCT) transform of $X$.}
{X? mpNum[]? A real vector.}
\end{mpFunctionsExtract}

\begin{mpFunctionsExtract}
\mpFunctionOne
{FFTW\_REDFT01? mpNum[]? a real vector containing the REDFT01 transform (type-III DCT) transform of $X$.}
{X? mpNum[]? A real vector.}
\end{mpFunctionsExtract}

\begin{mpFunctionsExtract}
\mpFunctionOne
{FFTW\_REDFT11? mpNum[]? a real vector containing the REDFT11 transform (type-IV DCT) transform of $X$.}
{X? mpNum[]? A real vector.}
\end{mpFunctionsExtract}

\begin{mpFunctionsExtract}
\mpFunctionOne
{FFTW\_RODFT00? mpNum[]? a real vector containing the RODFT00 transform (type-I DST) transform of $X$.}
{X? mpNum[]? A real vector.}
\end{mpFunctionsExtract}

\begin{mpFunctionsExtract}
\mpFunctionOne
{FFTW\_RODFT10? mpNum[]? a real vector containing the RODFT10 transform (type-II DST) transform of $X$.}
{X? mpNum[]? A real vector.}
\end{mpFunctionsExtract}

\begin{mpFunctionsExtract}
\mpFunctionOne
{FFTW\_RODFT01? mpNum[]? a real vector containing the RODFT01 transform (type-III DST) transform of $X$.}
{X? mpNum[]? A real vector.}
\end{mpFunctionsExtract}

\begin{mpFunctionsExtract}
\mpFunctionOne
{FFTW\_RODFT11? mpNum[]? a real vector containing the RODFT11 transform (type-IV DST) transform of $X$.}
{X? mpNum[]? A real vector.}
\end{mpFunctionsExtract}

\chapter{Minimization and Optimization: Procedures based on MINPACK}

\section{Multidimensional Rootfinding: Powell Hybrid}

\section{Nonlinear LeastSquares: Levenberg-Marquardt }

\chapter{Procedures based on NLOPT}

\section{Overview}

\section{Global optimization}

\section{Local derivative-free optimization}

\section{Local gradient-based optimization}

\section{NLOPT: Augmented Lagrangian algorithm}

\chapter{RandomNumbers}

\section{Definitions}

\section{The Random Number Generator Interface}

\section{Random number generator algorithms}

\begin{mpFunctionsExtract}
\mpFunctionOne
{SaveDefaultRngState? Boolean? a boolean value: TRUE if the state was successfully save, FALSE otherwise}
{FName? String? A String, containing the full path of the file.}
\end{mpFunctionsExtract}

\begin{mpFunctionsExtract}
\mpFunctionOne
{LoadDefaultRngState? Boolean? a boolean value: TRUE if the state was successfully loaded, FALSE otherwise}
{FName? String? A String, containing the full path of the file.}
\end{mpFunctionsExtract}

\section{Random number distributions}

\chapter{Special Functions (based on Boost)}

\section{Gamma and Beta Functions}

\begin{mpFunctionsExtract}
\mpFunctionOne
{TgammaBoost? mpNum? the gamma function for $x \neq 0, -1, -2,\ldots$.}
{x? mpNum? A real number.}
\end{mpFunctionsExtract}

\begin{mpFunctionsExtract}
\mpFunctionOne
{LgammaBoost? mpNum? the logarithm of the gamma function.}
{x? mpNum? A real number.}
\end{mpFunctionsExtract}

\begin{mpFunctionsExtract}
\mpFunctionTwo
{TgammaDeltaRatioBoost? mpNum?  the ratio of gamma functions.}
{x? mpNum? A real number.}
{$\delta$? mpNum? A real number.}
\end{mpFunctionsExtract}

\begin{mpFunctionsExtract}
\mpFunctionOne
{DigammaBoost? mpNum? the digamma function for $x \neq 0, -1, -2,\ldots$.}
{x? mpNum? A real number.}
\end{mpFunctionsExtract}

\begin{mpFunctionsExtract}
\mpFunctionTwo
{TgammaratioBoost? mpNum?  the ratio of gamma functions.}
{a? mpNum? A real number.}
{b? mpNum? A real number.}
\end{mpFunctionsExtract}

\begin{mpFunctionsExtract}
\mpFunctionTwo
{GammaPBoost? mpNum? the normalised incomplete gamma function $P(a,x)$.}
{a? mpNum? A real number.}
{x? mpNum? A real number.}
\end{mpFunctionsExtract}

\begin{mpFunctionsExtract}
\mpFunctionTwo
{GammaQBoost? mpNum? the normalised incomplete gamma function $Q(a,x)$.}
{a? mpNum? A real number.}
{x? mpNum? A real number.}
\end{mpFunctionsExtract}

\begin{mpFunctionsExtract}
\mpFunctionTwo
{NonNormalisedGammaPBoost? mpNum? the non-normalised incomplete gamma function $\Gamma(a,x)$.}
{a? mpNum? A real number.}
{x? mpNum? A real number.}
\end{mpFunctionsExtract}

\begin{mpFunctionsExtract}
\mpFunctionTwo
{NonNormalisedGammaQBoost? mpNum? the non-normalised incomplete gamma function $\gamma(a,x)$.}
{a? mpNum? A real number.}
{x? mpNum? A real number.}
\end{mpFunctionsExtract}

\begin{mpFunctionsExtract}
\mpFunctionTwo
{GammaPinvBoost? mpNum? the inverse of the normalised incomplete gamma function $P(a,x)$.}
{a? mpNum? A real number.}
{p? mpNum? A real number.}
\end{mpFunctionsExtract}

\begin{mpFunctionsExtract}
\mpFunctionTwo
{GammaQinvBoost? mpNum? the inverse of the normalised incomplete gamma function $Q(a,x)$.}
{a? mpNum? A real number.}
{q? mpNum? A real number.}
\end{mpFunctionsExtract}

\begin{mpFunctionsExtract}
\mpFunctionTwo
{GammaPinvaBoost? mpNum? the parameter $a$ of the normalised incomplete gamma function $P(a,x)$, such that $P(a,x) = p$.}
{x? mpNum? A real number.}
{p? mpNum? A real number.}
\end{mpFunctionsExtract}

\begin{mpFunctionsExtract}
\mpFunctionTwo
{GammaQinvaBoost? mpNum? the parameter $a$ of the normalised incomplete gamma function $Q(a,x)$, such that $Q(a,x) = q$.}
{x? mpNum? A real number.}
{q? mpNum? A real number.}
\end{mpFunctionsExtract}

\begin{mpFunctionsExtract}
\mpFunctionTwo
{GammaPDerivativeBoost? mpNum? the partial derivative with respect to $x$ of the incomplete gamma function $P(a,x)$.}
{a? mpNum? A real number.}
{x? mpNum? A real number.}
\end{mpFunctionsExtract}

\section{Factorials and Binomial Coefficient}

\begin{mpFunctionsExtract}
\mpFunctionOne
{FactorialBoost? mpNum? the factorial $n! = \Gamma(n+1) = n \times (n-1) \times \cdots \times 1$.}
{n? mpNum? An integer.}
\end{mpFunctionsExtract}

\begin{mpFunctionsExtract}
\mpFunctionOne
{DoubleFactorialBoost? mpNum? the double factorial $n!!$.}
{n? mpNum? An integer.}
\end{mpFunctionsExtract}

\begin{mpFunctionsExtract}
\mpFunctionTwo
{RisingFactorialBoost? mpNum? the rising factorial of $x$ and $i$.}
{n? mpNum? An integer.}
{i? mpNum? An integer.}
\end{mpFunctionsExtract}

\begin{mpFunctionsExtract}
\mpFunctionTwo
{FallingFactorialBoost? mpNum? the falling factorial of $x$ and $i$.}
{n? mpNum? An integer.}
{i? mpNum? An integer.}
\end{mpFunctionsExtract}

\begin{mpFunctionsExtract}
\mpFunctionTwo
{BinomialCoefficientBoost? mpNum? the binomial coefficient.}
{n? mpNum? An integer.}
{k? mpNum? An integer.}
\end{mpFunctionsExtract}

\section{Beta Functions}

\begin{mpFunctionsExtract}
\mpFunctionTwo
{BetaBoost? mpNum? the beta function.}
{a? mpNum? A real number.}
{b? mpNum? A real number.}
\end{mpFunctionsExtract}

\section{Error Function and Related Functions}

\begin{mpFunctionsExtract}
\mpFunctionOne
{ErfBoost? mpNum? the value of the error function.}
{x? mpNum? A real number.}
\end{mpFunctionsExtract}

\begin{mpFunctionsExtract}
\mpFunctionOne
{ErfcBoost? mpNum? the value of the complementary error function.}
{x? mpNum? A real number.}
\end{mpFunctionsExtract}

\begin{mpFunctionsExtract}
\mpFunctionOne
{ErfInvBoost? mpNum? the functional inverse of $\text{erf}(x)$}
{x? mpNum? A real number.}
\end{mpFunctionsExtract}

\begin{mpFunctionsExtract}
\mpFunctionOne
{ErfcInvBoost? mpNum? the functional inverse of $\text{erfc}(x)$}
{x? mpNum? A real number.}
\end{mpFunctionsExtract}

\section{Polynomials}

\begin{mpFunctionsExtract}
\mpFunctionTwo
{LegendrePBoost? mpNum? $P_l(x)$, the Legendre functions of the first kind.}
{l? mpNum? An Integer.}
{x? mpNum? A real number.}
\end{mpFunctionsExtract}

\begin{mpFunctionsExtract}
\mpFunctionFour
{LegendrePNextBoost? mpNum? the Legendre function of the first kind of degree $l+1$, using the results for degree $l$ and $l-1$.}
{l? mpNum? An Integer. The degree of the last polynomial calculated.}
{x? mpNum? A real number. The abscissa value.}
{Pl? mpNum? A real number. The value of the polynomial evaluated at degree $l$.}
{Plm1? mpNum? A real number. The value of the polynomial evaluated at degree $l-1$.}
\end{mpFunctionsExtract}

\begin{mpFunctionsExtract}
\mpFunctionThree
{AssociatedLegendrePlmBoost? mpNum? $L^m_n (x)$, the associated Legendre polynomials of degree $l \geq 0$ and order $m \geq 0$.}
{l? mpNum? An Integer.}
{m? mpNum? An Integer.}
{x? mpNum? A real number.}
\end{mpFunctionsExtract}

\begin{mpFunctionsExtract}
\mpFunctionFive
{AssociatedLegendrePlmNextBoost? mpNum? the Legendre function of the first kind of degree $l+1$, using the results for degree $l$ and $l-1$.}
{l? mpNum? An Integer. The degree of the last polynomial calculated.}
{m? mpNum? An Integer. The order of the Associated Polynomial.}
{x? mpNum? A real number. The abscissa value.}
{Pl? mpNum? A real number. The value of the polynomial evaluated at degree $l$.}
{Plm1? mpNum? A real number. The value of the polynomial evaluated at degree $l-1$.}
\end{mpFunctionsExtract}

\begin{mpFunctionsExtract}
\mpFunctionTwo
{LegendreQBoost? mpNum? $Q_l(x)$, the Legendre functions of the second kind of degree $l \geq 0$ and $|x| \neq 1$.}
{l? mpNum? An Integer.}
{x? mpNum? A real number.}
\end{mpFunctionsExtract}

\begin{mpFunctionsExtract}
\mpFunctionTwo
{LaguerreLBoost? mpNum? $L_n (x)$, the Laguerre polynomials of degree $n \geq 0$.}
{n? mpNum? An Integer.}
{x? mpNum? A real number.}
\end{mpFunctionsExtract}

\begin{mpFunctionsExtract}
\mpFunctionFour
{LaguerreLNextBoost? mpNum? the Laguerre polynomial of the first kind of degree $n+1$, using the results for degree $n$ and $n-1$.}
{n? mpNum? An Integer. The degree of the last polynomial calculated.}
{x? mpNum? A real number. The abscissa value.}
{Ln? mpNum? A real number. The value of the polynomial evaluated at degree $n$.}
{Lnm1? mpNum? A real number. The value of the polynomial evaluated at degree $n-1$.}
\end{mpFunctionsExtract}

\begin{mpFunctionsExtract}
\mpFunctionThree
{AssociatedLaguerreBoost? mpNum? $L^m_n (x)$, the associated Laguerre polynomials of degree $n \geq 0$ and order $m \geq 0$.}
{n? mpNum? An Integer.}
{m? mpNum? An Integer.}
{x? mpNum? A real number.}
\end{mpFunctionsExtract}

\begin{mpFunctionsExtract}
\mpFunctionFive
{AssociatedLaguerreLNextBoost? mpNum? the associated Laguerre polynomial of the first kind of degree $n+1$, using the results for degree $n$ and $n-1$.}
{n? mpNum? An Integer. The degree of the last polynomial calculated.}
{m? mpNum? An Integer. The order of the Associated Polynomial.}
{x? mpNum? A real number. The abscissa value.}
{Ln? mpNum? A real number. The value of the polynomial evaluated at degree $n$.}
{Lnm1? mpNum? A real number. The value of the polynomial evaluated at degree $n-1$.}
\end{mpFunctionsExtract}

\begin{mpFunctionsExtract}
\mpFunctionTwo
{HermiteHBoost? mpNum? $H_n(x)$, the Hermite polynomial of degree $n \geq 0$.}
{n? mpNum? An Integer.}
{x? mpNum? A real number.}
\end{mpFunctionsExtract}

\begin{mpFunctionsExtract}
\mpFunctionFour
{HermiteHNextBoost? mpNum? the Hermite polynomial of degree $n+1$, using the results for degree $n$ and $n-1$.}
{n? mpNum? An Integer. The degree of the last polynomial calculated.}
{x? mpNum? A real number. The abscissa value.}
{Hn? mpNum? A real number. The value of the polynomial evaluated at degree $n$.}
{Hnm1? mpNum? A real number. The value of the polynomial evaluated at degree $n-1$.}
\end{mpFunctionsExtract}

\begin{mpFunctionsExtract}
\mpFunctionFour
{SphericalHarmonicBoost? mpNum? the real and imaginary parts of the spherical harmonic function $Y_{lm}(\theta, \phi)$.}
{l? mpNumList[2]? An Integer.}
{m? mpNum? An Integer.}
{$\theta$? mpNum? A real number.}
{$\phi$? mpNum? A real number.}
\end{mpFunctionsExtract}

\section{Bessel Functions of Real Order}

\begin{mpFunctionsExtract}
\mpFunctionTwo
{BesselJBoost? mpNum? $J_{\nu}(z)$, the Bessel function of the first kind of real order $\nu$.}
{x? mpNum? A real number.}
{$\nu$? mpNum? A real number.}
\end{mpFunctionsExtract}

\begin{mpFunctionsExtract}
\mpFunctionTwo
{BesselYBoost? mpNum? $Y_{\nu}(z)$, the Bessel function of the second kind of order $\nu$.}
{x? mpNum? A real number.}
{$\nu$? mpNum? A real number.}
\end{mpFunctionsExtract}

\section{Modified Bessel Functions of Real Order}

\begin{mpFunctionsExtract}
\mpFunctionTwo
{BesselIBoost? mpNum? the modified Bessel function $I_{\nu}(z)$ of the first kind of order $\nu$.}
{x? mpNum? A real number.}
{$\nu$? mpNum? A real number.}
\end{mpFunctionsExtract}

\begin{mpFunctionsExtract}
\mpFunctionTwo
{BesselKBoost? mpNum? $K_{\nu}(x)$, the modified Bessel function of the second kind of order $\nu$.}
{x? mpNum? A real number.}
{$\nu$? mpNum? A real number.}
\end{mpFunctionsExtract}

\section{Spherical Bessel Functions}

\begin{mpFunctionsExtract}
\mpFunctionTwo
{BesselSphericaljBoost? mpNum? $j_n(x)$, the spherical Bessel function of the 1st kind, order $n$.}
{x? mpNum? A real number.}
{$\nu$? mpNum? A real number.}
\end{mpFunctionsExtract}

\begin{mpFunctionsExtract}
\mpFunctionTwo
{BesselSphericalyBoost? mpNum? $j_n(x)$, the spherical Bessel function of the 1st kind, order $n$.}
{x? mpNum? A real number.}
{$\nu$? mpNum? A real number.}
\end{mpFunctionsExtract}

\section{Hankel Functions}

\begin{mpFunctionsExtract}
\mpFunctionTwo
{cplxHankel1Boost? mpNum? the Hankel function of the first kind $H_{\nu}^{(1)}(x)$.}
{x? mpNum? A real number.}
{$\nu$? mpNum? A real number.}
\end{mpFunctionsExtract}

\begin{mpFunctionsExtract}
\mpFunctionTwo
{cplxHankel2Boost? mpNum? the Hankel function of the second kind $H_{\nu}^{(2)}(x)$.}
{x? mpNum? A real number.}
{$\nu$? mpNum? A real number.}
\end{mpFunctionsExtract}

\begin{mpFunctionsExtract}
\mpFunctionTwo
{cplxHankelSph1Boost? mpNum? the spherical Hankel function of the first kind $h_{\nu}^{(1)}(x)$.}
{x? mpNum? A real number.}
{$\nu$? mpNum? A real number.}
\end{mpFunctionsExtract}

\begin{mpFunctionsExtract}
\mpFunctionTwo
{cplxHankelSph2Boost? mpNum? the spherical Hankel function of the second kind $h_{\nu}^{(2)}(x)$.}
{x? mpNum? A real number.}
{$\nu$? mpNum? A real number.}
\end{mpFunctionsExtract}

\section{Airy Functions}

\begin{mpFunctionsExtract}
\mpFunctionOne
{AiryAiBoost? mpNum? the Airy function $\text{Ai}(x)$.}
{x? mpNum? A real number.}
\end{mpFunctionsExtract}

\begin{mpFunctionsExtract}
\mpFunctionOne
{AiryAiDerivativeBoost? mpNum? the Airy function $\text{Ai}'(x)$.}
{x? mpNum? A real number.}
\end{mpFunctionsExtract}

\begin{mpFunctionsExtract}
\mpFunctionOne
{AiryBiBoost? mpNum? the Airy function $\text{Bi}(x)$.}
{x? mpNum? A real number.}
\end{mpFunctionsExtract}

\begin{mpFunctionsExtract}
\mpFunctionOne
{AiryBiDerivativeBoost? mpNum? the Airy function $\text{Bi}'(x)$.}
{x? mpNum? A real number.}
\end{mpFunctionsExtract}

\section{Carlson-style Elliptic Integrals}

\begin{mpFunctionsExtract}
\mpFunctionTwo
{CarlsonRCBoost? mpNum? the value of the of Carlson's degenerate elliptic integral $R_C$.}
{x? mpNum? A real number.}
{y? mpNum? A real number.}
\end{mpFunctionsExtract}

\begin{mpFunctionsExtract}
\mpFunctionThree
{CarlsonRFBoost? mpNum? the value of the of Carlson's elliptic integral $R_F$ of the first kind.}
{x? mpNum? A real number.}
{y? mpNum? A real number.}
{z? mpNum? A real number.}
\end{mpFunctionsExtract}

\begin{mpFunctionsExtract}
\mpFunctionThree
{CarlsonRDBoost? mpNum? the value of the of Carlson's elliptic integral $R_D$ of the second kind.}
{x? mpNum? A real number.}
{y? mpNum? A real number.}
{z? mpNum? A real number.}
\end{mpFunctionsExtract}

\begin{mpFunctionsExtract}
\mpFunctionFour
{CarlsonRJBoost? mpNum? the value of the of Carlson's elliptic integral $R_J$ of the third kind.}
{x? mpNum? A real number.}
{y? mpNum? A real number.}
{z? mpNum? A real number.}
{r? mpNum? A real number.}
\end{mpFunctionsExtract}

\section{Legendre-style Elliptic Integrals}

\begin{mpFunctionsExtract}

\mpFunctionOne
{CompleteLegendreEllint1Boost? mpNum? the value of the complete elliptic integral of the first kind.}
{k? mpNum? A real number.}
\end{mpFunctionsExtract}

\begin{mpFunctionsExtract}
\mpFunctionOne
{CompleteLegendreEllint2Boost? mpNum? the value of the complete elliptic integral of the second kind.}
{k? mpNum? A real number.}
\end{mpFunctionsExtract}

\begin{mpFunctionsExtract}
\mpFunctionTwo
{CompleteLegendreEllint3Boost? the value of the complete elliptic integral of the third kind.}
{$\nu$? mpNum? A real number.}
{k? mpNum? A real number.}
\end{mpFunctionsExtract}

\begin{mpFunctionsExtract}
\mpFunctionTwo
{LegendreEllint1Boost? mpNum? the value of the incomplete Legendre elliptic integral of the first kind.}
{$\phi$? mpNum? A real number.}
{k? mpNum? A real number.}
\end{mpFunctionsExtract}

\begin{mpFunctionsExtract}
\mpFunctionTwo
{LegendreEllint2Boost? mpNum? the value of the incomplete Legendre elliptic integral of the second kind.}
{$\phi$? mpNum? A real number.}
{k? mpNum? A real number.}
\end{mpFunctionsExtract}

\begin{mpFunctionsExtract}
\mpFunctionThree
{LegendreEllint3Boost? mpNum? the value of the incomplete Legendre elliptic integral of the third kind.}
{$\phi$? mpNum? A real number.}
{$\nu$? mpNum? A real number.}
{k? mpNum? A real number.}
\end{mpFunctionsExtract}

\section{Jacobi Elliptic Functions}

\begin{mpFunctionsExtract}
\mpFunctionTwo
{JacobiSNBoost? mpNum? the Jacobi elliptic function $\text{sn}(x, k)$.}
{x? mpNum? A real number.}
{k? mpNum? A real number.}
\end{mpFunctionsExtract}

\begin{mpFunctionsExtract}
\mpFunctionTwo
{JacobiCNBoost? mpNum? the Jacobi elliptic function $\text{cn}(x, k)$.}
{x? mpNum? A real number.}
{k? mpNum? A real number.}
\end{mpFunctionsExtract}

\begin{mpFunctionsExtract}
\mpFunctionTwo
{JacobiDNBoost? mpNum? the Jacobi elliptic function $\text{dn}(x, k)$.}
{x? mpNum? A real number.}
{k? mpNum? A real number.}
\end{mpFunctionsExtract}

\section{Zeta Functions}

\begin{mpFunctionsExtract}
\mpFunctionOne
{RiemannZetaBoost? mpNum? the Riemann zeta function.}
{s? mpNum? A real number.}
\end{mpFunctionsExtract}

\section{Exponential Integral and Related Integrals}

\begin{mpFunctionsExtract}
\mpFunctionOne
{ExponentialIntegralE1Boost? mpNum? the exponential integral $\text{E}_1(x)$.}
{x? mpNum? A real number.}
\end{mpFunctionsExtract}

\begin{mpFunctionsExtract}
\mpFunctionOne
{ExponentialIntegralEiBoost? mpNum? the exponential integral $\text{Ei}(x)$.}
{x? mpNum? A real number.}
\end{mpFunctionsExtract}

\begin{mpFunctionsExtract}
\mpFunctionTwo
{ExponentialIntegralEnBoost? mpNum? the exponential integral  $\text{E}_n(x)$.}
{x? mpNum? A real number.}
{n? mpNum? A real number.}
\end{mpFunctionsExtract}

\section{Basic Functions}

\begin{mpFunctionsExtract}
\mpFunctionOne
{SinPiBoost? mpNum? the value of the sine of $\pi x$, with $x$ in radians.}
{x? mpNum? A real number.}
\end{mpFunctionsExtract}

\begin{mpFunctionsExtract}
\mpFunctionOne
{CosPiBoost? mpNum? the value of the cosine of $\pi x$, with $x$ in radians.}
{x? mpNum? A real number.}
\end{mpFunctionsExtract}

\begin{mpFunctionsExtract}
\mpFunctionOne
{Lnp1Boost? mpNum? the value of the function $\ln(1+x)$.}
{x? mpNum? A real number.}
\end{mpFunctionsExtract}

\begin{mpFunctionsExtract}
\mpFunctionOne
{Expm1Boost? mpNum? the value of the function $\text{expm1}(x) = e^{x}-1$.}
{x? mpNum? A real number.}
\end{mpFunctionsExtract}

\begin{mpFunctionsExtract}
\mpFunctionOne
{CbrtBoost? mpNum? the absolute value of the cube root of $x, \sqrt[3]{x}$.}
{x? mpNum? A real number.}
\end{mpFunctionsExtract}

\begin{mpFunctionsExtract}
\mpFunctionOne
{Sqrtp1m1Boost? mpNum? the value of $\sqrt{x+1}-1$.}
{x? mpNum? A real number.}
\end{mpFunctionsExtract}

\begin{mpFunctionsExtract}
\mpFunctionTwo
{Powm1Boost? mpNum? the value of $x^y-1, y \in  \mathbb{R}$.}
{x? mpNum? A real number.}
{y? mpNum? A real number.}
\end{mpFunctionsExtract}

\begin{mpFunctionsExtract}
\mpFunctionTwo
{HypotBoost? mpNum? the value of $\sqrt{x^2+y^2}$.}
{x? mpNum? A real number.}
{y? mpNum? A real number.}
\end{mpFunctionsExtract}

\section{Sinus Cardinal Function and Hyperbolic Sinus Cardinal Functions}

\begin{mpFunctionsExtract}
\mpFunctionOne
{SincaBoost? mpNum? the sinus cardinal function}
{x? mpNum? A real number.}
\end{mpFunctionsExtract}

\begin{mpFunctionsExtract}
\mpFunctionOne
{SinhcaBoost? mpNum? the hyperbolic sinus cardinal function.}
{x? mpNum? A real number.}
\end{mpFunctionsExtract}

\section{Inverse Hyperbolic Functions}

\begin{mpFunctionsExtract}
\mpFunctionOne
{AcoshBoost? mpNum? the value of the hyperbolic arc-cosine  of $x$ in radians.}
{x? mpNum? A real number.}
\end{mpFunctionsExtract}

\begin{mpFunctionsExtract}
\mpFunctionOne
{AsinhBoost? mpNum? the value of the hyperbolic arc-sine  of $x$ in radians.}
{x? mpNum? A real number.}
\end{mpFunctionsExtract}

\begin{mpFunctionsExtract}
\mpFunctionOne
{AtanhBoost? mpNum? the value of the hyperbolic arc-tangent  of $x$ in radians.}
{x? mpNum? A real number.}
\end{mpFunctionsExtract}

\chapter{Distribution Functions}

\section{Introduction to Distribution Functions}

\section{Beta-Distribution}

\begin{mpFunctionsExtract}
\mpFunctionFour
{BetaDist? mpNumList? returns pdf, CDF and related information for the central Beta-distribution}
{x? mpNum? A real number}
{a? mpNum? A real number greater 0, representing the numerator  degrees of freedom}
{b? mpNum? A real number greater 0, representing the denominator degrees of freedom}
{Output? String? A string describing the output choices}
\end{mpFunctionsExtract}

\begin{mpFunctionsExtract}
\mpFunctionFour
{BetaDistInv? mpNumList? returns quantiles and related information for the the central Beta-distribution}
{Prob? mpNum? A real number between 0 and 1.}
{m? mpNum? A real number greater 0, representing the numerator  degrees of freedom}
{n? mpNum? A real number greater 0, representing the denominator degrees of freedom}
{Output? String? A string describing the output choices}
\end{mpFunctionsExtract}

\begin{mpFunctionsExtract}
\mpFunctionThree
{BetaDistInfo? mpNumList? returns moments and related information for the central Beta-distribution}
{a? mpNum? A real number greater 0, representing the degrees of freedom}
{b? mpNum? A real number greater 0, representing the degrees of freedom}
{Output? String? A string describing the output choices}
\end{mpFunctionsExtract}

\begin{mpFunctionsExtract}
\mpFunctionFive
{BetaDistRandom? mpNumList? returns random numbers following a central Beta-distribution}
{Size? mpNum? A positive integer up to $10^7$}
{a? mpNum? A real number greater 0, representing the numerator  degrees of freedom}
{b? mpNum? A real number greater 0, representing the denominator degrees of freedom}
{Generator? String? A string describing the random generator}
{Output? String? A string describing the output choices}
\end{mpFunctionsExtract}

\section{Binomial Distribution}

\begin{mpFunctionsExtract}
\mpFunctionFour
{BinomialDist? mpNumList? returns pdf, CDF and related information for the central Binomial-distribution}
{x? mpNum? The number of successes in trials.}
{n? mpNum? The number of independent trials.}
{p? mpNum? The probability of success on each trial}
{Output? String? A string describing the output choices}
\end{mpFunctionsExtract}

\begin{mpFunctionsExtract}
\mpFunctionFour
{BinomialDistInv? mpNumList? returns quantiles and related information for the the central binomial-distribution}
{Prob? mpNum? A real number between 0 and 1.}
{n? mpNum? The number of Bernoulli trials.}
{p? mpNum? The probability of a success on each trial.}
{Output? String? A string describing the output choices}
\end{mpFunctionsExtract}

\begin{mpFunctionsExtract}
\mpFunctionThree
{BinomialDistInfo? mpNumList? returns moments and related information for the central Binomial-distribution}
{n? mpNum? The number of Bernoulli trials.}
{p? mpNum? The probability of a success on each trial.}
{Output? String? A string describing the output choices}
\end{mpFunctionsExtract}

\begin{mpFunctionsExtract}
\mpFunctionFive
{BinomialDistRandom? mpNumList? returns random numbers following a central Binomial-distribution}
{Size? mpNum? A positive integer up to $10^7$}
{n? mpNum? The number of Bernoulli trials.}
{p? mpNum? The probability of a success on each trial.}
{Generator? String? A string describing the random generator}
{Output? String? A string describing the output choices}
\end{mpFunctionsExtract}

\section{Chi-Square Distribution}

\begin{mpFunctionsExtract}
\mpFunctionThree
{CDist? mpNumList? returns pdf, CDF and related information for the central $\chi^2$-distribution}
{x? mpNum? A real number}
{n? mpNum? A real number greater 0, representing the degrees of freedom}
{Output? String? A string describing the output choices}
\end{mpFunctionsExtract}

\begin{mpFunctionsExtract}
\mpFunctionThree
{CDistInv? mpNumList? quantiles and related information for the the central $\chi^2$-distribution}
{Prob? mpNum? A real number between 0 and 1.}
{n? mpNum? A real number greater 0, representing the degrees of freedom}
{Output? String? A string describing the output choices}
\end{mpFunctionsExtract}

\begin{mpFunctionsExtract}
\mpFunctionTwo
{CDistInfo? mpNumList? moments and related information for the central $\chi^2$-distribution}
{n? mpNum? A real number greater 0, representing the degrees of freedom}
{Output? String? A string describing the output choices}
\end{mpFunctionsExtract}

\begin{mpFunctionsExtract}
\mpFunctionFour
{CDistRan? mpNumList? random numbers following a central $\chi^2$-distribution}
{Size? mpNum? A positive integer up to $10^7$}
{n? mpNum? A real number greater 0, representing the degrees of freedom}
{Generator? String? A string describing the random generator}
{Output? String? A string describing the output choices}
\end{mpFunctionsExtract}

\section{Exponential Distribution}

\begin{mpFunctionsExtract}
\mpFunctionThree
{ExponentialDist? mpNumList? returns pdf, CDF and related information for the central Exponential distribution}
{x? mpNum? The value of the distribution.}
{lambda? mpNum? The parameter of the distribution.}
{Output? String? A string describing the output choices}
\end{mpFunctionsExtract}

\begin{mpFunctionsExtract}
\mpFunctionThree
{ExponentialDistInv? mpNumList? returns quantiles and related information for the the central Exponential distribution}
{Prob? mpNum? A real number between 0 and 1.}
{lambda? mpNum? The number of Bernoulli trials.}
{Output? String? A string describing the output choices}
\end{mpFunctionsExtract}

\begin{mpFunctionsExtract}
\mpFunctionTwo
{ExponentialDistInfo? mpNumList? returns moments and related information for the central $t$-distribution}
{lambda? mpNum? A real number greater 0, representing the parameter of the distribution}
{Output? String? A string describing the output choices}
\end{mpFunctionsExtract}

\begin{mpFunctionsExtract}
\mpFunctionFour
{ExponentialDistRandom? mpNumList? returns random numbers following a central Beta-distribution}
{Size? mpNum? A positive integer up to $10^7$}
{lambda? mpNum? A real number greater 0, representing the numerator  degrees of freedom}
{Generator? String? A string describing the random generator}
{Output? String? A string describing the output choices}
\end{mpFunctionsExtract}

\section{Fisher's F-Distribution}

\begin{mpFunctionsExtract}
\mpFunctionFour
{FDist? mpNumList? returns pdf, CDF and related information for the central $F$-distribution}
{x? mpNum? A real number}
{m? mpNum? A real number greater 0, representing the numerator  degrees of freedom}
{n? mpNum? A real number greater 0, representing the denominator degrees of freedom}
{Output? String? A string describing the output choices}
\end{mpFunctionsExtract}

\begin{mpFunctionsExtract}
\mpFunctionThree
{FDistInv? mpNumList? returns quantiles and related information for the the central $t$-distribution}
{Prob? mpNum? A real number between 0 and 1.}
{m? mpNum? A real number greater 0, representing the numerator  degrees of freedom}
{n? mpNum? A real number greater 0, representing the denominator degrees of freedom}
{Output? String? A string describing the output choices}
\end{mpFunctionsExtract}

\begin{mpFunctionsExtract}
\mpFunctionThree
{FDistInfo? mpNumList? returns moments and related information for the central $t$-distribution}
{m? mpNum? A real number greater 0, representing the numerator  degrees of freedom}
{n? mpNum? A real number greater 0, representing the denominator degrees of freedom}
{Output? String? A string describing the output choices}
\end{mpFunctionsExtract}

\begin{mpFunctionsExtract}
\mpFunctionFive
{FDistRan? mpNumList? returns random numbers following a central $F$-distribution}
{Size? mpNum? A positive integer up to $10^7$}
{m? mpNum? A real number greater 0, representing the numerator  degrees of freedom}
{n? mpNum? A real number greater 0, representing the denominator degrees of freedom}
{Generator? String? A string describing the random generator}
{Output? String? A string describing the output choices}
\end{mpFunctionsExtract}

\section{Gamma (and Erlang) Distribution}

\begin{mpFunctionsExtract}
\mpFunctionFour
{GammaDist? mpNumList? returns pdf, CDF and related information for the central Gamma-distribution}
{x? mpNum? A real number}
{a? mpNum? A real number greater 0, a parameter to the distribution}
{b? mpNum? A real number greater 0, a parameter to the distribution}
{Output? String? A string describing the output choices}
\end{mpFunctionsExtract}

\begin{mpFunctionsExtract}
\mpFunctionThree
{GammaDistInv? mpNumList? returns quantiles and related information for the the central Gamma-distribution}
{Prob? mpNum? A real number between 0 and 1.}
{m? mpNum? A real number greater 0, a parameter to the distribution}
{n? mpNum? A real number greater 0, a parameter to the distribution}
{Output? String? A string describing the output choices}
\end{mpFunctionsExtract}

\begin{mpFunctionsExtract}
\mpFunctionTwo
{GammaDistInfo? mpNumList? returns moments and related information for the central Gamma-distribution}
{a? mpNum? A real number greater 0, representing the degrees of freedom}
{b? mpNum? A real number greater 0, representing the degrees of freedom}
{Output? String? A string describing the output choices}
\end{mpFunctionsExtract}

\begin{mpFunctionsExtract}
\mpFunctionFive
{GammaDistRandom? mpNumList? returns random numbers following a central Beta-distribution}
{Size? mpNum? A positive integer up to $10^7$}
{a? mpNum? A real number greater 0, a parameter to the distribution}
{b? mpNum? A real number greater 0, a parameter to the distribution}
{Generator? String? A string describing the random generator}
{Output? String? A string describing the output choices}
\end{mpFunctionsExtract}

\section{Hypergeometric Distribution}

\begin{mpFunctionsExtract}
\mpFunctionFive
{HypergeometricDist? mpNumList? returns pdf, CDF and related information for the central hypergeometric distribution}
{x? mpNum? The number of successes in the sample.}
{n? mpNum? The size of the sample.}
{M? mpNum? The number of successes in the population}
{N? mpNum? The population size}
{Output? String? A string describing the output choices}
\end{mpFunctionsExtract}

\begin{mpFunctionsExtract}
\mpFunctionFive
{HypergeometricDistInv? mpNumList? returns quantiles and related information for the the central hypergeometric distribution}
{Prob? mpNum? A real number between 0 and 1.}
{n? mpNum? The size of the sample.}
{M? mpNum? The number of successes in the population}
{N? mpNum? The population size}
{Output? String? A string describing the output choices}
\end{mpFunctionsExtract}

\begin{mpFunctionsExtract}
\mpFunctionFour
{HypergeometricDistInfo? mpNumList? returns moments and related information for the central hypergeometric distribution}
{n? mpNum? The size of the sample.}
{M? mpNum? The number of successes in the population}
{N? mpNum? The population size}
{Output? String? A string describing the output choices}
\end{mpFunctionsExtract}

\begin{mpFunctionsExtract}
\mpFunctionSix
{HypergeometricDistRandom? mpNumList? returns random numbers following a central hypergeometric distribution}
{Size? mpNum? A positive integer up to $10^7$}
{n? mpNum? The size of the sample.}
{M? mpNum? The number of successes in the population}
{N? mpNum? The population size}
{Generator? String? A string describing the random generator}
{Output? String? A string describing the output choices}
\end{mpFunctionsExtract}

\section{Lognormal Distribution}

\begin{mpFunctionsExtract}
\mpFunctionFour
{LogNormalDist? mpNumList? returns pdf, CDF and related information for the Lognormal-distribution}
{x? mpNum? A real number}
{mean? mpNum? A real number greater 0, representing the mean of the distribution}
{stdev? mpNum? A real number greater 0, representing the standard deviation of the distribution}
{Output? String? A string describing the output choices}
\end{mpFunctionsExtract}

\begin{mpFunctionsExtract}
\mpFunctionFour
{LognormalDistInv? mpNumList? returns quantiles and related information for the the Lognormal-distribution}
{Prob? mpNum? A real number between 0 and 1.}
{mean? mpNum? A real number greater 0, representing the mean of the distribution}
{stdev? mpNum? A real number greater 0, representing the standard deviation of the distribution}
{Output? String? A string describing the output choices}
\end{mpFunctionsExtract}

\begin{mpFunctionsExtract}
\mpFunctionThree
{LognormalDistInfo? mpNumList? returns moments and related information for the central Lognormal-distribution}
{mean? mpNum? A real number greater 0, representing the mean of the distribution}
{stdev? mpNum? A real number greater 0, representing the standard deviation of the distribution}
{Output? String? A string describing the output choices}
\end{mpFunctionsExtract}

\begin{mpFunctionsExtract}
\mpFunctionFive
{LognormalRandom? mpNumList? returns random numbers following a central Beta-distribution}
{Size? mpNum? A positive integer up to $10^7$}
{mean? mpNum? A real number greater 0, representing the mean of the distribution}
{stdev? mpNum? A real number greater 0, representing the standard deviation of the distribution}
{Generator? String? A string describing the random generator}
{Output? String? A string describing the output choices}
\end{mpFunctionsExtract}

\section{Negative Binomial Distribution}

\begin{mpFunctionsExtract}
\mpFunctionFour
{NegativeBinomialDist? mpNumList? returns pdf, CDF and related information for the central negative binomial distribution}
{x? mpNum? The number of failures in trials.}
{r? mpNum? The threshold number of successes.}
{p? mpNum? The probability of a success}
{Output? String? A string describing the output choices}
\end{mpFunctionsExtract}

\begin{mpFunctionsExtract}
\mpFunctionFour
{NegativeBinomialDistInv? mpNumList? returns quantiles and related information for the the central binomial-distribution}
{Prob? mpNum? A real number between 0 and 1.}
{r? mpNum? The threshold number of successes.}
{p? mpNum? The probability of a success}
{Output? String? A string describing the output choices}
\end{mpFunctionsExtract}

\begin{mpFunctionsExtract}
\mpFunctionThree
{NegativeBinomialDistInfo? mpNumList? returns moments and related information for the central Binomial-distribution}
{r? mpNum? The threshold number of successes.}
{p? mpNum? The probability of a success}
{Output? String? A string describing the output choices}
\end{mpFunctionsExtract}

\begin{mpFunctionsExtract}
\mpFunctionFive
{NegativeBinomialDistRandom? mpNumList? returns random numbers following a central Binomial-distribution}
{Size? mpNum? A positive integer up to $10^7$}
{r? mpNum? The threshold number of successes.}
{p? mpNum? The probability of a success}
{Generator? String? A string describing the random generator}
{Output? String? A string describing the output choices}
\end{mpFunctionsExtract}

\section{Normal Distribution}

\begin{mpFunctionsExtract}
\mpFunctionFour
{NDist? mpNumList? returns pdf, CDF and related information for the normal-distribution}
{x? mpNum? A real number}
{mean? mpNum? A real number greater 0, representing the mean of the distribution}
{stdev? mpNum? A real number greater 0, representing the standard deviation of the distribution}
{Output? String? A string describing the output choices}
\end{mpFunctionsExtract}

\begin{mpFunctionsExtract}
\mpFunctionFour
{NDistInv? mpNumList? returns quantiles and related information for the the Lognormal-distribution}
{Prob? mpNum? A real number between 0 and 1.}
{mean? mpNum? A real number greater 0, representing the mean of the distribution}
{stdev? mpNum? A real number greater 0, representing the standard deviation of the distribution}
{Output? String? A string describing the output choices}
\end{mpFunctionsExtract}

\begin{mpFunctionsExtract}
\mpFunctionThree
{NormalDistInfo? mpNumList? returns moments and related information for the central Lognormal-distribution}
{mean? mpNum? A real number greater 0, representing the mean of the distribution}
{stdev? mpNum? A real number greater 0, representing the standard deviation of the distribution}
{Output? String? A string describing the output choices}
\end{mpFunctionsExtract}

\begin{mpFunctionsExtract}
\mpFunctionFive
{NormalRandom? mpNumList? returns random numbers following a central Beta-distribution}
{Size? mpNum? A positive integer up to $10^7$}
{mean? mpNum? A real number greater 0, representing the mean of the distribution}
{stdev? mpNum? A real number greater 0, representing the standard deviation of the distribution}
{Generator? String? A string describing the random generator}
{Output? String? A string describing the output choices}
\end{mpFunctionsExtract}

\section{Poisson Distribution}

\begin{mpFunctionsExtract}
\mpFunctionThree
{PoissonDist? mpNumList? returns pdf, CDF and related information for the Poisson distribution}
{x? mpNum? A real number}
{lambda? mpNum? A real number greater 0, representing the degrees of freedom}
{Output? String? A string describing the output choices}
\end{mpFunctionsExtract}

\begin{mpFunctionsExtract}
\mpFunctionThree
{PoissonDistInv? mpNumList? quantiles and related information for the the Poisson distribution}
{Prob? mpNum? A real number between 0 and 1.}
{lambda? mpNum? A real number greater 0, representing the degrees of freedom}
{Output? String? A string describing the output choices}
\end{mpFunctionsExtract}

\begin{mpFunctionsExtract}
\mpFunctionTwo
{PoissonDistInfo? mpNumList? moments and related information for the Poisson distribution}
{lambda? mpNum? A real number greater 0, representing the degrees of freedom}
{Output? String? A string describing the output choices}
\end{mpFunctionsExtract}

\begin{mpFunctionsExtract}
\mpFunctionFour
{PoissonDistRan? mpNumList? random numbers following a Poisson distribution}
{Size? mpNum? A positive integer up to $10^7$}
{lambda? mpNum? A real number greater 0, representing the degrees of freedom}
{Generator? String? A string describing the random generator}
{Output? String? A string describing the output choices}
\end{mpFunctionsExtract}

\section{Student's t-Distribution}

\begin{mpFunctionsExtract}
\mpFunctionThree
{TDist? mpNumList? returns pdf, CDF and related information for the central $t$-distribution}
{x? mpNum? A real number}
{n? mpNum? A real number greater 0, representing the degrees of freedom}
{Output? String? A string describing the output choices}
\end{mpFunctionsExtract}

\begin{mpFunctionsExtract}
\mpFunctionThree
{TDistInv? mpNumList? returns quantiles and related information for the the central $t$-distribution}
{Prob? mpNum? A real number between 0 and 1.}
{n? mpNum? A real number greater 0, representing the degrees of freedom}
{Output? String? A string describing the output choices}
\end{mpFunctionsExtract}

\begin{mpFunctionsExtract}
\mpFunctionTwo
{TDistInfo? mpNumList? returns moments and related information for the central $t$-distribution}
{n? mpNum? A real number greater 0, representing the degrees of freedom}
{Output? String? A string describing the output choices}
\end{mpFunctionsExtract}

\begin{mpFunctionsExtract}
\mpFunctionFour
{TDistRan? mpNumList? returns random numbers following a central $t$-distribution}
{Size? mpNum? A positive integer up to $10^7$}
{n? mpNum? A real number greater 0, representing the degrees of freedom}
{Generator? String? A string describing the random generator}
{Output? String? A string describing the output choices}
\end{mpFunctionsExtract}

\section{Weibull Distribution}

\begin{mpFunctionsExtract}
\mpFunctionFour
{WeibullDist? mpNumList? returns pdf, CDF and related information for the Weibull distribution}
{x? mpNum? A real number}
{a? mpNum? A real number greater 0, representing the numerator  degrees of freedom}
{b? mpNum? A real number greater 0, representing the denominator degrees of freedom}
{Output? String? A string describing the output choices}
\end{mpFunctionsExtract}

\begin{mpFunctionsExtract}
\mpFunctionFour
{WeibullDistInv? mpNumList? returns quantiles and related information for the the central Beta-distribution}
{Prob? mpNum? A real number between 0 and 1.}
{a? mpNum? A real number greater 0, representing the numerator  degrees of freedom}
{b? mpNum? A real number greater 0, representing the denominator degrees of freedom}
{Output? String? A string describing the output choices}
\end{mpFunctionsExtract}

\begin{mpFunctionsExtract}
\mpFunctionThree
{WeibullDistInfo? mpNumList? returns moments and related information for the central Beta-distribution}
{a? mpNum? A real number greater 0, representing the degrees of freedom}
{b? mpNum? A real number greater 0, representing the degrees of freedom}
{Output? String? A string describing the output choices}
\end{mpFunctionsExtract}

\begin{mpFunctionsExtract}
\mpFunctionFive
{WeibullDistRandom? mpNumList? returns random numbers following a central Beta-distribution}
{Size? mpNum? A positive integer up to $10^7$}
{a? mpNum? A real number greater 0, representing the numerator  degrees of freedom}
{b? mpNum? A real number greater 0, representing the denominator degrees of freedom}
{Generator? String? A string describing the random generator}
{Output? String? A string describing the output choices}
\end{mpFunctionsExtract}

\section{Bernoulli Distribution}

\begin{mpFunctionsExtract}
\mpFunctionThree
{BernoulliDistBoost? mpNumList? returns pdf, CDF and related information for the central $t$-distribution}
{k? mpNum? A real number, 0 or 1}
{p? mpNum? A real number greater 0, representing the degrees of freedom}
{Output? String? A string describing the output choices}
\end{mpFunctionsExtract}

\begin{mpFunctionsExtract}
\mpFunctionThree
{BernoulliDistInvBoost? mpNumList? returns quantiles and related information for the the central $t$-distribution}
{Prob? mpNum? A real number between 0 and 1.}
{p? mpNum? A real number greater 0, representing the degrees of freedom}
{Output? String? A string describing the output choices}
\end{mpFunctionsExtract}

\begin{mpFunctionsExtract}
\mpFunctionTwo
{BernoulliDistInfoBoost? mpNumList? returns moments and related information for the central $t$-distribution}
{p? mpNum? A real number greater 0, representing the degrees of freedom}
{Output? String? A string describing the output choices}
\end{mpFunctionsExtract}

\begin{mpFunctionsExtract}
\mpFunctionFour
{BernoulliDistRandomBoost? mpNumList? returns random numbers following a central Binomial-distribution}
{Size? mpNum? A positive integer up to $10^7$}
{p? mpNum? The probability of a success on each trial.}
{Generator? String? A string describing the random generator}
{Output? String? A string describing the output choices}
\end{mpFunctionsExtract}

\section{Cauchy Distribution}

\begin{mpFunctionsExtract}
\mpFunctionFour
{CauchyDistBoost? mpNumList? returns pdf, CDF and related information for the Cauchy distribution}
{x? mpNum? A real number}
{a? mpNum? A real number greater 0, representing the numerator  degrees of freedom}
{b? mpNum? A real number greater 0, representing the denominator degrees of freedom}
{Output? String? A string describing the output choices}
\end{mpFunctionsExtract}

\begin{mpFunctionsExtract}
\mpFunctionFour
{CauchyDistInvBoost? mpNumList? returns quantiles and related information for the Cauchy distribution}
{Prob? mpNum? A real number between 0 and 1.}
{a? mpNum? A real number greater 0, representing the numerator  degrees of freedom}
{b? mpNum? A real number greater 0, representing the denominator degrees of freedom}
{Output? String? A string describing the output choices}
\end{mpFunctionsExtract}

\begin{mpFunctionsExtract}
\mpFunctionThree
{CauchyDistInfoBoost? mpNumList? returns moments and related information for the Cauchy distribution}
{a? mpNum? A real number greater 0, representing the degrees of freedom}
{b? mpNum? A real number greater 0, representing the degrees of freedom}
{Output? String? A string describing the output choices}
\end{mpFunctionsExtract}

\begin{mpFunctionsExtract}
\mpFunctionFive
{CauchyDistRandomBoost? mpNumList? returns random numbers following a Cauchy distribution}
{Size? mpNum? A positive integer up to $10^7$}
{a? mpNum? A real number greater 0, representing the numerator  degrees of freedom}
{b? mpNum? A real number greater 0, representing the denominator degrees of freedom}
{Generator? String? A string describing the random generator}
{Output? String? A string describing the output choices}
\end{mpFunctionsExtract}

\section{Extreme Value (or Gumbel) Distribution}

\begin{mpFunctionsExtract}
\mpFunctionFour
{ExtremevalueDistBoost? mpNumList? returns pdf, CDF and related information for the Extreme Value distribution}
{x? mpNum? A real number}
{a? mpNum? A real number greater 0, representing the numerator  degrees of freedom}
{b? mpNum? A real number greater 0, representing the denominator degrees of freedom}
{Output? String? A string describing the output choices}
\end{mpFunctionsExtract}

\begin{mpFunctionsExtract}
\mpFunctionFour
{ExtremevalueDistInvBoost? mpNumList? returns quantiles and related information for the the Extreme Value distribution}
{Prob? mpNum? A real number between 0 and 1.}
{a? mpNum? A real number greater 0, representing the numerator  degrees of freedom}
{b? mpNum? A real number greater 0, representing the denominator degrees of freedom}
{Output? String? A string describing the output choices}
\end{mpFunctionsExtract}

\begin{mpFunctionsExtract}
\mpFunctionThree
{ExtremevalueDistInfoBoost? mpNumList? returns moments and related information for the Extreme Value distribution}
{a? mpNum? A real number greater 0, representing the degrees of freedom}
{b? mpNum? A real number greater 0, representing the degrees of freedom}
{Output? String? A string describing the output choices}
\end{mpFunctionsExtract}

\begin{mpFunctionsExtract}
\mpFunctionFive
{ExtremevalueDistRandomBoost? mpNumList? returns random numbers following a Extreme Value distribution}
{Size? mpNum? A positive integer up to $10^7$}
{a? mpNum? A real number greater 0, representing the numerator  degrees of freedom}
{b? mpNum? A real number greater 0, representing the denominator degrees of freedom}
{Generator? String? A string describing the random generator}
{Output? String? A string describing the output choices}
\end{mpFunctionsExtract}

\section{Geometric Distribution}

\begin{mpFunctionsExtract}
\mpFunctionThree
{GeometricDistBoost? mpNumList? returns pdf, CDF and related information for the Geometric distribution}
{k? mpNum? A real number}
{p? mpNum? A real number greater 0, representing the numerator  degrees of freedom}
{Output? String? A string describing the output choices}
\end{mpFunctionsExtract}

\begin{mpFunctionsExtract}
\mpFunctionThree
{GeometricDistInvBoost? mpNumList? returns quantiles and related information for the Geometric distribution}
{Prob? mpNum? A real number between 0 and 1.}
{p? mpNum? A real number greater 0, representing the numerator  degrees of freedom}
{Output? String? A string describing the output choices}
\end{mpFunctionsExtract}

\begin{mpFunctionsExtract}
\mpFunctionTwo
{GeometricDistInfoBoost? mpNumList? returns moments and related information for the Geometric distribution}
{p? mpNum? A real number greater 0, representing the degrees of freedom}
{Output? String? A string describing the output choices}
\end{mpFunctionsExtract}

\begin{mpFunctionsExtract}
\mpFunctionFour
{GeometricDistRandomBoost? mpNumList? returns random numbers following a Geometric distribution}
{Size? mpNum? A positive integer up to $10^7$}
{p? mpNum? A real number greater 0, representing the denominator degrees of freedom}
{Generator? String? A string describing the random generator}
{Output? String? A string describing the output choices}
\end{mpFunctionsExtract}

\section{Inverse Chi Squared Distribution}

\begin{mpFunctionsExtract}
\mpFunctionThree
{InverseChiSquaredDistBoost? mpNumList? returns pdf, CDF and related information for the inverse-chi-squared -distribution}
{x? mpNum? A real number}
{n? mpNum? A real number greater 0, representing the degrees of freedom}
{Output? String? A string describing the output choices}
\end{mpFunctionsExtract}

\begin{mpFunctionsExtract}
\mpFunctionThree
{InverseChiSquaredDistInvBoost? mpNumList? quantiles and related information for the inverse-chi-squared distribution}
{Prob? mpNum? A real number between 0 and 1.}
{n? mpNum? A real number greater 0, representing the degrees of freedom}
{Output? String? A string describing the output choices}
\end{mpFunctionsExtract}

\begin{mpFunctionsExtract}
\mpFunctionTwo
{InverseChiSquaredDistInfoBoost? mpNumList? moments and related information for the inverse-chi-squared distribution}
{n? mpNum? A real number greater 0, representing the degrees of freedom}
{Output? String? A string describing the output choices}
\end{mpFunctionsExtract}

\begin{mpFunctionsExtract}
\mpFunctionFour
{InverseChiSquaredDistRanBoost? mpNumList? random numbers following a inverse-chi-squared distribution}
{Size? mpNum? A positive integer up to $10^7$}
{n? mpNum? A real number greater 0, representing the degrees of freedom}
{Generator? String? A string describing the random generator}
{Output? String? A string describing the output choices}
\end{mpFunctionsExtract}

\section{Inverse Gamma Distribution}

\begin{mpFunctionsExtract}
\mpFunctionFour
{InverseGammaDistBoost? mpNumList? returns pdf, CDF and related information for the inverse gamma distribution}
{x? mpNum? A real number}
{a? mpNum? A real number greater 0, representing the numerator  degrees of freedom}
{b? mpNum? A real number greater 0, representing the denominator degrees of freedom}
{Output? String? A string describing the output choices}
\end{mpFunctionsExtract}

\begin{mpFunctionsExtract}
\mpFunctionFour
{InverseGammaDistInvBoost? mpNumList? returns quantiles and related information for the the inverse gamma distribution}
{Prob? mpNum? A real number between 0 and 1.}
{m? mpNum? A real number greater 0, representing the numerator  degrees of freedom}
{n? mpNum? A real number greater 0, representing the denominator degrees of freedom}
{Output? String? A string describing the output choices}
\end{mpFunctionsExtract}

\begin{mpFunctionsExtract}
\mpFunctionThree
{InverseGammaDistInfoBoost? mpNumList? returns moments and related information for the inverse gamma distribution}
{a? mpNum? A real number greater 0, representing the degrees of freedom}
{b? mpNum? A real number greater 0, representing the degrees of freedom}
{Output? String? A string describing the output choices}
\end{mpFunctionsExtract}

\begin{mpFunctionsExtract}
\mpFunctionFive
{InverseGammaDistRanBoost? mpNumList? returns random numbers following a inverse gamma distribution}
{Size? mpNum? A positive integer up to $10^7$}
{a? mpNum? A real number greater 0, representing the numerator  degrees of freedom}
{b? mpNum? A real number greater 0, representing the denominator degrees of freedom}
{Generator? String? A string describing the random generator}
{Output? String? A string describing the output choices}
\end{mpFunctionsExtract}

\section{Inverse Gaussian (or Wald) Distribution}

\begin{mpFunctionsExtract}
\mpFunctionFour
{InverseGaussianDistBoost? mpNumList? returns pdf, CDF and related information for the inverse Gaussian distribution}
{x? mpNum? A real number}
{mu? mpNum? A real number greater 0, representing the numerator  degrees of freedom}
{lambda? mpNum? A real number greater 0, representing the denominator degrees of freedom}
{Output? String? A string describing the output choices}
\end{mpFunctionsExtract}

\begin{mpFunctionsExtract}
\mpFunctionFour
{InverseGaussianDistInvBoost? mpNumList? returns quantiles and related information for the the inverse Gaussian distribution}
{Prob? mpNum? A real number between 0 and 1.}
{mu? mpNum? A real number greater 0, representing the numerator  degrees of freedom}
{lambda? mpNum? A real number greater 0, representing the denominator degrees of freedom}
{Output? String? A string describing the output choices}
\end{mpFunctionsExtract}

\begin{mpFunctionsExtract}
\mpFunctionThree
{InverseGaussianDistInfoBoost? mpNumList? returns moments and related information for the inverse Gaussian distribution}
{mu? mpNum? A real number greater 0, representing the degrees of freedom}
{lambda? mpNum? A real number greater 0, representing the degrees of freedom}
{Output? String? A string describing the output choices}
\end{mpFunctionsExtract}

\begin{mpFunctionsExtract}
\mpFunctionFive
{InverseGaussianDistRanBoost? mpNumList? returns random numbers following a inverse Gaussian distribution}
{Size? mpNum? A positive integer up to $10^7$}
{mu? mpNum? A real number greater 0, representing the numerator  degrees of freedom}
{lambda? mpNum? A real number greater 0, representing the denominator degrees of freedom}
{Generator? String? A string describing the random generator}
{Output? String? A string describing the output choices}
\end{mpFunctionsExtract}

\section{Laplace Distribution}

\begin{mpFunctionsExtract}
\mpFunctionFour
{LaplaceDistBoost? mpNumList? returns pdf, CDF and related information for the Laplace distribution}
{x? mpNum? A real number}
{a? mpNum? A real number greater 0, representing the numerator  degrees of freedom}
{b? mpNum? A real number greater 0, representing the denominator degrees of freedom}
{Output? String? A string describing the output choices}
\end{mpFunctionsExtract}

\begin{mpFunctionsExtract}
\mpFunctionFour
{LaplaceDistInvBoost? mpNumList? returns quantiles and related information for the the Laplace distribution}
{Prob? mpNum? A real number between 0 and 1.}
{a? mpNum? A real number greater 0, representing the numerator  degrees of freedom}
{b? mpNum? A real number greater 0, representing the denominator degrees of freedom}
{Output? String? A string describing the output choices}
\end{mpFunctionsExtract}

\begin{mpFunctionsExtract}
\mpFunctionThree
{LaplaceDistInfoBoost? mpNumList? returns moments and related information for the Laplace distribution}
{a? mpNum? A real number greater 0, representing the degrees of freedom}
{b? mpNum? A real number greater 0, representing the degrees of freedom}
{Output? String? A string describing the output choices}
\end{mpFunctionsExtract}

\begin{mpFunctionsExtract}
\mpFunctionFive
{LaplaceDistRanBoost? mpNumList? returns random numbers following a Laplace distribution}
{Size? mpNum? A positive integer up to $10^7$}
{a? mpNum? A real number greater 0, representing the numerator  degrees of freedom}
{b? mpNum? A real number greater 0, representing the denominator degrees of freedom}
{Generator? String? A string describing the random generator}
{Output? String? A string describing the output choices}
\end{mpFunctionsExtract}

\section{Logistic Distribution}

\begin{mpFunctionsExtract}
\mpFunctionFour
{LogisticDistBoost? mpNumList? returns pdf, CDF and related information for the Logistic distribution}
{x? mpNum? A real number}
{a? mpNum? A real number greater 0, representing the numerator  degrees of freedom}
{b? mpNum? A real number greater 0, representing the denominator degrees of freedom}
{Output? String? A string describing the output choices}
\end{mpFunctionsExtract}

\begin{mpFunctionsExtract}
\mpFunctionFour
{LogisticDistInvBoost? mpNumList? returns quantiles and related information for the the Logistic distribution}
{Prob? mpNum? A real number between 0 and 1.}
{a? mpNum? A real number greater 0, representing the numerator  degrees of freedom}
{b? mpNum? A real number greater 0, representing the denominator degrees of freedom}
{Output? String? A string describing the output choices}
\end{mpFunctionsExtract}

\begin{mpFunctionsExtract}
\mpFunctionThree
{LogisticDistInfoBoost? mpNumList? returns moments and related information for the Logistic distribution}
{a? mpNum? A real number greater 0, representing the degrees of freedom}
{b? mpNum? A real number greater 0, representing the degrees of freedom}
{Output? String? A string describing the output choices}
\end{mpFunctionsExtract}

\begin{mpFunctionsExtract}
\mpFunctionFive
{LogisticDistRanBoost? mpNumList? returns random numbers following a Logistic distribution}
{Size? mpNum? A positive integer up to $10^7$}
{a? mpNum? A real number greater 0, representing the numerator  degrees of freedom}
{b? mpNum? A real number greater 0, representing the denominator degrees of freedom}
{Generator? String? A string describing the random generator}
{Output? String? A string describing the output choices}
\end{mpFunctionsExtract}

\section{Pareto Distribution}

\begin{mpFunctionsExtract}
\mpFunctionFour
{ParetoDistBoost? mpNumList? returns pdf, CDF and related information for the Pareto distribution}
{x? mpNum? A real number}
{a? mpNum? A real number greater 0, representing the numerator  degrees of freedom}
{b? mpNum? A real number greater 0, representing the denominator degrees of freedom}
{Output? String? A string describing the output choices}
\end{mpFunctionsExtract}

\begin{mpFunctionsExtract}
\mpFunctionFour
{ParetoDistInvBoost? mpNumList? returns quantiles and related information for the the Pareto distribution}
{Prob? mpNum? A real number between 0 and 1.}
{a? mpNum? A real number greater 0, representing the numerator  degrees of freedom}
{b? mpNum? A real number greater 0, representing the denominator degrees of freedom}
{Output? String? A string describing the output choices}
\end{mpFunctionsExtract}

\begin{mpFunctionsExtract}
\mpFunctionThree
{ParetoDistInfoBoost? mpNumList? returns moments and related information for the Pareto distribution}
{a? mpNum? A real number greater 0, representing the degrees of freedom}
{b? mpNum? A real number greater 0, representing the degrees of freedom}
{Output? String? A string describing the output choices}
\end{mpFunctionsExtract}

\begin{mpFunctionsExtract}
\mpFunctionFive
{ParetoDistRanBoost? mpNumList? returns random numbers following a Pareto distribution}
{Size? mpNum? A positive integer up to $10^7$}
{a? mpNum? A real number greater 0, representing the numerator  degrees of freedom}
{b? mpNum? A real number greater 0, representing the denominator degrees of freedom}
{Generator? String? A string describing the random generator}
{Output? String? A string describing the output choices}
\end{mpFunctionsExtract}

\section{Raleigh Distribution}

\begin{mpFunctionsExtract}
\mpFunctionThree
{RaleighDistBoost? mpNumList? returns pdf, CDF and related information for the Raleigh distribution}
{x? mpNum? A real number}
{n? mpNum? A real number greater 0, representing the degrees of freedom}
{Output? String? A string describing the output choices}
\end{mpFunctionsExtract}

\begin{mpFunctionsExtract}
\mpFunctionThree
{RaleighDistInvBoost? mpNumList? quantiles and related information for the Raleigh distribution}
{Prob? mpNum? A real number between 0 and 1.}
{n? mpNum? A real number greater 0, representing the degrees of freedom}
{Output? String? A string describing the output choices}
\end{mpFunctionsExtract}

\begin{mpFunctionsExtract}
\mpFunctionTwo
{RaleighDistInfoBoost? mpNumList? moments and related information for the Raleigh distribution}
{n? mpNum? A real number greater 0, representing the degrees of freedom}
{Output? String? A string describing the output choices}
\end{mpFunctionsExtract}

\begin{mpFunctionsExtract}
\mpFunctionFour
{RaleighDistRanBoost? mpNumList? random numbers following a Raleigh distribution}
{Size? mpNum? A positive integer up to $10^7$}
{n? mpNum? A real number greater 0, representing the degrees of freedom}
{Generator? String? A string describing the random generator}
{Output? String? A string describing the output choices}
\end{mpFunctionsExtract}

\section{Triangular Distribution}

\begin{mpFunctionsExtract}
\mpFunctionFive
{TriangularDistBoost? mpNumList? returns pdf, CDF and related information for the triangular distribution}
{x? mpNum? A real number.}
{a? mpNum? The left border parameter.}
{b? mpNum? The right border parameter.}
{c? mpNum? The mode parameter.}
{Output? String? A string describing the output choices}
\end{mpFunctionsExtract}

\begin{mpFunctionsExtract}
\mpFunctionFive
{TriangularDistInvBoost? mpNumList? returns quantiles and related information for the the triangular distribution}
{Prob? mpNum? A real number between 0 and 1.}
{a? mpNum? The left border parameter.}
{b? mpNum? The right border parameter.}
{c? mpNum? The mode parameter.}
{Output? String? A string describing the output choices}
\end{mpFunctionsExtract}

\begin{mpFunctionsExtract}
\mpFunctionFour
{TriangularDistInfoBoost? mpNumList? returns moments and related information for the triangular distribution}
{a? mpNum? The left border parameter.}
{b? mpNum? The right border parameter.}
{c? mpNum? The mode parameter.}
{Output? String? A string describing the output choices}
\end{mpFunctionsExtract}

\begin{mpFunctionsExtract}
\mpFunctionSix
{TriangularDistRanBoost? mpNumList? returns random numbers following a triangular distribution}
{Size? mpNum? A positive integer up to $10^7$}
{a? mpNum? The left border parameter.}
{b? mpNum? The right border parameter.}
{c? mpNum? The mode parameter.}
{Generator? String? A string describing the random generator}
{Output? String? A string describing the output choices}
\end{mpFunctionsExtract}

\section{Uniform Distribution}

\begin{mpFunctionsExtract}
\mpFunctionFour
{UniformDistBoost? mpNumList? returns pdf, CDF and related information for the uniform distribution}
{x? mpNum? A real number}
{a? mpNum? The left border parameter.}
{b? mpNum? The right border parameter.}
{Output? String? A string describing the output choices}
\end{mpFunctionsExtract}

\begin{mpFunctionsExtract}
\mpFunctionFour
{UniformDistInvBoost? mpNumList? returns quantiles and related information for the the uniform distribution}
{Prob? mpNum? A real number between 0 and 1.}
{a? mpNum? The left border parameter.}
{b? mpNum? The right border parameter.}
{Output? String? A string describing the output choices}
\end{mpFunctionsExtract}

\begin{mpFunctionsExtract}
\mpFunctionThree
{UniformDistInfoBoost? mpNumList? returns moments and related information for the uniform distribution}
{a? mpNum? A real number greater 0, representing the degrees of freedom}
{b? mpNum? A real number greater 0, representing the degrees of freedom}
{Output? String? A string describing the output choices}
\end{mpFunctionsExtract}

\begin{mpFunctionsExtract}
\mpFunctionFive
{UniformDistRanBoost? mpNumList? returns random numbers following a uniform distribution}
{Size? mpNum? A positive integer up to $10^7$}
{a? mpNum? A real number greater 0, representing the degrees of freedom}
{b? mpNum? A real number greater 0, representing the degrees of freedom}
{Generator? String? A string describing the random generator}
{Output? String? A string describing the output choices}
\end{mpFunctionsExtract}

\chapter{Noncentral Distribution Functions (based on Boost)}

\section{Noncentral Beta-Distribution}

\begin{mpFunctionsExtract}
\mpFunctionFive
{NoncentralBetaDistBoost? mpNumList? returns pdf, CDF and related information for the central Beta-distribution}
{x? mpNum? A real number}
{m? mpNum? A real number greater 0, representing the numerator  degrees of freedom}
{n? mpNum? A real number greater 0, representing the denominator degrees of freedom}
{lambda? mpNum? A real number greater 0, representing the noncentrality parameter}
{Output? String? A string describing the output choices}
\end{mpFunctionsExtract}

\begin{mpFunctionsExtract}
\mpFunctionFive
{NoncentralBetaDistInvBoost? mpNumList? returns quantiles and related information for the the noncentral Beta-distribution}
{Prob? mpNum? A real number between 0 and 1.}
{m? mpNum? A real number greater 0, representing the numerator  degrees of freedom}
{n? mpNum? A real number greater 0, representing the denominator degrees of freedom}
{lambda? mpNum? A real number greater 0, representing the noncentrality parameter}
{Output? String? A string describing the output choices}
\end{mpFunctionsExtract}

\begin{mpFunctionsExtract}
\mpFunctionFour
{NoncentralBetaDistInfoBoost? mpNumList? returns moments and related information for the noncentral Beta-distribution}
{m? mpNum? A real number greater 0, representing the numerator  degrees of freedom}
{n? mpNum? A real number greater 0, representing the denominator degrees of freedom}
{lambda? mpNum? A real number greater 0, representing the noncentrality parameter}
{Output? String? A string describing the output choices}
\end{mpFunctionsExtract}

\begin{mpFunctionsExtract}
\mpFunctionSix
{NoncentralBetaDistRanBoost? mpNumList? returns random numbers following a noncentral Beta-distribution}
{Size? mpNum? A positive integer up to $10^7$}
{m? mpNum? A real number greater 0, representing the numerator  degrees of freedom}
{n? mpNum? A real number greater 0, representing the denominator degrees of freedom}
{lambda? mpNum? A real number greater 0, representing the noncentrality parameter}
{Generator? String? A string describing the random generator}
{Output? String? A string describing the output choices}
\end{mpFunctionsExtract}

\section{Noncentral Chi-Square Distribution}

\begin{mpFunctionsExtract}
\mpFunctionFour
{NoncentralCDistBoost? mpNumList? returns pdf, CDF and related information for the noncentral $\chi^2$-distribution}
{x? mpNum? A real number}
{n? mpNum? A real number greater 0, representing the degrees of freedom}
{lambda? mpNum? A real number greater 0, representing the noncentrality parameter}
{Output? String? A string describing the output choices}
\end{mpFunctionsExtract}

\begin{mpFunctionsExtract}
\mpFunctionFour
{NoncentralCDistInvBoost? mpNumList? quantiles and related information for the noncentral $\chi^2$-distribution}
{Prob? mpNum? A real number between 0 and 1.}
{n? mpNum? A real number greater 0, representing the degrees of freedom}
{lambda? mpNum? A real number greater 0, representing the noncentrality parameter}
{Output? String? A string describing the output choices}
\end{mpFunctionsExtract}

\begin{mpFunctionsExtract}
\mpFunctionThree
{NoncentralCDistInfoBoost? mpNumList? moments and related information for the noncentral $\chi^2$-distribution}
{n? mpNum? A real number greater 0, representing the degrees of freedom}
{lambda? mpNum? A real number greater 0, representing the noncentrality parameter}
{Output? String? A string describing the output choices}
\end{mpFunctionsExtract}

\begin{mpFunctionsExtract}
\mpFunctionFive
{NoncentralCDistRanBoost? mpNumList? random numbers following a noncentral $\chi^2$-distribution}
{Size? mpNum? A positive integer up to $10^7$}
{n? mpNum? A real number greater 0, representing the degrees of freedom}
{lambda? mpNum? A real number greater 0, representing the noncentrality parameter}
{Generator? String? A string describing the random generator}
{Output? String? A string describing the output choices}
\end{mpFunctionsExtract}

\section{NonCentral F-Distribution}

\begin{mpFunctionsExtract}
\mpFunctionFive
{NoncentralFDistBoost? mpNumList? returns pdf, CDF and related information for the noncentral $F$-distribution}
{x? mpNum? A real number}
{m? mpNum? A real number greater 0, representing the numerator  degrees of freedom}
{n? mpNum? A real number greater 0, representing the denominator degrees of freedom}
{lambda? mpNum? A real number greater 0, representing the noncentrality parameter}
{Output? String? A string describing the output choices}
\end{mpFunctionsExtract}

\begin{mpFunctionsExtract}
\mpFunctionFive
{NoncentralFDistInvBoost? mpNumList? returns quantiles and related information for the the noncentral $F$-distribution}
{Prob? mpNum? A real number between 0 and 1.}
{m? mpNum? A real number greater 0, representing the numerator  degrees of freedom}
{n? mpNum? A real number greater 0, representing the denominator degrees of freedom}
{lambda? mpNum? A real number greater 0, representing the noncentrality parameter}
{Output? String? A string describing the output choices}
\end{mpFunctionsExtract}

\begin{mpFunctionsExtract}
\mpFunctionFour
{NoncentralFDistInfoBoost? mpNumList? returns moments and related information for the noncentral $F$-distribution}
{m? mpNum? A real number greater 0, representing the numerator  degrees of freedom}
{n? mpNum? A real number greater 0, representing the denominator degrees of freedom}
{lambda? mpNum? A real number greater 0, representing the noncentrality parameter}
{Output? String? A string describing the output choices}
\end{mpFunctionsExtract}

\begin{mpFunctionsExtract}
\mpFunctionSix
{NoncentralFDistRanBoost? mpNumList? returns random numbers following a noncentral $F$-distribution}
{Size? mpNum? A positive integer up to $10^7$}
{m? mpNum? A real number greater 0, representing the numerator  degrees of freedom}
{n? mpNum? A real number greater 0, representing the denominator degrees of freedom}
{lambda? mpNum? A real number greater 0, representing the noncentrality parameter}
{Generator? String? A string describing the random generator}
{Output? String? A string describing the output choices}
\end{mpFunctionsExtract}

\section{Noncentral Student's t-Distribution}

\begin{mpFunctionsExtract}
\mpFunctionFour
{NoncentralTDistBoost? mpNumList? returns pdf, CDF and related information for the noncentral $t$-distribution}
{x? mpNum? A real number}
{n? mpNum? A real number greater 0, representing the degrees of freedom}
{delta? mpNum? A real number greater 0, representing the noncentrality parameter}
{Output? String? A string describing the output choices}
\end{mpFunctionsExtract}

\begin{mpFunctionsExtract}
\mpFunctionFour
{NoncentralTDistInvBoost? mpNumList? quantiles and related information for the  noncentral $t$-distribution}
{Prob? mpNum? A real number between 0 and 1.}
{n? mpNum? A real number greater 0, representing the degrees of freedom}
{delta? mpNum? A real number greater 0, representing the noncentrality parameter}
{Output? String? A string describing the output choices}
\end{mpFunctionsExtract}

\begin{mpFunctionsExtract}
\mpFunctionThree
{NoncentralTDistInfoBoost? mpNumList? moments and related information for the noncentral $t$-distribution}
{n? mpNum? A real number greater 0, representing the degrees of freedom}
{delta? mpNum? A real number greater 0, representing the noncentrality parameter}
{Output? String? A string describing the output choices}
\end{mpFunctionsExtract}

\begin{mpFunctionsExtract}
\mpFunctionFive
{NoncentralTDistRanBoost? mpNumList? random numbers following a noncentral $t$-distribution}
{Size? mpNum? A positive integer up to $10^7$}
{n? mpNum? A real number greater 0, representing the degrees of freedom}
{delta? mpNum? A real number greater 0, representing the noncentrality parameter}
{Generator? String? A string describing the random generator}
{Output? String? A string describing the output choices}
\end{mpFunctionsExtract}

\section{Skew Normal Distribution}

\begin{mpFunctionsExtract}
\mpFunctionFive
{SkewNormalDistBoost? mpNumList? returns pdf, CDF and related information for the skew normal distribution}
{x? mpNum? A real number.}
{a? mpNum? The location parameter.}
{b? mpNum? The scale parameter}
{c? mpNum? The shape parameter}
{Output? String? A string describing the output choices}
\end{mpFunctionsExtract}

\begin{mpFunctionsExtract}
\mpFunctionFive
{SkewNormalDistInvBoost? mpNumList? returns quantiles and related information for the the skew normal distribution}
{Prob? mpNum? A real number between 0 and 1.}
{a? mpNum? The location parameter.}
{b? mpNum? The scale parameter}
{c? mpNum? The shape parameter}
{Output? String? A string describing the output choices}
\end{mpFunctionsExtract}

\begin{mpFunctionsExtract}
\mpFunctionFour
{SkewNormalDistInfoBoost? mpNumList? returns moments and related information for the skew normal distribution}
{a? mpNum? The location parameter.}
{b? mpNum? The scale parameter}
{c? mpNum? The shape parameter}
{Output? String? A string describing the output choices}
\end{mpFunctionsExtract}

\begin{mpFunctionsExtract}
\mpFunctionSix
{SkewNormalDistRanBoost? mpNumList? returns random numbers following a skew normal distribution}
{Size? mpNum? A positive integer up to $10^7$}
{a? mpNum? The location parameter.}
{b? mpNum? The scale parameter}
{c? mpNum? The shape parameter}
{Generator? String? A string describing the random generator}
{Output? String? A string describing the output choices}
\end{mpFunctionsExtract}

\section{Owen's T-Function}

\begin{mpFunctionsExtract}
\mpFunctionTwo
{TOwenBoost? mpNum? Owen's T-Function}
{h? mpNum? A real number.}
{a? mpNum? A real number.}
\end{mpFunctionsExtract}

\chapter{Ordinary Differential Equations}

\section{Defining the ODE System}

\section{Stepping Functions}

\section{Integrate functions: Evolution}

\chapter{Examples: Multivariate Special Functions}

\section{Functions Of Matrix Arguments}

\begin{mpFunctionsExtract}
\mpFunctionTwoNotImplemented
{Gammap? mpNum? the multivariate gamma function.}
{p? mpNum? An integer greater than 0.}
{x? mpNum? A real number or an array of real numbers.}
\end{mpFunctionsExtract}

\begin{mpFunctionsExtract}
\mpFunctionFourNotImplemented
{Hypergeometric2F1Matrix? mpNum? the Gauss hypergeometric function for matrix argument.}
{a? mpNum? A real number.}
{b? mpNum? A real number.}
{c? mpNum? A real number.}
{T? mpNum? A real matrix.}
\end{mpFunctionsExtract}

\begin{mpFunctionsExtract}
\mpFunctionThreeNotImplemented
{Hypergeometric1F1Matrix? mpNum? Kummer's confluent hypergeometric function  for matrix argument.}
{a? mpNum? A real number.}
{b? mpNum? A real number.}
{T? mpNum? A real matrix.}
\end{mpFunctionsExtract}

\begin{mpFunctionsExtract}
\mpFunctionTwoNotImplemented
{Hypergeometric0F1Matrix? mpNum? the confluent hypergeometric limit function for matrix argument.}
{n? mpNum? A real number.}
{T? mpNum? A real matrix.}
\end{mpFunctionsExtract}

\chapter{Examples: Moments, cumulants, and expansions}

\section{Moments and cumulants}

\begin{mpFunctionsExtract}
\mpFunctionOneNotImplemented
{CentralMomentsToRawMoments? mpNum? raw moments calculated from central moments.}
{Central? mpNum? A real vector.}
\end{mpFunctionsExtract}

\begin{mpFunctionsExtract}
\mpFunctionOneNotImplemented
{RawMomentsToCentralMoments? mpNum? central moments calculated from raw moments.}
{Raw? mpNum? A real vector.}
\end{mpFunctionsExtract}

\begin{mpFunctionsExtract}
\mpFunctionOneNotImplemented
{CentralMomentsToCumulants? mpNum? cumulants calculated from central moments.}
{Central? mpNum? A real vector.}
\end{mpFunctionsExtract}

\begin{mpFunctionsExtract}
\mpFunctionOneNotImplemented
{CumulantsToCentralMoments? mpNum? central moments calculated from cumulants.}
{Cumulants? mpNum? A real vector.}
\end{mpFunctionsExtract}

\begin{mpFunctionsExtract}
\mpFunctionOneNotImplemented
{RawMomentsToCumulants? mpNum? cumulants calculated from raw moments.}
{Raw? mpNum? A real vector.}
\end{mpFunctionsExtract}

\begin{mpFunctionsExtract}
\mpFunctionOneNotImplemented
{CumulantsToRawMoments? mpNum? raw moments calculated from cumulants.}
{Cumulants? mpNum? A real vector.}
\end{mpFunctionsExtract}

\section{The Edgeworth expansion}

\section{The Cornish-Fisher expansion}

\section{Saddlepoint approximations}

\section{Inverse Saddlepoint approximations}

\section[The Box-Davis expansion]{The Box-Davis expansion for a class of multivariate distributions}

\begin{mpFunctionsExtract}
\mpFunctionFourNotImplemented
{BoxDavisDist? mpNumList? pdf, CDF and related information for the Box-Davis-distribution}
{M? mpNum? A real number greater 0, defined as $M=-2 \rho \log(W)$}
{f? mpNum? A real number greater 0, representing the degrees of freedom for the $\chi^2$-expansion}
{omega? mpNum[]? An array of real numbers, representing the coefficients $\omega_i$}
{Output? String? A string describing the output choices}
\end{mpFunctionsExtract}

\begin{mpFunctionsExtract}
\mpFunctionFourNotImplemented
{BoxDavisDistInv? mpNumList? quantiles and related information for the Box-Davis-distribution}
{Prob? mpNum? A real number between 0 and 1.}
{f? mpNum? A real number greater 0, representing the degrees of freedom for the $\chi^2$-expansion}
{omega? mpNum[]? An array of real numbers, representing the coefficients$\omega_i$}
{Output? String? A string describing the output choices}
\end{mpFunctionsExtract}

\section{The Product of Independent Beta Variables}

\begin{mpFunctionsExtract}
\mpFunctionFiveNotImplemented
{BetaProductDist? mpNumList? pdf, CDF and related information for the central BetaProduct-distribution}
{x? mpNum? A real number}
{p? mpNum? An integer greater 0, representing the number of variates}
{a? mpNum[]? An array of real numbers greater 0, representing the numerator  degrees of freedom}
{b? mpNum[]? An array of real numbers greater 0, representing the denominator degrees of freedom}
{Output? String? A string describing the output choices}
\end{mpFunctionsExtract}

\begin{mpFunctionsExtract}
\mpFunctionFiveNotImplemented
{BetaProductDistInv? mpNumList? quantiles and related information for the the central BetaProduct-distribution}
{Prob? mpNum? A real number between 0 and 1.}
{p? mpNum? An integer greater 0, representing the number of variates}
{a? mpNum[]? An array of real numbers greater 0, representing the numerator  degrees of freedom}
{b? mpNum[]? An array of real numbers greater 0, representing the denominator degrees of freedom}
{Output? String? A string describing the output choices}
\end{mpFunctionsExtract}

\begin{mpFunctionsExtract}
\mpFunctionFourNotImplemented
{BetaProductDistInfo? mpNumList? moments and related information for the central BetaProduct-distribution}
{p? mpNum? An integer greater 0, representing the number of variates}
{a? mpNum[]? An array of real numbers greater 0, representing the numerator  degrees of freedom}
{b? mpNum[]? An array of real numbers greater 0, representing the denominator degrees of freedom}
{Output? String? A string describing the output choices}
\end{mpFunctionsExtract}

\begin{mpFunctionsExtract}
\mpFunctionSixNotImplemented
{BetaProductDistRandom? mpNumList? random numbers following a central BetaProduct-distribution}
{Size? mpNum? A positive integer up to $10^7$}
{p? mpNum? An integer greater 0, representing the number of variates}
{a? mpNum[]? An array of real numbers greater 0, representing the numerator  degrees of freedom}
{b? mpNum[]? An array of real numbers greater 0, representing the denominator degrees of freedom}
{Generator? String? A string describing the random generator}
{Output? String? A string describing the output choices}
\end{mpFunctionsExtract}

\chapter{Examples: Continuous Distribution Functions}

\section{Distribution of the Sample Correlation Coefficient}

\begin{mpFunctionsExtract}
\mpFunctionFourNotImplemented
{PearsonRhoDist? mpNumList? pdf, CDF and related information for the distribution of the sample correlation coefficient}
{x? mpNum? A real number}
{N? mpNum? A real number greater 2, representing the sample size}
{rho? mpNum? A real number greater 0, representing the correlation coefficient}
{Output? String? A string describing the output choices}
\end{mpFunctionsExtract}

\begin{mpFunctionsExtract}
\mpFunctionFourNotImplemented
{PearsonRhoDistInv? mpNumList? quantiles and related information for the distribution of the sample correlation coefficient}
{Prob? mpNum? A real number between 0 and 1.}
{N? mpNum? A real number greater 2, representing the sample size}
{rho? mpNum? A real number greater 0, representing the correlation coefficient}
{Output? String? A string describing the output choices}
\end{mpFunctionsExtract}

\begin{mpFunctionsExtract}
\mpFunctionThreeNotImplemented
{PearsonRhoDistInfo? mpNumList? moments and related information for the distribution of the sample correlation coefficient}
{N? mpNum? A real number greater 2, representing the sample size}
{rho? mpNum? A real number greater 0, representing the correlation coefficient}
{Output? String? A string describing the output choices}
\end{mpFunctionsExtract}

\begin{mpFunctionsExtract}
\mpFunctionFiveNotImplemented
{PearsonRhoDistRan? mpNumList? random numbers following the distribution of the sample correlation coefficient}
{Size? mpNum? A positive integer up to $10^7$}
{N? mpNum? A real number greater 2, representing the sample size}
{rho? mpNum? A real number greater 0, representing the correlation coefficient}
{Generator? String? A string describing the random generator}
{Output? String? A string describing the output choices}
\end{mpFunctionsExtract}

\begin{mpFunctionsExtract}
\mpFunctionFourNotImplemented
{PearsonRhoDistNoncentrality? mpNumList? confidence limits for the noncentrality parameter rhi and related information for the distribution of the sample correlation coefficient.}
{alpha? mpNum? A real number between 0 and 1, specifies the confidence level (or Type I error).}
{rho? mpNum? A real number between -1 and 1, representing the noncentrality parameter.}
{N? mpNum? A real number greater 2, representing the sample size}
{Output? String? A string describing the output choices}
\end{mpFunctionsExtract}

\begin{mpFunctionsExtract}
\mpFunctionFourNotImplemented
{PearsonRhoDistSampleSize? mpNumList? sample size estimates and related information for the distribution of the sample correlation coefficient.}
{alpha? mpNum? A real number between 0 and 1, specifies the confidence level (or Type I error).}
{beta? mpNum?  A real number between 0 and 1, specifies the Type II error (or 1 - Power).}
{ModifiedNoncentrality? mpNum? A real number greater 0, representing the modified noncentrality parameter.}
{Output? String? A string describing the output choices}
\end{mpFunctionsExtract}

\section{Distribution of the Sample Multiple Correlation Coefficient}

\begin{mpFunctionsExtract}
\mpFunctionFiveNotImplemented
{Rho2Dist? mpNumList? pdf, CDF and related information for the distribution of the squared population multiple correlation coefficient}
{x? mpNum? A real number}
{p? mpNum? An integer greater 2, representing the number of variates}
{N? mpNum? A real number greater p, representing the sample size}
{Rho2? mpNum? A real number greater or equal 0, representing the squared sample multiple correlation coefficient}
{Output? String? A string describing the output choices}
\end{mpFunctionsExtract}

\begin{mpFunctionsExtract}
\mpFunctionFiveNotImplemented
{Rho2DistInv? mpNumList? quantiles and related information for the distribution of the squared population multiple correlation coefficient}
{Prob? mpNum? A real number between 0 and 1.}
{p? mpNum? An integer greater 2, representing the number of variates}
{N? mpNum? A real number greater p, representing the sample size}
{Rho2? mpNum? A real number greater or equal 0, representing the squared sample multiple correlation coefficient}
{Output? String? A string describing the output choices}
\end{mpFunctionsExtract}

\begin{mpFunctionsExtract}
\mpFunctionFourNotImplemented
{Rho2DistInfo? mpNumList? moments and related information for the distribution of the squared population multiple correlation coefficient}
{p? mpNum? An integer greater 2, representing the number of variates}
{N? mpNum? A real number greater p, representing the sample size}
{Rho2? mpNum? A real number greater or equal 0, representing the squared sample multiple correlation coefficient}
{Output? String? A string describing the output choices}
\end{mpFunctionsExtract}

\begin{mpFunctionsExtract}
\mpFunctionSixNotImplemented
{Rho2DistRan? mpNumList? random numbers following the distribution of the squared population multiple correlation coefficient}
{Size? mpNum? A positive integer up to $10^7$}
{p? mpNum? An integer greater 2, representing the number of variates}
{N? mpNum? A real number greater p, representing the sample size}
{Rho2? mpNum? A real number greater or equal 0, representing the squared sample multiple correlation coefficient}
{Generator? String? A string describing the random generator}
{Output? String? A string describing the output choices}
\end{mpFunctionsExtract}

\begin{mpFunctionsExtract}
\mpFunctionFiveNotImplemented
{Rho2DistNoncentrality? mpNumList? confidence limits for the noncentrality parameter $\rho^2$ and related information for the distribution of the squared population multiple correlation coefficient.}
{alpha? mpNum? A real number between 0 and 1, specifies the confidence level (or Type I error).}
{Rho2? mpNum? A real number greater or equal 0, representing the squared sample multiple correlation coefficient}
{p? mpNum? An integer greater 2, representing the number of variates}
{N? mpNum? A real number greater p, representing the sample size}
{Output? String? A string describing the output choices}
\end{mpFunctionsExtract}

\begin{mpFunctionsExtract}
\mpFunctionFiveNotImplemented
{Rho2DistSampleSize? mpNumList? sample size estimates and related information for for the distribution of the squared population multiple correlation coefficient}
{alpha? mpNum? A real number between 0 and 1, specifies the confidence level (or Type I error).}
{beta? mpNum?  A real number between 0 and 1, specifies the Type II error (or 1 - Power).}
{p? mpNum? An integer greater 2, representing the number of variates}
{ModifiedNoncentrality? mpNum? A real number greater 0, representing the modified noncentrality parameter.}
{Output? String? A string describing the output choices}
\end{mpFunctionsExtract}

\section{Skew Normal Distribution}

\begin{mpFunctionsExtract}
\mpFunctionFiveNotImplemented
{SkewNormalDistBoost? mpNumList? returns pdf, CDF and related information for the skew normal distribution}
{x? mpNum? A real number.}
{a? mpNum? The location parameter.}
{b? mpNum? The scale parameter}
{c? mpNum? The shape parameter}
{Output? String? A string describing the output choices}
\end{mpFunctionsExtract}

\begin{mpFunctionsExtract}
\mpFunctionFiveNotImplemented
{SkewNormalDistInvBoost? mpNumList? returns quantiles and related information for the the skew normal distribution}
{Prob? mpNum? A real number between 0 and 1.}
{a? mpNum? The location parameter.}
{b? mpNum? The scale parameter}
{c? mpNum? The shape parameter}
{Output? String? A string describing the output choices}
\end{mpFunctionsExtract}

\begin{mpFunctionsExtract}
\mpFunctionFourNotImplemented
{SkewNormalDistInfoBoost? mpNumList? returns moments and related information for the skew normal distribution}
{a? mpNum? The location parameter.}
{b? mpNum? The scale parameter}
{c? mpNum? The shape parameter}
{Output? String? A string describing the output choices}
\end{mpFunctionsExtract}

\begin{mpFunctionsExtract}
\mpFunctionSixNotImplemented
{SkewNormalDistRanBoost? mpNumList? returns random numbers following a skew normal distribution}
{Size? mpNum? A positive integer up to $10^7$}
{a? mpNum? The location parameter.}
{b? mpNum? The scale parameter}
{c? mpNum? The shape parameter}
{Generator? String? A string describing the random generator}
{Output? String? A string describing the output choices}
\end{mpFunctionsExtract}

\begin{mpFunctionsExtract}
\mpFunctionTwo
{TOwenBoost? mpNum? Owen's T-Function}
{h? mpNum? A real number.}
{a? mpNum? A real number.}
\end{mpFunctionsExtract}

\section{Multivariate Normal Distribution}

\section{Multivariate t-Distribution}

\section{Noncentral Chi-Square Distribution}

\begin{mpFunctionsExtract}
\mpFunctionFourNotImplemented
{NoncentralCDistEx? mpNumList? pdf, CDF and related information for the noncentral $\chi^2$-distribution}
{x? mpNum? A real number}
{n? mpNum? A real number greater 0, representing the degrees of freedom}
{lambda? mpNum? A real number greater 0, representing the noncentrality parameter}
{Output? String? A string describing the output choices}
\end{mpFunctionsExtract}

\begin{mpFunctionsExtract}
\mpFunctionFourNotImplemented
{NoncentralCDistInvEx? mpNumList? quantiles and related information for the noncentral $\chi^2$-distribution}
{Prob? mpNum? A real number between 0 and 1.}
{n? mpNum? A real number greater 0, representing the degrees of freedom}
{lambda? mpNum? A real number greater 0, representing the noncentrality parameter}
{Output? String? A string describing the output choices}
\end{mpFunctionsExtract}

\begin{mpFunctionsExtract}
\mpFunctionThreeNotImplemented
{NoncentralCDistInfoEx? mpNumList? moments and related information for the noncentral $\chi^2$-distribution}
{n? mpNum? A real number greater 0, representing the degrees of freedom}
{lambda? mpNum? A real number greater 0, representing the noncentrality parameter}
{Output? String? A string describing the output choices}
\end{mpFunctionsExtract}

\begin{mpFunctionsExtract}
\mpFunctionFourNotImplemented
{NoncentralCDistNoncentralityEx? mpNumList? confidence limits for the noncentrality parameter lambda and related information for the noncentral $\chi^2$-distribution.}
{alpha? mpNum? A real number between 0 and 1, specifies the confidence level (or Type I error).}
{lambda? mpNum? A real number greater 0, representing the noncentrality parameter.}
{n? mpNum? A real number greater 0, representing the degrees of freedom.}
{Output? String? A string describing the output choices}
\end{mpFunctionsExtract}

\begin{mpFunctionsExtract}
\mpFunctionFiveNotImplemented
{NoncentralCDistSampleSizeEx? mpNumList? sample size estimates and related information for the noncentral $\chi^2$-distribution.}
{alpha? mpNum? A real number between 0 and 1, specifies the confidence level (or Type I error).}
{beta? mpNum?  A real number between 0 and 1, specifies the Type II error (or 1 - Power).}
{n? mpNum? A real number greater 0, representing the denominator degrees of freedom.}
{ModifiedNoncentrality? mpNum? A real number greater 0, representing the modified noncentrality parameter.}
{Output? String? A string describing the output choices}
\end{mpFunctionsExtract}

\begin{mpFunctionsExtract}
\mpFunctionFiveNotImplemented
{NoncentralCDistRanEx? mpNumList? random numbers following a noncentral $\chi^2$-distribution}
{Size? mpNum? A positive integer up to $10^7$}
{n? mpNum? A real number greater 0, representing the degrees of freedom}
{lambda? mpNum? A real number greater 0, representing the noncentrality parameter}
{Generator? String? A string describing the random generator}
{Output? String? A string describing the output choices}
\end{mpFunctionsExtract}

\section{Noncentral Beta-Distribution}

\begin{mpFunctionsExtract}
\mpFunctionFiveNotImplemented
{NoncentralBetaDistBoost? mpNumList? pdf, CDF and related information for the central Beta-distribution}
{x? mpNum? A real number}
{m? mpNum? A real number greater 0, representing the numerator  degrees of freedom}
{n? mpNum? A real number greater 0, representing the denominator degrees of freedom}
{lambda? mpNum? A real number greater 0, representing the noncentrality parameter}
{Output? String? A string describing the output choices}
\end{mpFunctionsExtract}

\begin{mpFunctionsExtract}
\mpFunctionFiveNotImplemented
{NoncentralBetaDistInvBoost? mpNumList? quantiles and related information for the the noncentral Beta-distribution}
{Prob? mpNum? A real number between 0 and 1.}
{m? mpNum? A real number greater 0, representing the numerator  degrees of freedom}
{n? mpNum? A real number greater 0, representing the denominator degrees of freedom}
{lambda? mpNum? A real number greater 0, representing the noncentrality parameter}
{Output? String? A string describing the output choices}
\end{mpFunctionsExtract}

\begin{mpFunctionsExtract}
\mpFunctionFourNotImplemented
{NoncentralBetaDistInfoBoost? mpNumList? moments and related information for the noncentral Beta-distribution}
{m? mpNum? A real number greater 0, representing the numerator  degrees of freedom}
{n? mpNum? A real number greater 0, representing the denominator degrees of freedom}
{lambda? mpNum? A real number greater 0, representing the noncentrality parameter}
{Output? String? A string describing the output choices}
\end{mpFunctionsExtract}

\begin{mpFunctionsExtract}
\mpFunctionSixNotImplemented
{NoncentralBetaDistRanBoost? mpNumList? random numbers following a noncentral Beta-distribution}
{Size? mpNum? A positive integer up to $10^7$}
{m? mpNum? A real number greater 0, representing the numerator  degrees of freedom}
{n? mpNum? A real number greater 0, representing the denominator degrees of freedom}
{lambda? mpNum? A real number greater 0, representing the noncentrality parameter}
{Generator? String? A string describing the random generator}
{Output? String? A string describing the output choices}
\end{mpFunctionsExtract}

\section{Noncentral Student's t-Distribution}

\begin{mpFunctionsExtract}
\mpFunctionFourNotImplemented
{NoncentralTDistBoost? mpNumList? pdf, CDF and related information for the noncentral $t$-distribution}
{x? mpNum? A real number}
{n? mpNum? A real number greater 0, representing the degrees of freedom}
{delta? mpNum? A real number greater 0, representing the noncentrality parameter}
{Output? String? A string describing the output choices}
\end{mpFunctionsExtract}

\begin{mpFunctionsExtract}
\mpFunctionFourNotImplemented
{NoncentralTDistInvBoost? mpNumList? quantiles and related information for the  noncentral $t$-distribution}
{Prob? mpNum? A real number between 0 and 1.}
{n? mpNum? A real number greater 0, representing the degrees of freedom}
{delta? mpNum? A real number greater 0, representing the noncentrality parameter}
{Output? String? A string describing the output choices}
\end{mpFunctionsExtract}

\begin{mpFunctionsExtract}
\mpFunctionThreeNotImplemented
{NoncentralTDistInfoBoost? mpNumList? moments and related information for the noncentral $t$-distribution}
{n? mpNum? A real number greater 0, representing the degrees of freedom}
{delta? mpNum? A real number greater 0, representing the noncentrality parameter}
{Output? String? A string describing the output choices}
\end{mpFunctionsExtract}

\begin{mpFunctionsExtract}
\mpFunctionFiveNotImplemented
{NoncentralTDistRanBoost? mpNumList? random numbers following a noncentral $t$-distribution}
{Size? mpNum? A positive integer up to $10^7$}
{n? mpNum? A real number greater 0, representing the degrees of freedom}
{delta? mpNum? A real number greater 0, representing the noncentrality parameter}
{Generator? String? A string describing the random generator}
{Output? String? A string describing the output choices}
\end{mpFunctionsExtract}

\begin{mpFunctionsExtract}
\mpFunctionFiveNotImplemented
{NoncentralTDistNoncentrality? mpNumList? confidence limits for the doubly noncentrality parameter delta and related information for the noncentral $t$-distribution.}
{alpha? mpNum? A real number between 0 and 1, specifies the confidence level (or Type I error).}
{delta? mpNum? A real number greater 0, representing the numerator noncentrality parameter}
{theta? mpNum? A real number greater 0, representing the denominator noncentrality parameter}
{n? mpNum? A real number greater 0, representing the degrees of freedom.}
{Output? String? A string describing the output choices}
\end{mpFunctionsExtract}

\begin{mpFunctionsExtract}
\mpFunctionFiveNotImplemented
{NoncentralTDistSampleSize? mpNumList? sample size estimates and related information for the doubly noncentral $t$-distribution.}
{alpha? mpNum? A real number between 0 and 1, specifies the confidence level (or Type I error).}
{beta? mpNum?  A real number between 0 and 1, specifies the Type II error (or 1 - Power).}
{ModifiedNoncentrality1? mpNum? A real number greater 0, representing the modified numerator noncentrality parameter.}
{ModifiedNoncentrality2? mpNum? A real number greater 0, representing the modified denominator noncentrality parameter.}
{Output? String? A string describing the output choices}
\end{mpFunctionsExtract}

\section{Doubly Noncentral Student's t-Distribution}

\begin{mpFunctionsExtract}
\mpFunctionFiveNotImplemented
{DoublyNoncentralTDist? mpNumList? pdf, CDF and related information for the doubly noncentral $t$-distribution}
{x? mpNum? A real number}
{n? mpNum? A real number greater 0, representing the degrees of freedom}
{delta? mpNum? A real number greater 0, representing the numerator noncentrality parameter}
{theta? mpNum? A real number greater 0, representing the denominator noncentrality parameter}
{Output? String? A string describing the output choices}
\end{mpFunctionsExtract}

\begin{mpFunctionsExtract}
\mpFunctionFiveNotImplemented
{DoublyNoncentralTDistInv? mpNumList? quantiles and related information for the  doubly noncentral $t$-distribution}
{Prob? mpNum? A real number between 0 and 1.}
{n? mpNum? A real number greater 0, representing the degrees of freedom}
{delta? mpNum? A real number greater 0, representing the numerator noncentrality parameter}
{theta? mpNum? A real number greater 0, representing the denominator noncentrality parameter}
{Output? String? A string describing the output choices}
\end{mpFunctionsExtract}

\begin{mpFunctionsExtract}
\mpFunctionFourNotImplemented
{DoublyNoncentralTDistInfo? mpNumList? moments and related information for the doubly noncentral $t$-distribution}
{n? mpNum? A real number greater 0, representing the degrees of freedom}
{delta? mpNum? A real number greater 0, representing the numerator noncentrality parameter}
{theta? mpNum? A real number greater 0, representing the denominator noncentrality parameter}
{Output? String? A string describing the output choices}
\end{mpFunctionsExtract}

\begin{mpFunctionsExtract}
\mpFunctionSixNotImplemented
{DoublyNoncentralTDistRan? mpNumList? random numbers following a doubly noncentral $t$-distribution}
{Size? mpNum? A positive integer up to $10^7$}
{n? mpNum? A real number greater 0, representing the degrees of freedom}
{delta? mpNum? A real number greater 0, representing the numerator noncentrality parameter}
{theta? mpNum? A real number greater 0, representing the denominator noncentrality parameter}
{Generator? String? A string describing the random generator}
{Output? String? A string describing the output choices}
\end{mpFunctionsExtract}

\begin{mpFunctionsExtract}
\mpFunctionFiveNotImplemented
{DoublyNoncentralTDistNoncentrality? mpNumList? confidence limits for the doubly noncentrality parameter delta and related information for the noncentral $t$-distribution.}
{alpha? mpNum? A real number between 0 and 1, specifies the confidence level (or Type I error).}
{delta? mpNum? A real number greater 0, representing the numerator noncentrality parameter}
{theta? mpNum? A real number greater 0, representing the denominator noncentrality parameter}
{n? mpNum? A real number greater 0, representing the degrees of freedom.}
{Output? String? A string describing the output choices}
\end{mpFunctionsExtract}

\begin{mpFunctionsExtract}
\mpFunctionFiveNotImplemented
{DoublyNoncentralTDistSampleSize? mpNumList? sample size estimates and related information for the doubly noncentral $t$-distribution.}
{alpha? mpNum? A real number between 0 and 1, specifies the confidence level (or Type I error).}
{beta? mpNum?  A real number between 0 and 1, specifies the Type II error (or 1 - Power).}
{ModifiedNoncentrality1? mpNum? A real number greater 0, representing the modified numerator noncentrality parameter.}
{ModifiedNoncentrality2? mpNum? A real number greater 0, representing the modified denominator noncentrality parameter.}
{Output? String? A string describing the output choices}
\end{mpFunctionsExtract}

\section{NonCentral F-Distribution}

\begin{mpFunctionsExtract}
\mpFunctionFiveNotImplemented
{NoncentralFDistBoost? mpNumList? pdf, CDF and related information for the noncentral $F$-distribution}
{x? mpNum? A real number}
{m? mpNum? A real number greater 0, representing the numerator  degrees of freedom}
{n? mpNum? A real number greater 0, representing the denominator degrees of freedom}
{lambda? mpNum? A real number greater 0, representing the noncentrality parameter}
{Output? String? A string describing the output choices}
\end{mpFunctionsExtract}

\begin{mpFunctionsExtract}
\mpFunctionFiveNotImplemented
{NoncentralFDistInvBoost? mpNumList? quantiles and related information for the the noncentral $F$-distribution}
{Prob? mpNum? A real number between 0 and 1.}
{m? mpNum? A real number greater 0, representing the numerator  degrees of freedom}
{n? mpNum? A real number greater 0, representing the denominator degrees of freedom}
{lambda? mpNum? A real number greater 0, representing the noncentrality parameter}
{Output? String? A string describing the output choices}
\end{mpFunctionsExtract}

\begin{mpFunctionsExtract}
\mpFunctionFourNotImplemented
{NoncentralFDistInfoBoost? mpNumList?  moments and related information for the noncentral $F$-distribution}
{m? mpNum? A real number greater 0, representing the numerator  degrees of freedom}
{n? mpNum? A real number greater 0, representing the denominator degrees of freedom}
{lambda? mpNum? A real number greater 0, representing the noncentrality parameter}
{Output? String? A string describing the output choices}
\end{mpFunctionsExtract}

\begin{mpFunctionsExtract}
\mpFunctionSixNotImplemented
{NoncentralFDistRanBoost? mpNumList? random numbers following a noncentral $F$-distribution}
{Size? mpNum? A positive integer up to $10^7$}
{m? mpNum? A real number greater 0, representing the numerator  degrees of freedom}
{n? mpNum? A real number greater 0, representing the denominator degrees of freedom}
{lambda? mpNum? A real number greater 0, representing the noncentrality parameter}
{Generator? String? A string describing the random generator}
{Output? String? A string describing the output choices}
\end{mpFunctionsExtract}

\begin{mpFunctionsExtract}
\mpFunctionSixNotImplemented
{NoncentralFDistNoncentralityEx? mpNumList? confidence limits for the noncentrality parameter lambda and related information for the noncentral $F$-distribution.}
{alpha? mpNum? A real number between 0 and 1, specifies the confidence level (or Type I error).}
{lambda1? mpNum? A real number greater 0, representing the numerator noncentrality parameter}
{lambda2? mpNum? A real number greater 0, representing the denominator noncentrality parameter}
{m? mpNum? A real number greater 0, representing the numerator  degrees of freedom.}
{n? mpNum? A real number greater 0, representing the denominator degrees of freedom.}
{Output? String? A string describing the output choices}
\end{mpFunctionsExtract}

\begin{mpFunctionsExtract}
\mpFunctionSixNotImplemented
{NoncentralFDistSampleSizeEx? mpNumList? sample size estimates and related information for the noncentral $F$-distribution.}
{alpha? mpNum? A real number between 0 and 1, specifies the confidence level (or Type I error).}
{beta? mpNum?  A real number between 0 and 1, specifies the Type II error (or 1 - Power).}
{m? mpNum? A real number greater 0, representing the numerator  degrees of freedom.}
{ModifiedNoncentrality1? mpNum? A real number greater 0, representing the modified numerator noncentrality parameter.}
{ModifiedNoncentrality1? mpNum? A real number greater 0, representing the modified denominator noncentrality parameter.}
{Output? String? A string describing the output choices}
\end{mpFunctionsExtract}

\section{Doubly NonCentral F-Distribution}

\begin{mpFunctionsExtract}
\mpFunctionSixNotImplemented
{DoublyNoncentralFDistEx? mpNumList? pdf, CDF and related information for the noncentral $F$-distribution}
{x? mpNum? A real number}
{m? mpNum? A real number greater 0, representing the numerator  degrees of freedom}
{n? mpNum? A real number greater 0, representing the denominator degrees of freedom}
{lambda1? mpNum? A real number greater 0, representing the numerator noncentrality parameter}
{lambda2? mpNum? A real number greater 0, representing the denominator noncentrality parameter}
{Output? String? A string describing the output choices}
\end{mpFunctionsExtract}

\begin{mpFunctionsExtract}
\mpFunctionSixNotImplemented
{DoublyNoncentralFDistInvEx? mpNumList? quantiles and related information for the the noncentral $F$-distribution}
{Prob? mpNum? A real number between 0 and 1.}
{m? mpNum? A real number greater 0, representing the numerator  degrees of freedom}
{n? mpNum? A real number greater 0, representing the denominator degrees of freedom}
{lambda1? mpNum? A real number greater 0, representing the numerator noncentrality parameter}
{lambda2? mpNum? A real number greater 0, representing the denominator noncentrality parameter}
{Output? String? A string describing the output choices}
\end{mpFunctionsExtract}

\begin{mpFunctionsExtract}
\mpFunctionFiveNotImplemented
{DoublyNoncentralFDistInfoEx? mpNumList? moments and related information for the noncentral $F$-distribution}
{m? mpNum? A real number greater 0, representing the numerator  degrees of freedom}
{n? mpNum? A real number greater 0, representing the denominator degrees of freedom}
{lambda1? mpNum? A real number greater 0, representing the numerator noncentrality parameter}
{lambda2? mpNum? A real number greater 0, representing the denominator noncentrality parameter}
{Output? String? A string describing the output choices}
\end{mpFunctionsExtract}

\begin{mpFunctionsExtract}
\mpFunctionSixNotImplemented
{DoublyNoncentralFDistRanEx? mpNumList? random numbers following a noncentral $F$-distribution}
{Size? mpNum? A positive integer up to $10^7$}
{m? mpNum? A real number greater 0, representing the numerator  degrees of freedom}
{n? mpNum? A real number greater 0, representing the denominator degrees of freedom}
{lambda1? mpNum? A real number greater 0, representing the numerator noncentrality parameter}
{lambda2? mpNum? A real number greater 0, representing the denominator noncentrality parameter}
{Generator? String? A string describing the random generator}
{Output? String? A string describing the output choices}
\end{mpFunctionsExtract}

\begin{mpFunctionsExtract}
\mpFunctionSixNotImplemented
{DoublyNoncentralFDistNoncentralityEx? mpNumList? confidence limits for the noncentrality parameter lambda and related information for the noncentral $F$-distribution.}
{alpha? mpNum? A real number between 0 and 1, specifies the confidence level (or Type I error).}
{lambda1? mpNum? A real number greater 0, representing the numerator noncentrality parameter}
{lambda2? mpNum? A real number greater 0, representing the denominator noncentrality parameter}
{m? mpNum? A real number greater 0, representing the numerator  degrees of freedom.}
{n? mpNum? A real number greater 0, representing the denominator degrees of freedom.}
{Output? String? A string describing the output choices}
\end{mpFunctionsExtract}

\begin{mpFunctionsExtract}
\mpFunctionSixNotImplemented
{DoublyNoncentralFDistSampleSizeEx? mpNumList? sample size estimates and related information for the noncentral $F$-distribution.}
{alpha? mpNum? A real number between 0 and 1, specifies the confidence level (or Type I error).}
{beta? mpNum?  A real number between 0 and 1, specifies the Type II error (or 1 - Power).}
{m? mpNum? A real number greater 0, representing the numerator  degrees of freedom.}
{ModifiedNoncentrality1? mpNum? A real number greater 0, representing the modified numerator noncentrality parameter.}
{ModifiedNoncentrality1? mpNum? A real number greater 0, representing the modified denominator noncentrality parameter.}
{Output? String? A string describing the output choices}
\end{mpFunctionsExtract}

\section{Noncentral Distribution of Roy's Largest Root}

\section{Noncentral Distribution of Wilks' Lambda}

\begin{mpFunctionsExtract}
\mpFunctionSixNotImplemented
{NoncentralWilksLambdaDist? mpNumList? pdf, CDF and related information for the noncentral WilksLambda-distribution}
{x? mpNum? A real number}
{p? mpNum? An integer greater 0, representing the number of variates}
{m? mpNum? A real number greater 0, representing the numerator  degrees of freedom}
{n? mpNum? A real number greater 0, representing the denominator degrees of freedom}
{Omega? mpNum? An array of real numbers representing the noncentrality parameter}
{Output? String? A string describing the output choices}
\end{mpFunctionsExtract}

\begin{mpFunctionsExtract}
\mpFunctionSixNotImplemented
{NoncentralWilksLambdaDistInv? mpNumList? quantiles and related information for the the noncentral WilksLambda-distribution}
{Prob? mpNum? A real number between 0 and 1.}
{p? mpNum? An integer greater 0, representing the number of variates}
{m? mpNum? A real number greater 0, representing the numerator  degrees of freedom}
{n? mpNum? A real number greater 0, representing the denominator degrees of freedom}
{Omega? mpNum? An array of real numbers representing the noncentrality parameter}
{Output? String? A string describing the output choices}
\end{mpFunctionsExtract}

\begin{mpFunctionsExtract}
\mpFunctionFiveNotImplemented
{NoncentralWilksLambdaDistInfo? mpNumList? moments and related information for the noncentral WilksLambda-distribution}
{p? mpNum? An integer greater 0, representing the number of variates}
{m? mpNum? A real number greater 0, representing the numerator  degrees of freedom}
{n? mpNum? A real number greater 0, representing the denominator degrees of freedom}
{Omega? mpNum? An array of real numbers representing the noncentrality parameter}
{Output? String? A string describing the output choices}
\end{mpFunctionsExtract}

\begin{mpFunctionsExtract}
\mpFunctionSevenNotImplemented
{NoncentralWilksLambdaDistRan? mpNumList? random numbers following a noncentral WilksLambda-distribution}
{Size? mpNum? A positive integer up to $10^7$}
{p? mpNum? An integer greater 0, representing the number of variates}
{m? mpNum? A real number greater 0, representing the numerator  degrees of freedom}
{n? mpNum? A real number greater 0, representing the denominator degrees of freedom}
{Generator? String? A string describing the random generator}
{Omega? mpNum? An array of real numbers representing the noncentrality parameter}
{Output? String? A string describing the output choices}
\end{mpFunctionsExtract}

\section{Noncentral Distribution of Hotelling's T2}

\section{Noncentral Distribution of Pillai's V}

\section{Noncentral Distribution of Bartlett's M (2 samples)}

\begin{mpFunctionsExtract}
\mpFunctionSixNotImplemented
{NoncentralBartlettsMDist? mpNumList? pdf, CDF and related information for the noncentral WilksLambda(CORR)-distribution}
{x? mpNum? A real number}
{p? mpNum? An integer greater 0, representing the number of variates}
{m? mpNum? A real number greater 0, representing the numerator  degrees of freedom}
{n? mpNum? A real number greater 0, representing the denominator degrees of freedom}
{Omega? mpNum? An array of real numbers representing the noncentrality parameter}
{Output? String? A string describing the output choices}
\end{mpFunctionsExtract}

\begin{mpFunctionsExtract}
\mpFunctionSixNotImplemented
{NoncentralBartlettsMDistInv? mpNumList? quantiles and related information for the the noncentral WilksLambda(CORR)-distribution}
{Prob? mpNum? A real number between 0 and 1.}
{p? mpNum? An integer greater 0, representing the number of variates}
{m? mpNum? A real number greater 0, representing the numerator  degrees of freedom}
{n? mpNum? A real number greater 0, representing the denominator degrees of freedom}
{Omega? mpNum? An array of real numbers representing the noncentrality parameter}
{Output? String? A string describing the output choices}
\end{mpFunctionsExtract}

\begin{mpFunctionsExtract}
\mpFunctionFiveNotImplemented
{NoncentralBartlettsMInfo? mpNumList? moments and related information for the noncentral WilksLambda(CORR)-distribution}
{p? mpNum? An integer greater 0, representing the number of variates}
{m? mpNum? A real number greater 0, representing the numerator  degrees of freedom}
{n? mpNum? A real number greater 0, representing the denominator degrees of freedom}
{Omega? mpNum? An array of real numbers representing the noncentrality parameter}
{Output? String? A string describing the output choices}
\end{mpFunctionsExtract}

\begin{mpFunctionsExtract}
\mpFunctionSevenNotImplemented
{NoncentralBartlettsMDistRan? mpNumList? random numbers following a noncentral WilksLambda(CORR)-distribution}
{Size? mpNum? A positive integer up to $10^7$}
{p? mpNum? An integer greater 0, representing the number of variates}
{m? mpNum? A real number greater 0, representing the numerator  degrees of freedom}
{n? mpNum? A real number greater 0, representing the denominator degrees of freedom}
{Generator? String? A string describing the random generator}
{Omega? mpNum? An array of real numbers representing the noncentrality parameter}
{Output? String? A string describing the output choices}
\end{mpFunctionsExtract}

\chapter{Examples: Discrete Distribution Functions}

\section{Noncentral Distribution of Mann-Whitney's U (with Stratification)}

\section{Noncentral Distribution of Wilcoxon's Signed Rank Test (with Stratification)}

\section{Noncentral Distribution of Kendall's Tau}

\section{Noncentral Distribution of Jonckheere-Terpsta's S}

\section{Noncentral Distribution of Spearman's Rho}

\section{Noncentral Distribution of Page's L}

\section{Noncentral Distribution of Kruskal-Wallis' H}

\section{Noncentral Distributions of Friedman's S}

\chapter{Examples: Statistical Procedures}

\section{Introduction to Inferential Statistics Functions}

\section{Tests for the mean from 1 sample (Student's t-test)}

\begin{mpFunctionsExtract}
\mpFunctionFiveNotImplemented
{StudentTTest1? mpNumList? p-values, confidence intervals and related information for Student's t-test}
{Type1Error? mpNum? A real number greater then 0 and less than 1.}
{N? mpNum[]? An array of integers greater than 1, representing the samples sizes}
{Mean? mpNum[]? An array of reals, representing the means}
{StDev? mpNum[]? An array of positive reals, representing the standard deviations}
{Output? String? A string describing the output choices}
\end{mpFunctionsExtract}

\begin{mpFunctionsExtract}
\mpFunctionFiveNotImplemented
{StudentTTestPower1? mpNumList? returns power estimations and related information for Student's t-test}
{Type1Error? mpNum? An real number greater then 0 and less than 1.}
{N? mpNum[]? An array of integers greater than 1, representing the samples sizes}
{Mean? mpNum[]? An array of reals, representing the means}
{StDev? mpNum[]? An array of positive reals, representing the standard deviations}
{Output? String? A string describing the output choices}
\end{mpFunctionsExtract}

\begin{mpFunctionsExtract}
\mpFunctionFiveNotImplemented
{StudentTTestSampleSize1? mpNumList? returns sample size estimations and related information for Student's t-test}
{Type1Error? mpNum? An real number greater then 0 and less than 1.}
{Type2Error? mpNum? An real number greater then 0 and less than 1.}
{Mean? mpNum[]? An array of reals, representing the means}
{StDev? mpNum[]? An array of positive reals, representing the standard deviations}
{Output? String? A string describing the output choices}
\end{mpFunctionsExtract}

\section{Tests for means from 2 independent samples (Student's t-test)}

\begin{mpFunctionsExtract}
\mpFunctionFiveNotImplemented
{StudentTTest2i? mpNumList? p-values, confidence intervals and related information for Student's t-test}
{Type1Error? mpNum? A real number greater then 0 and less than 1.}
{N? mpNum[]? An array of integers greater than 1, representing the samples sizes}
{Mean? mpNum[]? An array of reals, representing the means}
{StDev? mpNum[]? An array of positive reals, representing the standard deviations}
{Output? String? A string describing the output choices}
\end{mpFunctionsExtract}

\begin{mpFunctionsExtract}
\mpFunctionFiveNotImplemented
{StudentTTestPower2i? mpNumList? returns power estimations and related information for Student's t-test}
{Type1Error? mpNum? An real number greater then 0 and less than 1.}
{N? mpNum[]? An array of integers greater than 1, representing the samples sizes}
{Mean? mpNum[]? An array of reals, representing the means}
{StDev? mpNum[]? An array of positive reals, representing the standard deviations}
{Output? String? A string describing the output choices}
\end{mpFunctionsExtract}

\begin{mpFunctionsExtract}
\mpFunctionFiveNotImplemented
{StudentTTestSampleSize2i? mpNumList? returns sample size estimations and related information for Student's t-test}
{Type1Error? mpNum? An real number greater then 0 and less than 1.}
{Type2Error? mpNum? An real number greater then 0 and less than 1.}
{Mean? mpNum[]? An array of reals, representing the means}
{StDev? mpNum[]? An array of positive reals, representing the standard deviations}
{Output? String? A string describing the output choices}
\end{mpFunctionsExtract}

\chapter{Interfaces to the C family of languages}

\section{Windows, GNU/Linux, Mac OSX: GNU Compiler Collection}

\section{Windows: MSVC}

\section{Windows, GNU/Linux, Mac OSX: C}

\section{Windows, GNU/Linux, Mac OSX: C++}

\section{Mac OSX: Objective C}

\section{Mac OSX: Objective C++}

\chapter{Languages with CLR Support}

\section{Visual Basic .NET}

\section{\texorpdfstring {$\text {C\# 4.0 } $}{CSharp}}

\section{JScript 10.0}

\section{C++ 10.0, Visual Studio}

\section{\texorpdfstring {$\text {F\# 3.0 } $}{FSharp}}

\section{IronPython 2.7}

\section{MatLab (.NET interface)}

\chapter{Building the library}

\section{Building the Library, Part 1}

\section{Building the Library, Part 2}

\section{Building the documentation}

\section{Additional libraries}

\section{Working Notes}

\section{Where to find VB Code}

\section{How to run Permutation Code}

\chapter{Roadmap}

\section{CPython}

\section{R (Statistical System)}

\chapter{Acknowledgements}

\section{Contributors to libraries used in the numerical routines}

\chapter{Licenses}

\section{GNU Licenses}

\section{Other Licenses}

\chapter{\tocbibname }

\end{document}
