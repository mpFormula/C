\chapter{Preface}
\label{Preface} 

Arbitrary-precision arithmetic, also called bignum arithmetic, multiple precision arithmetic, or sometimes infinite-precision arithmetic, indicates that calculations are performed on numbers which digits of precision are limited only by the available memory of the host system. 

\vpara
Arbitrary precision is used in applications where the speed of arithmetic is not a limiting factor, or where precise results with very large numbers are required. It should not be confused with the symbolic computation provided by many computer algebra systems, which represent numbers by expressions such as , and can thus represent any computable number with infinite precision.

\vpara
The mpFormulaC Library and Toolbox are based on a number of well-established libraries, which implement multiprecision arithmetic.

\vpara
This manual is divided in various parts, which reflect different levels of confidence regarding the accuracy of the results.

\vpara
Part II: Arbitrary Precision with Guaranteed Error Bounds.

Functions in this part come optionally with a guaranteed error bound, which can (in principal) be made arbitrarily small.  Based mostly on GMP, MPFR, MPFI, MPC, FLINT, ARB and libmpdec.

\vpara
Part III: Arbitrary Precision with Error Tracking.

Functions in this part include functions which do not guarantee an error bound, but provide error tracking. This includes a comprehensive selections of complex and real linear algebra functions, based on Eigen.

\vpara
Part IV: Additional Functions. Work in progress

\vpara
The use of these function various environments is described in some detail in the appendices:

\vpara
Appendix A describes the interfaces to a number of popular programming languages and applications with built-in scripting languages.

\vpara
If you want to re-build or change the library and/or toolbox, have a look at appendices C and D.

\vpara
Finally, the mpFormula Library and Toolbox would not exist without the many authors and contributors of the underlying libraries. They are acknowledged in appendix E.


\vspace{0.6cm}
Dietrich Hadler

Helge Hadler

Thomas Hadler

