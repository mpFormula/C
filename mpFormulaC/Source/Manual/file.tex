%% 
%% This is file, `file.tex',
%% generated with the extract package.
%% 
%% Generated on :  2015/05/02,13:42
%% From source  :  mpFormulaC.tex
%% Using options:  active,generate=file,extract-cmd={chapter,section},extract-env={mpFunctionsExtract}
%% 
\documentclass[12pt,a4paper,openany]{book}

\begin{document}

\chapter{Preface}

\chapter{Introduction}

\section{Overview: Features and Setup}

\section{License}

\section{No Warranty}

\chapter{GMP and related libraries: an overview}

\section{Integer Types and Fractions}

\section{FloatingPoint Types}

\section{Arithmetic Operators}

\section{Comparison Operators and Sorting}

\section{Vectors, Matrices and Tables}

\chapter{MPZ}

\section{Arithmetic Operators }

\begin{mpFunctionsExtract}
\mpFunctionOne
{intNeg? mpNum?  $-n$}
{n? mpNum? An Integer.}
\end{mpFunctionsExtract}

\begin{mpFunctionsExtract}
\mpFunctionTwo
{intAdd? mpNum?  $n_1 + n_2.$.}
{n1? mpNum? An Integer.}
{n2? mpNum? An Integer.}
\end{mpFunctionsExtract}

\begin{mpFunctionsExtract}
\mpFunctionTwo
{intSub? mpNum?  $n_1 - n_2.$.}
{n1? mpNum? An Integer.}
{n2? mpNum? An Integer.}
\end{mpFunctionsExtract}

\begin{mpFunctionsExtract}
\mpFunctionTwo
{intMul? mpNum?  $n_1  \times n_2$.}
{n1? mpNum? An Integer.}
{n2? mpNum? An Integer.}
\end{mpFunctionsExtract}

\begin{mpFunctionsExtract}
\mpFunctionThree
{intFma? mpNum? $(n_1 \times n_2) + n_3$.}
{n1? mpNum? An Integer.}
{n2? mpNum? An Integer.}
{n3? mpNum? An Integer.}
\end{mpFunctionsExtract}

\begin{mpFunctionsExtract}
\mpFunctionThree
{intFms? mpNum? $(n_1 \times n_2) - n_3$.}
{n1? mpNum? An Integer.}
{n2? mpNum? An Integer.}
{n3? mpNum? An Integer.}
\end{mpFunctionsExtract}

\begin{mpFunctionsExtract}
\mpFunctionTwo
{intLSH? mpNum? the product of $n$ and $2^k$}
{n? mpNum? An Integer.}
{k? mpNum? An Integer.}
\end{mpFunctionsExtract}

\begin{mpFunctionsExtract}
\mpFunctionTwo
{intRSH? mpNum? the quotient of $n$ and $2^k$}
{n? mpNum? An Integer.}
{k? mpNum? An Integer.}
\end{mpFunctionsExtract}

\begin{mpFunctionsExtract}
\mpFunctionTwo
{intDivExact? mpNum? $n/d$}
{n? mpNum? An Integer.}
{d? mpNum? An Integer.}
\end{mpFunctionsExtract}

\begin{mpFunctionsExtract}
\mpFunctionTwo
{intMod? mpNum? $n$ mod $d$.}
{n? mpNum? An Integer.}
{d? mpNum? An Integer.}
\end{mpFunctionsExtract}

\section{Divisions, forming quotients and/or remainder}

\begin{mpFunctionsExtract}
\mpFunctionTwo
{intCDivQ? mpNum? the quotient of $n$ and $d$, rounded up towards $+\infty$.}
{n? mpNum? An Integer.}
{d? mpNum? An Integer.}
\end{mpFunctionsExtract}

\begin{mpFunctionsExtract}
\mpFunctionTwo
{intCDivQ2exp? mpNum? the quotient of $n$ and $2^b$, rounded up towards $+\infty$.}
{n? mpNum? An Integer.}
{b? mpNum? An Integer.}
\end{mpFunctionsExtract}

\begin{mpFunctionsExtract}
\mpFunctionTwo
{intCDivR? mpNum? the remainder, once the quotient of $n$ and $d$, rounded up towards $+\infty$, has been obtained.}
{n? mpNum? An Integer.}
{d? mpNum? An Integer.}
\end{mpFunctionsExtract}

\begin{mpFunctionsExtract}
\mpFunctionTwo
{intCDivR2exp? mpNum? the remainder, once the quotient of $n$ and $2^b$, rounded up towards $+\infty$, has been obtained.}
{n? mpNum? An Integer.}
{b? mpNum? An Integer.}
\end{mpFunctionsExtract}

\begin{mpFunctionsExtract}
\mpFunctionTwo
{intCDivQR? mpNumList[2]? the quotient of $n$ and $d$, rounded up towards $+\infty$, and the remainder.}
{n? mpNum? An Integer.}
{d? mpNum? An Integer.}
\end{mpFunctionsExtract}

\begin{mpFunctionsExtract}
\mpFunctionTwo
{intFDivQ? mpNum? the quotient of $n$ and $d$, rounded down towards $-\infty$.}
{n? mpNum? An Integer.}
{d? mpNum? An Integer.}
\end{mpFunctionsExtract}

\begin{mpFunctionsExtract}
\mpFunctionTwo
{intFDivQ2exp? mpNum? the quotient of $n$ and $2^b$, rounded down towards $-\infty$.}
{n? mpNum? An Integer.}
{b? mpNum? An Integer.}
\end{mpFunctionsExtract}

\begin{mpFunctionsExtract}
\mpFunctionTwo
{intFDivR? mpNum? the remainder, once the quotient of $n$ and $d$, rounded down towards $-\infty$, has been obtained.}
{n? mpNum? An Integer.}
{d? mpNum? An Integer.}
\end{mpFunctionsExtract}

\begin{mpFunctionsExtract}
\mpFunctionTwo
{intFDivR2exp? mpNum? the remainder, once the quotient of $n$ and $2^b$, rounded down towards $-\infty$, has been obtained.}
{n? mpNum? An Integer.}
{b? mpNum? An Integer.}
\end{mpFunctionsExtract}

\begin{mpFunctionsExtract}
\mpFunctionTwo
{intFDivQR? mpNumList[2]? the quotient of $n$ and $d$, rounded down towards $-\infty$, and the remainder.}
{n? mpNum? An Integer.}
{d? mpNum? An Integer.}
\end{mpFunctionsExtract}

\begin{mpFunctionsExtract}
\mpFunctionTwo
{intTDivQ? mpNum? the quotient of $n$ and $d$, rounded towards zero.}
{n? mpNum? An Integer.}
{d? mpNum? An Integer.}
\end{mpFunctionsExtract}

\begin{mpFunctionsExtract}
\mpFunctionTwo
{intTDivQ2exp? mpNum? the quotient of $n$ and $2^b$, rounded towards zero.}
{n? mpNum? An Integer.}
{b? mpNum? An Integer.}
\end{mpFunctionsExtract}

\begin{mpFunctionsExtract}
\mpFunctionTwo
{intTDivR? mpNum? the remainder, once the quotient of $n$ and $d$,rounded towards zero, has been obtained.}
{n? mpNum? An Integer.}
{d? mpNum? An Integer.}
\end{mpFunctionsExtract}

\begin{mpFunctionsExtract}
\mpFunctionTwo
{intTDivR2exp? mpNum? the remainder, once the quotient of $n$ and $2^b$, rounded towards zero, has been obtained.}
{n? mpNum? An Integer.}
{b? mpNum? An Integer.}
\end{mpFunctionsExtract}

\begin{mpFunctionsExtract}
\mpFunctionTwo
{intTDivQr? mpNumList[2]? the quotient of $n$ and $d$, rounded towards zero, and the remainder.}
{n? mpNum? An Integer.}
{d? mpNum? An Integer.}
\end{mpFunctionsExtract}

\section{Logical Operators }

\begin{mpFunctionsExtract}
\mpFunctionTwo
{intAND? mpNum? $n_1$ bitwise-and $n_2$.}
{n1? mpNum? An Integer.}
{n2? mpNum? An Integer.}
\end{mpFunctionsExtract}

\begin{mpFunctionsExtract}
\mpFunctionTwo
{intIOR? mpNum? $n_1$ bitwise-inclusive-or $n_2$.}
{n1? mpNum? An Integer.}
{n2? mpNum? An Integer.}
\end{mpFunctionsExtract}

\begin{mpFunctionsExtract}
\mpFunctionTwo
{intXOR? mpNum? $n_1$ bitwise-exclusive-or $n_2$.}
{n1? mpNum? An Integer.}
{n2? mpNum? An Integer.}
\end{mpFunctionsExtract}

\section{Bit-Oriented Functions}

\begin{mpFunctionsExtract}
\mpFunctionOne
{intComplement? mpNum? the one's complement of $n$.}
{n? mpNum? An Integer.}
\end{mpFunctionsExtract}

\begin{mpFunctionsExtract}
\mpFunctionTwo
{intHamDist? mpNum? the hamming distance between the two operands}
{n1? mpNum? An Integer.}
{n2? mpNum? An Integer.}
\end{mpFunctionsExtract}

\begin{mpFunctionsExtract}
\mpFunctionTwo
{intTestBit? mpNum? 1 or 0 according to whether bit $k$ in $n$ is set or not.}
{n? mpNum? An Integer.}
{k? mpNum? An Integer.}
\end{mpFunctionsExtract}

\begin{mpFunctionsExtract}
\mpFunctionTwo
{intComBit? mpNum? n with the complement bit $k$ set in $n$.}
{n? mpNum? An Integer.}
{k? mpNum? An Integer.}
\end{mpFunctionsExtract}

\begin{mpFunctionsExtract}
\mpFunctionTwo
{intClearBit? mpNum? $n$ with the bit $k$ cleared in $n$.}
{n? mpNum? An Integer.}
{k? mpNum? An Integer.}
\end{mpFunctionsExtract}

\begin{mpFunctionsExtract}
\mpFunctionTwo
{intSetBit? mpNum? $n$ with the bit $k$ set in $n$.}
{n? mpNum? An Integer.}
{k? mpNum? An Integer.}
\end{mpFunctionsExtract}

\begin{mpFunctionsExtract}
\mpFunctionTwo
{intScan0? mpNum? the index of the found bit 0, starting from bit $k$.}
{n? mpNum? An Integer.}
{k? mpNum? An Integer.}
\end{mpFunctionsExtract}

\begin{mpFunctionsExtract}
\mpFunctionTwo
{intScan1? mpNum? the index of the found bit 1, starting from bit $k$.}
{n? mpNum? An Integer.}
{k? mpNum? An Integer.}
\end{mpFunctionsExtract}

\begin{mpFunctionsExtract}
\mpFunctionOne
{intPopCount? mpNum? the population count of $n$.}
{n? mpNum? An Integer.}
\end{mpFunctionsExtract}

\section{Sign, Powers and Roots}

\begin{mpFunctionsExtract}
\mpFunctionOne
{intSgn? mpNum? the sign of $n$.}
{n? mpNum? An Integer.}
\end{mpFunctionsExtract}

\begin{mpFunctionsExtract}
\mpFunctionOne
{intAbs? mpNum? the absolute value of $n$.}
{n? mpNum? An Integer.}
\end{mpFunctionsExtract}

\begin{mpFunctionsExtract}
\mpFunctionTwo
{intPow? mpNum? the value of $n^k$. The case $0^0$ yields 1.}
{n? mpNum? An Integer.}
{k? mpNum? An Integer.}
\end{mpFunctionsExtract}

\begin{mpFunctionsExtract}
\mpFunctionThree
{intPowMod? mpNum? the value of $n^k \text{ mod } m$.}
{n? mpNum? An Integer.}
{k? mpNum? An Integer.}
{m? mpNum? An Integer.}
\end{mpFunctionsExtract}

\begin{mpFunctionsExtract}
\mpFunctionOne
{intSqrt? mpNum? the truncated integer part of the square root of $n$.}
{n? mpNum? An Integer.}
\end{mpFunctionsExtract}

\begin{mpFunctionsExtract}
\mpFunctionOne
{intSqrtRem? mpNumList[2]? the truncated integer part of the square root of $n$, and the remainder.}
{n? mpNum? An Integer.}
\end{mpFunctionsExtract}

\begin{mpFunctionsExtract}
\mpFunctionTwo
{intRoot? mpNum? the truncated integer part of the $n^{th}$ root of $m$}
{n? mpNum? An Integer.}
{m? mpNum? An Integer.}
\end{mpFunctionsExtract}

\begin{mpFunctionsExtract}
\mpFunctionTwo
{intRootRem? mpNumList[2]? the truncated integer part of the $n^{th}$ root of $m$, with remainder}
{n? mpNum? An Integer.}
{m? mpNum? An Integer.}
\end{mpFunctionsExtract}

\section{Numbertheoretic Functions}

\begin{mpFunctionsExtract}
\mpFunctionOne
{intFactorial? mpNum?  $n!$, the factorial of $n$}
{n? mpNum? An Integer.}
\end{mpFunctionsExtract}

\begin{mpFunctionsExtract}
\mpFunctionTwo
{intBinCoeff? mpNum? the binomial coefficient}
{n? mpNum? An Integer.}
{k? mpNum? An Integer.}
\end{mpFunctionsExtract}

\begin{mpFunctionsExtract}
\mpFunctionOne
{intNextprime? Integer?  the next prime greater than $n$.}
{n? Integer? An Integer.}
\end{mpFunctionsExtract}

\begin{mpFunctionsExtract}
\mpFunctionTwo
{intGcd? mpNum? the greatest common divisor of $n_1$ and $n_2$}
{n1? mpNum? An Integer.}
{n2? mpNum? An Integer.}
\end{mpFunctionsExtract}

\begin{mpFunctionsExtract}
\mpFunctionTwo
{intGcdExt? mpNumList[3]? the extended greatest common divisor of $n_1$ and $n_2$}
{n1? mpNum? An Integer.}
{n2? mpNum? An Integer.}
\end{mpFunctionsExtract}

\begin{mpFunctionsExtract}
\mpFunctionTwo
{intLcm? mpNum? the least common multiple of $n_1$ and $n_2$.}
{n1? mpNum? An Integer.}
{n2? mpNum? An Integer.}
\end{mpFunctionsExtract}

\begin{mpFunctionsExtract}
\mpFunctionTwo
{intInvertMod? mpNum? the inverse of $n_1$ modulo $n_2$}
{n1? mpNum? An Integer.}
{n2? mpNum? An Integer.}
\end{mpFunctionsExtract}

\begin{mpFunctionsExtract}
\mpFunctionTwo
{intRemoveFactor? mpNum? $n$ with all occurrences of the factor $f$ removed from $n$.}
{n? mpNum? An Integer.}
{f? mpNum? An Integer.}
\end{mpFunctionsExtract}

\begin{mpFunctionsExtract}
\mpFunctionTwo
{intLegendreSymbol? mpNum? the Legendre symbol $\left(\frac{a}{p}\right)$.}
{a? mpNum? An Integer.}
{p? mpNum? An Integer.}
\end{mpFunctionsExtract}

\begin{mpFunctionsExtract}
\mpFunctionTwo
{intJacobiSymbol? mpNum? the Jacobi symbol $\left(\frac{a}{b}\right)$}
{a? mpNum? An Integer.}
{b? mpNum? An Integer.}
\end{mpFunctionsExtract}

\begin{mpFunctionsExtract}
\mpFunctionTwo
{intKroneckerSymbol? mpNum? the Kronecker symbol $\left(\frac{a}{b}\right)$}
{a? mpNum? An Integer.}
{b? mpNum? An Integer.}
\end{mpFunctionsExtract}

\begin{mpFunctionsExtract}
\mpFunctionOne
{intFibonacci? mpNum? the $n^{th}$ Fibonacci number.}
{n? mpNum? An Integer.}
\end{mpFunctionsExtract}

\begin{mpFunctionsExtract}
\mpFunctionOne
{intLucas? mpNum? the $n^{th}$ Lucas number.}
{n? mpNum? An Integer.}
\end{mpFunctionsExtract}

\section{Additional Numbertheoretic Functions}

\begin{mpFunctionsExtract}
\mpFunctionOne
{intIsBpswPrp? mpNum? True if n is a Baillie-Pomerance-Selfridge-Wagstaff probable prime.}
{n? mpNum? An Integer.}
\end{mpFunctionsExtract}

\begin{mpFunctionsExtract}
\mpFunctionTwo
{intIsEulerPrp? mpNum? True if n is an Euler (also known as Solovay-Strassen) probable}
{n? mpNum? An Integer.}
{a? mpNum? An Integer.}
\end{mpFunctionsExtract}

\begin{mpFunctionsExtract}
\mpFunctionTwo
{intIsExtraStrongLucasPrp? mpNum? True if n is an extra strong Lucas probable prime}
{n? mpNum? An Integer.}
{p? mpNum? An Integer.}
\end{mpFunctionsExtract}

\begin{mpFunctionsExtract}
\mpFunctionTwo
{intIsFermatPrp? mpNum? True if n is a Fermat probable prime to the base a}
{n? mpNum? An Integer.}
{a? mpNum? An Integer.}
\end{mpFunctionsExtract}

\begin{mpFunctionsExtract}
\mpFunctionThree
{intIsFibonacciPrp? mpNum? True if n is an Fibonacci probable prime with parameters (p,q).}
{n? mpNum? An Integer.}
{p? mpNum? An Integer.}
{q? mpNum? An Integer.}
\end{mpFunctionsExtract}

\begin{mpFunctionsExtract}
\mpFunctionThree
{intIsLucasPrp? mpNum? True if n is a Lucas probable prime with parameters (p,q).}
{n? mpNum? An Integer.}
{p? mpNum? An Integer.}
{q? mpNum? An Integer.}
\end{mpFunctionsExtract}

\begin{mpFunctionsExtract}
\mpFunctionOne
{intIsSelfridgePrp? mpNum? True if n is a Lucas probable prime with Selfidge parameters (p,q).}
{a? mpNum? An Integer.}
\end{mpFunctionsExtract}

\begin{mpFunctionsExtract}
\mpFunctionOne
{intIsStrongBpswPrp? mpNum? True if n is a strong Baillie-Pomerance-Selfridge-Wagstaff probable prime}
{a? mpNum? An Integer.}
\end{mpFunctionsExtract}

\begin{mpFunctionsExtract}
\mpFunctionThree
{intIsStrongLucasPrp? mpNum? True if n is a strong Lucas probable prime with parameters (p,q).}
{n? mpNum? An Integer.}
{p? mpNum? An Integer.}
{q? mpNum? An Integer.}
\end{mpFunctionsExtract}

\begin{mpFunctionsExtract}
\mpFunctionTwo
{intIsStrongPrp? mpNum? True if n is an strong (also known as Miller-Rabin) probable prime}
{n? mpNum? An Integer.}
{a? mpNum? An Integer.}
\end{mpFunctionsExtract}

\begin{mpFunctionsExtract}
\mpFunctionOne
{intIsStrongSelfridgePrp? mpNum? True if n is a strong Lucas probable prime with Selfidge parameters}
{a? mpNum? An Integer.}
\end{mpFunctionsExtract}

\begin{mpFunctionsExtract}
\mpFunctionThree
{intLucasU? mpNum? the k-th element of the Lucas U sequence defined by p,q}
{p? mpNum? An Integer.}
{q? mpNum? An Integer.}
{k? mpNum? An Integer.}
\end{mpFunctionsExtract}

\begin{mpFunctionsExtract}
\mpFunctionFour
{intLucasModU? mpNum? the k-th element of the Lucas U sequence defined by p,q (mod n)}
{p? mpNum? An Integer.}
{q? mpNum? An Integer.}
{k? mpNum? An Integer.}
{n? mpNum? An Integer.}
\end{mpFunctionsExtract}

\begin{mpFunctionsExtract}
\mpFunctionThree
{intLucasV? mpNum? the k-th element of the Lucas V sequence defined by p,q}
{p? mpNum? An Integer.}
{q? mpNum? An Integer.}
{k? mpNum? An Integer.}
\end{mpFunctionsExtract}

\begin{mpFunctionsExtract}
\mpFunctionFour
{intLucasModV? mpNum? the k-th element of the Lucas V sequence defined by p,q (mod n)}
{p? mpNum? An Integer.}
{q? mpNum? An Integer.}
{k? mpNum? An Integer.}
{n? mpNum? An Integer.}
\end{mpFunctionsExtract}

\section{Random Numbers}

\begin{mpFunctionsExtract}
\mpFunctionOne
{intUrandomb? mpNum? a uniformly distributed random integer in the range 0 to $2^n - 1$, inclusive.}
{n? mpNum? An Integer.}
\end{mpFunctionsExtract}

\begin{mpFunctionsExtract}
\mpFunctionOne
{intUrandomm? mpNum? a uniformly distributed random integer in the range 0 to $n - 1$, inclusive.}
{n? mpNum? An Integer.}
\end{mpFunctionsExtract}

\begin{mpFunctionsExtract}
\mpFunctionOne
{intRrandomb? mpNum? a random integer with long strings of zeros and ones in the binary representation.}
{n? mpNum? An Integer.}
\end{mpFunctionsExtract}

\section{Information Functions for Integers}

\begin{mpFunctionsExtract}
\mpFunctionThree
{IsCongruent? mpNum? TRUE if $n$ is congruent to $c$ modulo $d$, and FALSE otherwise.}
{n? mpNum? An Integer.}
{d? mpNum? An Integer.}
{m? mpNum? An Integer.}
\end{mpFunctionsExtract}

\begin{mpFunctionsExtract}
\mpFunctionThree
{IsCongruent2exp? mpNum? TRUE if $n$ is congruent to $c$ modulo $d$, and FALSE otherwise.}
{n? mpNum? An Integer.}
{c? mpNum? An Integer.}
{b? mpNum? An Integer.}
\end{mpFunctionsExtract}

\begin{mpFunctionsExtract}
\mpFunctionTwo
{IsProbablyPrime? mpNum? 2 if $n$ is definitely prime, returns 1 if $n$ is probably prime (without being certain), and returns 0 if n is definitely composite.}
{n? mpNum? An Integer.}
{reps? mpNum? An Integer.}
\end{mpFunctionsExtract}

\begin{mpFunctionsExtract}
\mpFunctionTwo
{IsDivisible? mpNum? TRUE if $n$ is exactly divisible by $d$.}
{n? mpNum? An Integer.}
{d? mpNum? An Integer.}
\end{mpFunctionsExtract}

\begin{mpFunctionsExtract}
\mpFunctionTwo
{IsDivisible2exp? mpNum? TRUE if $n$ is exactly divisible by $2^b$.}
{n? mpNum? An Integer.}
{b? mpNum? An Integer.}
\end{mpFunctionsExtract}

\begin{mpFunctionsExtract}
\mpFunctionOne
{IsPerfectPower? mpNum? TRUE if $n$ is a perfect power.}
{n? mpNum? An Integer.}
\end{mpFunctionsExtract}

\begin{mpFunctionsExtract}
\mpFunctionOne
{IsPerfectSquare? mpNum? non-zero if $n$ is a perfect square.}
{n? mpNum? An Integer.}
\end{mpFunctionsExtract}

\chapter{MPQ}

\chapter{MPD}

\section{MPD Context}

\chapter{MPFR}

\section{MPFR Context}

\section{Constants}

\begin{mpFunctionsExtract}
\mpFunctionZero
{ConstLog2? mpNum? the value of the natural logarithm of 2, $\ln(2) = 0.69314718055994...$.}
\end{mpFunctionsExtract}

\begin{mpFunctionsExtract}
\mpFunctionZero
{Pi? mpNum? the value of $\pi = 3.1415926535897932...$.}
\end{mpFunctionsExtract}

\begin{mpFunctionsExtract}
\mpFunctionZero
{Catalan? mpNum? the value of Catalan's constant, $G = 0.9159655941772190...$.}
\end{mpFunctionsExtract}

\begin{mpFunctionsExtract}
\mpFunctionZero
{EulerGamma? mpNum? the value of Euler's Gamma, $\gamma = 0.57721566490153286...$.}
\end{mpFunctionsExtract}

\begin{mpFunctionsExtract}
\mpFunctionZero
{MachineEpsilon? mpNum? the value of the Machine Epsilon in the current precision}
\end{mpFunctionsExtract}

\begin{mpFunctionsExtract}
\mpFunctionZero
{MaxReal? mpNum? the value of the largest representable real number in the current precision.}
\end{mpFunctionsExtract}

\begin{mpFunctionsExtract}
\mpFunctionZero
{MaxInteger? mpNum? the value of the largest representable integer in the current precision.}
\end{mpFunctionsExtract}

\begin{mpFunctionsExtract}
\mpFunctionZero
{MinReal? mpNum? the value of the smallest representable positive real number in the current precision.}
\end{mpFunctionsExtract}

\begin{mpFunctionsExtract}
\mpFunctionZero
{MinInteger? mpNum? the value of the smallest representable positive integer in the current precision.}
\end{mpFunctionsExtract}

\begin{mpFunctionsExtract}
\mpFunctionZero
{PosInf? mpNum? the value of the representation of  $+\infty$ in the current precision.}
\end{mpFunctionsExtract}

\begin{mpFunctionsExtract}
\mpFunctionZero
{NegInf? mpNum? the value of the representation of  $-\infty$ in the current precision.}
\end{mpFunctionsExtract}

\begin{mpFunctionsExtract}
\mpFunctionZero
{NaN? mpNum? the value of the representation of Not a Number (NaN) in the current precision.}
\end{mpFunctionsExtract}

\section{Sign, Powers and Roots}

\begin{mpFunctionsExtract}
\mpFunctionOne
{Sign? mpNum? the value of the sign of $x, \text{sign}(x)$.}
{x? mpNum? A real number.}
\end{mpFunctionsExtract}

\begin{mpFunctionsExtract}
\mpFunctionTwo
{Copysign? mpNum? $|x|\cdot \text{sign(y)}$.}
{x? mpNum? A real number.}
{y? mpNum? A real number.}
\end{mpFunctionsExtract}

\begin{mpFunctionsExtract}
\mpFunctionOne
{Abs? mpNum? the absolute value of $x$, $|x| = \sqrt{x^2}$.}
{x? mpNum? A real number.}
\end{mpFunctionsExtract}

\begin{mpFunctionsExtract}
\mpFunctionOne
{Reci? mpNum? the absolute value of the reciprocal of $x,  1/x = x^{-1}$}
{x? mpNum? A real number.}
\end{mpFunctionsExtract}

\begin{mpFunctionsExtract}
\mpFunctionOne
{Square? mpNum? the absolute value of the square of $x,  x^2$.}
{x? mpNum? A real number.}
\end{mpFunctionsExtract}
Additional HTML Comment only in extracted file

\begin{mpFunctionsExtract}
\mpFunctionTwo
{Power\_k? mpNum? the value of $x^k, k \in  \mathbb{Z}$}
{x? mpNum? A real number.}
{k? mpNum? An integer.}
\end{mpFunctionsExtract}

\begin{mpFunctionsExtract}
\mpFunctionTwo
{Power? mpNum? the value of $x^y, y \in  \mathbb{R}$.}
{x? mpNum? A real number.}
{y? mpNum? A real number.}
\end{mpFunctionsExtract}

\begin{mpFunctionsExtract}
\mpFunctionTwo
{Powm1? mpNum? the value of $x^y-1, y \in  \mathbb{R}$.}
{x? mpNum? A real number.}
{y? mpNum? A real number.}
\end{mpFunctionsExtract}

\begin{mpFunctionsExtract}
\mpFunctionTwo
{X2pY2? mpNum? the value of $x^2+y^2$.}
{x? mpNum? A real number.}
{y? mpNum? A real number.}
\end{mpFunctionsExtract}

\begin{mpFunctionsExtract}
\mpFunctionTwo
{X2mY2? mpNum? the value of $x^2-y^2$.}
{x? mpNum? A real number.}
{y? mpNum? A real number.}
\end{mpFunctionsExtract}

\begin{mpFunctionsExtract}
\mpFunctionOne
{Sqrt? mpNum? the absolute value of the square root of $x, \sqrt{x}$.}
{x? mpNum? A real number.}
\end{mpFunctionsExtract}

\begin{mpFunctionsExtract}
\mpFunctionOne
{Sqrt\_n? mpNum? the absolute value of the square root of a nonnegative Integer$n, \sqrt{n}$.}
{x? mpNum? An integer.}
\end{mpFunctionsExtract}

\begin{mpFunctionsExtract}
\mpFunctionOne
{ReciSqrt? mpNum? the absolute value of the reciprocal square root of $x, \sqrt{x}$.}
{x? mpNum? A real number.}
\end{mpFunctionsExtract}

\begin{mpFunctionsExtract}
\mpFunctionOne
{Cbrt? mpNum? the absolute value of the cube root of $x, \sqrt[3]{x}$.}
{x? mpNum? A real number.}
\end{mpFunctionsExtract}

\begin{mpFunctionsExtract}
\mpFunctionOne
{Sqrtp1m1? mpNum? the value of $\sqrt{x+1}-1$.}
{x? mpNum? A real number.}
\end{mpFunctionsExtract}

\begin{mpFunctionsExtract}
\mpFunctionOne
{Sqrt1px2? mpNum? the value of $\sqrt{1+x^2}$.}
{x? mpNum? A real number.}
\end{mpFunctionsExtract}

\begin{mpFunctionsExtract}
\mpFunctionOne
{Sqrt1mx2? mpNum? the value of $\sqrt{1-x^2}$.}
{x? mpNum? A real number.}
\end{mpFunctionsExtract}

\begin{mpFunctionsExtract}
\mpFunctionOne
{Sqrtx2m1? mpNum? the value of $\sqrt{x^2-1}$.}
{x? mpNum? A real number.}
\end{mpFunctionsExtract}

\begin{mpFunctionsExtract}
\mpFunctionTwo
{Hypot? mpNum? the value of $\sqrt{x^2+y^2}$.}
{x? mpNum? A real number.}
{y? mpNum? A real number.}
\end{mpFunctionsExtract}

\begin{mpFunctionsExtract}
\mpFunctionTwo
{NthRoot? mpNum? the value of the $n^{th}$ root of $x$, $\sqrt[n]{x}, n=2,3,...$.}
{n? mpNum? An integer.}
{y? mpNum? A real number.}
\end{mpFunctionsExtract}

\section{Exponential, Logarithmic, and Lambert Functions}

\begin{mpFunctionsExtract}
\mpFunctionOne
{Exp? mpNum? the value of the exponential function,  $\text{exp}(x) = e^x = \exp(x)$.}
{x? mpNum? A real number.}
\end{mpFunctionsExtract}

\begin{mpFunctionsExtract}
\mpFunctionOne
{Exp10? mpNum? the value of the exponential function, $\text{exp10}(x) = 10^x = \exp_{10}(x)$.}
{x? mpNum? A real number.}
\end{mpFunctionsExtract}

\begin{mpFunctionsExtract}
\mpFunctionOne
{Exp2? mpNum? the value of the exponential function, $\text{exp2}(x) = 2^x = \exp_2(x)$.}
{x? mpNum? A real number.}
\end{mpFunctionsExtract}

\begin{mpFunctionsExtract}
\mpFunctionOne
{Expm1? mpNum? the value of the function $\text{expm1}(x) = e^{x}-1$.}
{x? mpNum? A real number.}
\end{mpFunctionsExtract}

\begin{mpFunctionsExtract}
\mpFunctionOne
{Expx2? mpNum? the value of the function $\text{expx2}(x) = e^{x^2}$.}
{x? mpNum? A real number.}
\end{mpFunctionsExtract}

\begin{mpFunctionsExtract}
\mpFunctionOne
{Expx2m1? mpNum? the value of the function $\text{expx2m1}(x) = e^{x^2}-1$.}
{x? mpNum? A real number.}
\end{mpFunctionsExtract}

\begin{mpFunctionsExtract}
\mpFunctionOne
{Expmx2? mpNum? the value of the function $\text{expmx2}(x) = e^{-x^2}$.}
{x? mpNum? A real number.}
\end{mpFunctionsExtract}

\begin{mpFunctionsExtract}
\mpFunctionOne
{Expmx2m1? mpNum? the value of the function $\text{expmx2m1}(x) = e^{-x^2}-1$.}
{x? mpNum? A real number.}
\end{mpFunctionsExtract}

\begin{mpFunctionsExtract}
\mpFunctionOne
{Ln? mpNum? the value of the natural logarithm $\text{ln}(x) = \log_e(x)$.}
{x? mpNum? A real number.}
\end{mpFunctionsExtract}

\begin{mpFunctionsExtract}
\mpFunctionOne
{Lnp1? mpNum? the value of the function $\ln(1+x)$.}
{x? mpNum? A real number.}
\end{mpFunctionsExtract}

\begin{mpFunctionsExtract}
\mpFunctionOne
{Log10? mpNum? the value of the decadic logarithm $\text{log10}(x) = \log_{10}(x)$.}
{x? mpNum? A real number.}
\end{mpFunctionsExtract}

\begin{mpFunctionsExtract}
\mpFunctionOne
{Log2? mpNum? the value of the binary logarithm $\text{log2}(x) = \log_{2}(x)$.}
{x? mpNum? A real number.}
\end{mpFunctionsExtract}

\begin{mpFunctionsExtract}
\mpFunctionOne
{Log? mpNum? the value of the logarithm  to base $b$: $\text{logb}(x) = \log_{b}(x)$.}
{x? mpNum? A real number.}
\end{mpFunctionsExtract}

\begin{mpFunctionsExtract}
\mpFunctionOne
{LnCos? mpNum? the value of the logarithm of the cosine of $x$: $\text{LnCos}(x) =\ln(\cos(x))$.}
{x? mpNum? A real number.}
\end{mpFunctionsExtract}

\begin{mpFunctionsExtract}
\mpFunctionOne
{LnSin? mpNum? the value of the logarithm of the sine of $x$: $\text{LnSin}(x) =\ln(\sin(x))$}
{x? mpNum? A real number.}
\end{mpFunctionsExtract}

\begin{mpFunctionsExtract}
\mpFunctionTwo
{LnSqrtx2y2? mpNum? the value of the function $\text{LnSqrtx2y2}(x) =\ln \left(\sqrt{x^2+y^2} \right)$.}
{x? mpNum? A real number.}
{y? mpNum? A real number.}
\end{mpFunctionsExtract}

\begin{mpFunctionsExtract}
\mpFunctionTwo
{LnSqrtxp1T2y2? mpNum? the value of the function $\text{LnSqrtxp1T2y2}(x) =\ln \left(\sqrt{(x+1)^2+y^2} \right)$.}
{x? mpNum? A real number.}
{y? mpNum? A real number.}
\end{mpFunctionsExtract}

\begin{mpFunctionsExtract}
\mpFunctionOne
{LambertW0? mpNum? the value of the Lambert functions $W_0(x)$}
{x? mpNum? A real number.}
\end{mpFunctionsExtract}

\begin{mpFunctionsExtract}
\mpFunctionOne
{LambertWm1? mpNum? the value of the Lambert functions $W_{-1}(x)$}
{x? mpNum? A real number.}
\end{mpFunctionsExtract}

\section{Trigonometric Functions}

\begin{mpFunctionsExtract}
\mpFunctionOne
{Sin? mpNum? the value of the sine of $x$, with $x$ in radians.}
{x? mpNum? A real number.}
\end{mpFunctionsExtract}

\begin{mpFunctionsExtract}
\mpFunctionOne
{SinDeg? mpNum? the value of the sine of $x$, with $x$ in degrees}
{x? mpNum? A real number.}
\end{mpFunctionsExtract}

\begin{mpFunctionsExtract}
\mpFunctionOne
{Cos? mpNum? the value of the cosine of $x$, with $x$ in radians.}
{x? mpNum? A real number.}
\end{mpFunctionsExtract}

\begin{mpFunctionsExtract}
\mpFunctionOne
{CosDeg? mpNum? the value of the cosine of $x$, with $x$ in degrees}
{x? mpNum? A real number.}
\end{mpFunctionsExtract}

\begin{mpFunctionsExtract}
\mpFunctionOne
{Tan? mpNum? the value of the tangent of $x$, with $x$ in radians.}
{x? mpNum? A real number.}
\end{mpFunctionsExtract}

\begin{mpFunctionsExtract}
\mpFunctionOne
{TanDeg? mpNum? the value of the tangent of $x$, with $x$ in degrees}
{x? mpNum? A real number.}
\end{mpFunctionsExtract}

\begin{mpFunctionsExtract}
\mpFunctionOne
{Csc? mpNum? the value of the cosecant of $x$, with $x$ in radians.}
{x? mpNum? A real number.}
\end{mpFunctionsExtract}

\begin{mpFunctionsExtract}
\mpFunctionOne
{CscDeg? mpNum? the value of the cosecant of $x$, with $x$ in degrees}
{x? mpNum? A real number.}
\end{mpFunctionsExtract}

\begin{mpFunctionsExtract}
\mpFunctionOne
{Sec? mpNum? the value of the secant of $x$, with $x$ in radians.}
{x? mpNum? A real number.}
\end{mpFunctionsExtract}

\begin{mpFunctionsExtract}
\mpFunctionOne
{SecDeg? mpNum? the value of the secant of $x$, with $x$ in degrees}
{x? mpNum? A real number.}
\end{mpFunctionsExtract}

\begin{mpFunctionsExtract}
\mpFunctionOne
{Cot? mpNum? the value of the cotangent of $x$, with $x$ in radians.}
{x? mpNum? A real number.}
\end{mpFunctionsExtract}

\begin{mpFunctionsExtract}
\mpFunctionOne
{CotDeg? mpNum? the value of the cotangent of $x$, with $x$ in degrees}
{x? mpNum? A real number.}
\end{mpFunctionsExtract}

\begin{mpFunctionsExtract}
\mpFunctionOne
{Sinca? mpNum? the sinus cardinal function}
{x? mpNum? A real number.}
\end{mpFunctionsExtract}

\begin{mpFunctionsExtract}
\mpFunctionOne
{Sinh? mpNum? the value of the hyperbolic sine of $x$, with $x$ in radians.}
{x? mpNum? A real number.}
\end{mpFunctionsExtract}

\begin{mpFunctionsExtract}
\mpFunctionOne
{SinhDeg? mpNum? the value of the hyperbolic sine of $x$, with $x$ in degrees}
{x? mpNum? A real number.}
\end{mpFunctionsExtract}

\begin{mpFunctionsExtract}
\mpFunctionOne
{Cosh? mpNum? the value of the hyperbolic cosine of $x$, with $x$ in radians.}
{x? mpNum? A real number.}
\end{mpFunctionsExtract}

\begin{mpFunctionsExtract}
\mpFunctionOne
{CoshDeg? mpNum? the value of the hyperbolic cosine of $x$, with $x$ in degrees}
{x? mpNum? A real number.}
\end{mpFunctionsExtract}

\begin{mpFunctionsExtract}
\mpFunctionOne
{Tanh? mpNum? the value of the hyperbolic cosine of $x$, with $x$ in radians.}
{x? mpNum? A real number.}
\end{mpFunctionsExtract}

\begin{mpFunctionsExtract}
\mpFunctionOne
{TanhDeg? mpNum? the value of the hyperbolic cosine of $x$, with $x$ in degrees}
{x? mpNum? A real number.}
\end{mpFunctionsExtract}

\begin{mpFunctionsExtract}
\mpFunctionOne
{Csch? mpNum? the value of the hyperbolic cosecant of $x$, with $x$ in radians.}
{x? mpNum? A real number.}
\end{mpFunctionsExtract}

\begin{mpFunctionsExtract}
\mpFunctionOne
{CschDeg? mpNum? the value of the hyperbolic cosecant of $x$, with $x$ in degrees}
{x? mpNum? A real number.}
\end{mpFunctionsExtract}

\begin{mpFunctionsExtract}
\mpFunctionOne
{Sech? mpNum? the value of the hyperbolic cosecant of $x$, with $x$ in radians.}
{x? mpNum? A real number.}
\end{mpFunctionsExtract}

\begin{mpFunctionsExtract}
\mpFunctionOne
{SechDeg? mpNum? the value of the hyperbolic cosecant of $x$, with $x$ in degrees}
{x? mpNum? A real number.}
\end{mpFunctionsExtract}

\begin{mpFunctionsExtract}
\mpFunctionOne
{Coth? mpNum? the value of the hyperbolic cotangent of $x$, with $x$ in radians.}
{x? mpNum? A real number.}
\end{mpFunctionsExtract}

\begin{mpFunctionsExtract}
\mpFunctionOne
{CothDeg? mpNum? the value of the hyperbolic cotangent of $x$, with $x$ in degrees}
{x? mpNum? A real number.}
\end{mpFunctionsExtract}

\begin{mpFunctionsExtract}
\mpFunctionOne
{Sinhca? mpNum? the hyperbolic sinus cardinal function.}
{x? mpNum? A real number.}
\end{mpFunctionsExtract}

\section{Inverse Trigonometric Functions}

\begin{mpFunctionsExtract}
\mpFunctionOne
{Asin? mpNum? the value of the arc-sine of $x$ in radians.}
{x? mpNum? A real number.}
\end{mpFunctionsExtract}

\begin{mpFunctionsExtract}
\mpFunctionOne
{AsinDeg? mpNum? the value of the arc-sine of $x$ in degrees}
{x? mpNum? A real number.}
\end{mpFunctionsExtract}

\begin{mpFunctionsExtract}
\mpFunctionOne
{Acos? mpNum? the value of the arc-cosine of $x$ in radians.}
{x? mpNum? A real number.}
\end{mpFunctionsExtract}

\begin{mpFunctionsExtract}
\mpFunctionOne
{AcosDeg? mpNum? the value of the arc-cosine of $x$ in degrees}
{x? mpNum? A real number.}
\end{mpFunctionsExtract}

\begin{mpFunctionsExtract}
\mpFunctionOne
{Atan? mpNum? the value of the arc-tangent of $x$ in radians.}
{x? mpNum? A real number.}
\end{mpFunctionsExtract}

\begin{mpFunctionsExtract}
\mpFunctionOne
{AtanDeg? mpNum? the value of the arc-tangent of $x$ in degrees}
{x? mpNum? A real number.}
\end{mpFunctionsExtract}

\begin{mpFunctionsExtract}
\mpFunctionTwo
{Atan2? mpNum? the value of the arc-tangent of $x$ in radians.}
{x? mpNum? A real number.}
{y? mpNum? A real number.}
\end{mpFunctionsExtract}

\begin{mpFunctionsExtract}
\mpFunctionTwo
{Atan2Deg? mpNum? the value of the arc-tangent of $x$ in degrees}
{x? mpNum? A real number.}
{y? mpNum? A real number.}
\end{mpFunctionsExtract}

\begin{mpFunctionsExtract}
\mpFunctionOne
{Acot? mpNum? the value of the arc-cotangent of $x$ in radians.}
{x? mpNum? A real number.}
\end{mpFunctionsExtract}

\begin{mpFunctionsExtract}
\mpFunctionOne
{AcotDeg? mpNum? the value of the arc-cotangent of $x$ in degrees}
{x? mpNum? A real number.}
\end{mpFunctionsExtract}

\begin{mpFunctionsExtract}
\mpFunctionOne
{Asinh? mpNum? the value of the hyperbolic arc-sine  of $x$ in radians.}
{x? mpNum? A real number.}
\end{mpFunctionsExtract}

\begin{mpFunctionsExtract}
\mpFunctionOne
{AsinhDeg? mpNum? the value of hyperbolic arc-sine  of $x$ in degrees}
{x? mpNum? A real number.}
\end{mpFunctionsExtract}

\begin{mpFunctionsExtract}
\mpFunctionOne
{Acosh? mpNum? the value of the hyperbolic arc-cosine  of $x$ in radians.}
{x? mpNum? A real number.}
\end{mpFunctionsExtract}

\begin{mpFunctionsExtract}
\mpFunctionOne
{AcoshDeg? mpNum? the value of hyperbolic arc-cosine  of $x$ in degrees}
{x? mpNum? A real number.}
\end{mpFunctionsExtract}

\begin{mpFunctionsExtract}
\mpFunctionOne
{Atanh? mpNum? the value of the hyperbolic arc-tangent  of $x$ in radians.}
{x? mpNum? A real number.}
\end{mpFunctionsExtract}

\begin{mpFunctionsExtract}
\mpFunctionOne
{AtanhDeg? mpNum? the value of hyperbolic arc-tangent  of $x$ in degrees}
{x? mpNum? A real number.}
\end{mpFunctionsExtract}

\begin{mpFunctionsExtract}
\mpFunctionOne
{Acoth? mpNum? the value of the hyperbolic arc-cotangent  of $x$ in radians.}
{x? mpNum? A real number.}
\end{mpFunctionsExtract}

\begin{mpFunctionsExtract}
\mpFunctionOne
{AcothDeg? mpNum? the value of hyperbolic arc-cotangent  of $x$ in degrees}
{x? mpNum? A real number.}
\end{mpFunctionsExtract}

\section{Elementary Functions of Mathematical Physics}

\begin{mpFunctionsExtract}
\mpFunctionOne
{BesselJ0? mpNum? $J_0(x)$, the Bessel function of the 1st kind, order zero.}
{x? mpNum? A real number.}
\end{mpFunctionsExtract}

\begin{mpFunctionsExtract}
\mpFunctionOne
{BesselJ1? mpNum? $J_1(x)$, the Bessel function of the 1st kind, order one.}
{x? mpNum? A real number.}
\end{mpFunctionsExtract}

\begin{mpFunctionsExtract}
\mpFunctionTwo
{BesselJn? mpNum? $J_n(x)$, the Bessel function of the 1st kind, order $n$.}
{x? mpNum? A real number.}
{n? mpNum? An Integer.}
\end{mpFunctionsExtract}

\begin{mpFunctionsExtract}
\mpFunctionOne
{BesselY0? mpNum? $Y_0(x)$, the Bessel function of the second kind, order zero.}
{x? mpNum? A real number.}
\end{mpFunctionsExtract}

\begin{mpFunctionsExtract}
\mpFunctionOne
{BesselY1? mpNum? $Y_1(x)$, the Bessel function of the second kind, order one.}
{x? mpNum? A real number.}
\end{mpFunctionsExtract}

\begin{mpFunctionsExtract}
\mpFunctionTwo
{BesselYn? mpNum? $Y_n(x)$, the Bessel function of the second kind, order $n$.}
{x? mpNum? A real number.}
{n? mpNum? An Integer.}
\end{mpFunctionsExtract}

\begin{mpFunctionsExtract}
\mpFunctionOne
{Erf? mpNum? the value of the error function.}
{x? mpNum? A real number.}
\end{mpFunctionsExtract}

\begin{mpFunctionsExtract}
\mpFunctionOne
{Erfc? mpNum? the value of the complementary error function.}
{x? mpNum? A real number.}
\end{mpFunctionsExtract}

\begin{mpFunctionsExtract}
\mpFunctionOne
{Tgamma? mpNum? the gamma function for $x \neq 0, -1, -2,\ldots$.}
{x? mpNum? A real number.}
\end{mpFunctionsExtract}

\begin{mpFunctionsExtract}
\mpFunctionOne
{Lgamma? mpNum? the logarithm of the gamma function.}
{x? mpNum? A real number.}
\end{mpFunctionsExtract}

\begin{mpFunctionsExtract}
\mpFunctionTwo
{Pochhammer? mpNum? the Pochhammer symbol.}
{a? mpNum? An integer.}
{x? mpNum? An integer.}
\end{mpFunctionsExtract}

\begin{mpFunctionsExtract}
\mpFunctionTwo
{Beta? mpNum? the Beta function.}
{a? mpNum? A real number.}
{b? mpNum? A real number.}
\end{mpFunctionsExtract}

\begin{mpFunctionsExtract}
\mpFunctionTwo
{LnBetaBoost? mpNum? the logarithm of the beta function $\ln B(a,b)|$ with $a,b \neq 0,-1,-2,\ldots$.}
{a? mpNum? A real number.}
{b? mpNum? A real number.}
\end{mpFunctionsExtract}

\begin{mpFunctionsExtract}
\mpFunctionThree
{IBetaBoost? mpNum? the normalised incomplete beta function.}
{a? mpNum? A real number.}
{b? mpNum? A real number.}
{x? mpNum? A real number.}
\end{mpFunctionsExtract}

\begin{mpFunctionsExtract}
\mpFunctionThree
{IBetacBoost? mpNum? the normalised complement of the incomplete beta function, $1 - I_x(a,b)$.}
{a? mpNum? A real number.}
{b? mpNum? A real number.}
{x? mpNum? A real number.}
\end{mpFunctionsExtract}

\begin{mpFunctionsExtract}
\mpFunctionThree
{IBetaNonNormalizedBoost? mpNum? the non-normalised incomplete beta function.}
{a? mpNum? A real number.}
{b? mpNum? A real number.}
{x? mpNum? A real number.}
\end{mpFunctionsExtract}

\begin{mpFunctionsExtract}
\mpFunctionThree
{IBetacNonNormalizedBoost? mpNum? the non-normalised complement of the incomplete beta function, $1 - B_x(a,b)$.}
{a? mpNum? A real number.}
{b? mpNum? A real number.}
{x? mpNum? A real number.}
\end{mpFunctionsExtract}

\begin{mpFunctionsExtract}
\mpFunctionThree
{IBetaInvBoost? mpNum? the inverse of the normalised incomplete beta function $I_x(a,b)$.}
{a? mpNum? A real number.}
{b? mpNum? A real number.}
{p? mpNum? A real number.}
\end{mpFunctionsExtract}

\begin{mpFunctionsExtract}
\mpFunctionThree
{IBetacInvBoost? mpNum? the inverse of the complement of the normalised incomplete beta function $1 - I_x(a,b)$.}
{a? mpNum? A real number.}
{b? mpNum? A real number.}
{q? mpNum? A real number.}
\end{mpFunctionsExtract}

\begin{mpFunctionsExtract}
\mpFunctionThree
{IBetaInvaBoost? mpNum? the parameter $a$ of the normalised incomplete beta function $I_x(a,b)$, such that $I_x(a,b) = p$.}
{x? mpNum? A real number.}
{b? mpNum? A real number.}
{p? mpNum? A real number.}
\end{mpFunctionsExtract}

\begin{mpFunctionsExtract}
\mpFunctionThree
{IBetacInvaBoost? mpNum? the parameter $a$ of the complement of the normalised incomplete beta function $1-I_x(a,b)$, such that $1-I_x(a,b) = q$.}
{x? mpNum? A real number.}
{b? mpNum? A real number.}
{q? mpNum? A real number.}
\end{mpFunctionsExtract}

\begin{mpFunctionsExtract}
\mpFunctionThree
{IBetaInvbBoost? mpNum? the parameter $b$ of the normalised incomplete beta function $I_x(a,b)$, such that $I_x(a,b) = p$.}
{x? mpNum? A real number.}
{a? mpNum? A real number.}
{p? mpNum? A real number.}
\end{mpFunctionsExtract}

\begin{mpFunctionsExtract}
\mpFunctionThree
{IBetacInvbBoost? mpNum? the parameter $b$ of the complement of the normalised incomplete beta function $1-I_x(a,b)$, such that $1-I_x(a,b) = q$.}
{x? mpNum? A real number.}
{a? mpNum? A real number.}
{q? mpNum? A real number.}
\end{mpFunctionsExtract}

\begin{mpFunctionsExtract}
\mpFunctionThree
{IBetaDerivativeBoost? mpNum? the partial derivative with respect to $x$ of the incomplete beta function.}
{x? mpNum? A real number.}
{a? mpNum? A real number.}
{b? mpNum? A real number.}
\end{mpFunctionsExtract}

\begin{mpFunctionsExtract}
\mpFunctionOne
{RiemannZeta? mpNum? the Riemann zeta function.}
{s? mpNum? A real number.}
\end{mpFunctionsExtract}

\begin{mpFunctionsExtract}
\mpFunctionOne
{Dilogarithm? mpNum? the dilogarithm function $\text{Li}_2(x)$.}
{x? mpNum? A real number.}
\end{mpFunctionsExtract}

\section{Integer and Remainder Related Functions}

\begin{mpFunctionsExtract}
\mpFunctionOne
{Rint? mpNum? the rounded value of $x$.}
{x? mpNum? A real number.}
{RoundingMode? mpNum? An integer.}
\end{mpFunctionsExtract}

\begin{mpFunctionsExtract}
\mpFunctionOne
{RintRound? mpNum? the rounded value of $x$, rounded to the nearest integer, rounding halfway cases away from zero.}
{x? mpNum? A real number.}
\end{mpFunctionsExtract}

\begin{mpFunctionsExtract}
\mpFunctionOne
{RintCeil? mpNum? the rounded value of $x$, rounded to the next higher or equal integer.}
{x? mpNum? A real number.}
\end{mpFunctionsExtract}

\begin{mpFunctionsExtract}
\mpFunctionOne
{RintFloor? mpNum? the rounded value of $x$, rounded to the next lower or equal integer.}
{x? mpNum? A real number.}
\end{mpFunctionsExtract}

\begin{mpFunctionsExtract}
\mpFunctionOne
{RintTrunc? mpNum? the rounded value of $x$, rounded to the next integer toward zero.}
{x? mpNum? A real number.}
\end{mpFunctionsExtract}

\begin{mpFunctionsExtract}
\mpFunctionOne
{Frac? mpNum? the fractional part of $x$}
{x? mpNum? A real number.}
\end{mpFunctionsExtract}

\begin{mpFunctionsExtract}
\mpFunctionOne
{Modf? mpNumList? simultaneously the integer and fractional part of $x$}
{x? mpNum? A real number.}
\end{mpFunctionsExtract}

\begin{mpFunctionsExtract}
\mpFunctionTwo
{Fmod? mpNum? the remainder of $x/y$}
{x? mpNum? A real number.}
{y? mpNum? A real number.}
\end{mpFunctionsExtract}

\begin{mpFunctionsExtract}
\mpFunctionTwo
{Remainder? mpNum? the remainder of $x/y$}
{x? mpNum? A real number.}
{y? mpNum? A real number.}
\end{mpFunctionsExtract}

\begin{mpFunctionsExtract}
\mpFunctionTwo
{Remquo? mpNum? the remainder of $x/y$}
{x? mpNum? A real number.}
{y? mpNum? A real number.}
\end{mpFunctionsExtract}

\section{Miscellaneous Functions}

\begin{mpFunctionsExtract}
\mpFunctionTwo
{Nexttoward? mpNum? the next floating-point number (with the precision of $x$ and the current exponent range) in the direction of $y$}
{x? mpNum? A real number.}
{y? mpNum? A real number.}
\end{mpFunctionsExtract}

\begin{mpFunctionsExtract}
\mpFunctionOne
{Nextabove? mpNum? the next floating-point number (with the precision of $x$ and the current exponent range) in the direction of plus infinity.}
{x? mpNum? A real number.}
\end{mpFunctionsExtract}

\begin{mpFunctionsExtract}
\mpFunctionOne
{Nextbelow? mpNum? the next floating-point number (with the precision of $x$ and the current exponent range) in the direction of minus infinity.}
{x? mpNum? A real number.}
\end{mpFunctionsExtract}

\begin{mpFunctionsExtract}
\mpFunctionOne
{Frexp? mpNumList? returns simultaneously significand and exponent of $x$}
{x? mpNum? A real number.}
\end{mpFunctionsExtract}

\begin{mpFunctionsExtract}
\mpFunctionTwo
{Ldexp? mpNum? $x \cdot 2^{y}$}
{x? mpNum? A real number.}
{y? mpNum? A real number.}
\end{mpFunctionsExtract}

\begin{mpFunctionsExtract}
\mpFunctionThree
{Fma? mpNum? $(a \times b) + c$.}
{a? mpNum? A real number.}
{b? mpNum? A real number.}
{c? mpNum? A real number.}
\end{mpFunctionsExtract}

\begin{mpFunctionsExtract}
\mpFunctionThree
{Fms? mpNum? $(a \times b) - c$.}
{a? mpNum? A real number.}
{b? mpNum? A real number.}
{c? mpNum? A real number.}
\end{mpFunctionsExtract}

\section{Numerical Information Functions}

\begin{mpFunctionsExtract}
\mpFunctionOne
{IsInf? Boolean? TRUE if $x$ is infinity (positive or negative), and FALSE otherwise.}
{x? mpNum? A real number.}
\end{mpFunctionsExtract}

\begin{mpFunctionsExtract}
\mpFunctionOne
{IsInteger? Boolean? TRUE if $x$ is an integer, and FALSE otherwise.}
{x? mpNum? A real number.}
\end{mpFunctionsExtract}

\begin{mpFunctionsExtract}
\mpFunctionOne
{IsNan? Boolean? TRUE if $x$ is an NaN (Not a Number), and FALSE otherwise.}
{x? mpNum? A real number.}
\end{mpFunctionsExtract}

\begin{mpFunctionsExtract}
\mpFunctionOne
{IsNeg? Boolean? TRUE if $x$ is negative, and FALSE otherwise.}
{x? mpNum? A real number.}
\end{mpFunctionsExtract}

\begin{mpFunctionsExtract}
\mpFunctionOne
{IsNonNeg? Boolean? TRUE if $x \geq 0$, and FALSE otherwise.}
{x? mpNum? A real number.}
\end{mpFunctionsExtract}

\begin{mpFunctionsExtract}
\mpFunctionOne
{IsNonPos? Boolean? TRUE if $x \leq 0$, and FALSE otherwise.}
{x? mpNum? A real number.}
\end{mpFunctionsExtract}

\begin{mpFunctionsExtract}
\mpFunctionOne
{IsPos? Boolean? TRUE if $x > 0$, and FALSE otherwise.}
{x? mpNum? A real number.}
\end{mpFunctionsExtract}

\begin{mpFunctionsExtract}
\mpFunctionOne
{IsRegular? Boolean? TRUE if $x$ is an regular number (i.e. neither NaN nor an infinity nor zero), and FALSE otherwise.}
{x? mpNum? A real number.}
\end{mpFunctionsExtract}

\begin{mpFunctionsExtract}
\mpFunctionTwo
{IsUnordered? Boolean? TRUE if $x$ or $y$ is NaN (i.e. they cannot be compared), and FALSE otherwise.}
{x? mpNum? A real number.}
{y? mpNum? A real number.}
\end{mpFunctionsExtract}

\begin{mpFunctionsExtract}
\mpFunctionOne
{IsZero? Boolean? TRUE if $x = 0$, and FALSE otherwise.}
{x? mpNum? A real number.}
\end{mpFunctionsExtract}

\chapter{MPC}

\section{Conversion between Real and Complex Numbers}

\begin{mpFunctionsExtract}
\mpFunctionTwo
{cplxRect? mpNum? a complex number $z$ build from the real components $x$ and $y$ as $z=x+iy$.}
{x? mpNum? A real number.}
{y? mpNum? A real number.}
\end{mpFunctionsExtract}

\begin{mpFunctionsExtract}
\mpFunctionOne
{cplxReal? mpNum? the real component $x$ of $z=x+iy$.}
{z? mpNum? A complex number.}
\end{mpFunctionsExtract}

\begin{mpFunctionsExtract}
\mpFunctionOne
{cplxImag? mpNum? the imaginary component $y$ of $z=x+iy$.}
{z? mpNum? A complex number.}
\end{mpFunctionsExtract}

\begin{mpFunctionsExtract}
\mpFunctionOne
{cplxAbs? mpNum? the absolute value of $z=x+iy$}
{z? mpNum? A complex number.}
\end{mpFunctionsExtract}

\begin{mpFunctionsExtract}
\mpFunctionOne
{cplxArg? mpNum? the argument of $z=x+iy$}
{z? mpNum? A complex number.}
\end{mpFunctionsExtract}

\section{Unary and Arithmetic Operators }

\begin{mpFunctionsExtract}
\mpFunctionOne
{cplxNeg? mpNum?  $-z=-x-iy$.}
{z? mpNum? A complex number.}
\end{mpFunctionsExtract}

\begin{mpFunctionsExtract}
\mpFunctionOne
{cplxConj? mpNum? the conjugate of $z$, $\overline{z}=x-iy$}
{z? mpNum? A complex number.}
\end{mpFunctionsExtract}

\begin{mpFunctionsExtract}
\mpFunctionTwo
{cplxAdd? mpNum?  $-z=-x-iy$.}
{z1? mpNum? A complex number.}
{z2? mpNum? A complex number.}
\end{mpFunctionsExtract}

\begin{mpFunctionsExtract}
\mpFunctionOne
{cplxSum? mpNum? the sum of up to 255 complex numbers.}
{z? mpNum[]? An array of complex numbers.}
\end{mpFunctionsExtract}

\begin{mpFunctionsExtract}
\mpFunctionTwo
{cplxSub? mpNum? the difference of $z1$ and $z2$.}
{z1? mpNum? A complex number.}
{z2? mpNum? A complex number.}
\end{mpFunctionsExtract}

\begin{mpFunctionsExtract}
\mpFunctionTwo
{cplxMul? mpNum? the product of $z1$ and $z2$.}
{z1? mpNum? A complex number.}
{z2? mpNum? A complex number.}
\end{mpFunctionsExtract}

\begin{mpFunctionsExtract}
\mpFunctionOne
{cplxProduct? mpNum? the product of up to 255 complex numbers.}
{z? mpNum[]? An array of complex numbers.}
\end{mpFunctionsExtract}

\begin{mpFunctionsExtract}
\mpFunctionTwo
{cplxDiv? mpNum? the difference of $z1$ and $z2$.}
{z1? mpNum? A complex number.}
{z2? mpNum? A complex number.}
\end{mpFunctionsExtract}

\section{Roots and Power Functions}

\begin{mpFunctionsExtract}
\mpFunctionOne
{cplxSqr? mpNum? the square of $z$.}
{z? mpNum? A complex number.}
\end{mpFunctionsExtract}

\begin{mpFunctionsExtract}
\mpFunctionTwo
{cplxPower? mpNum? an integer power of $z$}
{z? mpNum? A complex number.}
{k? mpNum? An integer.}
\end{mpFunctionsExtract}

\begin{mpFunctionsExtract}
\mpFunctionTwo
{cplxPowR? mpNum? an real power of $z$}
{z? mpNum? A complex number.}
{a? mpNum? A real number.}
\end{mpFunctionsExtract}

\begin{mpFunctionsExtract}
\mpFunctionTwo
{cplxPowC? mpNum? an complex power of $z$}
{z1? mpNum? A complex number.}
{z2? mpNum? A complex number.}
\end{mpFunctionsExtract}

\begin{mpFunctionsExtract}
\mpFunctionOne
{cplxSqrt? mpNum? the square root of $z$}
{z? mpNum? A complex number.}
\end{mpFunctionsExtract}

\begin{mpFunctionsExtract}
\mpFunctionTwo
{cplxNthRoot? mpNum? an integer power of $z$}
{z? mpNum? A complex number.}
{n? mpNum? An integer.}
\end{mpFunctionsExtract}

\section{Exponential and Logarithmic Functions}

\begin{mpFunctionsExtract}
\mpFunctionOne
{cplxExp? mpNum? the complex exponential of $z$}
{z? mpNum? A complex number.}
\end{mpFunctionsExtract}

\begin{mpFunctionsExtract}
\mpFunctionOne
{cplxExp10? mpNum?  $10^z$}
{z? mpNum? A complex number.}
\end{mpFunctionsExtract}

\begin{mpFunctionsExtract}
\mpFunctionOne
{cplxExp2? mpNum?  $2^z$}
{z? mpNum? A complex number.}
\end{mpFunctionsExtract}

\begin{mpFunctionsExtract}
\mpFunctionOne
{cplxLn? mpNum? the complex natural logarithm of $z$}
{z? mpNum? A complex number.}
\end{mpFunctionsExtract}

\begin{mpFunctionsExtract}
\mpFunctionOne
{cplxLog10? mpNum? $\log_{10}(z)$}
{z? mpNum? A complex number.}
\end{mpFunctionsExtract}

\begin{mpFunctionsExtract}
\mpFunctionOne
{cplxLog2? mpNum? $\log_{2}(z)$}
{z? mpNum? A complex number.}
\end{mpFunctionsExtract}

\section{Trigonometric Functions}

\begin{mpFunctionsExtract}
\mpFunctionOne
{cplxSin? mpNum? complex sine of $z$}
{z? mpNum? A complex number.}
\end{mpFunctionsExtract}

\begin{mpFunctionsExtract}
\mpFunctionOne
{cplxCos? mpNum? complex cosine of $z$}
{z? mpNum? A complex number.}
\end{mpFunctionsExtract}

\begin{mpFunctionsExtract}
\mpFunctionOne
{cplxTan? mpNum? complex tangent of $z$}
{z? mpNum? A complex number.}
\end{mpFunctionsExtract}

\begin{mpFunctionsExtract}
\mpFunctionOne
{cplxSec? mpNum? the complex secant of $z$}
{z? mpNum? A complex number.}
\end{mpFunctionsExtract}

\begin{mpFunctionsExtract}
\mpFunctionOne
{cplxCsc? mpNum? the complex cosecant of $z$}
{z? mpNum? A complex number.}
\end{mpFunctionsExtract}

\begin{mpFunctionsExtract}
\mpFunctionOne
{cplxCot? mpNum? the complex cotangent of $z$}
{z? mpNum? A complex number.}
\end{mpFunctionsExtract}

\begin{mpFunctionsExtract}
\mpFunctionOne
{cplxSinh? mpNum? the complex hyperbolic sine of $z$}
{z? mpNum? A complex number.}
\end{mpFunctionsExtract}

\begin{mpFunctionsExtract}
\mpFunctionOne
{cplxCosh? mpNum? the complex hyperbolic cosine of $z$}
{z? mpNum? A complex number.}
\end{mpFunctionsExtract}

\begin{mpFunctionsExtract}
\mpFunctionOne
{cplxTanh? mpNum? the complex hyperbolic tangent of $z$}
{z? mpNum? A complex number.}
\end{mpFunctionsExtract}

\begin{mpFunctionsExtract}
\mpFunctionOne
{cplxSech? mpNum? the complex hyperbolic secant of $z$}
{z? mpNum? A complex number.}
\end{mpFunctionsExtract}

\begin{mpFunctionsExtract}
\mpFunctionOne
{cplxCsch? mpNum? the complex hyperbolic cosecant of $z$}
{z? mpNum? A complex number.}
\end{mpFunctionsExtract}

\begin{mpFunctionsExtract}
\mpFunctionOne
{cplxCoth? mpNum? the complex hyperbolic cotangent of $z$}
{z? mpNum? A complex number.}
\end{mpFunctionsExtract}

\section{Inverse Trigonometric Functions}

\begin{mpFunctionsExtract}
\mpFunctionOne
{cplxASin? mpNum? the inverse complex sine of $z$}
{z? mpNum? A complex number.}
\end{mpFunctionsExtract}

\begin{mpFunctionsExtract}
\mpFunctionOne
{cplxACos? mpNum? the inverse complex cosine of $z$}
{z? mpNum? A complex number.}
\end{mpFunctionsExtract}

\begin{mpFunctionsExtract}
\mpFunctionOne
{cplxATan? mpNum? the inverse complex tangent of $z$}
{z? mpNum? A complex number.}
\end{mpFunctionsExtract}

\begin{mpFunctionsExtract}
\mpFunctionOne
{cplxACot? mpNum? the inverse complex cotangent of $z$}
{z? mpNum? A complex number.}
\end{mpFunctionsExtract}

\begin{mpFunctionsExtract}
\mpFunctionOne
{cplxASinh? mpNum? the inverse complex hyperbolic sine of $z$}
{z? mpNum? A complex number.}
\end{mpFunctionsExtract}

\begin{mpFunctionsExtract}
\mpFunctionOne
{cplxACosh? mpNum? the inverse complex hyperbolic cosine of $z$}
{z? mpNum? A complex number.}
\end{mpFunctionsExtract}

\begin{mpFunctionsExtract}
\mpFunctionOne
{cplxATanh? mpNum? the inverse complex hyperbolic tangent of $z$}
{z? mpNum? A complex number.}
\end{mpFunctionsExtract}

\begin{mpFunctionsExtract}
\mpFunctionOne
{cplxACoth? mpNum? the inverse complex hyperbolic cotangent of $z$}
{z? mpNum? A complex number.}
\end{mpFunctionsExtract}

\chapter{MPFI}

\section{Information Functions for Intervals}

\begin{mpFunctionsExtract}
\mpFunctionOne
{IsEmpty? mpNum? TRUE  if $x$ is empty (its endpoints are in reverse order), and FALSE otherwise.}
{x? mpNum? A real number.}
\end{mpFunctionsExtract}

\begin{mpFunctionsExtract}
\mpFunctionTwo
{IsInside? mpNum? TRUE  if $x$ is contained in $y$, and FALSE otherwise.}
{x? mpNum? A real number.}
{y? mpNum? A real number.}
\end{mpFunctionsExtract}

\begin{mpFunctionsExtract}
\mpFunctionTwo
{IsStrictlyInside? mpNum? TRUE  if the second interval $y$ is contained in the interior of $x$, and FALSE otherwise.}
{x? mpNum? A real number.}
{y? mpNum? A real number.}
\end{mpFunctionsExtract}

\begin{mpFunctionsExtract}
\mpFunctionOne
{IsStrictlyNeg? mpNum? TRUE if $x$ contains only negative numbers, and FALSE otherwise.}
{x? mpNum? A real number.}
\end{mpFunctionsExtract}

\begin{mpFunctionsExtract}
\mpFunctionOne
{IsStrictlyPos? mpNum? TRUE if $x$ contains only positive numbers, and FALSE otherwise.}
{x? mpNum? A real number.}
\end{mpFunctionsExtract}

\chapter{MPFCI}

\chapter{BLAS Support (based on Eigen)}

\section{BLAS Level 1 Support and related Functions}

\begin{mpFunctionsExtract}
\mpFunctionTwo
{RDot? mpNum? the real scalar product $\boldsymbol{x}^T \boldsymbol{y}$ for the real vectors $\boldsymbol{x}$ and $\boldsymbol{y}$.}
{x? mpNum[]? A vector of real numbers.}
{y? mpNum[]? A vector of real numbers.}
\end{mpFunctionsExtract}

\begin{mpFunctionsExtract}
\mpFunctionTwo
{cplxDotu? mpNum? the complex  scalar product $\boldsymbol{x}^T \boldsymbol{y}$ for the complex  vectors $\boldsymbol{x}$ and $\boldsymbol{y}$.}
{x? mpNum[]? A vector of complex numbers.}
{y? mpNum[]? A vector of complex numbers.}
\end{mpFunctionsExtract}

\begin{mpFunctionsExtract}
\mpFunctionTwo
{cplxDotc? mpNum? the complex conjugate scalar product $\boldsymbol{x}^H \boldsymbol{y}$ for the complex  vectors $\boldsymbol{x}$ and $\boldsymbol{y}$.}
{x? mpNum[]? A vector of complex numbers.}
{y? mpNum[]? A vector of complex numbers.}
\end{mpFunctionsExtract}

\begin{mpFunctionsExtract}
\mpFunctionTwo
{RNrm2? mpNum? the Euclidean norm $||\boldsymbol{x}||_2$ of the real vector $\boldsymbol{x}$.}
{x? mpNum[]? A vector of real numbers.}
{y? mpNum[]? A vector of real numbers.}
\end{mpFunctionsExtract}

\begin{mpFunctionsExtract}
\mpFunctionTwo
{cplxNrm2? mpNum? the Euclidean norm $||\boldsymbol{x}||_2$ of the complex vector $\boldsymbol{x}$.}
{x? mpNum[]? A vector of complex numbers.}
{y? mpNum[]? A vector of complex numbers.}
\end{mpFunctionsExtract}

\begin{mpFunctionsExtract}
\mpFunctionTwo
{RAsum? mpNum? the the absolute sum of the elements of the real vector $\boldsymbol{x}$.}
{x? mpNum[]? A vector of real numbers.}
{y? mpNum[]? A vector of real numbers.}
\end{mpFunctionsExtract}

\begin{mpFunctionsExtract}
\mpFunctionTwo
{cplxAsum? mpNum? the  sum of the magnitudes of the real and imaginary parts of the complex vector $\boldsymbol{x}$.}
{x? mpNum[]? A vector of complex numbers.}
{y? mpNum[]? A vector of complex numbers.}
\end{mpFunctionsExtract}

\begin{mpFunctionsExtract}
\mpFunctionThree
{RAxpy? mpNum?  the sum $\alpha \boldsymbol{x} + \boldsymbol{y}$ for the real scalar $\alpha$ and the real vectors $\boldsymbol{x}$ and $\boldsymbol{y}$.}
{$\alpha$? mpNum? A real scalar.}
{x? mpNum[]? A vector of real numbers.}
{y? mpNum[]? A vector of real numbers.}
\end{mpFunctionsExtract}

\begin{mpFunctionsExtract}
\mpFunctionThree
{cplxAxpy? mpNum? the sum $\alpha \boldsymbol{x} + \boldsymbol{y}$ for the complex scalar $\alpha$ and the complex vectors $\boldsymbol{x}$ and $\boldsymbol{y}$.}
{$\alpha$? mpNum? A complex scalar.}
{x? mpNum[]? A vector of complex numbers.}
{y? mpNum[]? A vector of complex numbers.}
\end{mpFunctionsExtract}

\section{BLAS Level 2 Support}

\begin{mpFunctionsExtract}
\mpFunctionSix
{RGemv? mpNum? the matrix-vector product and sum for a general matrix.}
{TransA? Integer? An indicator specifying the multiplication.}
{$\alpha$? mpNum? A real scalar.}
{A? mpNum[,]? A matrix of real numbers.}
{x? mpNum[]? A vector of real numbers.}
{$\beta$? mpNum? A real scalar.}
{y? mpNum[]? A vector of real numbers.}
\end{mpFunctionsExtract}

\begin{mpFunctionsExtract}
\mpFunctionSix
{cplxGemv? mpNum? the matrix-vector product and sum for a general matrix.}
{TransA? Integer? An indicator specifying the multiplication.}
{$\alpha$? mpNum? A complex scalar.}
{A? mpNum[,]? A matrix of complex numbers.}
{x? mpNum[]? A vector of complex numbers.}
{$\beta$? mpNum? A complex scalar.}
{y? mpNum[]? A vector of complex numbers.}
\end{mpFunctionsExtract}

\begin{mpFunctionsExtract}
\mpFunctionFive
{RTrmv? mpNum?  the matrix-vector product for a triangular matrix.}
{Uplo? Integer? An indicator specifying whether the upper or lower triangle will be used.}
{TransA? Integer? An indicator specifying the multiplication.}
{Diag? Integer? An indicator specifying the use of the diagonal.}
{A? mpNum[,]? A matrix of real numbers.}
{x? mpNum[]? A vector of real numbers.}
\end{mpFunctionsExtract}

\begin{mpFunctionsExtract}
\mpFunctionFive
{cplxTrmv? mpNum?  the matrix-vector product for a triangular matrix.}
{Uplo? Integer? An indicator specifying whether the upper or lower triangle will be used.}
{TransA? Integer? An indicator specifying the multiplication.}
{Diag? Integer? An indicator specifying the use of the diagonal.}
{A? mpNum[,]? A matrix of complex numbers.}
{x? mpNum[]? A vector of complex numbers.}
\end{mpFunctionsExtract}

\begin{mpFunctionsExtract}
\mpFunctionFive
{RTrsv? mpNum?  the inverse matrix-vector product for a triangular matrix.}
{Uplo? Integer? An indicator specifying whether the upper or lower triangle will be used.}
{TransA? Integer? An indicator specifying the multiplication.}
{Diag? Integer? An indicator specifying the use of the diagonal.}
{A? mpNum[,]? A matrix of real numbers.}
{x? mpNum[]? A vector of real numbers.}
\end{mpFunctionsExtract}

\begin{mpFunctionsExtract}
\mpFunctionFive
{cplxTrsv? mpNum?  the inverse matrix-vector product for a triangular matrix.}
{Uplo? Integer? An indicator specifying whether the upper or lower triangle will be used.}
{TransA? Integer? An indicator specifying the multiplication.}
{Diag? Integer? An indicator specifying the use of the diagonal.}
{A? mpNum[,]? A matrix of complex numbers.}
{x? mpNum[]? A vector of complex numbers.}
\end{mpFunctionsExtract}

\begin{mpFunctionsExtract}
\mpFunctionSix
{RSymv? mpNum? the matrix-vector product and sum for a symmetric matrix.}
{Uplo? Integer? An indicator specifying whether the upper or lower triangle will be used.}
{$\alpha$? mpNum? A real scalar.}
{A? mpNum[,]? A matrix of real numbers.}
{x? mpNum[]? A vector of real numbers.}
{$\beta$? mpNum? A real scalar.}
{y? mpNum[]? A vector of real numbers.}
\end{mpFunctionsExtract}

\begin{mpFunctionsExtract}
\mpFunctionSix
{cplxHemv? mpNum? the matrix-vector product and sum for a hermitian matrix.}
{Uplo? Integer? An indicator specifying whether the upper or lower triangle will be used.}
{$\alpha$? mpNum? A complex scalar.}
{A? mpNum[,]? A matrix of complex numbers.}
{x? mpNum[]? A vector of complex numbers.}
{$\beta$? mpNum? A complex scalar.}
{y? mpNum[]? A vector of complex numbers.}
\end{mpFunctionsExtract}

\begin{mpFunctionsExtract}
\mpFunctionFour
{RGer? mpNum? the rank-1 update for a general matrix}
{$\alpha$? mpNum? A real scalar.}
{x? mpNum[]? A vector of real numbers.}
{y? mpNum[]? A vector of real numbers.}
{A? mpNum[,]? A matrix of real numbers.}
\end{mpFunctionsExtract}

\begin{mpFunctionsExtract}
\mpFunctionFour
{cplxGeru? mpNum? the rank-1 update for a general matrix}
{$\alpha$? mpNum? A complex scalar.}
{x? mpNum[]? A vector of complex numbers.}
{y? mpNum[]? A vector of complex numbers.}
{A? mpNum[,]? A matrix of complex numbers.}
\end{mpFunctionsExtract}

\begin{mpFunctionsExtract}
\mpFunctionFour
{cplxGerc? mpNum? the rank-1 update for a general matrix}
{$\alpha$? mpNum? A complex scalar.}
{x? mpNum[]? A vector of complex numbers.}
{y? mpNum[]? A vector of complex numbers.}
{A? mpNum[,]? A matrix of complex numbers.}
\end{mpFunctionsExtract}

\begin{mpFunctionsExtract}
\mpFunctionFour
{RSyr? mpNum? the Rank-1 update for a symmetric matrix.}
{Uplo? Integer? An indicator specifying whether the upper or lower triangle will be used.}
{$\alpha$? mpNum? A real scalar.}
{x? mpNum[]? A vector of real numbers.}
{A? mpNum[,]? A matrix of real numbers.}
\end{mpFunctionsExtract}

\begin{mpFunctionsExtract}
\mpFunctionFour
{cplxHer? mpNum? the Rank-1 update for a hermitian matrix.}
{Uplo? Integer? An indicator specifying whether the upper or lower triangle will be used.}
{$\alpha$? mpNum? A complex scalar.}
{x? mpNum[]? A vector of complex numbers.}
{A? mpNum[,]? A matrix of complex numbers.}
\end{mpFunctionsExtract}

\begin{mpFunctionsExtract}
\mpFunctionFive
{RSyr2? mpNum? the Rank-1 update for a symmetric matrix.}
{Uplo? Integer? An indicator specifying whether the upper or lower triangle will be used.}
{$\alpha$? mpNum? A real scalar.}
{x? mpNum[]? A vector of real numbers.}
{y? mpNum[]? A vector of real numbers.}
{A? mpNum[,]? A matrix of real numbers.}
\end{mpFunctionsExtract}

\begin{mpFunctionsExtract}
\mpFunctionFive
{cplxHer2? mpNum? the Rank-1 update for a hermitian matrix.}
{Uplo? Integer? An indicator specifying whether the upper or lower triangle will be used.}
{$\alpha$? mpNum? A complex scalar.}
{x? mpNum[]? A vector of complex numbers.}
{y? mpNum[]? A vector of complex numbers.}
{A? mpNum[,]? A matrix of complex numbers.}
\end{mpFunctionsExtract}

\section{BLAS Level 3 Support}

\begin{mpFunctionsExtract}
\mpFunctionSeven
{RGemm? mpNum? the matrix-matrix product and sum for a general matrix.}
{TransA? Integer? An indicator specifying the multiplication.}
{TransB? Integer? An indicator specifying the multiplication.}
{$\alpha$? mpNum? A real scalar.}
{A? mpNum[,]? A matrix of real numbers.}
{B? mpNum[,]? A matrix of real numbers.}
{$\beta$? mpNum? A real scalar.}
{C? mpNum[,]? A matrix of real numbers.}
\end{mpFunctionsExtract}

\begin{mpFunctionsExtract}
\mpFunctionSeven
{cplxGemm? mpNum? the matrix-matrix product and sum for a general matrix.}
{TransA? Integer? An indicator specifying the multiplication.}
{TransB? Integer? An indicator specifying the multiplication.}
{$\alpha$? mpNum? A real scalar.}
{A? mpNum[,]? A matrix of real numbers.}
{B? mpNum[,]? A matrix of real numbers.}
{$\beta$? mpNum? A real scalar.}
{C? mpNum[,]? A matrix of real numbers.}
\end{mpFunctionsExtract}

\begin{mpFunctionsExtract}
\mpFunctionSeven
{RSymm? mpNum? the matrix-matrix product and sum for a symmetric matrix.}
{Side? Integer? An indicator specifying the order of the multiplication.}
{Uplo? Integer? An indicator specifying whether the upper or lower triangle will be used.}
{$\alpha$? mpNum? A real scalar.}
{A? mpNum[,]? A matrix of real numbers.}
{B? mpNum[,]? A matrix of real numbers.}
{$\beta$? mpNum? A real scalar.}
{C? mpNum[,]? A matrix of real numbers.}
\end{mpFunctionsExtract}

\begin{mpFunctionsExtract}
\mpFunctionSeven
{cplxSymm? mpNum? the matrix-matrix product and sum for a symmetric matrix.}
{Side? Integer? An indicator specifying the order of the multiplication.}
{Uplo? Integer? An indicator specifying whether the upper or lower triangle will be used.}
{$\alpha$? mpNum? A complex scalar.}
{A? mpNum[,]? A matrix of complex numbers.}
{B? mpNum[,]? A matrix of complex numbers.}
{$\beta$? mpNum? A complex scalar.}
{C? mpNum[,]? A matrix of complex numbers.}
\end{mpFunctionsExtract}

\begin{mpFunctionsExtract}
\mpFunctionSeven
{cplxHemm? mpNum? the matrix-matrix product and sum for a hermitian matrix.}
{Side? Integer? An indicator specifying the order of the multiplication.}
{Uplo? Integer? An indicator specifying whether the upper or lower triangle will be used.}
{$\alpha$? mpNum? A complex scalar.}
{A? mpNum[,]? A matrix of complex numbers.}
{B? mpNum[,]? A matrix of complex numbers.}
{$\beta$? mpNum? A complex scalar.}
{C? mpNum[,]? A matrix of complex numbers.}
\end{mpFunctionsExtract}

\begin{mpFunctionsExtract}
\mpFunctionSix
{RTrmm? mpNum? the matrix-matrix produc for a triangular matrix.}
{Side? Integer? An indicator specifying the order of the multiplication.}
{Uplo? Integer? An indicator specifying whether the upper or lower triangle will be used.}
{TransA? Integer? An indicator specifying the multiplication.}
{$\alpha$? mpNum? A real scalar.}
{A? mpNum[,]? A matrix of real numbers.}
{B? mpNum[,]? A matrix of real numbers.}
\end{mpFunctionsExtract}

\begin{mpFunctionsExtract}
\mpFunctionSix
{cplxTrmm? mpNum? the matrix-matrix product for a triangular matrix.}
{Side? Integer? An indicator specifying the order of the multiplication.}
{Uplo? Integer? An indicator specifying whether the upper or lower triangle will be used.}
{TransA? Integer? An indicator specifying the multiplication.}
{$\alpha$? mpNum? A complex scalar.}
{A? mpNum[,]? A matrix of complex numbers.}
{B? mpNum[,]? A matrix of complex numbers.}
\end{mpFunctionsExtract}

\begin{mpFunctionsExtract}
\mpFunctionSix
{RTrsm? mpNum? the inverse matrix-matrix product for a triangular matrix.}
{Side? Integer? An indicator specifying the order of the multiplication.}
{Uplo? Integer? An indicator specifying whether the upper or lower triangle will be used.}
{TransA? Integer? An indicator specifying the multiplication.}
{$\alpha$? mpNum? A real scalar.}
{A? mpNum[,]? A matrix of real numbers.}
{B? mpNum[,]? A matrix of real numbers.}
\end{mpFunctionsExtract}

\begin{mpFunctionsExtract}
\mpFunctionSix
{cplxTrsm? mpNum? the inverse matrix-matrix product for a triangular matrix.}
{Side? Integer? An indicator specifying the order of the multiplication.}
{Uplo? Integer? An indicator specifying whether the upper or lower triangle will be used.}
{TransA? Integer? An indicator specifying the multiplication.}
{$\alpha$? mpNum? A complex scalar.}
{A? mpNum[,]? A matrix of complex numbers.}
{B? mpNum[,]? A matrix of complex numbers.}
\end{mpFunctionsExtract}

\begin{mpFunctionsExtract}
\mpFunctionSix
{Rsyrk? mpNum? a rank-k update for a symmetric matrix.}
{Uplo? Integer? An indicator specifying whether the upper or lower triangle will be used.}
{Trans? Integer? An indicator specifying the multiplication.}
{$\alpha$? mpNum? A real scalar.}
{A? mpNum[,]? A matrix of real numbers.}
{$\beta$? mpNum? A real scalar.}
{C? mpNum[,]? A matrix of real numbers.}
\end{mpFunctionsExtract}

\begin{mpFunctionsExtract}
\mpFunctionSix
{cplxSyrk? mpNum? a rank-k update for a symmetric matrix.}
{Uplo? Integer? An indicator specifying whether the upper or lower triangle will be used.}
{Trans? Integer? An indicator specifying the multiplication.}
{$\alpha$? mpNum? A complex scalar.}
{A? mpNum[,]? A matrix of complex numbers.}
{$\beta$? mpNum? A complex scalar.}
{C? mpNum[,]? A matrix of complex numbers.}
\end{mpFunctionsExtract}

\begin{mpFunctionsExtract}
\mpFunctionSix
{cplxHerk? mpNum? a rank-k update for a hermitian matrix.}
{Uplo? Integer? An indicator specifying whether the upper or lower triangle will be used.}
{Trans? Integer? An indicator specifying the multiplication.}
{$\alpha$? mpNum? A complex scalar.}
{A? mpNum[,]? A matrix of complex numbers.}
{$\beta$? mpNum? A complex scalar.}
{C? mpNum[,]? A matrix of complex numbers.}
\end{mpFunctionsExtract}

\begin{mpFunctionsExtract}
\mpFunctionSeven
{Rsyr2k? mpNum? a rank-k update for a symmetric matrix.}
{Uplo? Integer? An indicator specifying whether the upper or lower triangle will be used.}
{Trans? Integer? An indicator specifying the multiplication.}
{$\alpha$? mpNum? A real scalar.}
{A? mpNum[,]? A matrix of real numbers.}
{B? mpNum[,]? A matrix of real numbers.}
{$\beta$? mpNum? A real scalar.}
{C? mpNum[,]? A matrix of real numbers.}
\end{mpFunctionsExtract}

\begin{mpFunctionsExtract}
\mpFunctionSeven
{cplxSyr2k? mpNum? a rank-k update for a symmetric matrix.}
{Uplo? Integer? An indicator specifying whether the upper or lower triangle will be used.}
{Trans? Integer? An indicator specifying the multiplication.}
{$\alpha$? mpNum? A complex scalar.}
{A? mpNum[,]? A matrix of complex numbers.}
{B? mpNum[,]? A matrix of complex numbers.}
{$\beta$? mpNum? A complex scalar.}
{C? mpNum[,]? A matrix of complex numbers.}
\end{mpFunctionsExtract}

\begin{mpFunctionsExtract}
\mpFunctionSeven
{cplxHer2k? mpNum? a rank-k update for a hermitian matrix.}
{Uplo? Integer? An indicator specifying whether the upper or lower triangle will be used.}
{Trans? Integer? An indicator specifying the multiplication.}
{$\alpha$? mpNum? A complex scalar.}
{A? mpNum[,]? A matrix of complex numbers.}
{B? mpNum[,]? A matrix of complex numbers.}
{$\beta$? mpNum? A complex scalar.}
{C? mpNum[,]? A matrix of complex numbers.}
\end{mpFunctionsExtract}

\chapter{Linear Solvers (based on Eigen)}

\section{Cholesky Decomposition without Pivoting}

\begin{mpFunctionsExtract}
\mpFunctionFour
{DecompCholeskyLLT? mpNumList? the Cholesky decomposition $A = LL^* = U^*U$ of a matrix.}
{A? mpNum[,]? the real matrix of which we are computing the $LL^T$ Cholesky decomposition.}
{B? mpNum[,]? A vector or matrix of real numbers.}
{UpLo? Integer? the triangular part that will be used for the decompositon: Lower (default) or Upper. The other triangular part won't be read.}
{Output? String? A string specifying the output options.}
\end{mpFunctionsExtract}

\begin{mpFunctionsExtract}
\mpFunctionFour
{cplxDecompCholeskyLLT? mpNumList? the Cholesky decomposition $A = LL^* = U^*U$ of a matrix.}
{A? mpNum[,]? the complex matrix of which we are computing the $LL^T$ Cholesky decomposition.}
{B? mpNum[,]? A vector or complex of real numbers.}
{UpLo? Integer? the triangular part that will be used for the decompositon: Lower (default) or Upper. The other triangular part won't be read.}
{Output? String? A string specifying the output options.}
\end{mpFunctionsExtract}

\begin{mpFunctionsExtract}
\mpFunctionThree
{SolveCholeskyLLT? mpNum[]? the solution $x$ of $A x = b$ , based on a Cholesky decomposition.}
{A? mpNum[,]? A symmetric positive definite real matrix.}
{B? mpNum[,]? A real vector or matrix.}
{UpLo? Integer? the triangular part that will be used for the decompositon: Lower (default) or Upper. The other triangular part won't be read.}
\end{mpFunctionsExtract}

\begin{mpFunctionsExtract}
\mpFunctionThree
{cplxSolveCholeskyLLT? mpNum[]? the solution $x$ of $A x = b$ , based on a Cholesky decomposition.}
{A? mpNum[,]? A symmetric positive definite complex matrix.}
{B? mpNum[,]? A complex vector or matrix.}
{UpLo? Integer? the triangular part that will be used for the decompositon: Lower (default) or Upper. The other triangular part won't be read.}
\end{mpFunctionsExtract}

\begin{mpFunctionsExtract}
\mpFunctionTwo
{InvertCholeskyLLT? mpNum[]? $A^{-1}$, the inverse of $A$, based on a Cholesky decomposition.}
{A? mpNum[,]? A symmetric positive definite real matrix.}
{UpLo? Integer? the triangular part that will be used for the decompositon: Lower (default) or Upper. The other triangular part won't be read.}
\end{mpFunctionsExtract}

\begin{mpFunctionsExtract}
\mpFunctionTwo
{cplxInvertCholeskyLLT? mpNum[]? $A^{-1}$, the inverse of $A$, based on a Cholesky decomposition.}
{A? mpNum[,]? A symmetric positive definite complex matrix.}
{UpLo? Integer? the triangular part that will be used for the decompositon: Lower (default) or Upper. The other triangular part won't be read.}
\end{mpFunctionsExtract}

\begin{mpFunctionsExtract}
\mpFunctionTwo
{DetCholeskyLLT? mpNum? $|A|$, the determinant of $A$, based on a Cholesky decomposition.}
{A? mpNum[,]? A symmetric positive definite real matrix.}
{UpLo? Integer? the triangular part that will be used for the decompositon: Lower (default) or Upper. The other triangular part won't be read.}
\end{mpFunctionsExtract}

\begin{mpFunctionsExtract}
\mpFunctionTwo
{cplxDetCholeskyLLT? mpNum? $|A|$, the determinant of $A$, based on a Cholesky decomposition.}
{A? mpNum[,]? A symmetric positive definite complex matrix.}
{UpLo? Integer? the triangular part that will be used for the decompositon: Lower (default) or Upper. The other triangular part won't be read.}
\end{mpFunctionsExtract}

\section{Cholesky Decomposition with Pivoting}

\begin{mpFunctionsExtract}
\mpFunctionFour
{DecompCholeskyLDLT? mpNumList? the Cholesky decomposition with pivoting of $A = LL^* = U^*U$.}
{A? mpNum[,]? the real matrix of which we are computing the $LL^T$ Cholesky decomposition.}
{B? mpNum[,]? A vector or matrix of real numbers.}
{UpLo? Integer? the triangular part that will be used for the decompositon: Lower (default) or Upper. The other triangular part won't be read.}
{Output? String? A string specifying the output options.}
\end{mpFunctionsExtract}

\begin{mpFunctionsExtract}
\mpFunctionFour
{cplxDecompCholeskyLDLT? mpNumList? the Cholesky decomposition with pivoting of $A = LL^* = U^*U$.}
{A? mpNum[,]? the complex matrix of which we are computing the $LL^T$ Cholesky decomposition.}
{B? mpNum[,]? A vector or complex of real numbers.}
{UpLo? Integer? the triangular part that will be used for the decompositon: Lower (default) or Upper. The other triangular part won't be read.}
{Output? String? A string specifying the output options.}
\end{mpFunctionsExtract}

\begin{mpFunctionsExtract}
\mpFunctionThree
{SolveCholeskyLDLT? mpNum[]? the solution $x$ of $A x = b$ , based on a Cholesky decomposition with pivoting.}
{A? mpNum[,]? A symmetric positive definite real matrix.}
{B? mpNum[,]? A real vector or matrix.}
{UpLo? Integer? the triangular part that will be used for the decompositon: Lower (default) or Upper. The other triangular part won't be read.}
\end{mpFunctionsExtract}

\begin{mpFunctionsExtract}
\mpFunctionThree
{cplxSolveCholeskyLDLT? mpNum[]? the solution $x$ of $A x = b$ , based on a Cholesky decomposition with pivoting.}
{A? mpNum[,]? A symmetric positive definite complex matrix.}
{B? mpNum[,]? A complex vector or matrix.}
{UpLo? Integer? the triangular part that will be used for the decompositon: Lower (default) or Upper. The other triangular part won't be read.}
\end{mpFunctionsExtract}

\begin{mpFunctionsExtract}
\mpFunctionTwo
{InvertCholeskyLDLT? mpNum[]? $A^{-1}$, the inverse of $A$, based on a Cholesky decomposition with pivoting.}
{A? mpNum[,]? A symmetric positive definite real matrix.}
{UpLo? Integer? the triangular part that will be used for the decompositon: Lower (default) or Upper. The other triangular part won't be read.}
\end{mpFunctionsExtract}

\begin{mpFunctionsExtract}
\mpFunctionTwo
{cplxInvertCholeskyLDLT? mpNum[]? $A^{-1}$, the inverse of $A$, based on a Cholesky decomposition with pivoting.}
{A? mpNum[,]? A symmetric positive definite complex matrix.}
{UpLo? Integer? the triangular part that will be used for the decompositon: Lower (default) or Upper. The other triangular part won't be read.}
\end{mpFunctionsExtract}

\begin{mpFunctionsExtract}
\mpFunctionTwo
{DetCholeskyLDLT? mpNum? $|A|$, the determinant of $A$, based on a Cholesky decomposition.}
{A? mpNum[,]? A symmetric positive definite real matrix.}
{UpLo? Integer? the triangular part that will be used for the decompositon: Lower (default) or Upper. The other triangular part won't be read.}
\end{mpFunctionsExtract}

\begin{mpFunctionsExtract}
\mpFunctionTwo
{cplxDetCholeskyLDLT? mpNum? $|A|$, the determinant of $A$, based on a Cholesky decomposition.}
{A? mpNum[,]? A symmetric positive definite complex matrix.}
{UpLo? Integer? the triangular part that will be used for the decompositon: Lower (default) or Upper. The other triangular part won't be read.}
\end{mpFunctionsExtract}

\section{LU Decomposition with partial Pivoting}

\begin{mpFunctionsExtract}
\mpFunctionThree
{DecompPartialPivLU? mpNumList? the LU decomposition with partial pivoting of $A = PLU$.}
{A? mpNum[,]? the square real matrix of which we are computing the $LU$ decomposition.}
{B? mpNum[,]? A vector or matrix of real numbers.}
{Output? String? A string specifying the output options.}
\end{mpFunctionsExtract}

\begin{mpFunctionsExtract}
\mpFunctionThree
{cplxDecompPartialPivLU? mpNumList? the LU decomposition with partial pivoting of $A = PLU$.}
{A? mpNum[,]? the square complex matrix of which we are computing the $LU$ decomposition.}
{B? mpNum[,]? A vector or complex of real numbers.}
{Output? String? A string specifying the output options.}
\end{mpFunctionsExtract}

\begin{mpFunctionsExtract}
\mpFunctionTwo
{SolvePartialPivLU? mpNum[]? the solution $x$ of $A x = b$ , based on a LU decomposition with partial pivoting.}
{A? mpNum[,]? A square real matrix.}
{B? mpNum[,]? A real vector or matrix.}
\end{mpFunctionsExtract}

\begin{mpFunctionsExtract}
\mpFunctionTwo
{cplxSolvePartialPivLU? mpNum[]? the solution $x$ of $A x = b$ , based on a LU decomposition with partial pivoting.}
{A? mpNum[,]? A square complex matrix.}
{B? mpNum[,]? A complex vector or matrix.}
\end{mpFunctionsExtract}

\begin{mpFunctionsExtract}
\mpFunctionOne
{InvertPartialPivLU? mpNum[]? $A^{-1}$, the inverse of $A$, based on a LU decomposition with partial pivoting.}
{A? mpNum[,]? A square real matrix.}
\end{mpFunctionsExtract}

\begin{mpFunctionsExtract}
\mpFunctionOne
{cplxInvertPartialPivLU? mpNum[]? $A^{-1}$, the inverse of $A$, based on a LU decomposition with partial pivoting.}
{A? mpNum[,]? A square complex matrix.}
\end{mpFunctionsExtract}

\begin{mpFunctionsExtract}
\mpFunctionOne
{DetPartialPivLU? mpNum? $|A|$, the determinant of $A$, based on a LU decomposition with partial pivoting.}
{A? mpNum[,]? A square real matrix.}
\end{mpFunctionsExtract}

\begin{mpFunctionsExtract}
\mpFunctionOne
{cplxDetPartialPivLU? mpNum? $|A|$, the determinant of $A$, based on a LU decomposition with partial pivoting.}
{A? mpNum[,]? A square complex matrix.}
\end{mpFunctionsExtract}

\section{LU Decomposition with full Pivoting}

\begin{mpFunctionsExtract}
\mpFunctionThree
{DecompFullPivLU? mpNumList? the LU decomposition with full pivoting of $A = PLUQ$.}
{A? mpNum[,]? the square real matrix of which we are computing the $LU$ decomposition.}
{B? mpNum[,]? A vector or matrix of real numbers.}
{Output? String? A string specifying the output options.}
\end{mpFunctionsExtract}

\begin{mpFunctionsExtract}
\mpFunctionThree
{cplxDecompFullPivLU? mpNumList? the LU decomposition with full pivoting of $A = PLUQ$.}
{A? mpNum[,]? the square complex matrix of which we are computing the $LU$ decomposition.}
{B? mpNum[,]? A vector or complex of real numbers.}
{Output? String? A string specifying the output options.}
\end{mpFunctionsExtract}

\begin{mpFunctionsExtract}
\mpFunctionTwo
{SolveFullPivLU? mpNum[]? the solution $x$ of $A x = b$ , based on a LU decomposition with full pivoting.}
{A? mpNum[,]? A square real matrix.}
{B? mpNum[,]? A real vector or matrix.}
\end{mpFunctionsExtract}

\begin{mpFunctionsExtract}
\mpFunctionTwo
{cplxSolveFullPivLU? mpNum[]? the solution $x$ of $A x = b$ , based on a LU decomposition with full pivoting.}
{A? mpNum[,]? A square complex matrix.}
{B? mpNum[,]? A complex vector or matrix.}
\end{mpFunctionsExtract}

\begin{mpFunctionsExtract}
\mpFunctionOne
{InvertFullPivLU? mpNum[]? $A^{-1}$, the inverse of $A$, based on a LU decomposition with full pivoting.}
{A? mpNum[,]? A square real matrix.}
\end{mpFunctionsExtract}

\begin{mpFunctionsExtract}
\mpFunctionOne
{cplxInvertFullPivLU? mpNum[]? $A^{-1}$, the inverse of $A$, based on a LU decomposition with full pivoting.}
{A? mpNum[,]? A square complex matrix.}
\end{mpFunctionsExtract}

\begin{mpFunctionsExtract}
\mpFunctionOne
{DetFullPivLU? mpNum? $|A|$, the determinant of $A$, based on a LU decomposition with full pivoting.}
{A? mpNum[,]? A square real matrix.}
\end{mpFunctionsExtract}

\begin{mpFunctionsExtract}
\mpFunctionOne
{cplxDetFullPivLU? mpNum? $|A|$, the determinant of $A$, based on a LU decomposition with full pivoting.}
{A? mpNum[,]? A square complex matrix.}
\end{mpFunctionsExtract}

\section{QR Decomposition without Pivoting}

\begin{mpFunctionsExtract}
\mpFunctionThree
{DecompQR? mpNumList? the QR decomposition without pivoting of $A = QR$.}
{A? mpNum[,]? the square real matrix of which we are computing the $LU$ decomposition.}
{B? mpNum[,]? A vector or matrix of real numbers.}
{Output? String? A string specifying the output options.}
\end{mpFunctionsExtract}

\begin{mpFunctionsExtract}
\mpFunctionThree
{cplxDecompQR? mpNumList? the QR decomposition without pivoting of $A = QR$.}
{A? mpNum[,]? the square complex matrix of which we are computing the $LU$ decomposition.}
{B? mpNum[,]? A vector or complex of real numbers.}
{Output? String? A string specifying the output options.}
\end{mpFunctionsExtract}

\begin{mpFunctionsExtract}
\mpFunctionTwo
{SolveQR? mpNum[]? the solution $x$ of $A x = b$ , based on a QR decomposition without pivoting.}
{A? mpNum[,]? A square real matrix.}
{B? mpNum[,]? A real vector or matrix.}
\end{mpFunctionsExtract}

\begin{mpFunctionsExtract}
\mpFunctionTwo
{cplxSolveQR? mpNum[]? the solution $x$ of $A x = b$ , based on a QR decomposition without pivoting.}
{A? mpNum[,]? A square complex matrix.}
{B? mpNum[,]? A complex vector or matrix.}
\end{mpFunctionsExtract}

\begin{mpFunctionsExtract}
\mpFunctionOne
{InvertQR? mpNum[]? $A^{-1}$, the inverse of $A$, based on a QR decomposition without pivoting.}
{A? mpNum[,]? A square real matrix.}
\end{mpFunctionsExtract}

\begin{mpFunctionsExtract}
\mpFunctionOne
{cplxInvertQR? mpNum[]? $A^{-1}$, the inverse of $A$, based on a QR decomposition without pivoting.}
{A? mpNum[,]? A square complex matrix.}
\end{mpFunctionsExtract}

\begin{mpFunctionsExtract}
\mpFunctionOne
{DetQR? mpNum? $|A|$, the determinant of $A$, based on a QR decomposition without pivoting.}
{A? mpNum[,]? A square real matrix.}
\end{mpFunctionsExtract}

\begin{mpFunctionsExtract}
\mpFunctionOne
{cplxDetQR? mpNum? $|A|$, the determinant of $A$, based on a QR decomposition without pivoting.}
{A? mpNum[,]? A square complex matrix.}
\end{mpFunctionsExtract}

\section{QR Decomposition with column Pivoting}

\begin{mpFunctionsExtract}
\mpFunctionThree
{DecompColPivQR? mpNumList? the QR decomposition with column-pivoting of $A = QR$.}
{A? mpNum[,]? the square real matrix of which we are computing the $LU$ decomposition.}
{B? mpNum[,]? A vector or matrix of real numbers.}
{Output? String? A string specifying the output options.}
\end{mpFunctionsExtract}

\begin{mpFunctionsExtract}
\mpFunctionThree
{cplxDecompColPivQR? mpNumList? the QR decomposition with column-pivoting of $A = QR$.}
{A? mpNum[,]? the square complex matrix of which we are computing the $LU$ decomposition.}
{B? mpNum[,]? A vector or complex of real numbers.}
{Output? String? A string specifying the output options.}
\end{mpFunctionsExtract}

\begin{mpFunctionsExtract}
\mpFunctionTwo
{SolveColPivQR? mpNum[]? the solution $x$ of $A x = b$ , based on a QR decomposition with column-pivoting.}
{A? mpNum[,]? A square real matrix.}
{B? mpNum[,]? A real vector or matrix.}
\end{mpFunctionsExtract}

\begin{mpFunctionsExtract}
\mpFunctionTwo
{cplxSolveColPivQR? mpNum[]? the solution $x$ of $A x = b$ , based on a QR decomposition with column-pivoting.}
{A? mpNum[,]? A square complex matrix.}
{B? mpNum[,]? A complex vector or matrix.}
\end{mpFunctionsExtract}

\begin{mpFunctionsExtract}
\mpFunctionOne
{InvertColPivQR? mpNum[]? $A^{-1}$, the inverse of $A$, based on a QR decomposition with column-pivoting.}
{A? mpNum[,]? A square real matrix.}
\end{mpFunctionsExtract}

\begin{mpFunctionsExtract}
\mpFunctionOne
{cplxInvertColPivQR? mpNum[]? $A^{-1}$, the inverse of $A$, based on a QR decomposition with column-pivoting.}
{A? mpNum[,]? A square complex matrix.}
\end{mpFunctionsExtract}

\begin{mpFunctionsExtract}
\mpFunctionOne
{DetColPivQR? mpNum? $|A|$, the determinant of $A$, based on a QR decomposition with column-pivoting.}
{A? mpNum[,]? A square real matrix.}
\end{mpFunctionsExtract}

\begin{mpFunctionsExtract}
\mpFunctionOne
{cplxDetColPivQR? mpNum? $|A|$, the determinant of $A$, based on a QR decomposition with column-pivoting.}
{A? mpNum[,]? A square complex matrix.}
\end{mpFunctionsExtract}

\section{QR Decomposition with full Pivoting}

\begin{mpFunctionsExtract}
\mpFunctionThree
{DecompFullPivQR? mpNumList? the QR decomposition with full pivoting of $AP = QR$.}
{A? mpNum[,]? the square real matrix of which we are computing the $LU$ decomposition.}
{B? mpNum[,]? A vector or matrix of real numbers.}
{Output? String? A string specifying the output options.}
\end{mpFunctionsExtract}

\begin{mpFunctionsExtract}
\mpFunctionThree
{cplxDecompFullPivQR? mpNumList? the QR decomposition with full pivoting of $AP = QR$.}
{A? mpNum[,]? the square complex matrix of which we are computing the $LU$ decomposition.}
{B? mpNum[,]? A vector or complex of real numbers.}
{Output? String? A string specifying the output options.}
\end{mpFunctionsExtract}

\begin{mpFunctionsExtract}
\mpFunctionTwo
{SolveFullPivQR? mpNum[]? the solution $x$ of $A x = b$ , based on a QR decomposition with full pivoting.}
{A? mpNum[,]? A square real matrix.}
{B? mpNum[,]? A real vector or matrix.}
\end{mpFunctionsExtract}

\begin{mpFunctionsExtract}
\mpFunctionTwo
{cplxSolveFullPivQR? mpNum[]? the solution $x$ of $A x = b$ , based on a QR decomposition with full pivoting.}
{A? mpNum[,]? A square complex matrix.}
{B? mpNum[,]? A complex vector or matrix.}
\end{mpFunctionsExtract}

\begin{mpFunctionsExtract}
\mpFunctionOne
{InvertFullPivQR? mpNum[]? $A^{-1}$, the inverse of $A$, based on a QR decomposition with full pivoting.}
{A? mpNum[,]? A square real matrix.}
\end{mpFunctionsExtract}

\begin{mpFunctionsExtract}
\mpFunctionOne
{cplxInvertFullPivQR? mpNum[]? $A^{-1}$, the inverse of $A$, based on a QR decomposition with full pivoting.}
{A? mpNum[,]? A square complex matrix.}
\end{mpFunctionsExtract}

\begin{mpFunctionsExtract}
\mpFunctionOne
{DetFullPivQR? mpNum? $|A|$, the determinant of $A$, based on a QR decomposition with full pivoting.}
{A? mpNum[,]? A square real matrix.}
\end{mpFunctionsExtract}

\begin{mpFunctionsExtract}
\mpFunctionOne
{cplxDetFullPivQR? mpNum? $|A|$, the determinant of $A$, based on a QR decomposition with full pivoting.}
{A? mpNum[,]? A square complex matrix.}
\end{mpFunctionsExtract}

\section{Singular Value Decomposition}

\begin{mpFunctionsExtract}
\mpFunctionFour
{DecompJacobiSVD? mpNumList? the Cholesky decomposition $A = LL^* = U^*U$ of a matrix.}
{A? mpNum[,]? the real matrix of which we are computing the $LL^T$ Cholesky decomposition.}
{B? mpNum[,]? A vector or matrix of real numbers.}
{computationOptions? Integer? An optional parameter allowing to specify if you want full or thin U or V unitaries to be computed.}
{Output? String? A string specifying the output options.}
\end{mpFunctionsExtract}

\begin{mpFunctionsExtract}
\mpFunctionFour
{cplxDecompJacobiSVD? mpNumList? the Cholesky decomposition $A = LL^* = U^*U$ of a matrix.}
{A? mpNum[,]? the complex matrix of which we are computing the $LL^T$ Cholesky decomposition.}
{B? mpNum[,]? A vector or complex of real numbers.}
{computationOptions? Integer? An optional parameter allowing to specify if you want full or thin U or V unitaries to be computed.}
{Output? String? A string specifying the output options.}
\end{mpFunctionsExtract}

\begin{mpFunctionsExtract}
\mpFunctionTwo
{SolveJacobiSVD? mpNum[]? the solution $x$ of $A x = b$ , based on a singular value decomposition.}
{A? mpNum[,]? A symmetric positive definite real matrix.}
{B? mpNum[,]? A real vector or matrix.}
\end{mpFunctionsExtract}

\begin{mpFunctionsExtract}
\mpFunctionTwo
{cplxSolveJacobiSVD? mpNum[]? the solution $x$ of $A x = b$ , based on a singular value decomposition.}
{A? mpNum[,]? A symmetric positive definite complex matrix.}
{B? mpNum[,]? A complex vector or matrix.}
\end{mpFunctionsExtract}

\begin{mpFunctionsExtract}
\mpFunctionOne
{InvertJacobiSVD? mpNum[]? $A^{-1}$, the inverse of $A$, based on a singular value decomposition.}
{A? mpNum[,]? A square real matrix.}
\end{mpFunctionsExtract}

\begin{mpFunctionsExtract}
\mpFunctionOne
{cplxInvertJacobiSVD? mpNum[]? $A^{-1}$, the inverse of $A$, based on a singular value decomposition.}
{A? mpNum[,]? A square complex matrix.}
\end{mpFunctionsExtract}

\begin{mpFunctionsExtract}
\mpFunctionOne
{DetJacobiSVD? mpNum? $|A|$, the determinant of $A$, based on a singular value decomposition.}
{A? mpNum[,]? A square real matrix.}
\end{mpFunctionsExtract}

\begin{mpFunctionsExtract}
\mpFunctionOne
{cplxDetJacobiSVD? mpNum? $|A|$, the determinant of $A$, based on a singular value decomposition.}
{A? mpNum[,]? A square complex matrix.}
\end{mpFunctionsExtract}

\section{Householder Transformations}

\chapter{Eigensystems, (based on Eigen)}

\section{Symmetric/Hermitian Eigensystems}

\begin{mpFunctionsExtract}
\mpFunctionOne
{EigenSymm? mpNum? the eigenvalues of a real symmetric matrix.}
{A? mpNum[,]? the real matrix of which we are computing the eigenvalues.}
\end{mpFunctionsExtract}

\begin{mpFunctionsExtract}
\mpFunctionOne
{EigenSymmv? mpNum? the eigenvalues and eigenvectors of a real symmetric matrix.}
{A? mpNum[,]? the real matrix of which we are computing the eigenvalues.}
\end{mpFunctionsExtract}

\begin{mpFunctionsExtract}
\mpFunctionOne
{MatSymmInverseSqrt? mpNum? the inverse matrix square root of a real symmetric matrix.}
{A? mpNum[,]? the real matrix of which we are computing the eigenvalues.}
\end{mpFunctionsExtract}

\begin{mpFunctionsExtract}
\mpFunctionOne
{MatSymmSqrt? mpNum? the matrix square root of a real symmetric matrix.}
{A? mpNum[,]? the real matrix of which we are computing the eigenvalues.}
\end{mpFunctionsExtract}

\begin{mpFunctionsExtract}
\mpFunctionOne
{cplxEigenHerm? mpNum? the eigenvalues of a complex hermitian matrix.}
{A? mpNum[,]? the complex hermitian  matrix of which we are computing the eigenvalues.}
\end{mpFunctionsExtract}

\begin{mpFunctionsExtract}
\mpFunctionOne
{cplxEigenHermv? mpNum? the eigenvalues and eigenvectors of a complex hermitian matrix.}
{A? mpNum[,]? the complex hermitian  matrix of which we are computing the eigenvalues.}
\end{mpFunctionsExtract}

\section{General (Nonsymmetric) Eigensystems}

\begin{mpFunctionsExtract}
\mpFunctionOne
{EigenNonsymm? mpNum[]? the eigenvalues of a real general (non-symmetric) matrix.}
{A? mpNum[,]? the real general (non-symmetric) matrix of which we are computing the eigenvalues.}
\end{mpFunctionsExtract}

\begin{mpFunctionsExtract}
\mpFunctionOne
{EigenNonsymmv? mpNumList[2]? the eigenvalues and eigenvectors of a real general (non-symmetric) matrix.}
{A? mpNum[,]? the real general (non-symmetric) matrix of which we are computing the eigenvalues.}
\end{mpFunctionsExtract}

\begin{mpFunctionsExtract}
\mpFunctionOne
{PseudoEigenNonsymm? mpNum[]? the pseudoeigenvalues of a real general (non-symmetric) matrix.}
{A? mpNum[,]? the real general (non-symmetric) matrix of which we are computing the pseudoeigenvalues.}
\end{mpFunctionsExtract}

\begin{mpFunctionsExtract}
\mpFunctionOne
{PseudoEigenNonsymmv? mpNumList[2]? the pseudoeigenvalues and pseudoeigenvectors of a real general (non-symmetric) matrix.}
{A? mpNum[,]? the real general (non-symmetric) matrix of which we are computing the pseudoeigenvalues and pseudoeigenvectors.}
\end{mpFunctionsExtract}

\begin{mpFunctionsExtract}
\mpFunctionOne
{cplxEigenNonsymm? mpNum[]? the eigenvalues of a complex general (non-symmetric) matrix.}
{A? mpNum[,]? the complex general (non-symmetric) matrix of which we are computing the eigenvalues.}
\end{mpFunctionsExtract}

\begin{mpFunctionsExtract}
\mpFunctionOne
{cplxEigenNonsymmv? mpNumList[2]? the eigenvalues and eigenvectors of a complex general (non-symmetric) matrix.}
{A? mpNum[,]? the complex general (non-symmetric) matrix of which we are computing the eigenvalues.}
\end{mpFunctionsExtract}

\section{Generalized Eigensystems}

\begin{mpFunctionsExtract}
\mpFunctionTwo
{EigenGensymm? mpNum? the eigenvalues of a real Generalized Symmetric-Definite Eigensystem.}
{A? mpNum[,]? Selfadjoint matrix in matrix pencil. Only the lower triangular part of the matrix is referenced.}
{B? mpNum[,]? Positive-definite matrix in matrix pencil. Only the lower triangular part of the matrix is referenced.}
\end{mpFunctionsExtract}

\begin{mpFunctionsExtract}
\mpFunctionTwo
{EigenGensymmv? mpNum? the eigenvalues and eigenvectors of a real Generalized Symmetric-Definite Eigensystem.}
{A? mpNum[,]? Selfadjoint matrix in matrix pencil. Only the lower triangular part of the matrix is referenced.}
{B? mpNum[,]? Positive-definite matrix in matrix pencil. Only the lower triangular part of the matrix is referenced.}
\end{mpFunctionsExtract}

\begin{mpFunctionsExtract}
\mpFunctionTwo
{cplxEigenGenherm? mpNum? the eigenvalues of a Complex Hermitian Generalized Symmetric-Definite Eigensystem.}
{A? mpNum[,]? Selfadjoint matrix in matrix pencil. Only the lower triangular part of the matrix is referenced.}
{B? mpNum[,]? Positive-definite matrix in matrix pencil. Only the lower triangular part of the matrix is referenced.}
\end{mpFunctionsExtract}

\begin{mpFunctionsExtract}
\mpFunctionTwo
{cplxEigenGenhermv? mpNum? the eigenvalues and eigenvectors of a Complex Hermitian Generalized Symmetric-Definite Eigensystem.}
{A? mpNum[,]? Selfadjoint matrix in matrix pencil. Only the lower triangular part of the matrix is referenced.}
{B? mpNum[,]? Positive-definite matrix in matrix pencil. Only the lower triangular part of the matrix is referenced.}
\end{mpFunctionsExtract}

\begin{mpFunctionsExtract}
\mpFunctionTwo
{EigenGenNonsymm? mpNum? the eigenvalues of a real Generalized Non-Symmetric Eigensystem.}
{A? mpNum[,]? Selfadjoint matrix in matrix pencil. Only the lower triangular part of the matrix is referenced.}
{B? mpNum[,]? Positive-definite matrix in matrix pencil. Only the lower triangular part of the matrix is referenced.}
\end{mpFunctionsExtract}

\begin{mpFunctionsExtract}
\mpFunctionTwo
{EigenGenNonsymmv? mpNum? the eigenvalues and eigenvectors of a real Generalized Non-Symmetric Eigensystem.}
{A? mpNum[,]? Selfadjoint matrix in matrix pencil. Only the lower triangular part of the matrix is referenced.}
{B? mpNum[,]? Positive-definite matrix in matrix pencil. Only the lower triangular part of the matrix is referenced.}
\end{mpFunctionsExtract}

\section{Decompositions}

\section{Matrix Functions}

\begin{mpFunctionsExtract}
\mpFunctionOne
{MatSqrt? mpNum? an expression representing the matrix square root of the real matrix M.}
{M? mpNum[,]? the real matrix of which we are computing the matrix square root.}
\end{mpFunctionsExtract}

\begin{mpFunctionsExtract}
\mpFunctionOne
{cplxMatSqrt? mpNum? an expression representing the matrix square root of the complex matrix M.}
{M? mpNum[,]? the complex matrix of which we are computing the matrix square root.}
\end{mpFunctionsExtract}

\begin{mpFunctionsExtract}
\mpFunctionOne
{MatExp? mpNum? an expression representing the matrix exponential of the real matrix M.}
{M? mpNum[,]? the real matrix of which we are computing the matrix exponential.}
\end{mpFunctionsExtract}

\begin{mpFunctionsExtract}
\mpFunctionOne
{cplxMatExp? mpNum? an expression representing the matrix exponential of the complex matrix M.}
{M? mpNum[,]? the complex matrix of which we are computing the matrix exponential.}
\end{mpFunctionsExtract}

\begin{mpFunctionsExtract}
\mpFunctionOne
{MatLog? mpNum? an expression representing the matrix logarithm of the real matrix M.}
{M? mpNum[,]? the real matrix of which we are computing the matrix logarithm.}
\end{mpFunctionsExtract}

\begin{mpFunctionsExtract}
\mpFunctionOne
{cplxMatLog? mpNum? an expression representing the matrix logarithm of the complex matrix M.}
{M? mpNum[,]? the complex matrix of which we are computing the matrix logarithm.}
\end{mpFunctionsExtract}

\begin{mpFunctionsExtract}
\mpFunctionTwo
{MatPow? mpNum? an expression representing the matrix power of the real matrix M.}
{M? mpNum[,]? M base of the matrix power, should be a square matrix.}
{p? mpNum? exponent of the matrix power, should be real.}
\end{mpFunctionsExtract}

\begin{mpFunctionsExtract}
\mpFunctionTwo
{cplxMatPow? mpNum? an expression representing the matrix power of the complex matrix M.}
{M? mpNum[,]? M base of the matrix power, should be a square matrix.}
{p? mpNum? exponent of the matrix power, should be real.}
\end{mpFunctionsExtract}

\begin{mpFunctionsExtract}
\mpFunctionTwo
{MatGeneralFunction? mpNum? an expression representing f applied to the real matrix M.}
{M? mpNum[,]? argument of matrix function, should be a square matrix.}
{f? mpFunction? f an entire function; f(x,n) should compute the n-th derivative of f at x.}
\end{mpFunctionsExtract}

\begin{mpFunctionsExtract}
\mpFunctionTwo
{cplxMatGeneralFunction? mpNum? an expression representing f applied to the complex matrix M.}
{M? mpNum[,]? argument of matrix function, should be a square matrix.}
{f? mpFunction? f an entire function; f(x,n) should compute the n-th derivative of f at x.}
\end{mpFunctionsExtract}

\begin{mpFunctionsExtract}
\mpFunctionOne
{MatSin? mpNum? an expression representing the matrix sine of the real matrix M.}
{M? mpNum[,]? the real matrix of which we are computing the matrix sine.}
\end{mpFunctionsExtract}

\begin{mpFunctionsExtract}
\mpFunctionOne
{cplxMatSin? mpNum? an expression representing the matrix sine of the complex matrix M.}
{M? mpNum[,]? the complex matrix of which we are computing the matrix sine.}
\end{mpFunctionsExtract}

\begin{mpFunctionsExtract}
\mpFunctionOne
{MatCos? mpNum? an expression representing the matrix cosine of the real matrix M.}
{M? mpNum[,]? the real matrix of which we are computing the matrix cosine.}
\end{mpFunctionsExtract}

\begin{mpFunctionsExtract}
\mpFunctionOne
{cplxMatCos? mpNum? an expression representing the matrix cosine of the complex matrix M.}
{M? mpNum[,]? the complex matrix of which we are computing the matrix cosine.}
\end{mpFunctionsExtract}

\begin{mpFunctionsExtract}
\mpFunctionOne
{MatSinh? mpNum? an expression representing the matrix hyperbolic sine of the real matrix M.}
{M? mpNum[,]? the real matrix of which we are computing the matrix hyperbolic sine.}
\end{mpFunctionsExtract}

\begin{mpFunctionsExtract}
\mpFunctionOne
{cplxMatSinh? mpNum? an expression representing the matrix hyperbolic sine of the complex matrix M.}
{M? mpNum[,]? the complex matrix of which we are computing the matrix hyperbolic sine.}
\end{mpFunctionsExtract}

\begin{mpFunctionsExtract}
\mpFunctionOne
{MatCosh? mpNum? an expression representing the matrix hyperbolic cosine of the real matrix M.}
{M? mpNum[,]? the real matrix of which we are computing the matrix hyperbolic cosine.}
\end{mpFunctionsExtract}

\begin{mpFunctionsExtract}
\mpFunctionOne
{cplxMatCosh? mpNum? an expression representing the matrix hyperbolic cosine of the complex matrix M.}
{M? mpNum[,]? the complex matrix of which we are computing the matrix hyperbolic cosine.}
\end{mpFunctionsExtract}

\chapter{Polynomials (based on Eigen)}

\section{Polynomial Evaluation}

\begin{mpFunctionsExtract}
\mpFunctionTwo
{PolynomialEvaluation? mpNum? the value of a polynomial for the real variable $x$ with real coefficients $c$.}
{x? mpNum? A real number.}
{c? mpNum[]? A vector of real coefficients.}
\end{mpFunctionsExtract}

\begin{mpFunctionsExtract}
\mpFunctionTwo
{cplxPolynomialEvaluation? mpNum? the value of a polynomial for the complex variable $z$ with complex coefficients $c$.}
{z? mpNum? A complex number.}
{c? mpNum[]? A vector of complex coefficients.}
\end{mpFunctionsExtract}

\section{Quadratic Equations}

\begin{mpFunctionsExtract}
\mpFunctionThree
{QuadraticEquation? mpNum[]? a real vector containing the real roots of the quadratic equation.}
{a? mpNum? A real number.}
{b? mpNum? A real number.}
{c? mpNum? A real number.}
\end{mpFunctionsExtract}

\begin{mpFunctionsExtract}
\mpFunctionThree
{cplxQuadraticEquation? mpNum[]? a complex vector containing the complex roots of the quadratic equation.}
{a? mpNum? A real or complex number.}
{b? mpNum? A real or complex number.}
{c? mpNum? A real or complex number.}
\end{mpFunctionsExtract}

\section{Cubic Equations}

\begin{mpFunctionsExtract}
\mpFunctionFour
{CubicEquation? mpNum[]? a real vector containing the real roots of the cubic equation.}
{a? mpNum? A real number.}
{b? mpNum? A real number.}
{c? mpNum? A real number.}
{d? mpNum? A real number.}
\end{mpFunctionsExtract}

\begin{mpFunctionsExtract}
\mpFunctionFour
{cplxCubicEquation? mpNum[]? a complex vector containing the complex roots of the cubic equation.}
{a? mpNum? A real or complex number.}
{b? mpNum? A real or complex number.}
{c? mpNum? A real or complex number.}
{d? mpNum? A real or complex number.}
\end{mpFunctionsExtract}

\section{Quartic Equations}

\begin{mpFunctionsExtract}
\mpFunctionFive
{QuarticEquation? mpNum[]? a real vector containing the real roots of the quartic equation.}
{a? mpNum? A real number.}
{b? mpNum? A real number.}
{c? mpNum? A real number.}
{d? mpNum? A real number.}
{e? mpNum? A real number.}
\end{mpFunctionsExtract}

\begin{mpFunctionsExtract}
\mpFunctionFive
{cplxQuarticEquation? mpNum[]? a complex vector containing the complex roots of the quartic equation.}
{a? mpNum? A real or complex number.}
{b? mpNum? A real or complex number.}
{c? mpNum? A real or complex number.}
{d? mpNum? A real or complex number.}
{e? mpNum? A real or complex number.}
\end{mpFunctionsExtract}

\section{General Polynomial Equations}

\begin{mpFunctionsExtract}
\mpFunctionOne
{GeneralPolynomialEquation? mpNum[]? a real vector containing the real roots of the general real polynomial.}
{a? mpNum[]? The real coefficients of the polynomial.}
\end{mpFunctionsExtract}

\begin{mpFunctionsExtract}
\mpFunctionOne
{cplxGeneralPolynomialEquation? mpNum[]? a complex vector containing the complex roots of the general complex polynomial.}
{c? mpNum[]? The complex coefficients of the polynomial.}
\end{mpFunctionsExtract}

\chapter{Fast Fourier Transform (based on Eigen)}

\section{Discrete Fourier Transforms}

\begin{mpFunctionsExtract}
\mpFunctionOne
{FFTW\_FORWARD? mpNum[]? a complex vector containing the forward complex discrete Fourier transform of $X$.}
{X? mpNum[]? A complex vector.}
\end{mpFunctionsExtract}

\begin{mpFunctionsExtract}
\mpFunctionOne
{FFTW\_BACKWARD? mpNum[]? a complex vector containing the backward complex discrete Fourier transform of $X$.}
{X? mpNum[]? A complex vector.}
\end{mpFunctionsExtract}

\begin{mpFunctionsExtract}
\mpFunctionOne
{FFTW\_R2C? mpNum[]? a complex vector containing the forward complex discrete Fourier transform of $X$.}
{X? mpNum[]? A real vector.}
\end{mpFunctionsExtract}

\begin{mpFunctionsExtract}
\mpFunctionOne
{FFTW\_C2R? mpNum[]? a real vector containing the backward  discrete Fourier transform of $X$.}
{X? mpNum[]? A complex hermitian vector.}
\end{mpFunctionsExtract}

\begin{mpFunctionsExtract}
\mpFunctionOne
{FFTW\_REDFT00? mpNum[]? a real vector containing the REDFT00 transform (type-I DCT) transform of $X$.}
{X? mpNum[]? A real vector.}
\end{mpFunctionsExtract}

\begin{mpFunctionsExtract}
\mpFunctionOne
{FFTW\_REDFT10? mpNum[]? a real vector containing the REDFT10 transform (type-II DCT) transform of $X$.}
{X? mpNum[]? A real vector.}
\end{mpFunctionsExtract}

\begin{mpFunctionsExtract}
\mpFunctionOne
{FFTW\_REDFT01? mpNum[]? a real vector containing the REDFT01 transform (type-III DCT) transform of $X$.}
{X? mpNum[]? A real vector.}
\end{mpFunctionsExtract}

\begin{mpFunctionsExtract}
\mpFunctionOne
{FFTW\_REDFT11? mpNum[]? a real vector containing the REDFT11 transform (type-IV DCT) transform of $X$.}
{X? mpNum[]? A real vector.}
\end{mpFunctionsExtract}

\begin{mpFunctionsExtract}
\mpFunctionOne
{FFTW\_RODFT00? mpNum[]? a real vector containing the RODFT00 transform (type-I DST) transform of $X$.}
{X? mpNum[]? A real vector.}
\end{mpFunctionsExtract}

\begin{mpFunctionsExtract}
\mpFunctionOne
{FFTW\_RODFT10? mpNum[]? a real vector containing the RODFT10 transform (type-II DST) transform of $X$.}
{X? mpNum[]? A real vector.}
\end{mpFunctionsExtract}

\begin{mpFunctionsExtract}
\mpFunctionOne
{FFTW\_RODFT01? mpNum[]? a real vector containing the RODFT01 transform (type-III DST) transform of $X$.}
{X? mpNum[]? A real vector.}
\end{mpFunctionsExtract}

\begin{mpFunctionsExtract}
\mpFunctionOne
{FFTW\_RODFT11? mpNum[]? a real vector containing the RODFT11 transform (type-IV DST) transform of $X$.}
{X? mpNum[]? A real vector.}
\end{mpFunctionsExtract}

\chapter{RandomNumbers}

\section{Definitions}

\section{The Random Number Generator Interface}

\section{Random number generator algorithms}

\begin{mpFunctionsExtract}
\mpFunctionOne
{SaveDefaultRngState? Boolean? a boolean value: TRUE if the state was successfully save, FALSE otherwise}
{FName? String? A String, containing the full path of the file.}
\end{mpFunctionsExtract}

\begin{mpFunctionsExtract}
\mpFunctionOne
{LoadDefaultRngState? Boolean? a boolean value: TRUE if the state was successfully loaded, FALSE otherwise}
{FName? String? A String, containing the full path of the file.}
\end{mpFunctionsExtract}

\section{Random number distributions}

\chapter{Special Functions (based on Boost)}

\section{Gamma and Beta Functions}

\begin{mpFunctionsExtract}
\mpFunctionOne
{TgammaBoost? mpNum? the gamma function for $x \neq 0, -1, -2,\ldots$.}
{x? mpNum? A real number.}
\end{mpFunctionsExtract}

\begin{mpFunctionsExtract}
\mpFunctionOne
{LgammaBoost? mpNum? the logarithm of the gamma function.}
{x? mpNum? A real number.}
\end{mpFunctionsExtract}

\begin{mpFunctionsExtract}
\mpFunctionTwo
{TgammaDeltaRatioBoost? mpNum?  the ratio of gamma functions.}
{x? mpNum? A real number.}
{$\delta$? mpNum? A real number.}
\end{mpFunctionsExtract}

\begin{mpFunctionsExtract}
\mpFunctionOne
{DigammaBoost? mpNum? the digamma function for $x \neq 0, -1, -2,\ldots$.}
{x? mpNum? A real number.}
\end{mpFunctionsExtract}

\begin{mpFunctionsExtract}
\mpFunctionTwo
{TgammaratioBoost? mpNum?  the ratio of gamma functions.}
{a? mpNum? A real number.}
{b? mpNum? A real number.}
\end{mpFunctionsExtract}

\begin{mpFunctionsExtract}
\mpFunctionTwo
{GammaPBoost? mpNum? the normalised incomplete gamma function $P(a,x)$.}
{a? mpNum? A real number.}
{x? mpNum? A real number.}
\end{mpFunctionsExtract}

\begin{mpFunctionsExtract}
\mpFunctionTwo
{GammaQBoost? mpNum? the normalised incomplete gamma function $Q(a,x)$.}
{a? mpNum? A real number.}
{x? mpNum? A real number.}
\end{mpFunctionsExtract}

\begin{mpFunctionsExtract}
\mpFunctionTwo
{NonNormalisedGammaPBoost? mpNum? the non-normalised incomplete gamma function $\Gamma(a,x)$.}
{a? mpNum? A real number.}
{x? mpNum? A real number.}
\end{mpFunctionsExtract}

\begin{mpFunctionsExtract}
\mpFunctionTwo
{NonNormalisedGammaQBoost? mpNum? the non-normalised incomplete gamma function $\gamma(a,x)$.}
{a? mpNum? A real number.}
{x? mpNum? A real number.}
\end{mpFunctionsExtract}

\begin{mpFunctionsExtract}
\mpFunctionTwo
{GammaPinvBoost? mpNum? the inverse of the normalised incomplete gamma function $P(a,x)$.}
{a? mpNum? A real number.}
{p? mpNum? A real number.}
\end{mpFunctionsExtract}

\begin{mpFunctionsExtract}
\mpFunctionTwo
{GammaQinvBoost? mpNum? the inverse of the normalised incomplete gamma function $Q(a,x)$.}
{a? mpNum? A real number.}
{q? mpNum? A real number.}
\end{mpFunctionsExtract}

\begin{mpFunctionsExtract}
\mpFunctionTwo
{GammaPinvaBoost? mpNum? the parameter $a$ of the normalised incomplete gamma function $P(a,x)$, such that $P(a,x) = p$.}
{x? mpNum? A real number.}
{p? mpNum? A real number.}
\end{mpFunctionsExtract}

\begin{mpFunctionsExtract}
\mpFunctionTwo
{GammaQinvaBoost? mpNum? the parameter $a$ of the normalised incomplete gamma function $Q(a,x)$, such that $Q(a,x) = q$.}
{x? mpNum? A real number.}
{q? mpNum? A real number.}
\end{mpFunctionsExtract}

\begin{mpFunctionsExtract}
\mpFunctionTwo
{GammaPDerivativeBoost? mpNum? the partial derivative with respect to $x$ of the incomplete gamma function $P(a,x)$.}
{a? mpNum? A real number.}
{x? mpNum? A real number.}
\end{mpFunctionsExtract}

\section{Factorials and Binomial Coefficient}

\begin{mpFunctionsExtract}
\mpFunctionOne
{FactorialBoost? mpNum? the factorial $n! = \Gamma(n+1) = n \times (n-1) \times \cdots \times 1$.}
{n? mpNum? An integer.}
\end{mpFunctionsExtract}

\begin{mpFunctionsExtract}
\mpFunctionOne
{DoubleFactorialBoost? mpNum? the double factorial $n!!$.}
{n? mpNum? An integer.}
\end{mpFunctionsExtract}

\begin{mpFunctionsExtract}
\mpFunctionTwo
{RisingFactorialBoost? mpNum? the rising factorial of $x$ and $i$.}
{n? mpNum? An integer.}
{i? mpNum? An integer.}
\end{mpFunctionsExtract}

\begin{mpFunctionsExtract}
\mpFunctionTwo
{FallingFactorialBoost? mpNum? the falling factorial of $x$ and $i$.}
{n? mpNum? An integer.}
{i? mpNum? An integer.}
\end{mpFunctionsExtract}

\begin{mpFunctionsExtract}
\mpFunctionTwo
{BinomialCoefficientBoost? mpNum? the binomial coefficient.}
{n? mpNum? An integer.}
{k? mpNum? An integer.}
\end{mpFunctionsExtract}

\section{Beta Functions}

\begin{mpFunctionsExtract}
\mpFunctionTwo
{BetaBoost? mpNum? the beta function.}
{a? mpNum? A real number.}
{b? mpNum? A real number.}
\end{mpFunctionsExtract}

\section{Error Function and Related Functions}

\begin{mpFunctionsExtract}
\mpFunctionOne
{ErfBoost? mpNum? the value of the error function.}
{x? mpNum? A real number.}
\end{mpFunctionsExtract}

\begin{mpFunctionsExtract}
\mpFunctionOne
{ErfcBoost? mpNum? the value of the complementary error function.}
{x? mpNum? A real number.}
\end{mpFunctionsExtract}

\begin{mpFunctionsExtract}
\mpFunctionOne
{ErfInvBoost? mpNum? the functional inverse of $\text{erf}(x)$}
{x? mpNum? A real number.}
\end{mpFunctionsExtract}

\begin{mpFunctionsExtract}
\mpFunctionOne
{ErfcInvBoost? mpNum? the functional inverse of $\text{erfc}(x)$}
{x? mpNum? A real number.}
\end{mpFunctionsExtract}

\section{Polynomials}

\begin{mpFunctionsExtract}
\mpFunctionTwo
{LegendrePBoost? mpNum? $P_l(x)$, the Legendre functions of the first kind.}
{l? mpNum? An Integer.}
{x? mpNum? A real number.}
\end{mpFunctionsExtract}

\begin{mpFunctionsExtract}
\mpFunctionFour
{LegendrePNextBoost? mpNum? the Legendre function of the first kind of degree $l+1$, using the results for degree $l$ and $l-1$.}
{l? mpNum? An Integer. The degree of the last polynomial calculated.}
{x? mpNum? A real number. The abscissa value.}
{Pl? mpNum? A real number. The value of the polynomial evaluated at degree $l$.}
{Plm1? mpNum? A real number. The value of the polynomial evaluated at degree $l-1$.}
\end{mpFunctionsExtract}

\begin{mpFunctionsExtract}
\mpFunctionThree
{AssociatedLegendrePlmBoost? mpNum? $L^m_n (x)$, the associated Legendre polynomials of degree $l \geq 0$ and order $m \geq 0$.}
{l? mpNum? An Integer.}
{m? mpNum? An Integer.}
{x? mpNum? A real number.}
\end{mpFunctionsExtract}

\begin{mpFunctionsExtract}
\mpFunctionFive
{AssociatedLegendrePlmNextBoost? mpNum? the Legendre function of the first kind of degree $l+1$, using the results for degree $l$ and $l-1$.}
{l? mpNum? An Integer. The degree of the last polynomial calculated.}
{m? mpNum? An Integer. The order of the Associated Polynomial.}
{x? mpNum? A real number. The abscissa value.}
{Pl? mpNum? A real number. The value of the polynomial evaluated at degree $l$.}
{Plm1? mpNum? A real number. The value of the polynomial evaluated at degree $l-1$.}
\end{mpFunctionsExtract}

\begin{mpFunctionsExtract}
\mpFunctionTwo
{LegendreQBoost? mpNum? $Q_l(x)$, the Legendre functions of the second kind of degree $l \geq 0$ and $|x| \neq 1$.}
{l? mpNum? An Integer.}
{x? mpNum? A real number.}
\end{mpFunctionsExtract}

\begin{mpFunctionsExtract}
\mpFunctionTwo
{LaguerreLBoost? mpNum? $L_n (x)$, the Laguerre polynomials of degree $n \geq 0$.}
{n? mpNum? An Integer.}
{x? mpNum? A real number.}
\end{mpFunctionsExtract}

\begin{mpFunctionsExtract}
\mpFunctionFour
{LaguerreLNextBoost? mpNum? the Laguerre polynomial of the first kind of degree $n+1$, using the results for degree $n$ and $n-1$.}
{n? mpNum? An Integer. The degree of the last polynomial calculated.}
{x? mpNum? A real number. The abscissa value.}
{Ln? mpNum? A real number. The value of the polynomial evaluated at degree $n$.}
{Lnm1? mpNum? A real number. The value of the polynomial evaluated at degree $n-1$.}
\end{mpFunctionsExtract}

\begin{mpFunctionsExtract}
\mpFunctionThree
{AssociatedLaguerreBoost? mpNum? $L^m_n (x)$, the associated Laguerre polynomials of degree $n \geq 0$ and order $m \geq 0$.}
{n? mpNum? An Integer.}
{m? mpNum? An Integer.}
{x? mpNum? A real number.}
\end{mpFunctionsExtract}

\begin{mpFunctionsExtract}
\mpFunctionFive
{AssociatedLaguerreLNextBoost? mpNum? the associated Laguerre polynomial of the first kind of degree $n+1$, using the results for degree $n$ and $n-1$.}
{n? mpNum? An Integer. The degree of the last polynomial calculated.}
{m? mpNum? An Integer. The order of the Associated Polynomial.}
{x? mpNum? A real number. The abscissa value.}
{Ln? mpNum? A real number. The value of the polynomial evaluated at degree $n$.}
{Lnm1? mpNum? A real number. The value of the polynomial evaluated at degree $n-1$.}
\end{mpFunctionsExtract}

\begin{mpFunctionsExtract}
\mpFunctionTwo
{HermiteHBoost? mpNum? $H_n(x)$, the Hermite polynomial of degree $n \geq 0$.}
{n? mpNum? An Integer.}
{x? mpNum? A real number.}
\end{mpFunctionsExtract}

\begin{mpFunctionsExtract}
\mpFunctionFour
{HermiteHNextBoost? mpNum? the Hermite polynomial of degree $n+1$, using the results for degree $n$ and $n-1$.}
{n? mpNum? An Integer. The degree of the last polynomial calculated.}
{x? mpNum? A real number. The abscissa value.}
{Hn? mpNum? A real number. The value of the polynomial evaluated at degree $n$.}
{Hnm1? mpNum? A real number. The value of the polynomial evaluated at degree $n-1$.}
\end{mpFunctionsExtract}

\begin{mpFunctionsExtract}
\mpFunctionFour
{SphericalHarmonicBoost? mpNum? the real and imaginary parts of the spherical harmonic function $Y_{lm}(\theta, \phi)$.}
{l? mpNumList[2]? An Integer.}
{m? mpNum? An Integer.}
{$\theta$? mpNum? A real number.}
{$\phi$? mpNum? A real number.}
\end{mpFunctionsExtract}

\section{Bessel Functions of Real Order}

\begin{mpFunctionsExtract}
\mpFunctionTwo
{BesselJBoost? mpNum? $J_{\nu}(z)$, the Bessel function of the first kind of real order $\nu$.}
{x? mpNum? A real number.}
{$\nu$? mpNum? A real number.}
\end{mpFunctionsExtract}

\begin{mpFunctionsExtract}
\mpFunctionTwo
{BesselYBoost? mpNum? $Y_{\nu}(z)$, the Bessel function of the second kind of order $\nu$.}
{x? mpNum? A real number.}
{$\nu$? mpNum? A real number.}
\end{mpFunctionsExtract}

\section{Modified Bessel Functions of Real Order}

\begin{mpFunctionsExtract}
\mpFunctionTwo
{BesselIBoost? mpNum? the modified Bessel function $I_{\nu}(z)$ of the first kind of order $\nu$.}
{x? mpNum? A real number.}
{$\nu$? mpNum? A real number.}
\end{mpFunctionsExtract}

\begin{mpFunctionsExtract}
\mpFunctionTwo
{BesselKBoost? mpNum? $K_{\nu}(x)$, the modified Bessel function of the second kind of order $\nu$.}
{x? mpNum? A real number.}
{$\nu$? mpNum? A real number.}
\end{mpFunctionsExtract}

\section{Spherical Bessel Functions}

\begin{mpFunctionsExtract}
\mpFunctionTwo
{BesselSphericaljBoost? mpNum? $j_n(x)$, the spherical Bessel function of the 1st kind, order $n$.}
{x? mpNum? A real number.}
{$\nu$? mpNum? A real number.}
\end{mpFunctionsExtract}

\begin{mpFunctionsExtract}
\mpFunctionTwo
{BesselSphericalyBoost? mpNum? $j_n(x)$, the spherical Bessel function of the 1st kind, order $n$.}
{x? mpNum? A real number.}
{$\nu$? mpNum? A real number.}
\end{mpFunctionsExtract}

\section{Hankel Functions}

\begin{mpFunctionsExtract}
\mpFunctionTwo
{cplxHankel1Boost? mpNum? the Hankel function of the first kind $H_{\nu}^{(1)}(x)$.}
{x? mpNum? A real number.}
{$\nu$? mpNum? A real number.}
\end{mpFunctionsExtract}

\begin{mpFunctionsExtract}
\mpFunctionTwo
{cplxHankel2Boost? mpNum? the Hankel function of the second kind $H_{\nu}^{(2)}(x)$.}
{x? mpNum? A real number.}
{$\nu$? mpNum? A real number.}
\end{mpFunctionsExtract}

\begin{mpFunctionsExtract}
\mpFunctionTwo
{cplxHankelSph1Boost? mpNum? the spherical Hankel function of the first kind $h_{\nu}^{(1)}(x)$.}
{x? mpNum? A real number.}
{$\nu$? mpNum? A real number.}
\end{mpFunctionsExtract}

\begin{mpFunctionsExtract}
\mpFunctionTwo
{cplxHankelSph2Boost? mpNum? the spherical Hankel function of the second kind $h_{\nu}^{(2)}(x)$.}
{x? mpNum? A real number.}
{$\nu$? mpNum? A real number.}
\end{mpFunctionsExtract}

\section{Airy Functions}

\begin{mpFunctionsExtract}
\mpFunctionOne
{AiryAiBoost? mpNum? the Airy function $\text{Ai}(x)$.}
{x? mpNum? A real number.}
\end{mpFunctionsExtract}

\begin{mpFunctionsExtract}
\mpFunctionOne
{AiryAiDerivativeBoost? mpNum? the Airy function $\text{Ai}'(x)$.}
{x? mpNum? A real number.}
\end{mpFunctionsExtract}

\begin{mpFunctionsExtract}
\mpFunctionOne
{AiryBiBoost? mpNum? the Airy function $\text{Bi}(x)$.}
{x? mpNum? A real number.}
\end{mpFunctionsExtract}

\begin{mpFunctionsExtract}
\mpFunctionOne
{AiryBiDerivativeBoost? mpNum? the Airy function $\text{Bi}'(x)$.}
{x? mpNum? A real number.}
\end{mpFunctionsExtract}

\section{Carlson-style Elliptic Integrals}

\begin{mpFunctionsExtract}
\mpFunctionTwo
{CarlsonRCBoost? mpNum? the value of the of Carlson's degenerate elliptic integral $R_C$.}
{x? mpNum? A real number.}
{y? mpNum? A real number.}
\end{mpFunctionsExtract}

\begin{mpFunctionsExtract}
\mpFunctionThree
{CarlsonRFBoost? mpNum? the value of the of Carlson's elliptic integral $R_F$ of the first kind.}
{x? mpNum? A real number.}
{y? mpNum? A real number.}
{z? mpNum? A real number.}
\end{mpFunctionsExtract}

\begin{mpFunctionsExtract}
\mpFunctionThree
{CarlsonRDBoost? mpNum? the value of the of Carlson's elliptic integral $R_D$ of the second kind.}
{x? mpNum? A real number.}
{y? mpNum? A real number.}
{z? mpNum? A real number.}
\end{mpFunctionsExtract}

\begin{mpFunctionsExtract}
\mpFunctionFour
{CarlsonRJBoost? mpNum? the value of the of Carlson's elliptic integral $R_J$ of the third kind.}
{x? mpNum? A real number.}
{y? mpNum? A real number.}
{z? mpNum? A real number.}
{r? mpNum? A real number.}
\end{mpFunctionsExtract}

\section{Legendre-style Elliptic Integrals}

\begin{mpFunctionsExtract}

\mpFunctionOne
{CompleteLegendreEllint1Boost? mpNum? the value of the complete elliptic integral of the first kind.}
{k? mpNum? A real number.}
\end{mpFunctionsExtract}

\begin{mpFunctionsExtract}
\mpFunctionOne
{CompleteLegendreEllint2Boost? mpNum? the value of the complete elliptic integral of the second kind.}
{k? mpNum? A real number.}
\end{mpFunctionsExtract}

\begin{mpFunctionsExtract}
\mpFunctionTwo
{CompleteLegendreEllint3Boost? the value of the complete elliptic integral of the third kind.}
{$\nu$? mpNum? A real number.}
{k? mpNum? A real number.}
\end{mpFunctionsExtract}

\begin{mpFunctionsExtract}
\mpFunctionTwo
{LegendreEllint1Boost? mpNum? the value of the incomplete Legendre elliptic integral of the first kind.}
{$\phi$? mpNum? A real number.}
{k? mpNum? A real number.}
\end{mpFunctionsExtract}

\begin{mpFunctionsExtract}
\mpFunctionTwo
{LegendreEllint2Boost? mpNum? the value of the incomplete Legendre elliptic integral of the second kind.}
{$\phi$? mpNum? A real number.}
{k? mpNum? A real number.}
\end{mpFunctionsExtract}

\begin{mpFunctionsExtract}
\mpFunctionThree
{LegendreEllint3Boost? mpNum? the value of the incomplete Legendre elliptic integral of the third kind.}
{$\phi$? mpNum? A real number.}
{$\nu$? mpNum? A real number.}
{k? mpNum? A real number.}
\end{mpFunctionsExtract}

\section{Jacobi Elliptic Functions}

\begin{mpFunctionsExtract}
\mpFunctionTwo
{JacobiSNBoost? mpNum? the Jacobi elliptic function $\text{sn}(x, k)$.}
{x? mpNum? A real number.}
{k? mpNum? A real number.}
\end{mpFunctionsExtract}

\begin{mpFunctionsExtract}
\mpFunctionTwo
{JacobiCNBoost? mpNum? the Jacobi elliptic function $\text{cn}(x, k)$.}
{x? mpNum? A real number.}
{k? mpNum? A real number.}
\end{mpFunctionsExtract}

\begin{mpFunctionsExtract}
\mpFunctionTwo
{JacobiDNBoost? mpNum? the Jacobi elliptic function $\text{dn}(x, k)$.}
{x? mpNum? A real number.}
{k? mpNum? A real number.}
\end{mpFunctionsExtract}

\section{Zeta Functions}

\begin{mpFunctionsExtract}
\mpFunctionOne
{RiemannZetaBoost? mpNum? the Riemann zeta function.}
{s? mpNum? A real number.}
\end{mpFunctionsExtract}

\section{Exponential Integral and Related Integrals}

\begin{mpFunctionsExtract}
\mpFunctionOne
{ExponentialIntegralE1Boost? mpNum? the exponential integral $\text{E}_1(x)$.}
{x? mpNum? A real number.}
\end{mpFunctionsExtract}

\begin{mpFunctionsExtract}
\mpFunctionOne
{ExponentialIntegralEiBoost? mpNum? the exponential integral $\text{Ei}(x)$.}
{x? mpNum? A real number.}
\end{mpFunctionsExtract}

\begin{mpFunctionsExtract}
\mpFunctionTwo
{ExponentialIntegralEnBoost? mpNum? the exponential integral  $\text{E}_n(x)$.}
{x? mpNum? A real number.}
{n? mpNum? A real number.}
\end{mpFunctionsExtract}

\section{Basic Functions}

\begin{mpFunctionsExtract}
\mpFunctionOne
{SinPiBoost? mpNum? the value of the sine of $\pi x$, with $x$ in radians.}
{x? mpNum? A real number.}
\end{mpFunctionsExtract}

\begin{mpFunctionsExtract}
\mpFunctionOne
{CosPiBoost? mpNum? the value of the cosine of $\pi x$, with $x$ in radians.}
{x? mpNum? A real number.}
\end{mpFunctionsExtract}

\begin{mpFunctionsExtract}
\mpFunctionOne
{Lnp1Boost? mpNum? the value of the function $\ln(1+x)$.}
{x? mpNum? A real number.}
\end{mpFunctionsExtract}

\begin{mpFunctionsExtract}
\mpFunctionOne
{Expm1Boost? mpNum? the value of the function $\text{expm1}(x) = e^{x}-1$.}
{x? mpNum? A real number.}
\end{mpFunctionsExtract}

\begin{mpFunctionsExtract}
\mpFunctionOne
{CbrtBoost? mpNum? the absolute value of the cube root of $x, \sqrt[3]{x}$.}
{x? mpNum? A real number.}
\end{mpFunctionsExtract}

\begin{mpFunctionsExtract}
\mpFunctionOne
{Sqrtp1m1Boost? mpNum? the value of $\sqrt{x+1}-1$.}
{x? mpNum? A real number.}
\end{mpFunctionsExtract}

\begin{mpFunctionsExtract}
\mpFunctionTwo
{Powm1Boost? mpNum? the value of $x^y-1, y \in  \mathbb{R}$.}
{x? mpNum? A real number.}
{y? mpNum? A real number.}
\end{mpFunctionsExtract}

\begin{mpFunctionsExtract}
\mpFunctionTwo
{HypotBoost? mpNum? the value of $\sqrt{x^2+y^2}$.}
{x? mpNum? A real number.}
{y? mpNum? A real number.}
\end{mpFunctionsExtract}

\section{Sinus Cardinal Function and Hyperbolic Sinus Cardinal Functions}

\begin{mpFunctionsExtract}
\mpFunctionOne
{SincaBoost? mpNum? the sinus cardinal function}
{x? mpNum? A real number.}
\end{mpFunctionsExtract}

\begin{mpFunctionsExtract}
\mpFunctionOne
{SinhcaBoost? mpNum? the hyperbolic sinus cardinal function.}
{x? mpNum? A real number.}
\end{mpFunctionsExtract}

\section{Inverse Hyperbolic Functions}

\begin{mpFunctionsExtract}
\mpFunctionOne
{AcoshBoost? mpNum? the value of the hyperbolic arc-cosine  of $x$ in radians.}
{x? mpNum? A real number.}
\end{mpFunctionsExtract}

\begin{mpFunctionsExtract}
\mpFunctionOne
{AsinhBoost? mpNum? the value of the hyperbolic arc-sine  of $x$ in radians.}
{x? mpNum? A real number.}
\end{mpFunctionsExtract}

\begin{mpFunctionsExtract}
\mpFunctionOne
{AtanhBoost? mpNum? the value of the hyperbolic arc-tangent  of $x$ in radians.}
{x? mpNum? A real number.}
\end{mpFunctionsExtract}

\chapter{Distribution Functions}

\section{Introduction to Distribution Functions}

\section{Beta-Distribution}

\begin{mpFunctionsExtract}
\mpFunctionFour
{BetaDist? mpNumList? returns pdf, CDF and related information for the central Beta-distribution}
{x? mpNum? A real number}
{a? mpNum? A real number greater 0, representing the numerator  degrees of freedom}
{b? mpNum? A real number greater 0, representing the denominator degrees of freedom}
{Output? String? A string describing the output choices}
\end{mpFunctionsExtract}

\begin{mpFunctionsExtract}
\mpFunctionFour
{BetaDistInv? mpNumList? returns quantiles and related information for the the central Beta-distribution}
{Prob? mpNum? A real number between 0 and 1.}
{m? mpNum? A real number greater 0, representing the numerator  degrees of freedom}
{n? mpNum? A real number greater 0, representing the denominator degrees of freedom}
{Output? String? A string describing the output choices}
\end{mpFunctionsExtract}

\begin{mpFunctionsExtract}
\mpFunctionThree
{BetaDistInfo? mpNumList? returns moments and related information for the central Beta-distribution}
{a? mpNum? A real number greater 0, representing the degrees of freedom}
{b? mpNum? A real number greater 0, representing the degrees of freedom}
{Output? String? A string describing the output choices}
\end{mpFunctionsExtract}

\begin{mpFunctionsExtract}
\mpFunctionFive
{BetaDistRandom? mpNumList? returns random numbers following a central Beta-distribution}
{Size? mpNum? A positive integer up to $10^7$}
{a? mpNum? A real number greater 0, representing the numerator  degrees of freedom}
{b? mpNum? A real number greater 0, representing the denominator degrees of freedom}
{Generator? String? A string describing the random generator}
{Output? String? A string describing the output choices}
\end{mpFunctionsExtract}

\section{Binomial Distribution}

\begin{mpFunctionsExtract}
\mpFunctionFour
{BinomialDist? mpNumList? returns pdf, CDF and related information for the central Binomial-distribution}
{x? mpNum? The number of successes in trials.}
{n? mpNum? The number of independent trials.}
{p? mpNum? The probability of success on each trial}
{Output? String? A string describing the output choices}
\end{mpFunctionsExtract}

\begin{mpFunctionsExtract}
\mpFunctionFour
{BinomialDistInv? mpNumList? returns quantiles and related information for the the central binomial-distribution}
{Prob? mpNum? A real number between 0 and 1.}
{n? mpNum? The number of Bernoulli trials.}
{p? mpNum? The probability of a success on each trial.}
{Output? String? A string describing the output choices}
\end{mpFunctionsExtract}

\begin{mpFunctionsExtract}
\mpFunctionThree
{BinomialDistInfo? mpNumList? returns moments and related information for the central Binomial-distribution}
{n? mpNum? The number of Bernoulli trials.}
{p? mpNum? The probability of a success on each trial.}
{Output? String? A string describing the output choices}
\end{mpFunctionsExtract}

\begin{mpFunctionsExtract}
\mpFunctionFive
{BinomialDistRandom? mpNumList? returns random numbers following a central Binomial-distribution}
{Size? mpNum? A positive integer up to $10^7$}
{n? mpNum? The number of Bernoulli trials.}
{p? mpNum? The probability of a success on each trial.}
{Generator? String? A string describing the random generator}
{Output? String? A string describing the output choices}
\end{mpFunctionsExtract}

\section{Chi-Square Distribution}

\begin{mpFunctionsExtract}
\mpFunctionThree
{CDist? mpNumList? returns pdf, CDF and related information for the central $\chi^2$-distribution}
{x? mpNum? A real number}
{n? mpNum? A real number greater 0, representing the degrees of freedom}
{Output? String? A string describing the output choices}
\end{mpFunctionsExtract}

\begin{mpFunctionsExtract}
\mpFunctionThree
{CDistInv? mpNumList? quantiles and related information for the the central $\chi^2$-distribution}
{Prob? mpNum? A real number between 0 and 1.}
{n? mpNum? A real number greater 0, representing the degrees of freedom}
{Output? String? A string describing the output choices}
\end{mpFunctionsExtract}

\begin{mpFunctionsExtract}
\mpFunctionTwo
{CDistInfo? mpNumList? moments and related information for the central $\chi^2$-distribution}
{n? mpNum? A real number greater 0, representing the degrees of freedom}
{Output? String? A string describing the output choices}
\end{mpFunctionsExtract}

\begin{mpFunctionsExtract}
\mpFunctionFour
{CDistRan? mpNumList? random numbers following a central $\chi^2$-distribution}
{Size? mpNum? A positive integer up to $10^7$}
{n? mpNum? A real number greater 0, representing the degrees of freedom}
{Generator? String? A string describing the random generator}
{Output? String? A string describing the output choices}
\end{mpFunctionsExtract}

\section{Exponential Distribution}

\begin{mpFunctionsExtract}
\mpFunctionThree
{ExponentialDist? mpNumList? returns pdf, CDF and related information for the central Exponential distribution}
{x? mpNum? The value of the distribution.}
{lambda? mpNum? The parameter of the distribution.}
{Output? String? A string describing the output choices}
\end{mpFunctionsExtract}

\begin{mpFunctionsExtract}
\mpFunctionThree
{ExponentialDistInv? mpNumList? returns quantiles and related information for the the central Exponential distribution}
{Prob? mpNum? A real number between 0 and 1.}
{lambda? mpNum? The number of Bernoulli trials.}
{Output? String? A string describing the output choices}
\end{mpFunctionsExtract}

\begin{mpFunctionsExtract}
\mpFunctionTwo
{ExponentialDistInfo? mpNumList? returns moments and related information for the central $t$-distribution}
{lambda? mpNum? A real number greater 0, representing the parameter of the distribution}
{Output? String? A string describing the output choices}
\end{mpFunctionsExtract}

\begin{mpFunctionsExtract}
\mpFunctionFour
{ExponentialDistRandom? mpNumList? returns random numbers following a central Beta-distribution}
{Size? mpNum? A positive integer up to $10^7$}
{lambda? mpNum? A real number greater 0, representing the numerator  degrees of freedom}
{Generator? String? A string describing the random generator}
{Output? String? A string describing the output choices}
\end{mpFunctionsExtract}

\section{Fisher's F-Distribution}

\begin{mpFunctionsExtract}
\mpFunctionFour
{FDist? mpNumList? returns pdf, CDF and related information for the central $F$-distribution}
{x? mpNum? A real number}
{m? mpNum? A real number greater 0, representing the numerator  degrees of freedom}
{n? mpNum? A real number greater 0, representing the denominator degrees of freedom}
{Output? String? A string describing the output choices}
\end{mpFunctionsExtract}

\begin{mpFunctionsExtract}
\mpFunctionThree
{FDistInv? mpNumList? returns quantiles and related information for the the central $t$-distribution}
{Prob? mpNum? A real number between 0 and 1.}
{m? mpNum? A real number greater 0, representing the numerator  degrees of freedom}
{n? mpNum? A real number greater 0, representing the denominator degrees of freedom}
{Output? String? A string describing the output choices}
\end{mpFunctionsExtract}

\begin{mpFunctionsExtract}
\mpFunctionThree
{FDistInfo? mpNumList? returns moments and related information for the central $t$-distribution}
{m? mpNum? A real number greater 0, representing the numerator  degrees of freedom}
{n? mpNum? A real number greater 0, representing the denominator degrees of freedom}
{Output? String? A string describing the output choices}
\end{mpFunctionsExtract}

\begin{mpFunctionsExtract}
\mpFunctionFive
{FDistRan? mpNumList? returns random numbers following a central $F$-distribution}
{Size? mpNum? A positive integer up to $10^7$}
{m? mpNum? A real number greater 0, representing the numerator  degrees of freedom}
{n? mpNum? A real number greater 0, representing the denominator degrees of freedom}
{Generator? String? A string describing the random generator}
{Output? String? A string describing the output choices}
\end{mpFunctionsExtract}

\section{Gamma (and Erlang) Distribution}

\begin{mpFunctionsExtract}
\mpFunctionFour
{GammaDist? mpNumList? returns pdf, CDF and related information for the central Gamma-distribution}
{x? mpNum? A real number}
{a? mpNum? A real number greater 0, a parameter to the distribution}
{b? mpNum? A real number greater 0, a parameter to the distribution}
{Output? String? A string describing the output choices}
\end{mpFunctionsExtract}

\begin{mpFunctionsExtract}
\mpFunctionThree
{GammaDistInv? mpNumList? returns quantiles and related information for the the central Gamma-distribution}
{Prob? mpNum? A real number between 0 and 1.}
{m? mpNum? A real number greater 0, a parameter to the distribution}
{n? mpNum? A real number greater 0, a parameter to the distribution}
{Output? String? A string describing the output choices}
\end{mpFunctionsExtract}

\begin{mpFunctionsExtract}
\mpFunctionTwo
{GammaDistInfo? mpNumList? returns moments and related information for the central Gamma-distribution}
{a? mpNum? A real number greater 0, representing the degrees of freedom}
{b? mpNum? A real number greater 0, representing the degrees of freedom}
{Output? String? A string describing the output choices}
\end{mpFunctionsExtract}

\begin{mpFunctionsExtract}
\mpFunctionFive
{GammaDistRandom? mpNumList? returns random numbers following a central Beta-distribution}
{Size? mpNum? A positive integer up to $10^7$}
{a? mpNum? A real number greater 0, a parameter to the distribution}
{b? mpNum? A real number greater 0, a parameter to the distribution}
{Generator? String? A string describing the random generator}
{Output? String? A string describing the output choices}
\end{mpFunctionsExtract}

\section{Hypergeometric Distribution}

\begin{mpFunctionsExtract}
\mpFunctionFive
{HypergeometricDist? mpNumList? returns pdf, CDF and related information for the central hypergeometric distribution}
{x? mpNum? The number of successes in the sample.}
{n? mpNum? The size of the sample.}
{M? mpNum? The number of successes in the population}
{N? mpNum? The population size}
{Output? String? A string describing the output choices}
\end{mpFunctionsExtract}

\begin{mpFunctionsExtract}
\mpFunctionFive
{HypergeometricDistInv? mpNumList? returns quantiles and related information for the the central hypergeometric distribution}
{Prob? mpNum? A real number between 0 and 1.}
{n? mpNum? The size of the sample.}
{M? mpNum? The number of successes in the population}
{N? mpNum? The population size}
{Output? String? A string describing the output choices}
\end{mpFunctionsExtract}

\begin{mpFunctionsExtract}
\mpFunctionFour
{HypergeometricDistInfo? mpNumList? returns moments and related information for the central hypergeometric distribution}
{n? mpNum? The size of the sample.}
{M? mpNum? The number of successes in the population}
{N? mpNum? The population size}
{Output? String? A string describing the output choices}
\end{mpFunctionsExtract}

\begin{mpFunctionsExtract}
\mpFunctionSix
{HypergeometricDistRandom? mpNumList? returns random numbers following a central hypergeometric distribution}
{Size? mpNum? A positive integer up to $10^7$}
{n? mpNum? The size of the sample.}
{M? mpNum? The number of successes in the population}
{N? mpNum? The population size}
{Generator? String? A string describing the random generator}
{Output? String? A string describing the output choices}
\end{mpFunctionsExtract}

\section{Lognormal Distribution}

\begin{mpFunctionsExtract}
\mpFunctionFour
{LogNormalDist? mpNumList? returns pdf, CDF and related information for the Lognormal-distribution}
{x? mpNum? A real number}
{mean? mpNum? A real number greater 0, representing the mean of the distribution}
{stdev? mpNum? A real number greater 0, representing the standard deviation of the distribution}
{Output? String? A string describing the output choices}
\end{mpFunctionsExtract}

\begin{mpFunctionsExtract}
\mpFunctionFour
{LognormalDistInv? mpNumList? returns quantiles and related information for the the Lognormal-distribution}
{Prob? mpNum? A real number between 0 and 1.}
{mean? mpNum? A real number greater 0, representing the mean of the distribution}
{stdev? mpNum? A real number greater 0, representing the standard deviation of the distribution}
{Output? String? A string describing the output choices}
\end{mpFunctionsExtract}

\begin{mpFunctionsExtract}
\mpFunctionThree
{LognormalDistInfo? mpNumList? returns moments and related information for the central Lognormal-distribution}
{mean? mpNum? A real number greater 0, representing the mean of the distribution}
{stdev? mpNum? A real number greater 0, representing the standard deviation of the distribution}
{Output? String? A string describing the output choices}
\end{mpFunctionsExtract}

\begin{mpFunctionsExtract}
\mpFunctionFive
{LognormalRandom? mpNumList? returns random numbers following a central Beta-distribution}
{Size? mpNum? A positive integer up to $10^7$}
{mean? mpNum? A real number greater 0, representing the mean of the distribution}
{stdev? mpNum? A real number greater 0, representing the standard deviation of the distribution}
{Generator? String? A string describing the random generator}
{Output? String? A string describing the output choices}
\end{mpFunctionsExtract}

\section{Negative Binomial Distribution}

\begin{mpFunctionsExtract}
\mpFunctionFour
{NegativeBinomialDist? mpNumList? returns pdf, CDF and related information for the central negative binomial distribution}
{x? mpNum? The number of failures in trials.}
{r? mpNum? The threshold number of successes.}
{p? mpNum? The probability of a success}
{Output? String? A string describing the output choices}
\end{mpFunctionsExtract}

\begin{mpFunctionsExtract}
\mpFunctionFour
{NegativeBinomialDistInv? mpNumList? returns quantiles and related information for the the central binomial-distribution}
{Prob? mpNum? A real number between 0 and 1.}
{r? mpNum? The threshold number of successes.}
{p? mpNum? The probability of a success}
{Output? String? A string describing the output choices}
\end{mpFunctionsExtract}

\begin{mpFunctionsExtract}
\mpFunctionThree
{NegativeBinomialDistInfo? mpNumList? returns moments and related information for the central Binomial-distribution}
{r? mpNum? The threshold number of successes.}
{p? mpNum? The probability of a success}
{Output? String? A string describing the output choices}
\end{mpFunctionsExtract}

\begin{mpFunctionsExtract}
\mpFunctionFive
{NegativeBinomialDistRandom? mpNumList? returns random numbers following a central Binomial-distribution}
{Size? mpNum? A positive integer up to $10^7$}
{r? mpNum? The threshold number of successes.}
{p? mpNum? The probability of a success}
{Generator? String? A string describing the random generator}
{Output? String? A string describing the output choices}
\end{mpFunctionsExtract}

\section{Normal Distribution}

\begin{mpFunctionsExtract}
\mpFunctionFour
{NDist? mpNumList? returns pdf, CDF and related information for the normal-distribution}
{x? mpNum? A real number}
{mean? mpNum? A real number greater 0, representing the mean of the distribution}
{stdev? mpNum? A real number greater 0, representing the standard deviation of the distribution}
{Output? String? A string describing the output choices}
\end{mpFunctionsExtract}

\begin{mpFunctionsExtract}
\mpFunctionFour
{NDistInv? mpNumList? returns quantiles and related information for the the Lognormal-distribution}
{Prob? mpNum? A real number between 0 and 1.}
{mean? mpNum? A real number greater 0, representing the mean of the distribution}
{stdev? mpNum? A real number greater 0, representing the standard deviation of the distribution}
{Output? String? A string describing the output choices}
\end{mpFunctionsExtract}

\begin{mpFunctionsExtract}
\mpFunctionThree
{NormalDistInfo? mpNumList? returns moments and related information for the central Lognormal-distribution}
{mean? mpNum? A real number greater 0, representing the mean of the distribution}
{stdev? mpNum? A real number greater 0, representing the standard deviation of the distribution}
{Output? String? A string describing the output choices}
\end{mpFunctionsExtract}

\begin{mpFunctionsExtract}
\mpFunctionFive
{NormalRandom? mpNumList? returns random numbers following a central Beta-distribution}
{Size? mpNum? A positive integer up to $10^7$}
{mean? mpNum? A real number greater 0, representing the mean of the distribution}
{stdev? mpNum? A real number greater 0, representing the standard deviation of the distribution}
{Generator? String? A string describing the random generator}
{Output? String? A string describing the output choices}
\end{mpFunctionsExtract}

\section{Poisson Distribution}

\begin{mpFunctionsExtract}
\mpFunctionThree
{PoissonDist? mpNumList? returns pdf, CDF and related information for the Poisson distribution}
{x? mpNum? A real number}
{lambda? mpNum? A real number greater 0, representing the degrees of freedom}
{Output? String? A string describing the output choices}
\end{mpFunctionsExtract}

\begin{mpFunctionsExtract}
\mpFunctionThree
{PoissonDistInv? mpNumList? quantiles and related information for the the Poisson distribution}
{Prob? mpNum? A real number between 0 and 1.}
{lambda? mpNum? A real number greater 0, representing the degrees of freedom}
{Output? String? A string describing the output choices}
\end{mpFunctionsExtract}

\begin{mpFunctionsExtract}
\mpFunctionTwo
{PoissonDistInfo? mpNumList? moments and related information for the Poisson distribution}
{lambda? mpNum? A real number greater 0, representing the degrees of freedom}
{Output? String? A string describing the output choices}
\end{mpFunctionsExtract}

\begin{mpFunctionsExtract}
\mpFunctionFour
{PoissonDistRan? mpNumList? random numbers following a Poisson distribution}
{Size? mpNum? A positive integer up to $10^7$}
{lambda? mpNum? A real number greater 0, representing the degrees of freedom}
{Generator? String? A string describing the random generator}
{Output? String? A string describing the output choices}
\end{mpFunctionsExtract}

\section{Student's t-Distribution}

\begin{mpFunctionsExtract}
\mpFunctionThree
{TDist? mpNumList? returns pdf, CDF and related information for the central $t$-distribution}
{x? mpNum? A real number}
{n? mpNum? A real number greater 0, representing the degrees of freedom}
{Output? String? A string describing the output choices}
\end{mpFunctionsExtract}

\begin{mpFunctionsExtract}
\mpFunctionThree
{TDistInv? mpNumList? returns quantiles and related information for the the central $t$-distribution}
{Prob? mpNum? A real number between 0 and 1.}
{n? mpNum? A real number greater 0, representing the degrees of freedom}
{Output? String? A string describing the output choices}
\end{mpFunctionsExtract}

\begin{mpFunctionsExtract}
\mpFunctionTwo
{TDistInfo? mpNumList? returns moments and related information for the central $t$-distribution}
{n? mpNum? A real number greater 0, representing the degrees of freedom}
{Output? String? A string describing the output choices}
\end{mpFunctionsExtract}

\begin{mpFunctionsExtract}
\mpFunctionFour
{TDistRan? mpNumList? returns random numbers following a central $t$-distribution}
{Size? mpNum? A positive integer up to $10^7$}
{n? mpNum? A real number greater 0, representing the degrees of freedom}
{Generator? String? A string describing the random generator}
{Output? String? A string describing the output choices}
\end{mpFunctionsExtract}

\section{Weibull Distribution}

\begin{mpFunctionsExtract}
\mpFunctionFour
{WeibullDist? mpNumList? returns pdf, CDF and related information for the Weibull distribution}
{x? mpNum? A real number}
{a? mpNum? A real number greater 0, representing the numerator  degrees of freedom}
{b? mpNum? A real number greater 0, representing the denominator degrees of freedom}
{Output? String? A string describing the output choices}
\end{mpFunctionsExtract}

\begin{mpFunctionsExtract}
\mpFunctionFour
{WeibullDistInv? mpNumList? returns quantiles and related information for the the central Beta-distribution}
{Prob? mpNum? A real number between 0 and 1.}
{a? mpNum? A real number greater 0, representing the numerator  degrees of freedom}
{b? mpNum? A real number greater 0, representing the denominator degrees of freedom}
{Output? String? A string describing the output choices}
\end{mpFunctionsExtract}

\begin{mpFunctionsExtract}
\mpFunctionThree
{WeibullDistInfo? mpNumList? returns moments and related information for the central Beta-distribution}
{a? mpNum? A real number greater 0, representing the degrees of freedom}
{b? mpNum? A real number greater 0, representing the degrees of freedom}
{Output? String? A string describing the output choices}
\end{mpFunctionsExtract}

\begin{mpFunctionsExtract}
\mpFunctionFive
{WeibullDistRandom? mpNumList? returns random numbers following a central Beta-distribution}
{Size? mpNum? A positive integer up to $10^7$}
{a? mpNum? A real number greater 0, representing the numerator  degrees of freedom}
{b? mpNum? A real number greater 0, representing the denominator degrees of freedom}
{Generator? String? A string describing the random generator}
{Output? String? A string describing the output choices}
\end{mpFunctionsExtract}

\section{Bernoulli Distribution}

\begin{mpFunctionsExtract}
\mpFunctionThree
{BernoulliDistBoost? mpNumList? returns pdf, CDF and related information for the central $t$-distribution}
{k? mpNum? A real number, 0 or 1}
{p? mpNum? A real number greater 0, representing the degrees of freedom}
{Output? String? A string describing the output choices}
\end{mpFunctionsExtract}

\begin{mpFunctionsExtract}
\mpFunctionThree
{BernoulliDistInvBoost? mpNumList? returns quantiles and related information for the the central $t$-distribution}
{Prob? mpNum? A real number between 0 and 1.}
{p? mpNum? A real number greater 0, representing the degrees of freedom}
{Output? String? A string describing the output choices}
\end{mpFunctionsExtract}

\begin{mpFunctionsExtract}
\mpFunctionTwo
{BernoulliDistInfoBoost? mpNumList? returns moments and related information for the central $t$-distribution}
{p? mpNum? A real number greater 0, representing the degrees of freedom}
{Output? String? A string describing the output choices}
\end{mpFunctionsExtract}

\begin{mpFunctionsExtract}
\mpFunctionFour
{BernoulliDistRandomBoost? mpNumList? returns random numbers following a central Binomial-distribution}
{Size? mpNum? A positive integer up to $10^7$}
{p? mpNum? The probability of a success on each trial.}
{Generator? String? A string describing the random generator}
{Output? String? A string describing the output choices}
\end{mpFunctionsExtract}

\section{Cauchy Distribution}

\begin{mpFunctionsExtract}
\mpFunctionFour
{CauchyDistBoost? mpNumList? returns pdf, CDF and related information for the Cauchy distribution}
{x? mpNum? A real number}
{a? mpNum? A real number greater 0, representing the numerator  degrees of freedom}
{b? mpNum? A real number greater 0, representing the denominator degrees of freedom}
{Output? String? A string describing the output choices}
\end{mpFunctionsExtract}

\begin{mpFunctionsExtract}
\mpFunctionFour
{CauchyDistInvBoost? mpNumList? returns quantiles and related information for the Cauchy distribution}
{Prob? mpNum? A real number between 0 and 1.}
{a? mpNum? A real number greater 0, representing the numerator  degrees of freedom}
{b? mpNum? A real number greater 0, representing the denominator degrees of freedom}
{Output? String? A string describing the output choices}
\end{mpFunctionsExtract}

\begin{mpFunctionsExtract}
\mpFunctionThree
{CauchyDistInfoBoost? mpNumList? returns moments and related information for the Cauchy distribution}
{a? mpNum? A real number greater 0, representing the degrees of freedom}
{b? mpNum? A real number greater 0, representing the degrees of freedom}
{Output? String? A string describing the output choices}
\end{mpFunctionsExtract}

\begin{mpFunctionsExtract}
\mpFunctionFive
{CauchyDistRandomBoost? mpNumList? returns random numbers following a Cauchy distribution}
{Size? mpNum? A positive integer up to $10^7$}
{a? mpNum? A real number greater 0, representing the numerator  degrees of freedom}
{b? mpNum? A real number greater 0, representing the denominator degrees of freedom}
{Generator? String? A string describing the random generator}
{Output? String? A string describing the output choices}
\end{mpFunctionsExtract}

\section{Extreme Value (or Gumbel) Distribution}

\begin{mpFunctionsExtract}
\mpFunctionFour
{ExtremevalueDistBoost? mpNumList? returns pdf, CDF and related information for the Extreme Value distribution}
{x? mpNum? A real number}
{a? mpNum? A real number greater 0, representing the numerator  degrees of freedom}
{b? mpNum? A real number greater 0, representing the denominator degrees of freedom}
{Output? String? A string describing the output choices}
\end{mpFunctionsExtract}

\begin{mpFunctionsExtract}
\mpFunctionFour
{ExtremevalueDistInvBoost? mpNumList? returns quantiles and related information for the the Extreme Value distribution}
{Prob? mpNum? A real number between 0 and 1.}
{a? mpNum? A real number greater 0, representing the numerator  degrees of freedom}
{b? mpNum? A real number greater 0, representing the denominator degrees of freedom}
{Output? String? A string describing the output choices}
\end{mpFunctionsExtract}

\begin{mpFunctionsExtract}
\mpFunctionThree
{ExtremevalueDistInfoBoost? mpNumList? returns moments and related information for the Extreme Value distribution}
{a? mpNum? A real number greater 0, representing the degrees of freedom}
{b? mpNum? A real number greater 0, representing the degrees of freedom}
{Output? String? A string describing the output choices}
\end{mpFunctionsExtract}

\begin{mpFunctionsExtract}
\mpFunctionFive
{ExtremevalueDistRandomBoost? mpNumList? returns random numbers following a Extreme Value distribution}
{Size? mpNum? A positive integer up to $10^7$}
{a? mpNum? A real number greater 0, representing the numerator  degrees of freedom}
{b? mpNum? A real number greater 0, representing the denominator degrees of freedom}
{Generator? String? A string describing the random generator}
{Output? String? A string describing the output choices}
\end{mpFunctionsExtract}

\section{Geometric Distribution}

\begin{mpFunctionsExtract}
\mpFunctionThree
{GeometricDistBoost? mpNumList? returns pdf, CDF and related information for the Geometric distribution}
{k? mpNum? A real number}
{p? mpNum? A real number greater 0, representing the numerator  degrees of freedom}
{Output? String? A string describing the output choices}
\end{mpFunctionsExtract}

\begin{mpFunctionsExtract}
\mpFunctionThree
{GeometricDistInvBoost? mpNumList? returns quantiles and related information for the Geometric distribution}
{Prob? mpNum? A real number between 0 and 1.}
{p? mpNum? A real number greater 0, representing the numerator  degrees of freedom}
{Output? String? A string describing the output choices}
\end{mpFunctionsExtract}

\begin{mpFunctionsExtract}
\mpFunctionTwo
{GeometricDistInfoBoost? mpNumList? returns moments and related information for the Geometric distribution}
{p? mpNum? A real number greater 0, representing the degrees of freedom}
{Output? String? A string describing the output choices}
\end{mpFunctionsExtract}

\begin{mpFunctionsExtract}
\mpFunctionFour
{GeometricDistRandomBoost? mpNumList? returns random numbers following a Geometric distribution}
{Size? mpNum? A positive integer up to $10^7$}
{p? mpNum? A real number greater 0, representing the denominator degrees of freedom}
{Generator? String? A string describing the random generator}
{Output? String? A string describing the output choices}
\end{mpFunctionsExtract}

\section{Inverse Chi Squared Distribution}

\begin{mpFunctionsExtract}
\mpFunctionThree
{InverseChiSquaredDistBoost? mpNumList? returns pdf, CDF and related information for the inverse-chi-squared -distribution}
{x? mpNum? A real number}
{n? mpNum? A real number greater 0, representing the degrees of freedom}
{Output? String? A string describing the output choices}
\end{mpFunctionsExtract}

\begin{mpFunctionsExtract}
\mpFunctionThree
{InverseChiSquaredDistInvBoost? mpNumList? quantiles and related information for the inverse-chi-squared distribution}
{Prob? mpNum? A real number between 0 and 1.}
{n? mpNum? A real number greater 0, representing the degrees of freedom}
{Output? String? A string describing the output choices}
\end{mpFunctionsExtract}

\begin{mpFunctionsExtract}
\mpFunctionTwo
{InverseChiSquaredDistInfoBoost? mpNumList? moments and related information for the inverse-chi-squared distribution}
{n? mpNum? A real number greater 0, representing the degrees of freedom}
{Output? String? A string describing the output choices}
\end{mpFunctionsExtract}

\begin{mpFunctionsExtract}
\mpFunctionFour
{InverseChiSquaredDistRanBoost? mpNumList? random numbers following a inverse-chi-squared distribution}
{Size? mpNum? A positive integer up to $10^7$}
{n? mpNum? A real number greater 0, representing the degrees of freedom}
{Generator? String? A string describing the random generator}
{Output? String? A string describing the output choices}
\end{mpFunctionsExtract}

\section{Inverse Gamma Distribution}

\begin{mpFunctionsExtract}
\mpFunctionFour
{InverseGammaDistBoost? mpNumList? returns pdf, CDF and related information for the inverse gamma distribution}
{x? mpNum? A real number}
{a? mpNum? A real number greater 0, representing the numerator  degrees of freedom}
{b? mpNum? A real number greater 0, representing the denominator degrees of freedom}
{Output? String? A string describing the output choices}
\end{mpFunctionsExtract}

\begin{mpFunctionsExtract}
\mpFunctionFour
{InverseGammaDistInvBoost? mpNumList? returns quantiles and related information for the the inverse gamma distribution}
{Prob? mpNum? A real number between 0 and 1.}
{m? mpNum? A real number greater 0, representing the numerator  degrees of freedom}
{n? mpNum? A real number greater 0, representing the denominator degrees of freedom}
{Output? String? A string describing the output choices}
\end{mpFunctionsExtract}

\begin{mpFunctionsExtract}
\mpFunctionThree
{InverseGammaDistInfoBoost? mpNumList? returns moments and related information for the inverse gamma distribution}
{a? mpNum? A real number greater 0, representing the degrees of freedom}
{b? mpNum? A real number greater 0, representing the degrees of freedom}
{Output? String? A string describing the output choices}
\end{mpFunctionsExtract}

\begin{mpFunctionsExtract}
\mpFunctionFive
{InverseGammaDistRanBoost? mpNumList? returns random numbers following a inverse gamma distribution}
{Size? mpNum? A positive integer up to $10^7$}
{a? mpNum? A real number greater 0, representing the numerator  degrees of freedom}
{b? mpNum? A real number greater 0, representing the denominator degrees of freedom}
{Generator? String? A string describing the random generator}
{Output? String? A string describing the output choices}
\end{mpFunctionsExtract}

\section{Inverse Gaussian (or Wald) Distribution}

\begin{mpFunctionsExtract}
\mpFunctionFour
{InverseGaussianDistBoost? mpNumList? returns pdf, CDF and related information for the inverse Gaussian distribution}
{x? mpNum? A real number}
{mu? mpNum? A real number greater 0, representing the numerator  degrees of freedom}
{lambda? mpNum? A real number greater 0, representing the denominator degrees of freedom}
{Output? String? A string describing the output choices}
\end{mpFunctionsExtract}

\begin{mpFunctionsExtract}
\mpFunctionFour
{InverseGaussianDistInvBoost? mpNumList? returns quantiles and related information for the the inverse Gaussian distribution}
{Prob? mpNum? A real number between 0 and 1.}
{mu? mpNum? A real number greater 0, representing the numerator  degrees of freedom}
{lambda? mpNum? A real number greater 0, representing the denominator degrees of freedom}
{Output? String? A string describing the output choices}
\end{mpFunctionsExtract}

\begin{mpFunctionsExtract}
\mpFunctionThree
{InverseGaussianDistInfoBoost? mpNumList? returns moments and related information for the inverse Gaussian distribution}
{mu? mpNum? A real number greater 0, representing the degrees of freedom}
{lambda? mpNum? A real number greater 0, representing the degrees of freedom}
{Output? String? A string describing the output choices}
\end{mpFunctionsExtract}

\begin{mpFunctionsExtract}
\mpFunctionFive
{InverseGaussianDistRanBoost? mpNumList? returns random numbers following a inverse Gaussian distribution}
{Size? mpNum? A positive integer up to $10^7$}
{mu? mpNum? A real number greater 0, representing the numerator  degrees of freedom}
{lambda? mpNum? A real number greater 0, representing the denominator degrees of freedom}
{Generator? String? A string describing the random generator}
{Output? String? A string describing the output choices}
\end{mpFunctionsExtract}

\section{Laplace Distribution}

\begin{mpFunctionsExtract}
\mpFunctionFour
{LaplaceDistBoost? mpNumList? returns pdf, CDF and related information for the Laplace distribution}
{x? mpNum? A real number}
{a? mpNum? A real number greater 0, representing the numerator  degrees of freedom}
{b? mpNum? A real number greater 0, representing the denominator degrees of freedom}
{Output? String? A string describing the output choices}
\end{mpFunctionsExtract}

\begin{mpFunctionsExtract}
\mpFunctionFour
{LaplaceDistInvBoost? mpNumList? returns quantiles and related information for the the Laplace distribution}
{Prob? mpNum? A real number between 0 and 1.}
{a? mpNum? A real number greater 0, representing the numerator  degrees of freedom}
{b? mpNum? A real number greater 0, representing the denominator degrees of freedom}
{Output? String? A string describing the output choices}
\end{mpFunctionsExtract}

\begin{mpFunctionsExtract}
\mpFunctionThree
{LaplaceDistInfoBoost? mpNumList? returns moments and related information for the Laplace distribution}
{a? mpNum? A real number greater 0, representing the degrees of freedom}
{b? mpNum? A real number greater 0, representing the degrees of freedom}
{Output? String? A string describing the output choices}
\end{mpFunctionsExtract}

\begin{mpFunctionsExtract}
\mpFunctionFive
{LaplaceDistRanBoost? mpNumList? returns random numbers following a Laplace distribution}
{Size? mpNum? A positive integer up to $10^7$}
{a? mpNum? A real number greater 0, representing the numerator  degrees of freedom}
{b? mpNum? A real number greater 0, representing the denominator degrees of freedom}
{Generator? String? A string describing the random generator}
{Output? String? A string describing the output choices}
\end{mpFunctionsExtract}

\section{Logistic Distribution}

\begin{mpFunctionsExtract}
\mpFunctionFour
{LogisticDistBoost? mpNumList? returns pdf, CDF and related information for the Logistic distribution}
{x? mpNum? A real number}
{a? mpNum? A real number greater 0, representing the numerator  degrees of freedom}
{b? mpNum? A real number greater 0, representing the denominator degrees of freedom}
{Output? String? A string describing the output choices}
\end{mpFunctionsExtract}

\begin{mpFunctionsExtract}
\mpFunctionFour
{LogisticDistInvBoost? mpNumList? returns quantiles and related information for the the Logistic distribution}
{Prob? mpNum? A real number between 0 and 1.}
{a? mpNum? A real number greater 0, representing the numerator  degrees of freedom}
{b? mpNum? A real number greater 0, representing the denominator degrees of freedom}
{Output? String? A string describing the output choices}
\end{mpFunctionsExtract}

\begin{mpFunctionsExtract}
\mpFunctionThree
{LogisticDistInfoBoost? mpNumList? returns moments and related information for the Logistic distribution}
{a? mpNum? A real number greater 0, representing the degrees of freedom}
{b? mpNum? A real number greater 0, representing the degrees of freedom}
{Output? String? A string describing the output choices}
\end{mpFunctionsExtract}

\begin{mpFunctionsExtract}
\mpFunctionFive
{LogisticDistRanBoost? mpNumList? returns random numbers following a Logistic distribution}
{Size? mpNum? A positive integer up to $10^7$}
{a? mpNum? A real number greater 0, representing the numerator  degrees of freedom}
{b? mpNum? A real number greater 0, representing the denominator degrees of freedom}
{Generator? String? A string describing the random generator}
{Output? String? A string describing the output choices}
\end{mpFunctionsExtract}

\section{Pareto Distribution}

\begin{mpFunctionsExtract}
\mpFunctionFour
{ParetoDistBoost? mpNumList? returns pdf, CDF and related information for the Pareto distribution}
{x? mpNum? A real number}
{a? mpNum? A real number greater 0, representing the numerator  degrees of freedom}
{b? mpNum? A real number greater 0, representing the denominator degrees of freedom}
{Output? String? A string describing the output choices}
\end{mpFunctionsExtract}

\begin{mpFunctionsExtract}
\mpFunctionFour
{ParetoDistInvBoost? mpNumList? returns quantiles and related information for the the Pareto distribution}
{Prob? mpNum? A real number between 0 and 1.}
{a? mpNum? A real number greater 0, representing the numerator  degrees of freedom}
{b? mpNum? A real number greater 0, representing the denominator degrees of freedom}
{Output? String? A string describing the output choices}
\end{mpFunctionsExtract}

\begin{mpFunctionsExtract}
\mpFunctionThree
{ParetoDistInfoBoost? mpNumList? returns moments and related information for the Pareto distribution}
{a? mpNum? A real number greater 0, representing the degrees of freedom}
{b? mpNum? A real number greater 0, representing the degrees of freedom}
{Output? String? A string describing the output choices}
\end{mpFunctionsExtract}

\begin{mpFunctionsExtract}
\mpFunctionFive
{ParetoDistRanBoost? mpNumList? returns random numbers following a Pareto distribution}
{Size? mpNum? A positive integer up to $10^7$}
{a? mpNum? A real number greater 0, representing the numerator  degrees of freedom}
{b? mpNum? A real number greater 0, representing the denominator degrees of freedom}
{Generator? String? A string describing the random generator}
{Output? String? A string describing the output choices}
\end{mpFunctionsExtract}

\section{Raleigh Distribution}

\begin{mpFunctionsExtract}
\mpFunctionThree
{RaleighDistBoost? mpNumList? returns pdf, CDF and related information for the Raleigh distribution}
{x? mpNum? A real number}
{n? mpNum? A real number greater 0, representing the degrees of freedom}
{Output? String? A string describing the output choices}
\end{mpFunctionsExtract}

\begin{mpFunctionsExtract}
\mpFunctionThree
{RaleighDistInvBoost? mpNumList? quantiles and related information for the Raleigh distribution}
{Prob? mpNum? A real number between 0 and 1.}
{n? mpNum? A real number greater 0, representing the degrees of freedom}
{Output? String? A string describing the output choices}
\end{mpFunctionsExtract}

\begin{mpFunctionsExtract}
\mpFunctionTwo
{RaleighDistInfoBoost? mpNumList? moments and related information for the Raleigh distribution}
{n? mpNum? A real number greater 0, representing the degrees of freedom}
{Output? String? A string describing the output choices}
\end{mpFunctionsExtract}

\begin{mpFunctionsExtract}
\mpFunctionFour
{RaleighDistRanBoost? mpNumList? random numbers following a Raleigh distribution}
{Size? mpNum? A positive integer up to $10^7$}
{n? mpNum? A real number greater 0, representing the degrees of freedom}
{Generator? String? A string describing the random generator}
{Output? String? A string describing the output choices}
\end{mpFunctionsExtract}

\section{Triangular Distribution}

\begin{mpFunctionsExtract}
\mpFunctionFive
{TriangularDistBoost? mpNumList? returns pdf, CDF and related information for the triangular distribution}
{x? mpNum? A real number.}
{a? mpNum? The left border parameter.}
{b? mpNum? The right border parameter.}
{c? mpNum? The mode parameter.}
{Output? String? A string describing the output choices}
\end{mpFunctionsExtract}

\begin{mpFunctionsExtract}
\mpFunctionFive
{TriangularDistInvBoost? mpNumList? returns quantiles and related information for the the triangular distribution}
{Prob? mpNum? A real number between 0 and 1.}
{a? mpNum? The left border parameter.}
{b? mpNum? The right border parameter.}
{c? mpNum? The mode parameter.}
{Output? String? A string describing the output choices}
\end{mpFunctionsExtract}

\begin{mpFunctionsExtract}
\mpFunctionFour
{TriangularDistInfoBoost? mpNumList? returns moments and related information for the triangular distribution}
{a? mpNum? The left border parameter.}
{b? mpNum? The right border parameter.}
{c? mpNum? The mode parameter.}
{Output? String? A string describing the output choices}
\end{mpFunctionsExtract}

\begin{mpFunctionsExtract}
\mpFunctionSix
{TriangularDistRanBoost? mpNumList? returns random numbers following a triangular distribution}
{Size? mpNum? A positive integer up to $10^7$}
{a? mpNum? The left border parameter.}
{b? mpNum? The right border parameter.}
{c? mpNum? The mode parameter.}
{Generator? String? A string describing the random generator}
{Output? String? A string describing the output choices}
\end{mpFunctionsExtract}

\section{Uniform Distribution}

\begin{mpFunctionsExtract}
\mpFunctionFour
{UniformDistBoost? mpNumList? returns pdf, CDF and related information for the uniform distribution}
{x? mpNum? A real number}
{a? mpNum? The left border parameter.}
{b? mpNum? The right border parameter.}
{Output? String? A string describing the output choices}
\end{mpFunctionsExtract}

\begin{mpFunctionsExtract}
\mpFunctionFour
{UniformDistInvBoost? mpNumList? returns quantiles and related information for the the uniform distribution}
{Prob? mpNum? A real number between 0 and 1.}
{a? mpNum? The left border parameter.}
{b? mpNum? The right border parameter.}
{Output? String? A string describing the output choices}
\end{mpFunctionsExtract}

\begin{mpFunctionsExtract}
\mpFunctionThree
{UniformDistInfoBoost? mpNumList? returns moments and related information for the uniform distribution}
{a? mpNum? A real number greater 0, representing the degrees of freedom}
{b? mpNum? A real number greater 0, representing the degrees of freedom}
{Output? String? A string describing the output choices}
\end{mpFunctionsExtract}

\begin{mpFunctionsExtract}
\mpFunctionFive
{UniformDistRanBoost? mpNumList? returns random numbers following a uniform distribution}
{Size? mpNum? A positive integer up to $10^7$}
{a? mpNum? A real number greater 0, representing the degrees of freedom}
{b? mpNum? A real number greater 0, representing the degrees of freedom}
{Generator? String? A string describing the random generator}
{Output? String? A string describing the output choices}
\end{mpFunctionsExtract}

\chapter{Noncentral Distribution Functions (based on Boost)}

\section{Noncentral Beta-Distribution}

\begin{mpFunctionsExtract}
\mpFunctionFive
{NoncentralBetaDistBoost? mpNumList? returns pdf, CDF and related information for the central Beta-distribution}
{x? mpNum? A real number}
{m? mpNum? A real number greater 0, representing the numerator  degrees of freedom}
{n? mpNum? A real number greater 0, representing the denominator degrees of freedom}
{lambda? mpNum? A real number greater 0, representing the noncentrality parameter}
{Output? String? A string describing the output choices}
\end{mpFunctionsExtract}

\begin{mpFunctionsExtract}
\mpFunctionFive
{NoncentralBetaDistInvBoost? mpNumList? returns quantiles and related information for the the noncentral Beta-distribution}
{Prob? mpNum? A real number between 0 and 1.}
{m? mpNum? A real number greater 0, representing the numerator  degrees of freedom}
{n? mpNum? A real number greater 0, representing the denominator degrees of freedom}
{lambda? mpNum? A real number greater 0, representing the noncentrality parameter}
{Output? String? A string describing the output choices}
\end{mpFunctionsExtract}

\begin{mpFunctionsExtract}
\mpFunctionFour
{NoncentralBetaDistInfoBoost? mpNumList? returns moments and related information for the noncentral Beta-distribution}
{m? mpNum? A real number greater 0, representing the numerator  degrees of freedom}
{n? mpNum? A real number greater 0, representing the denominator degrees of freedom}
{lambda? mpNum? A real number greater 0, representing the noncentrality parameter}
{Output? String? A string describing the output choices}
\end{mpFunctionsExtract}

\begin{mpFunctionsExtract}
\mpFunctionSix
{NoncentralBetaDistRanBoost? mpNumList? returns random numbers following a noncentral Beta-distribution}
{Size? mpNum? A positive integer up to $10^7$}
{m? mpNum? A real number greater 0, representing the numerator  degrees of freedom}
{n? mpNum? A real number greater 0, representing the denominator degrees of freedom}
{lambda? mpNum? A real number greater 0, representing the noncentrality parameter}
{Generator? String? A string describing the random generator}
{Output? String? A string describing the output choices}
\end{mpFunctionsExtract}

\section{Noncentral Chi-Square Distribution}

\begin{mpFunctionsExtract}
\mpFunctionFour
{NoncentralCDistBoost? mpNumList? returns pdf, CDF and related information for the noncentral $\chi^2$-distribution}
{x? mpNum? A real number}
{n? mpNum? A real number greater 0, representing the degrees of freedom}
{lambda? mpNum? A real number greater 0, representing the noncentrality parameter}
{Output? String? A string describing the output choices}
\end{mpFunctionsExtract}

\begin{mpFunctionsExtract}
\mpFunctionFour
{NoncentralCDistInvBoost? mpNumList? quantiles and related information for the noncentral $\chi^2$-distribution}
{Prob? mpNum? A real number between 0 and 1.}
{n? mpNum? A real number greater 0, representing the degrees of freedom}
{lambda? mpNum? A real number greater 0, representing the noncentrality parameter}
{Output? String? A string describing the output choices}
\end{mpFunctionsExtract}

\begin{mpFunctionsExtract}
\mpFunctionThree
{NoncentralCDistInfoBoost? mpNumList? moments and related information for the noncentral $\chi^2$-distribution}
{n? mpNum? A real number greater 0, representing the degrees of freedom}
{lambda? mpNum? A real number greater 0, representing the noncentrality parameter}
{Output? String? A string describing the output choices}
\end{mpFunctionsExtract}

\begin{mpFunctionsExtract}
\mpFunctionFive
{NoncentralCDistRanBoost? mpNumList? random numbers following a noncentral $\chi^2$-distribution}
{Size? mpNum? A positive integer up to $10^7$}
{n? mpNum? A real number greater 0, representing the degrees of freedom}
{lambda? mpNum? A real number greater 0, representing the noncentrality parameter}
{Generator? String? A string describing the random generator}
{Output? String? A string describing the output choices}
\end{mpFunctionsExtract}

\section{NonCentral F-Distribution}

\begin{mpFunctionsExtract}
\mpFunctionFive
{NoncentralFDistBoost? mpNumList? returns pdf, CDF and related information for the noncentral $F$-distribution}
{x? mpNum? A real number}
{m? mpNum? A real number greater 0, representing the numerator  degrees of freedom}
{n? mpNum? A real number greater 0, representing the denominator degrees of freedom}
{lambda? mpNum? A real number greater 0, representing the noncentrality parameter}
{Output? String? A string describing the output choices}
\end{mpFunctionsExtract}

\begin{mpFunctionsExtract}
\mpFunctionFive
{NoncentralFDistInvBoost? mpNumList? returns quantiles and related information for the the noncentral $F$-distribution}
{Prob? mpNum? A real number between 0 and 1.}
{m? mpNum? A real number greater 0, representing the numerator  degrees of freedom}
{n? mpNum? A real number greater 0, representing the denominator degrees of freedom}
{lambda? mpNum? A real number greater 0, representing the noncentrality parameter}
{Output? String? A string describing the output choices}
\end{mpFunctionsExtract}

\begin{mpFunctionsExtract}
\mpFunctionFour
{NoncentralFDistInfoBoost? mpNumList? returns moments and related information for the noncentral $F$-distribution}
{m? mpNum? A real number greater 0, representing the numerator  degrees of freedom}
{n? mpNum? A real number greater 0, representing the denominator degrees of freedom}
{lambda? mpNum? A real number greater 0, representing the noncentrality parameter}
{Output? String? A string describing the output choices}
\end{mpFunctionsExtract}

\begin{mpFunctionsExtract}
\mpFunctionSix
{NoncentralFDistRanBoost? mpNumList? returns random numbers following a noncentral $F$-distribution}
{Size? mpNum? A positive integer up to $10^7$}
{m? mpNum? A real number greater 0, representing the numerator  degrees of freedom}
{n? mpNum? A real number greater 0, representing the denominator degrees of freedom}
{lambda? mpNum? A real number greater 0, representing the noncentrality parameter}
{Generator? String? A string describing the random generator}
{Output? String? A string describing the output choices}
\end{mpFunctionsExtract}

\section{Noncentral Student's t-Distribution}

\begin{mpFunctionsExtract}
\mpFunctionFour
{NoncentralTDistBoost? mpNumList? returns pdf, CDF and related information for the noncentral $t$-distribution}
{x? mpNum? A real number}
{n? mpNum? A real number greater 0, representing the degrees of freedom}
{delta? mpNum? A real number greater 0, representing the noncentrality parameter}
{Output? String? A string describing the output choices}
\end{mpFunctionsExtract}

\begin{mpFunctionsExtract}
\mpFunctionFour
{NoncentralTDistInvBoost? mpNumList? quantiles and related information for the  noncentral $t$-distribution}
{Prob? mpNum? A real number between 0 and 1.}
{n? mpNum? A real number greater 0, representing the degrees of freedom}
{delta? mpNum? A real number greater 0, representing the noncentrality parameter}
{Output? String? A string describing the output choices}
\end{mpFunctionsExtract}

\begin{mpFunctionsExtract}
\mpFunctionThree
{NoncentralTDistInfoBoost? mpNumList? moments and related information for the noncentral $t$-distribution}
{n? mpNum? A real number greater 0, representing the degrees of freedom}
{delta? mpNum? A real number greater 0, representing the noncentrality parameter}
{Output? String? A string describing the output choices}
\end{mpFunctionsExtract}

\begin{mpFunctionsExtract}
\mpFunctionFive
{NoncentralTDistRanBoost? mpNumList? random numbers following a noncentral $t$-distribution}
{Size? mpNum? A positive integer up to $10^7$}
{n? mpNum? A real number greater 0, representing the degrees of freedom}
{delta? mpNum? A real number greater 0, representing the noncentrality parameter}
{Generator? String? A string describing the random generator}
{Output? String? A string describing the output choices}
\end{mpFunctionsExtract}

\section{Skew Normal Distribution}

\begin{mpFunctionsExtract}
\mpFunctionFive
{SkewNormalDistBoost? mpNumList? returns pdf, CDF and related information for the skew normal distribution}
{x? mpNum? A real number.}
{a? mpNum? The location parameter.}
{b? mpNum? The scale parameter}
{c? mpNum? The shape parameter}
{Output? String? A string describing the output choices}
\end{mpFunctionsExtract}

\begin{mpFunctionsExtract}
\mpFunctionFive
{SkewNormalDistInvBoost? mpNumList? returns quantiles and related information for the the skew normal distribution}
{Prob? mpNum? A real number between 0 and 1.}
{a? mpNum? The location parameter.}
{b? mpNum? The scale parameter}
{c? mpNum? The shape parameter}
{Output? String? A string describing the output choices}
\end{mpFunctionsExtract}

\begin{mpFunctionsExtract}
\mpFunctionFour
{SkewNormalDistInfoBoost? mpNumList? returns moments and related information for the skew normal distribution}
{a? mpNum? The location parameter.}
{b? mpNum? The scale parameter}
{c? mpNum? The shape parameter}
{Output? String? A string describing the output choices}
\end{mpFunctionsExtract}

\begin{mpFunctionsExtract}
\mpFunctionSix
{SkewNormalDistRanBoost? mpNumList? returns random numbers following a skew normal distribution}
{Size? mpNum? A positive integer up to $10^7$}
{a? mpNum? The location parameter.}
{b? mpNum? The scale parameter}
{c? mpNum? The shape parameter}
{Generator? String? A string describing the random generator}
{Output? String? A string describing the output choices}
\end{mpFunctionsExtract}

\section{Owen's T-Function}

\begin{mpFunctionsExtract}
\mpFunctionTwo
{TOwenBoost? mpNum? Owen's T-Function}
{h? mpNum? A real number.}
{a? mpNum? A real number.}
\end{mpFunctionsExtract}

\chapter{Ordinary Differential Equations}

\section{Defining the ODE System}

\section{Stepping Functions}

\section{Integrate functions: Evolution}

\chapter{Nonlinear Root-finding, Minimization and Optimization}

\section{One-Dimensional Root-finding}

\section{One-Dimensional Minimization}

\section{Procedures based on MINPACK}

\section{Procedures based on NLOPT: Overview}

\section{NLOPT: Global optimization}

\section{NLOPT: Local derivative-free optimization}

\section{NLOPT: Local gradient-based optimization}

\section{NLOPT: Augmented Lagrangian algorithm}

\chapter{Interfaces}

\section{Interfaces to the C family of languages}

\section{Component Object Model (COM) Interface}

\section{Languages with CLR Support}

\section{Java (via jni4net)}

\section{SQLite and System.Data.SQLite}

\section{gnuplot}

\chapter{Building the library}

\section{Building the Library, Part 1}

\section{Building the Library, Part 2}

\section{Building the Library, Part 3}

\section{Building the documentation}

\section{Additional libraries}

\section{Working Notes}

\section{Where to find VB Code}

\section{How to run Permutation Code}

\chapter{Acknowledgements}

\section{Contributors to libraries used in the numerical routines}

\chapter{Licenses}

\section{GNU Licenses}

\section{Other Licenses}

\chapter{\tocbibname }

\end{document}
